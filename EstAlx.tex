\documentclass[twoside]{report}
\usepackage[utf8]{inputenc}
\usepackage[galician,english]{babel}
\usepackage[dvipsnames]{xcolor}
\usepackage{amssymb}
\usepackage{amsthm,xcolor}
\usepackage{mathtools}
\usepackage{fancyhdr}
\usepackage[Conny]{fncychap} %--Paquete para os encabezados dos capítulos
\usepackage{tikz} %--Para os gráficos
\usepackage[all]{xy} %--Diagramas conmutativos
\usepackage{mdframed} %--Cadros dos teoremas
\usepackage{tocloft} %--Para xogar coa táboa de contidos
\usepackage{hyperref}
\usepackage{mathrsfs}
\usepackage{xstring}
\usepackage{cancel} %--Riscar elementos
\usepackage{multicol}%--Distribución en columnas
\usepackage{fix-cm} %--Cambiar o tamaño de letra
\usepackage{geometry}

 \geometry{
 a4paper,
 left=25mm,
 top=25mm,
 right =25mm,
 bottom=20mm,
 }

 
 \usetikzlibrary{matrix,decorations.pathreplacing, calc, positioning,fit} %-Debuxar e señalar sobre as matrices
 
%-----------------------------------------------------------------------------------------%

%--------------SE O DOCUMENTO É ARTICLE, NON ADMITE CAPÍTULOS----------------------------%

\usepackage{xpatch}
\newlength{\chaptertopskip}
\setlength{\chaptertopskip}{0pt}
\makeatletter
\xpatchcmd{\@makechapterhead}{\vspace*{50\p@}}{\vspace*{\chaptertopskip}}{\typeout{Success}}{\typeout{Failure!!!}} 
\makeatother %----Move o título do capítulo para arriba---%

\renewcommand{\cfttoctitlefont}{\huge\bfseries} %Fonte do título da táboa de contidos

\addto\captionsgalician{% Replace "english" with the language you use
  \renewcommand{\contentsname}%
    {\magbf{Estruturas Alxébricas. Curso 2019-2020}}%
}

% Texto do título da táboa de contidos
\addto\captionsenglish{% Replace "english" with the language you use
  \renewcommand{\contentsname}%
    {\magbf{Estruturas Alxébricas. Curso 2019-2020}}%
}

\makeatletter
\renewcommand{\@chapapp}{Unidade} %--Pón Unidade en vez de Capítulo--%
\makeatother

\makeatletter %--Eliminar numeración dos capítulos na táboa de contidos
\renewcommand{\cftchappresnum}{\begin{lrbox}{\@tempboxa}}
\renewcommand{\cftchapaftersnum}{\end{lrbox}}
\makeatother

\renewcommand{\thesection}{\arabic{section}} %---Numérase con 1 en vez de 1.1---%

\makeatletter
\renewcommand{\DOCH}{%--Coloréase o título do capítulo (Solo aplica a estilo Conny)
    \color{magenta}\mghrulefill{3\RW}\par\nobreak
    \vskip -0.5\baselineskip
    \mghrulefill{\RW}\par\nobreak
    \CNV\FmN{\@chapapp}\space \CNoV \thechapter
    \par\nobreak
   \vskip -0.5\baselineskip
  }
\makeatother

\renewcommand{\cfttoctitlefont}{\hspace*{\fill}\huge}
\renewcommand{\cftaftertoctitle}{\hspace*{\fill}}

\setlength{\cftbeforetoctitleskip}{-4em}
\setlength\cftaftertoctitleskip{1pt}
\setlength{\cftchapindent}{-1pt}% Just some value...
\setlength{\cftbeforechapskip}{1pt}

%------------------------------------------------------------------------------------------%

\usepackage{etoolbox}

\makeatletter
\pretocmd{\chapter}{\addtocontents{toc}{\protect\addvspace{15\p@}}}{}{}
\pretocmd{\section}{\addtocontents{toc}{\protect\addvspace{5\p@}}}{}{}
\makeatother

%-----------------------------------------------------------------------------------------%

\title{\textbf{ESTRUTURAS ALXÉBRICAS}}
\author{Héctor Varela Rodríguez [hector.varela@rai.usc.es]}
\date{}

%-----------------------------------------------------------------------------------------%
 
\definecolor{classicrose}{rgb}{0.98, 0.8, 0.91} %Definicion do color
\definecolor{deeppink}{rgb}{1.0, 0.08, 0.58}
\definecolor{fashionfuchsia}{rgb}{0.96, 0.0, 0.63}

%----------------------------------------------------------------------------------------%
%Este comando permite deformar os iconos dos items do comando itemize. Neste caso, faríase un cadrado

\usepackage{scalerel}
\def\sq{\mathbin{\scalerel*{\strut\rule[-.5ex]{2ex}{2ex}}{\cdot}}}

\renewcommand{\labelitemi}{$\sq$}

%----------------------------------------------------------------------------------------%

\newtagform{brackets}{(}{)} %--Referencias das ecuacións
\usetagform{brackets}

\newcommand{\magbf}[1]{\textcolor{magenta}{\textbf{#1}}} %----Entre corchetes aparece o número de parámetros. #1 indica onde se inclúe o parámetro----%

\newcommand{\magen}[1]{\textcolor{magenta}{#1}}

\newcommand{\almostall}{\mathrel{\vee\mkern-19mu\frown}} %Para case todo (novo símbolo, non sei de onde saleu pero vale). mkern sirve para poñer espacio negativo

\newcommand{\injarrow}{\mathrel{\longrightarrow\mkern-19mu \tiny |}} %Para case todo (novo símbolo, non sei de onde saleu pero vale). mkern sirve para poñer espacio negativo

%Símbolo = en vertical
\newcommand{\verteq}{\rotatebox{90}{$\,=$}}
\newcommand{\equalto}[2]{\underset{\scriptstyle\overset{\mkern4mu\verteq}{#2}}{#1}}

\newcommand{\chaptitle}[1]{
    \IfEqCase{#1}{ 
        {1}{\textbf{Unidade 1. Grupos}}
        {2}{\textbf{Unidade 2. Aneis}}
        {3}{\textbf{Unidade 3. Módulos}}
        {4}{\textbf{Unidade 4. Módulos de tipo finito sobre un DIP}}
    }[\PackageError{chaptitle}{Undefined option to chaptitle: #1}{}]
}

% - Estilo de páxina sen encabezado
\fancypagestyle{noheader}{
    \fancyfoot[C]{\textbf{\thepage}}
    \fancyhead{}
    \renewcommand{\headrulewidth}{0pt}
}

\pagestyle{fancy} %--Estilo do encabezado e pé de página: NECESARIO ESPECIFICAR twoside EN documentclass
\fancyhf{}
\fancyhead[LE]{\textbf{Estruturas Alxébricas}}
\fancyhead[RE]{Curso 2019 - 2020}
\fancyhead[LO]{Curso 2019 - 2020}
\fancyhead[RO]{\chaptitle{\thechapter}}
\fancyfoot[C]{\textbf{\thepage}}
\renewcommand{\footrulewidth}{0.4pt} %-----Liña no pé de páxina-------%
\renewcommand{\headrule}{\hbox to\headwidth{\color{magenta}\leaders\hrule height \headrulewidth\hfill}}
\renewcommand{\footrule}{\hbox to\headwidth{\color{magenta}\leaders\hrule height \footrulewidth\hfill}}

\makeatletter
\renewcommand{\@seccntformat}[1]{\csname the#1\endcsname.\quad}
\makeatother %--Engádeselle o punto ós índices dos apartados 

\AtBeginDocument{\addtocontents{toc}{\protect\thispagestyle{noheader}}} %--Cambiar o estilo de páxina para a táboa de contidos

% declare a new theorem style
\newtheoremstyle{mystyle}%
{0pt}% Space above
{0pt}% Space below 
{\itshape\color{magenta}}% Body font
{0pt}% Indent amount
{\bfseries\color{magenta}}% Theorem head font
{.}% Punctuation after theorem head
{.5em}% Space after theorem head
{}% Theorem head spec (can be left empty, meaning ‘normal’)

\theoremstyle{mystyle}

%Teoremas, proposicións, lemas e corolarios
\newtheorem{theo}{\magbf{Teorema}}[chapter]
\newtheorem{prop}{\magbf{Proposición}}[chapter]
\newtheorem{lem}{\magbf{Lema}}[chapter]
\newtheorem{cor}{\magbf{Corolario}}[chapter]

\newenvironment{theorem}
{\begin{mdframed}[linecolor = magenta,backgroundcolor = classicrose, linewidth = 2mm]\begin{theo}}
{\end{theo}\end{mdframed}}

\newenvironment{proposition}
{\begin{mdframed}[linecolor = magenta,backgroundcolor = classicrose, linewidth = 2mm]\begin{prop}}
{\end{prop}\end{mdframed}}

\newenvironment{lemma}
{\begin{mdframed}[linecolor = magenta,backgroundcolor = classicrose, linewidth = 2mm]\begin{lem}}
{\end{lem}\end{mdframed}}

\newenvironment{corollary}
{\begin{mdframed}[linecolor = magenta,backgroundcolor = classicrose, linewidth = 2mm]\begin{cor}}
{\end{cor}\end{mdframed}}

%%%-------------------------------------EMPEZAMOS O DOCUMENTO-------------------------------------%%%

\begin{document}

\tableofcontents

\thispagestyle{noheader}

\newpage

\chapter[Unidade 1. Grupos]{\textbf{Grupos}}

\thispagestyle{noheader}

    \noindent A teoría de grupos ten a súa orixe no traballo de Évariste Galois (1811 - 1832) sobre a existencia de raíces da ecuación $a_{n}x^{n} + \dots + a_{1}x + a_{0} = 0$. Non obstante, algúns dos resultados da teoría de grupos xa apareceran anteriormente en traballos doutros matemáticos, entre os que se atopa Cauchy. Por iso, cómpre sinalar que o termo \textit{grupo} foi acuñado por Galois no seu traballo: \textit{Mémoire sur les conditions de résolubilité des équations par radicaux}. Con todo, a formulación axiomática da teoría de grupos como se coñece actualmente inicíase co traballo de Heinrich Martin Weber (1842-1913): \textit{Die allgemeinen Grundlagender Galois’schen Gleichungstheorie}.\\
    
    \noindent Hoxe en día, a teoría de grupos é unha das áreas das matemáticas que máis aplicacións ten. En particular, dentro das ciencias exactas, inclúese áreas tales coma xeometría alxébrica, teoría de números ou topoloxía alxébrica; mentres que na física e na química, por exemplo, axuda no estudo das simetrías das estruturas moleculares.

\textcolor{magenta}{\section{Grupos e subgrupos}}

\textcolor{magenta}{\subsection{Grupos: xeneralidades}}

\vspace{5mm}

\noindent \textbf{Definición 1.1.} Un \textbf{\textcolor{magenta}{grupo}} é un par $( \hspace{1mm} G, \hspace{1mm} \cdot \hspace{1mm})$ , sendo $G$ un conxunto e  $\cdot$ unha operación interna definida nel, isto é, unha aplicación:

\vspace{2mm}

\begin{center}
    $\cdot : G \times G \longrightarrow G$ \\
    \vspace{2mm}
    \hspace{6mm} $(x,y) \leadsto x \cdot y$
\end{center}  


\vspace{2mm}

\noindent a cal verifica os seguintes axiomas:

\vspace{2mm}

\begin{itemize}
    
    \item \textbf{Asociatividade}
    
    \vspace{1mm}
    
    $\forall \hspace{1mm} x,y \in G, \hspace{3mm} x \cdot (y \cdot z) = (x \cdot y) \cdot z$
    
    \item \textbf{Elemento neutro}
    
    \vspace{1mm}
    
    $\exists \hspace{1mm} e \in G \hspace{1mm} | \hspace{1mm} e \cdot x = x = x \cdot e$
    
    \item \textbf{Elemento simétrico}
    
    \vspace{1mm}
    
    $\forall \hspace{1mm} x \in G, \hspace{2mm} \exists \hspace{1mm} x' \in G \hspace{1mm} | \hspace{1mm} x \cdot x' = e = x' \cdot x$
    
\end{itemize}

\vspace{2mm}

\noindent Se, ademais, a operación $\cdot$  é \textbf{conmutativa}, isto é:

\vspace{2mm}

\begin{center}
    $x \cdot y = y \cdot x \hspace{2mm} \forall \hspace{1mm} x,y \in G$
\end{center}

\vspace{2mm}

\noindent dirase que $( \hspace{1mm} G, \hspace{1mm} \cdot \hspace{1mm})$ é un \textcolor{magenta}{\textbf{grupo conmutativo}} ou \textcolor{magenta}{\textbf{abeliano}}.

\vspace{5mm}

\noindent Vexamos a continuación unha serie de exemplos para ilustrar esta definición:

\begin{enumerate}
    \item $(\mathbb{N}, +)$ non é un grupo
    
    \vspace{1mm}
    
    Cúmprese que a suma en $\mathbb{N}$ é asociativa. Ademais, existe elemento neutro para esta operación no conxunto considerado: o cero (0). Porén, non todo número natural posúe elemento simétrico respecto da suma en $\mathbb{N}$; de feito, só o cero cumpre tal propiedade.
    
    \item $(\mathbb{Z}, +)$ si é un grupo, en particular abeliano
    
    \vspace{1mm}
    
    En efecto, a suma de números enteiros é asociativa e conmutativa. O cero é o elemento neutro desta operación, e está en $\mathbb{Z}$. Ademais, todo número enteiro posúe elemento simétrico respecto da suma, que se denomina \textbf{elemento oposto}.
    
    De igual xeito, $(\mathbb{Q},+), \hspace{1mm} (\mathbb{R},+)$ e $(\mathbb{C},+)$ son grupos abelianos.
    
    \item $(\mathbb{Z}, \cdot)$ non é un grupo
    
    \vspace{1mm}
    
    No produto de números enteiros, existe elemento neutro: 1. Ademais, a operación é asociativa. Non obstante, existen números enteiros cuxo simétrico non existe en $\mathbb{Z}$. Incluso existe un elemento sen simétrico respecto do produto: o cero (0).
    
    \item $(\mathbb{Q}, \cdot)$ non é un grupo
    
    \vspace{1mm}
    
    O cero está no conxunto dos números racionais, e non posúe simétrico respecto do produto.
    
    \item $(\mathbb{Q}^{*}, \cdot)$ é un grupo abeliano
    
    \vspace{1mm}
    
    En efecto, no conxunto dos racionais sen o cero, o produto segue sendo unha operación asociativa e conmutativa, con elemento neutro (o 1). Ademais, todo elemento do conxunto posúe simétrico respecto desta operación.
    
    De igual xeito, $(\mathbb{R}^{*}, \cdot)$ e $(\mathbb{C}^{*}, \cdot)$ son grupos abelianos.
    
    \item $(M_{n}(\mathbb{R}), \cdot)$ non é un grupo
    
    \vspace{1mm}
    
    En efecto, non toda matriz cadrada ten inversa.
    
    \item $(GL(n, \mathbb{R}), \cdot)$ é un grupo non abeliano
    
    \vspace{1mm}
    
    Por definición, o grupo linear de matrices reais de orde $n$ engloba tódalas matrices inversibles desa orde. Ademais, o produto de matrices é unha operación asociativa. Non obstante, non é conmutativo.
    
    \item Sexan $(G, \cdot)$ un grupo e X un conxunto. Considérese entón o conxunto:
    
    \begin{center}
        $H = \{ \hspace{1mm} f: X \longrightarrow G \hspace{1mm} | \hspace{1mm} f$ é aplicación$ \hspace{1mm}\}$
    \end{center}
    
    Dadas $f,g \in H$, defínese a operación interna $(f \cdot g)(x) = f(x) \cdot g(x) \hspace{2mm} \forall \hspace{1mm} x \in X$.
    
    O conxunto $H$, coa operación anterior, é un grupo.
    
    \vspace{1mm}
    
    \item Sexa $X$ un conxunto. En xeral, o conxunto $\{ f: X \longrightarrow X \hspace{1MM} | \hspace{1MM} f$ é aplicación$\}$, coa operación composición, $\circ$, non é un grupo, pois non toda aplicación posúe inversa. 
    
    Non obstante, o conxunto $\{ f: X \longrightarrow X \hspace{1mm} | \hspace{1mm} f$ é bixectiva$\}$ si é un grupo con esa operación, aínda que non é abeliano.
    
\end{enumerate}

\vspace{5mm}

\noindent \textbf{Observación 1.1.} No que segue, empregarase a notación que se introduce a continuación:

\begin{itemize}
    
    \item Para un grupo arbitrario, faise uso da \textbf{notación multiplicativa}:
    
    \begin{itemize}
        \item O grupo denótase por $(G, \cdot)$
        \item O elemento neutro denotarase por 1
        \item Sendo $x \in G$, $x^{-1}$ representará o elemento simétrico de $x$
        \item Dado $n \in \mathbb{Z}$ e $x \in G$:
        
\pagebreak
        
        \[
        nx \equiv 
        \begin{cases}
        x \cdot \stackrel{n}{\dots} \cdot x & \text{se } n > 0 \\
        0 & \text{se } n = 0 \\
        (-x) \cdot \stackrel{-n}{\dots} \cdot (-x) & \text{se } n < 0
        \end{cases}
        \]
    \end{itemize}
    
    
    \item Para un grupo conmutativo, faise uso da \textbf{notación aditiva}:
    
    \begin{itemize}
        \item O grupo denótase por $(G, +)$
        \item O elemento neutro denotarase por 0
        \item Sendo $x \in G$, $(-x)$ representará o elemento simétrico de $x$, que se denominará \textbf{oposto}
        
        \item Dado $n \in \mathbb{Z}$ e $x \in G$:
        
        \vspace{1mm}
        
        \[
        nx \equiv 
        \begin{cases}
        x + \stackrel{n}{\dots} + x & \text{se } n > 0 \\
        0 & \text{se } n = 0 \\
        (-x) + \stackrel{-n}{\dots} + (-x) & \text{se } n < 0
        \end{cases}
        \]

    \end{itemize}
\end{itemize}

\vspace{7mm}

\noindent \textbf{Observación 1.2.} Convén puntualizar que na definición de grupo, o conxunto considerado debe ser \textbf{distinto do conxunto baleiro}, $\varnothing$, pois tense que verificar a existencia de, cando menos, un elemento: o neutro respecto da operación interna considerada.

\textcolor{magenta}{\subsection{Subgrupos}}

\vspace{5mm}

\noindent \textbf{Definición 1.2.} Sexa $(G, \cdot \hspace{1mm})$ un grupo. Considérese un subconxunto $H \subset G$, con $H \neq \varnothing$. Dirase que $H$ é \textcolor{magenta}{\textbf{subgrupo}} de $G$, e denotarase por $\textcolor{magenta}{H < G}$, se o par $(H, \cdot \hspace{1mm})$ é un grupo.\\

\noindent Da propia definición de subgrupo despréndense certas propiedades que debe cumprir esta estrutura:

\vspace{2mm}

\begin{itemize}
    \item Dados $x,y \in H \implies x \cdot y \in H$
    \item 1 $\in H$
    \item Se $x \in H \implies x^{-1} \in H$
\end{itemize}

\noindent A asociatividade en $H$ é inmediata pola condición de grupo de $G$.

\vspace{3mm}

\noindent Exemplos de subgrupos son os seguintes:

\begin{enumerate}
    \item O conxunto de números enteiros múltiplos de 3, 3$\mathbb{Z}$, coa operación suma, é un subgrupo de $(\mathbb{Z}, +)$.
    \item Máis en xeral, dado calquera $n \in \mathbb{Z}$, cúmprese que $n\mathbb{Z} < \mathbb{Z}$ coa operación suma.
    \item Dado un grupo $G$ arbitrario, o conxunto formado polo elemento neutro $\{1\}$ coa operación interna considerada é un subgrupo de $G$, que se denomina \textbf{subgrupo trivial}.
    \item Dado un grupo $G$ arbitrario, é un subgrupo de si mesmo.
\end{enumerate}

\vspace{2mm}

\noindent \textbf{Observación 1.3.} Para determinar que subconxuntos dun grupo son subgrupos deste, non sempre é conveniente ter que facer uso da definición. Por iso, cómpre dispoñer de condicións que axuden nesta tarefa.

\noindent En particular, o seguinte resultado será de moita utilidade en demostracións e exercicios posteriores:

\vspace{1mm}

\begin{theorem} \label{th1.1}
Sexa $G$ un grupo arbitrario. Supóñase $H$ un subconxunto non baleiro de $G$. Tense: 
\begin{center}
    H é subgrupo de G $\Longleftrightarrow$ $\forall \hspace{1mm} x,y \in H$, $x \cdot y^{-1} \in H$
\end{center} 
\end{theorem}

\vspace{2mm}

\noindent \textbf{\textit{\underline{Demostración}}}

\vspace{2mm}

\noindent \fcolorbox{magenta}{white}{$\Longrightarrow$/} Supóñase $H$ subgrupo de $G$. Entón, como $y \in H$, tense que $y^{-1} \in H$. Logo, xa que $x,y^{-1} \in H$, sendo $H$ subgrupo, cúmprese que, efectivamente, $x \cdot y^{-1} \in H$.

\vspace{2mm}

\noindent \fcolorbox{magenta}{white}{$\Longleftarrow$/} Reciprocamente, supóñase agora que $\forall \hspace{1mm} x,y \in H$, $x \cdot y^{-1} \in H$, e demóstrese que, efectivamente, $H$ é subgrupo de G. \\
Véxase, en primeiro lugar, que $1 \in H$. Sendo $H$ non baleiro, $\exists \hspace{1mm} x \in H$. Como $x \in H$ e $x \in H$, aplicando a hipótese, $x \cdot x^{-1} = 1 \in H$. \\
Próbese agora que todo elemento de $H$ posúe simétrico nese conxunto. Tendo demostrado que $1 \in H$, e sendo $x \in H$, por hipótese, garántese que $1 \cdot x^{-1} = x^{-1} \in H.$ \newline
\noindent Queda entón demostrar que $\forall \hspace{1mm} x,y \in H$, $x \cdot y \in H$. Tal e como se acaba de ver, como $y \in H$, entón $y^{-1} \in H$. Así, cumpríndose que $x, y^{-1} \in H$, garántese que $x \cdot (y^{-1})^{-1} = x \cdot y \in H$. $\square$

\vspace{5mm}

\noindent Dado un subconxunto arbitrario dentro dun grupo, este non ten por que posuír estrutura de subgrupo coa operación interna considerada. Por iso, interesa coñecer que subgrupos conteñen tal subconxunto de elementos. En particular:

\vspace{3mm}

\noindent \textbf{Definición 1.3.} Sexa $G$ un grupo. Considérese un subconxunto $X \subset G$ non baleiro. Defínese o \textcolor{magenta}{\textbf{subgrupo xerado por $X$}}, denotado por \textcolor{magenta}{$\langle X \rangle$}, como o menor subgrupo de $G$ que contén a $X$.\\

\begin{proposition} \label{prop1.1}
Sexa G un grupo arbitrario. Considérese unha familia de subgrupos de G, $\{H_{i}\}_{i \in I}$. Cúmprese que a súa intersección, $\underset{i \in I}{\bigcap}H_{i}$, é tamén un subgrupo de G.
\end{proposition}

\vspace{2mm}

\noindent \textbf{\textit{\underline{Demostración}}}

\vspace{2mm}

\noindent O primeiro que hai que probar é que tal intersección é non baleira. En efecto, todo subgrupo posúe o elemento neutro da operación interna definida nel. Así, a intersección de todos eles contén cando menos o elemento neutro, cumpríndose así que $\underset{i \in I}{\bigcap}H_{i} \neq  \varnothing$.

\noindent A continuación, empregarase o \hyperref[th1.1]{\magbf{Teorema 1.1}} para demostrar a condición de subgrupo da intersección. Dados $x,y \in \underset{i \in I}{\bigcap}H_{i}$, tratarase de probar que $x \cdot y^{-1} \in \underset{i \in I}{\bigcap}H_{i}$.

\noindent Como $x,y \in \underset{i \in I}{\bigcap}H_{i}$, en particular $x,y \in H_{i} \hspace{2mm} \forall \hspace{1mm} i \in I$, logo tense que $y^{-1} \in I \hspace{2mm} \forall \hspace{1mm} i \in I$. Pola condición de subgrupo de $H_{i}$ para cada $i$, cúmprese que $x \cdot y^{-1} \in H_{i} \hspace{2mm} \forall \hspace{1mm} i \in I$. Así, tense que $x \cdot y^{-1} \in \underset{i \in I}{\bigcap}H_{i}$. $\square$

\vspace{3mm}

\noindent A partir da proposición anterior pódese establecer outra definición para o subgrupo xerado por un subconxunto $X$ nun grupo $G$: este é a intersección de tódolos subgrupos que conteñen a $X$.


    $$\langle X \rangle = \bigcap{ \{H \hspace{1MM} | \hspace{1MM} H < G, H \supset X\} } $$


\vspace{3mm}

\noindent \textbf{Observación 1.4.} A condición expresada na proposición anterior non se cumpre para a unión de subgrupos. Como contraexemplo: $2\mathbb{Z}, 3\mathbb{Z} < \mathbb{Z}$, pero $2\mathbb{Z} \cup 3\mathbb{Z} \nless \mathbb{Z}$. Por exemplo, $5 \notin 2\mathbb{Z} \cup 3\mathbb{Z}$.

\vspace{3mm}

\begin{proposition} \label{prop1.2}
Sexa G un grupo arbitrario. Considérese X un subconxunto non baleiro de G. Entón, cúmprese:
\begin{center}
    $\langle X \rangle = \{a_{1} \cdot \ldots \cdot a_{n} \hspace{1mm} | \hspace{1mm} n \in \mathbb{N^{*}}$, para cada $i \in \{1, \dots, n\}$, $(a_{i} \in X) \lor (a_{i}^{-1} \in X) \}$
\end{center}
\end{proposition}

\vspace{2mm}

\noindent \textbf{\textit{\underline{Demostración}}}

\vspace{2mm}

\noindent No que segue, asúmese a notación $A$ para o conxunto descrito na proposición. É preciso demostrar os seguintes puntos:

\begin{itemize}
    \item $X \subset A$
    
    Isto é inmediato, pois $A$ contén tódolos elementos de $X$ operados entre eles e os seus inversos; en particular, contenos a eles mesmos.
    
    \item $A$ é subgrupo
    
    Como $X \subset A$, con $X \neq \varnothing$, tense que $A \neq \varnothing$. Facendo uso do \hyperref[th1.1]{\magbf{Teorema 1.1}}, sexan $a_{1} \cdot \dots \cdot a_{n}$, $b_{1} \cdot \dots \cdot b_{m} \in A$. Tense:
    
    $(a_{1} \cdot \dots \cdot a_{n}) \cdot (b_{1} \cdot \dots \cdot b_{m})^{-1} = a_{1} \cdot \dots \cdot a_{n} \cdot b_{m}^{-1} \cdot \dots \cdot b_{1}^{-1} \in A$
    
    \item $A$ é o menor subgrupo que contén a $X$
    
    Considérese outro subgrupo $H < G$ tal que $X \subset H$. Verifícase que $A \subset H$?
    
    Sexa $a_{1} \cdot \dots \cdot a_{n} \in A$. Por definición de $A$, $a_{i} \in X \subset H$ ou $a_{i}^{-1} \in X \subset H \hspace{2mm} \forall i \in \{1,\dots,n\}$. Sendo $H$ subgrupo, cúmprese que $a_{1} \cdot \dots \cdot a_{n} \in H$, probando así que $A \in H$. $\square$
\end{itemize}

\vspace{-5mm}

\textcolor{magenta}{\subsection{Subgrupos notables}}

\vspace{3mm}

\noindent Os seguintes son exemplos destacados de subgrupos:

\vspace{3mm}

\noindent \textbf{Definición 1.4.} Sexa $G$ un grupo arbitrario. Supóñase $a \in G$ e $H$ un subgrupo de $G$. Defínese o \textbf{\textcolor{magenta}{subgrupo conxugado de $H$ por $a$}}, e denótase por \textcolor{magenta}{$aHa^{-1}$}, como o seguinte conxunto:

\vspace{2mm}

\begin{center}
    \fcolorbox{magenta}{white}{$aHa^{-1} = \{aha^{-1} \hspace{1mm} | \hspace{1mm} h \in H \}$}
\end{center}

\vspace{1mm}

\noindent Cúmprese que este conxunto é un subgrupo coa operación interna definida en $G$. En efecto:

\begin{itemize}
    \item Sendo $H$ subgrupo, en particular $H \neq \varnothing$. Entón, $\exists \hspace{1mm} h \in H$. Así, calquera que sexa $a \in G$, $aha^{-1} \in aHa^{-1}$, cumpríndose que $aHa^{-1} \neq \varnothing$
    \item Dados $x,y \in H$, empregarase o \hyperref[th1.1]{\magbf{Teorema 1.1}} para ver que o conxunto é realmente un subgrupo. Tense:
    
    $x \in aHa^{-1} \implies \exists \hspace{1mm} h \in H \hspace{1mm} | \hspace{1mm} x = aha^{-1}$\\
    $y \in aHa^{-1} \implies \exists \hspace{1mm} h' \in H \hspace{1mm} | \hspace{1mm} y = ah'a^{-1}$
    
    Entón:
    
    $x \cdot y^{-1} = (aha^{-1}) \cdot (ah'a^{-1})^{-1} = (aha^{-1}) \cdot (a(h')^{-1}a^{-1}) = aha^{-1}a(h')^{-1}a^{-1} = ah(h')^{-1}a^{-1}$ 
    
    Como $H$ é un subgrupo e $h' \in H$, cúmprese que $(h')^{-1} \in H$, polo que $h \cdot (h')^{-1} \in H$. Logo, $x \cdot y^{-1} = ah(h')^{-1}a^{-1} \in aHa^{-1}$, obtendo así que o conxunto é, efectivamente, un subgrupo de $G$. $\square$
\end{itemize}

\vspace{3mm}

\noindent \textbf{Definición 1.5.} Sexa $G$ un grupo. Denomínase \textcolor{magenta}{\textbf{centro de $G$}}, e denótase por $\textcolor{magenta}{Z(G)}$, o seguinte subconxunto de $G$:

\begin{center}
    \fcolorbox{magenta}{white}{$Z(G) = \{x \in G \hspace{1mm} | \hspace{1mm} x \cdot y = y \cdot x \hspace{2mm} \forall \hspace{1mm} y \in G$ \}}
\end{center}

\noindent Dotándoo da operación interna definida en $G$, cúmprese que $Z(G) < G$. Tense:

\begin{itemize}
    \item $Z(G) \neq \varnothing$, pois en particular o elemento neutro da operación conmuta con calquera elemento do grupo.
    \item Empregando o \hyperref[th1.1]{\magbf{Teorema 1.1}}, véxase que, dados $x,y \in Z(G)$, $x \cdot y^{-1} \in Z(G)$. Certamente, dado calquera $g \in G$:
    
    $(x \cdot y^{-1}) \cdot g \parallel g \cdot (x \cdot y^{-1}) \Longleftrightarrow x \cdot y^{-1} \cdot g \cdot y \parallel g \cdot x \Longleftrightarrow y^{-1} \cdot g \cdot y \parallel x^{-1} \cdot g \cdot x$
    
    Agora ben, como $x,y \in Z(G)$, pódese escribir:
    
    $y^{-1} \cdot y \cdot g \parallel x^{-1} \cdot x \cdot g \Longleftrightarrow g \parallel g$
    
    É dicir, facendo as mesmas operacións en ámbalas dúas expresións, chégase ó mesmo resultado. Así, $(x \cdot y^{-1}) \cdot g = g \cdot (x \cdot y^{-1})$, cumpríndose que $x \cdot y^{-1} \in Z(G)$. $\square$
    
\end{itemize}

\vspace{4mm}

\noindent \textbf{Definición 1.6.} Sexa $G$ un grupo. Considérese un elemento arbitrario $a \in G$. Defínese o \textcolor{magenta}{\textbf{subgrupo cíclico xerado por $a$}}, denotado por \textcolor{magenta}{$\langle a \rangle$}, como o seguinte subconxunto de $G$:

\begin{center}
    \fcolorbox{magenta}{white}{$\langle a \rangle = \{a^{n} \hspace{1mm} | \hspace{1mm} n \in \mathbb{Z} \}$}
\end{center}

\noindent Se $G$ é un grupo tal que $\exists \hspace{1mm} a \in G \hspace{1mm} | \hspace{1mm} \langle a \rangle = G$, dirase que $G$ é un \textcolor{magenta}{\textbf{grupo cíclico}}.

\vspace{3mm}

\noindent Cúmprese que estes subconxuntos son subgrupos $\forall \hspace{1mm} a \in G$. Certamente:

\begin{itemize}
    \item Que $\langle a \rangle \neq \varnothing$ é trivial, pois en particular $a^{0} = 1$, $a^{1} = a \in \langle a \rangle$.  
    \item Hai que probar a condición de subgrupo, i.e. dados $x,y \in \langle a \rangle$, $x \cdot y^{-1} \in \langle a \rangle$. Tense: 
    
    \vspace{1mm}
    
    Como $x,y \in \langle a \rangle$, $\exists \hspace{1mm} n,m \in \mathbb{Z} \hspace{1mm} | \hspace{1mm} x = a^{n}$, $y = a^{m}$.
    
    Se $y = a^{m}$, $y^{-1} = a^{-m}$. Así, $x \cdot y^{-1} = a^{n} \cdot a^{-m} = a^{n-m}$. Sendo $n,m \in \mathbb{Z}$, $n-m \in \mathbb{Z}$, polo que $x \cdot y^{-1} \in \langle a \rangle$, como se quería probar. $\square$
    
\end{itemize}

\vspace{3mm}

\noindent Ilustrarase esta última definición cun exemplo moi familiar. Considérese o grupo $(\mathbb{Z}, +)$. Dado calquera $n \in \mathbb{Z}$, o subconxunto de enteiros $n\mathbb{Z}$, i.e. os múltiplos de $n$, é un subgrupo cíclico. 

\vspace{3mm}

\noindent \textbf{Observación 1.5.} Sexa $a \in G$, con $G$ un grupo arbitrario. O subgrupo cíclico de $a$ coincide co subgrupo xerado polo subconxunto $\{a\} \subset G$: $\langle a \rangle = \langle \{a\} \rangle$.

\textcolor{magenta}{\subsection{Ordes}}

\vspace{3mm}

\noindent \textbf{Definición 1.7.} Sexa $G$ un grupo. Defínese a \textcolor{magenta}{\textbf{orde de $G$}} de dous xeitos:

\vspace{2mm}

\begin{itemize}
    \item Se $G$ é finito, a súa orde correspóndese co seu cardinal.
    \item Se pola contra é infinito, dise que a orde de $G$ é infinita.
\end{itemize}

\vspace{2mm}

\noindent As posibles notacións para a orde dun grupo son as seguintes: \textcolor{magenta}{$|G|$} (esta será a máis habitual), \textcolor{magenta}{$ord(G)$} ou \textcolor{magenta}{$o(G)$}.

\vspace{3mm}

\noindent \textbf{Definición 1.8.} Sexa $G$ un grupo. Dado $a \in G$ arbitrario, defínese a \textcolor{magenta}{\textbf{orde do elemento $a$}} como a orde do subgrupo cíclico $\langle a \rangle$. Denótase por \textcolor{magenta}{$|a|$}.

\vspace{3mm}

\noindent O seguinte resultado é fundamental para o estudo de subgrupos cíclicos de orde finita:

\vspace{3mm}

\begin{proposition} \label{prop1.3}
Sexa $G$ un grupo e $a \in G$ un elemento arbitrario con orde $n$. Verifícanse:
\begin{enumerate}
        \item $\langle a \rangle = \{ 1, a, \dots, a^{n-1} \}$. Tódolos elementos do subgrupo son distintos 2 a 2, e ademais, cúmprese que $a^{n} = 1$.
        \item $a^{m} = 1 \Longleftrightarrow n|m$. En particular, $n$ é o menor enteiro positivo tal que $a^{n} = 1$.
\end{enumerate}
\end{proposition}

\vspace{2mm}

\noindent \textbf{\textit{\underline{Demostración}}} 

\vspace{2mm}

\noindent \magbf{(1)} En primeiro lugar, cómpre notar que, por definición de subgrupo cíclico de $a$, é inmediato que $\{1,a,\dots,a^{n-1}\} \subset \langle a \rangle$. Probarase a continuación a outra inclusión. \\

\noindent Sendo $\langle a \rangle$ un grupo finito formado por tódalas potencias enteiras de $a$, como $\mathbb{Z}$ é un conxunto infinito, necesariamente, $\exists \hspace{1mm} i,j \in \mathbb{Z}^{+} \hspace{1mm} | \hspace{1mm} 0 \leq i < j$, $a^{i} = a^{j} \Longleftrightarrow a^{j-i} = 1$, onde $j-i > 0$. É dicir, existen enteiros (en particular, positivos) para os cales operando $a$ consigo mesmo tal número de veces resulta no neutro da operación. Sexa $s$ o menor enteiro positivo tal que $a^{s} = 1$, e véxase que $s = n$. \\

\noindent Dado $m \in \mathbb{Z}$, se $a^{m} \in \langle a \rangle$, entón $a^{m} \in \{1,a,\dots,a^{s-1}\}$. En efecto, polo teorema da división enteira, $\exists \hspace{1mm} q,r \in \mathbb{Z} \hspace{1mm} | \hspace{1mm} m = sq + r$, con $0 \leq r < s$. En base a isto, tense:
\begin{equation}\label{ec1}
    a^{m} = {(a^{s})}^{q} \cdot a^{r} \overset{\overset{a^{s} = 1}{\Downarrow}}{=} a^{r} \implies a^{m} \in \{1,a,\dots,a^{s-1}\}
\end{equation}

\noindent Demóstrase, con isto, que $\langle a \rangle \subset \{1,a,\dots,a^{s-1}\}$. E pola propia definición de subgrupo cíclico, tense que $\{1,a,\dots,a^{s-1}\} \subset \langle a \rangle$, probando así a igualdade entre ámbolos dous conxuntos. Agora ben, recórdese que o conxunto ten $n$ elementos. Así, necesariamente, $\langle a \rangle = \{1,a,\dots,a^{n-1}\}$, obtendo así que $s = n$ e , en consecuencia, que $a^{n} = 1$. 

\hspace{2mm}

\noindent \magbf{(2)} Emprégase o mesmo razoamento ca en \eqref{ec1}: $\exists \hspace{1mm} q,r \in \mathbb{Z} \hspace{1mm} | \hspace{1mm} m = nq + r$, con $0 \leq r < n$. Así:

$$1 = a^{m} = {(a^{n})}^{q} \cdot a^{r} = a^{r} \overset{\overset{0 \leq r < n}{\Downarrow}}{\Longleftrightarrow} r = 0 \Longleftrightarrow m = nq \Longleftrightarrow n|m \hspace{5mm}\square$$ 

\textcolor{magenta}{\section{Grupos cociente}}

\textcolor{magenta}{\subsection{As clases laterais de equivalencia. O teorema de Lagrange}}

\vspace{3mm}

\noindent Sexan $G$ un grupo e $H$ un subgrupo arbitrario de $G$. Dados $a,b \in G$, defínese a seguinte relación: 

$$\fcolorbox{magenta}{white}{$a \sim_{\leftarrow} b :\Longleftrightarrow a^{-1}b \in H$}$$

\vspace{3mm}

\noindent Pódese comprobar facilmente que esta é unha relación de equivalencia. Como tal, define un conxunto cociente de clases de equivalencia, denotado neste caso por $G/\sim_{\leftarrow} \hspace{1mm} \equiv \hspace{1mm} G/H$. \\

\noindent Considérese $a \in G$ de xeito arbitrario. Vaise calcular a súa clase de equivalencia dentro deste conxunto cociente: 

$$[a] = \{ b \in G \hspace{1MM} | \hspace{1MM} a \sim_{\leftarrow} b \} = \{ b \in G \hspace{1MM} | \hspace{1MM} a^{-1}b = h \in H \} = \{ b \in G \hspace{1MM} | \hspace{1MM} b = ah, h \in H \} = aH$$

\vspace{2mm}

\noindent A estes conxuntos dáselles o nome de \textcolor{magenta}{\textbf{clases laterais pola esquerda de $H$}}.\\

\noindent Analogamente, dados $a,b \in G$ arbitrarios, pódese definir outra relación binaria:

$$\fcolorbox{magenta}{white}{$a \sim_{\rightarrow} b :\Longleftrightarrow ba^{-1} \in H$}$$

\vspace{3mm} 

\noindent a cal tamén é relación de equivalencia en $G$, dando lugar a un conxunto cociente $G/\sim_{\rightarrow}$ de tal xeito que, considerando $a \in G$, a súa clase de equivalencia resulta ser:

$$[a] = \{ b \in G \hspace{1MM} | \hspace{1MM} a \sim_{\rightarrow} b \} = \{ b \in G \hspace{1MM} | \hspace{1MM} ba^{-1} = h \in H \} = \{ b \in G \hspace{1MM} | \hspace{1MM} b = ha, h \in H \} = Ha$$

\vspace{2mm}

\noindent Estes conxuntos denomínanse \textcolor{magenta}{\textbf{clases laterais pola dereita de $H$}}.\\

\noindent As respectivas particións de $G$ que definen estas dúas relacións de equivalencia son denotadas por $G:H$ e $H:G$. \\

\noindent En teoría de grupos, existe un resultado moi importante que relaciona a orde dun grupo finito coa orde de calquera dos seus subgrupos: o \textbf{teorema de Lagrange}, e faino empregando a relación de equivalencia definida anteriormente. O resultado di o seguinte:\\

\begin{theorem}[\textbf{Teorema de Lagrange}] \label{th1.2}
Sexan $G$ un grupo finito e $H$ un subgrupo de $G$. Considérense $a_{1}H$, $a_{2}H$, \dots, $a_{r}H$ as clases pola esquerda que $H$ determina en $G$. Cúmprese a igualdade: 
\begin{center}
    $|G| = |H| \cdot r$ 
\end{center} 
\end{theorem}

\vspace{2mm}

\noindent \textbf{\textit{\underline{Demostración}}}

\vspace{2mm}

\noindent Para demostrar o teorema, probarase primeiramente que toda clase lateral pola esquerda posúe o mesmo número de elementos ca $H$. Con tal fin, escollendo $a \in G$, considérese a seguinte aplicación:

\vspace{2mm}

\begin{center}
    $\cdot : H \longrightarrow aH$ \\
    \vspace{2mm}
    \hspace{3mm} $h \leadsto a \cdot h$
\end{center}  

\vspace{2mm}

\noindent A aplicación é bixectiva. En efecto:

\begin{itemize}
    \item É inxectiva, pois $\forall \hspace{1mm} x,y \in H$, $a \cdot x = a \cdot y \overset{\overset{G \text{ grupo}}{\Downarrow}}{\implies} x = y$
    \item É sobrexectiva, pois todo elemento de $G$, en particular calquera elemento de $H$, pertence a unha (e só unha) clase de equivalencia do conxunto cociente. Toda clase posúe elementos.
\end{itemize} 

\noindent Así, $|H| = |aH| \hspace{1mm} \forall \hspace{1mm} a \in G$; en particular, para os $a_{1}, \dots, a_{r}$ do teorema. Sendo $G:H$ unha partición, $G = \bigcup\limits_{i = 1}^{r} a_{i}H$, con $a_{i}H \cap a_{j}H = \varnothing$ se $i \neq j$. Polo tanto, pódese escribir:

$$|G| = \sum_{i=1}^{r}{|a_{i}H|} = |H| \cdot r \hspace{2mm} \square$$ 

\vspace{3mm}

\noindent O enunciado do teorema de Lagrange pode formularse, de xeito completamente análogo, para as clases laterais pola dereita.\\

\noindent Isto permite garantir que o número de clases laterais pola esquerda determinadas por un subgrupo é igual ó número de clases laterais pola dereita do mesmo (pódese construír unha bixección entre ámbolos dous conxuntos, como se recolle en \cite{rotman}). Por esta razón, ten sentido definir: \\

\noindent \textbf{Definición 1.9.} Sexan $G$ un grupo e $H$ un subgrupo deste. Defínese o \textcolor{magenta}{\textbf{índice de $H$ en $G$}} como o número de clases de equivalencia que $H$ determina sobre $G$. Denótase por \textcolor{magenta}{$[G:H] \hspace{1mm} \equiv \hspace{1mm} [H:G]$}. 

\vspace{4mm}

\noindent Como consecuencia inmediata do teorema de Lagrange tense: \\
\begin{corollary} \label{cor1.1}
Sexa $G$ un grupo finito. Cúmprese:
\begin{enumerate}
    \item $H < G \implies$ A orde de $H$ é divisor da orde de $G$
    \item $a \in G \implies$ A orde de $a$ é divisor da orde de $G$
    \item $a \in G \implies a^{|G|} = 1$
    \item Se a orde de $G$ é un número primo, entón $G$ é un grupo cíclico
\end{enumerate}
\end{corollary}

\vspace{2mm}

\noindent \textbf{\textit{\underline{Demostración}}}

\vspace{2mm}

\noindent \magbf{(1)} É inmediato a partir do propio enunciado do teorema \\

\noindent \magbf{(2)} A orde de $a$ é a do seu subgrupo cíclico, reducindo así o problema ó caso \magbf{(1)} \\

\noindent \magbf{(3)} O teorema permite afirmar que $|G| = |a|[G:\langle a \rangle]$. Entón:

$$a^{|G|} = {(a^{|a|})}^{[G:\langle a \rangle]} = 1^{[G:\langle a \rangle]} = 1$$

\noindent \magbf{(4)} Hai que probar que $\exists \hspace{1mm} a \in G \hspace{1mm} | \hspace{1mm} \langle a \rangle = G$.\\

\noindent Considérese $a \in G$ distinto do elemento neutro. Facendo $p$ a orde de $G$, tense:

\[ 
\left. \begin{array}{r} 
|a| \text{ divide a } p\\[1ex]
p \text{ é primo}
\end{array} \right\} 
\underset{a \neq 1}{\implies} |a| = p \implies G = \langle a \rangle \hspace{2mm} \square
\]

\hspace{3mm}

\noindent \textbf{Observación 1.6.} O conxunto cociente $G/H$, a priori, non ten estrutura de grupo.

\textcolor{magenta}{\subsection{Subgrupos normais. Grupo cociente}}

\vspace{3mm}

\noindent \textbf{Definición 1.10.} Sexa $G$ un grupo e $H$ un subgrupo. Dirase que $H$ é un \textcolor{magenta}{\textbf{subgrupo normal}} de $G$, e denotarase por \textcolor{magenta}{$H \triangleleft G$}, se contén a tódolos seus subgrupos conxugados.

\hspace{2MM}

\begin{center}
    $H$ é \textcolor{magenta}{\textbf{normal}} \hspace{1MM} :$\Longleftrightarrow \hspace{1Mm} aHa^{-1} \subset H \hspace{3MM} \forall \hspace{1mm} a \in G \hspace{1mm} \Longleftrightarrow \hspace{1mm} aha^{-1} \in H \hspace{2MM} \forall \hspace{1Mm} h \in H \hspace{2MM} \forall \hspace{1mm} a \in G$
\end{center}

\hspace{2mm}

\begin{proposition}[\textbf{Definición equivalente de subgrupo normal}] \label{prop1.4}
Sexa $G$ un grupo e $H$ un subgrupo seu. Equivalen:
\begin{enumerate}
    \item $H \triangleleft G$
    \item $aH = Ha \hspace{2mm} \forall \hspace{1mm} a \in G$
    \item $aH \cdot bH = abH \hspace{2MM} \forall \hspace{1MM} a,b \in G$
\end{enumerate}
\end{proposition}

\vspace{2mm}

\noindent \textbf{\textit{\underline{Demostración}}}

\vspace{2mm}

\noindent \fcolorbox{magenta}{white}{\textbf{$(1) \implies (2)$}}\\

\noindent Como $H \triangleleft G$, dado $a \in G$:

\begin{itemize}
    \item $aHa^{-1} \subset H \implies aH \subset Ha$
    \item $a^{-1}Ha \subset H \implies Ha \subset aH$
\end{itemize}

\noindent \fcolorbox{magenta}{white}{\textbf{$(2) \implies (3)$}} \\

\noindent Sexan $h, h' \in H$. Entón: 

$$aH \cdot bH \ni ah \cdot bh' = ahbh'= a(hb)h' \underset{\underset{\textbf{(2)}}{\Uparrow}}{=} a(bh'')h' = ab \cdot h''h' \in abH $$

\noindent \fcolorbox{magenta}{white}{\textbf{$(3) \implies (1)$}} \\

\noindent Dado $a \in G$ arbitrario, tense: 

$$aHa^{-1}H \underset{\underset{\textbf{(3)}}{\Uparrow}}{=} aa^{-1}H = H \implies aHa^{-1} \subset H$$ 

\noindent Así, cúmprese que $H$ é un subgrupo normal. $\square$ \\

\noindent Como xa se apuntaba nunha observación anterior, en xeral, o conxunto cociente das clases laterais, tanto pola esquerda coma pola dereita, non é un grupo. Porén, cando o subgrupo que determina tales clases é normal, tal conxunto si está dotado desa estrutura, como así o recolle o seguinte teorema: \\

\begin{theorem}
Sexa $G$ un grupo e $H$ un subgrupo normal seu. Baixo tales condicións, o conxunto cociente $G/H$ está dotado de estrutura de grupo coa operación:
\begin{center}
    $aH \cdot bH = ab \cdot H$
\end{center}
A esta estrutura chámaselle \textbf{grupo cociente de $G$ por $H$}.
\end{theorem}

\vspace{2mm}

\noindent \textbf{\textit{\underline{Demostración}}}

\vspace{2mm}

\noindent Primeiramente, véxase que a operación considerada está ben definida no conxunto cociente, é dicir, que non depende do representante da clase de equivalencia. \\

\noindent Considérense así $a_{1},a_{2},b_{1},b_{2} \in G \hspace{1mm}| \hspace{1mm} a_{1}H = a_{2}H$, $b_{1}H = b_{2}H$, e compróbese que $a_{1}H \cdot b_{1}H = a_{2}H \cdot b_{2}H$. \\

\noindent Recórdese que $a_{1}H = a_{2}H \Longleftrightarrow a_{1}^{-1}a_{2} \in H$ \textbf{(i)} e, analogamente, $b_{1}H = b_{2}H \Longleftrightarrow b_{1}^{-1}b_{2} \in H$ \textbf{(ii)}. Tense:

$$a_{1}H \cdot b_{1}H = a_{2}H \cdot b_{2}H \Longleftrightarrow a_{1}b_{1}H = a_{2}b_{2}H \underset{\underset{\text{Def clases}}{\Uparrow}}{\Longleftrightarrow} (a_{1}b_{1})^{-1}a_{2}b_{2} \in H$$

\noindent Véxase que, certamente, $(a_{1}b_{1})^{-1}a_{2}b_{2} \in H$:

$$(a_{1}b_{1})^{-1}a_{2}b_{2} = b_{1}^{-1}a_{1}^{-1}a_{2}b_{2} \overset{\textbf{(i)}}{=} b_{1}^{-1}hb_{2} \underset{\underset{\substack{H \triangleleft G \\ \hyperref[prop1.4]{\magbf{Prop 1.4}}}}{\Uparrow}}{=} h'b_{1}^{-1}b_{2} \overset{\textbf{(ii)}}{=} h'h'' \in H$$

\noindent Demostrouse así que a operación está ben definida en $G/H$. Queda por ver que cumpre as propiedades para que $G/H$ teña estrutura de grupo:

\begin{itemize}
    \item Existe elemento neutro da operación: $[1] = 1H = H$
    \item O elemento simétrico de $aH$ é $a^{-1}H$
    \item A operación é asociativa
\end{itemize}

\noindent Compróbase facilmente que se verifican as propiedades anteriores, obtendo así que $G/H$ é un grupo coa operación considerada. $\square$

\textcolor{magenta}{\section{O grupo de permutacións de $n$ elementos, $S_{n}$}}

\hspace{5mm}

\noindent Cando se presentou a definición de grupo, un dos exemplos que a ilustraban era o seguinte:\\

\noindent \textit{Considérese $A = \{ f: X \longrightarrow X \hspace{1mm} | \hspace{1mm}$f é bixectiva$\}$. Denotando por $\circ$ a operación composición, como é habitual, cúmprese que o par $(A, \circ)$ é un grupo, non abeliano en xeral.} \\

\noindent Un caso de particular interese dentro deste exemplo é aquel no que se ten o conxunto $X = \{1,2, \dots, n\}$. En tal situación, o conxunto de aplicacións considerado é o seguinte:
$$S_{n} = \{ \sigma: X \longrightarrow X\}$$

\noindent O par $(S_{n}, \hspace{1mm} \circ)$ denomínase \textcolor{magenta}{\textbf{grupo simétrico}} ou \textcolor{magenta}{\textbf{grupo de permutacións de orde $n$}}. Para o caso particular onde $n = 2$, este grupo é abeliano. Cando $n > 2$, en xeral, non se dá tal condición.\\

\noindent Os elementos deste conxunto, que se denominan \textcolor{magenta}{\textbf{permutacións}}, represéntanse como matrices de 2 filas e $n$ columnas, estruturadas do seguinte xeito:

$$\sigma = \begin{pmatrix}
1 & 2 & \dots & n\\
\sigma(1) & \sigma(2) & \dots & \sigma(n)\\
\end{pmatrix}$$

\vspace{3mm}

\noindent Véxanse agora algúns exemplos, neste caso tomados en $S_{5}$:

$$\sigma = \begin{pmatrix}
1 & 2 & 3 & 4 & 5 \\
2 & 3 & 5 & 1 & 4 \\
\end{pmatrix}
\hspace{8mm}
\tau = \begin{pmatrix}
1 & 2 & 3 & 4 & 5 \\
5 & 4 & 3 & 2 & 1 \\
\end{pmatrix} 
$$
\\
$$
\tau \circ \sigma = \begin{pmatrix}
1 & 2 & 3 & 4 & 5 \\
4 & 3 & 1 & 5 & 2 \\
\end{pmatrix}
\hspace{8mm}
\sigma \circ \tau = \begin{pmatrix}
1 & 2 & 3 & 4 & 5 \\
4 & 1 & 5 & 3 & 2 \\
\end{pmatrix}
$$
\\
$$
\sigma^{-1} = \begin{pmatrix}
1 & 2 & 3 & 4 & 5 \\
4 & 1 & 2 & 5 & 3 \\
\end{pmatrix}
\hspace{8mm}
\tau^{-1} = \begin{pmatrix}
1 & 2 & 3 & 4 & 5 \\
5 & 4 & 3 & 2 & 1 \\
\end{pmatrix}
$$

\vspace{3mm}

\noindent \textbf{Observación 1.7.} Se nunha permutación existen elementos invariantes, estes adoitan ser suprimidos da notación matricial. \\

\noindent Así, por exemplo, para a permutación $\tau$ anterior, teríase:

$$
\tau = \begin{pmatrix}
1 & 2 & 3 & 4 & 5 \\
5 & 4 & 3 & 2 & 1 \\
\end{pmatrix}
\equiv
\begin{pmatrix}
1 & 2 & 4 & 5 \\
5 & 4 & 2 & 1 \\
\end{pmatrix}
$$

\vspace{5mm}

\noindent Dentro do grupo de permutacións, existe un certo tipo que se define a continuación, de especial relevancia no estudo de $S_{n}$, como se verá inmediatamente: \\

\noindent \textbf{Definición 1.11}. Dise que unha permutación $\sigma \in S_{n}$ é un \textcolor{magenta}{\textbf{ciclo de orde $r$}}, ou \textcolor{magenta}{\textbf{permutación circular de orde $r$}}, con $r \leq n$, se posúe $r$ elementos non invariantes, i.e. se $ \exists \hspace{1mm} a_{1}, \dots, a_{r}$ tales que $\sigma(a_{i}) = a_{i+1}$, $i \in \{1, \dots, r-1 \}$ e $\sigma(a_{r}) = a_{1}$.\\

\noindent Habitualmente, os ciclos adóitanse denotar por un vector fila da forma: 

$$
\begin{pmatrix}
a_{1} & a_{2} & \dots & a_{r} \\
\end{pmatrix}
$$

\vspace{2mm}

\noindent Por exemplo, o seguinte ciclo de $S_{6}$:

$$ \tau = \begin{pmatrix}
1 & 2 & 3 & 4 & 5 & 6 \\
6 & 1 & 2 & 3 & 4 & 5 \\
\end{pmatrix}
\equiv 
\begin{pmatrix}
1 & 6 & 5 & 4 & 3 & 2
\end{pmatrix}
=
\begin{pmatrix}
5 & 4 & 3 & 2 & 1 & 6
\end{pmatrix}
$$

\noindent Como xa se adiantaba, os ciclos son fundamentais no estudo do grupo das permutacións. O seguinte resultado resume por que:

\vspace{5mm}

\begin{proposition} \label{prop1.5}
Toda permutación pódese escribir como composición de ciclos actuando sobre conxuntos disxuntos. Esta factorización é única, salvo a orde dos factores.
\end{proposition}

\vspace{2mm}

\noindent \textbf{\textit{\underline{Demostración}}}

\vspace{2mm}

\noindent Pódese ver en \cite{rotman}. $\square$

\vspace{5mm}

\noindent Como exemplo, considérese a seguinte permutación de $S_{8}$: 

$$
\sigma = 
\begin{pmatrix}
1 & 2 & 3 & 4 & 5 & 6 & 7 & 8 \\
8 & 3 & 4 & 1 & 6 & 5 & 2 & 7 \\
\end{pmatrix}
=
\begin{pmatrix}
1 & 8 & 7 & 2 & 3 & 4
\end{pmatrix}
\begin{pmatrix}
5 & 6
\end{pmatrix}
$$

\vspace{3mm}

\noindent Este resultado, sendo de especial relevancia, aínda se pode estender máis. Para iso, apoiarémonos na seguinte definición e no resultado que a sucede:\\

\noindent \textbf{Definición 1.12.} Un ciclo de orde 2 denomínase \textcolor{magenta}{\textbf{trasposición}}.\\

\begin{proposition} \label{prop1.6}
Todo ciclo pódese factorizar como produto de trasposicións. Ademais, dúas factorizacións mediante produto de trasposicións do mesmo ciclo posúen a mesma paridade.
\end{proposition}

\vspace{2mm}

\noindent \textbf{\textit{\underline{Demostración}}}

\vspace{2mm}

\noindent Pódese ver en \cite{rotman}. $\square$

\vspace{5mm}

\noindent Xuntando as dúas proposicións anteriores, obtense:\\

\begin{theorem} \label{th1.4}
Toda permutación pódese factorizar como produto de trasposicións. Ademais, dúas factorizacións mediante produto de trasposicións da mesma permutación posúen a mesma paridade.
\end{theorem}

\vspace{5mm}

\noindent Como exemplo de factorización mediante trasposicións, considérese:

$$
\sigma = 
\begin{pmatrix}
1 & 6 & 5 & 4 & 3 & 2
\end{pmatrix}
= 
\begin{pmatrix}
1 & 6
\end{pmatrix}
\begin{pmatrix}
6 & 5
\end{pmatrix}
\begin{pmatrix}
5 & 4
\end{pmatrix}
\begin{pmatrix}
4 & 3
\end{pmatrix}
\begin{pmatrix}
3 & 2
\end{pmatrix}
\begin{pmatrix}
2 & 1
\end{pmatrix}
= 
\begin{pmatrix}
1 & 2
\end{pmatrix}
\begin{pmatrix}
1 & 3
\end{pmatrix}
\begin{pmatrix}
1 & 4
\end{pmatrix}
\begin{pmatrix}
1 & 5
\end{pmatrix}
\begin{pmatrix}
1 & 6
\end{pmatrix}
$$

\vspace{4mm}

\noindent \textbf{Observación 1.8.} Os ciclos disxuntos conmutan coa operación de composición. \\

\noindent Outro concepto de interese no grupo $S_{n}$ é o seguinte: \\

\noindent \textbf{Definición 1.13.} Sexa $\sigma \in S_{n}$. Defínese a \textcolor{magenta}{\textbf{signatura de $\sigma$}}, denotada por \textcolor{magenta}{$\varepsilon(\sigma)$}, como:

\begin{center}
\fcolorbox{magenta}{white}{
$\varepsilon(\sigma) = (-1)^{\text{n.º trasposicións na factorización de } \sigma}$
}  
\end{center}

\vspace{3mm}

\noindent A partir esta definición xorde unha partición de $S_{n}$ formada polos seguintes subconxuntos:

$$I_{n} = \{ \sigma \in S_{n} \hspace{1mm} | \hspace{1mm} \varepsilon(\sigma) = -1 \}
\hspace{8Mm}
A_{n} = \{ \sigma \in S_{n} \hspace{1mm} | \hspace{1mm} \varepsilon(\sigma) = 1 \}$$

\vspace{2mm}

\noindent O segundo deles, coa operación composición, resulta ser un subgrupo normal de $S_{n}$ coñecido como \textcolor{magenta}{\textbf{subgrupo alternado}}, o cal é moi importante dentro da teoría de grupos. Os elementos de $A_{n}$ denomínanse \textcolor{magenta}{\textbf{permutacións pares}}. \\

\vspace{2mm}

\noindent \textbf{Exercicio.} Comprobar que, efectivamente, $A_{n} \triangleleft S_{n}$.\\

\vspace{2mm}

\noindent Pola contra, $I_{n}$ non é subgrupo, mais posúe o mesmo cardinal ca $A_{n}$, podendo establecerse unha bixección entre eles. Recórdese que o número de permutacións de $n$ elementos é $n!$, polo que $|A_{n}| = \displaystyle \frac{n!}{2} = |I_{n}|$.

\textcolor{magenta}{\section{Homomorfismos de grupos}}

\textcolor{magenta}{\subsection{Definición e propiedades}}

\hspace{5mm}

\noindent \textbf{Definición 1.14.} Sexan $(G, \cdot)$, $(H, \circ)$ senllos grupos entre os cales se establece unha aplicación $f: G \longrightarrow H$. Tal aplicación dirase un \textcolor{magenta}{\textbf{homomorfismo dos grupos $G$ e $H$}} se cumpre: \\

\begin{center}
\fcolorbox{magenta}{white}{
$f(a \cdot b) = f(a) \circ f(b) \hspace{2mm} \forall \hspace{1mm} a,b \in G$
}
\end{center}

\vspace{3mm}

\noindent A partir da definición despréndense unha serie de \textbf{propiedades} destas aplicacións: 

\begin{enumerate}
    \item $f(1_{G}) = 1_{H}$
    \item $f(x^{-1}) = [f(x)]^{-1} \hspace{2mm} \forall \hspace{1mm} x \in G$
    \item $f(x^{m}) = [f(x)]^{m} \hspace{2mm} \forall \hspace{1mm} x \in G \hspace{2mm} \forall m \in \mathbb{Z}$
\end{enumerate}

\vspace{2mm}

\noindent \textbf{\textit{\underline{Demostración}}}

\vspace{2mm}

\noindent \magbf{(1)} Emprégase a seguinte igualdade:\\ 
\begin{equation}\label{ec2}
    f(1_{G}) = f(1_{G} \cdot 1_{G}) = f(1_{G}) \circ f(1_{G})
\end{equation}

Entón, pódese escribir:
$$1_{H} = f(1_{G}) \circ [f(1_{G})]^{-1} \overset{\overset{\eqref{ec2}}{\Downarrow}}{=} f(1_{G}) \circ f(1_{G}) \circ [f(1_{G})]^{-1} = f(1_{G}) \circ 1_{H} = f(1_{G})$$

\vspace{2mm}

Outra maneira de demostralo é empregando a idempotencia do elemento neutro respecto da operación.\\


\noindent \magbf{(2)}
\[ 
\left. \begin{array}{r} 
f(x^{-1}) \circ f(x) = f(x^{-1} \cdot x) = f(1_{G}) = 1_{H}\\[1ex]
f(x) \circ f(x^{-1}) = f(x \cdot x^{-1}) = f(1_{G}) = 1_{H}
\end{array} \right\} 
\implies f(x^{-1}) = [f(x)]^{-1}
\]

\vspace{3mm}

\noindent \magbf{(3)} Déixase como exercicio proposto. Como idea, empregar a asociatividade da operación xunto cun razoamento indutivo no expoñente. $\square$

\vspace{3mm}

\noindent Amósanse a continuación algúns exemplos de homomorfismos de grupos: 

\begin{enumerate}
    \item Dado un grupo $G$ arbitrario, a identidade:
    
    $$ id: G \longrightarrow G $$
    
    é un homomorfismo.
    
    \vspace{2mm}
    
    \item Dado un subgrupo $H$ nun grupo $G$, a aplicación inclusión:
    
    $$ i : H \xhookrightarrow{} G$$
    
    é un homomorfismo entre $H$ e $G$.
    
    \vspace{2mm}
    
    \item A aplicación:
    
    \begin{center}
        $\mathbb{Z} \longrightarrow 5\mathbb{Z}$ \\
        \vspace{2mm}
        $n \leadsto 5n$
    \end{center}  
    
    é un homomorfismo entre os grupos $\mathbb{Z}$ e $5\mathbb{Z}$.
    
    \vspace{2mm}
    
    \item A aplicación:
    
    \begin{center}
        $\mathbb{Z} \longrightarrow \mathbb{Z}/5\mathbb{Z}$ \\
        \vspace{2mm}
        $n \leadsto [n]$
    \end{center} 
    
    é un homomorfismo entre os grupos $\mathbb{Z}$ e $\mathbb{Z}/5\mathbb{Z}$.
    
    \vspace{2mm}
    
    \item A aplicación:
    
    \begin{center}
        log: $(\mathbb{R}^{+}, \cdot)  \longrightarrow (\mathbb{R}, +) $\\
        \vspace{2mm}
        $\hspace{6mm} x \leadsto$ log $x$
    \end{center} 
    
    é un homomorfismo de grupos.
    
    \vspace{2mm}
    
    \item A aplicación: %-------------Consultar con Rosa o {-1,1} en vez de Q*
    
    \begin{center}
        $(S_{n}, \circ) \longrightarrow (\{-1,1\}, \cdot)$ \\
        \vspace{2mm}
        $\hspace{6mm} \sigma \leadsto \varepsilon(\sigma)$
    \end{center} 
    
    é un homomorfismo de grupos.
    
    \vspace{2mm}
    
    \item Dado un grupo $G$ arbitrario, a aplicación:
    
    \begin{center}
        $f: G \longrightarrow G$ \\
        \vspace{2mm}
        $\hspace{6mm} x \leadsto x^{-1}$
    \end{center} 
    
    é un homomorfismo $\Longleftrightarrow G$ é un grupo abeliano. 
    
    \item Sexan $G_{1}$ e $G_{2}$ grupos. Cúmprese que as aplicacións proxección:
    
    \begin{center}
        $G_{1} \times G_{2} \longrightarrow G_{1}$ \\
        \vspace{2mm}
        $(g_{1}, g_{2}) \leadsto g_{1}$
    \end{center} 
    
    \begin{center}
        $G_{1} \times G_{2} \longrightarrow G_{2}$ \\
        \vspace{2mm}
        $(g_{1}, g_{2}) \leadsto g_{2}$
    \end{center} 
    
    son homomorfismos de grupos.
    
\end{enumerate}

\vspace{3mm}

\noindent \textbf{Exercicio.} Demostrar que o produto cartesiano de dous grupos $(G_{1}, \cdot)$ e $(G_{2}, \circ)$ é un grupo coa operación:

    \begin{center}
        $\hspace{-45mm} \bullet : (G_{1} \times G_{2}) \times (G_{1} \times G_{2}) \longrightarrow G_{1} \times G_{2}$ \\
        \vspace{2mm}
        $((g_{1}, g_{2}),(g'_{1},g'_{2})) \hspace{2mm} \leadsto \hspace{2mm} (g_{1},g_{2}) \bullet (g'_{1}, g'_{2}) := (g_{1} \cdot g'_{1}, g_{2} \circ g'_{2})$
    \end{center} 

\textcolor{magenta}{\subsection{Subgrupos asociados a un homomorfismo}}

\vspace{5mm}

\noindent Neste apartado estúdase o comportamento dos subgrupos do dominio e o rango dun homomorfismo a través do mesmo, analizando en particular se conservan a súa condición de subgrupo.\\

\noindent Na materia de \textit{Espazos vectoriais e cálculo matricial}, cando se estudaban as aplicacións lineares, aparecían dous subespazos destacados que permitían clasificar tales aplicacións e estudar as súas propiedades: o \textbf{núcleo} e a \textbf{imaxe}. Estes dous conceptos vólvense definir agora, no marco teoría de grupos e, máis en particular, do estudo dos homomorfismos. Se no caso dos espazos vectoriais se empregaba a palabra \textit{subespazo}, aquí será substituída por \textit{subgrupo}. En concreto, o núcleo e a imaxe serán respectivos subgrupos do dominio e o rango do homomorfismo considerado, como se probará máis adiante.\\

\noindent Definimos entón:\\

\noindent \textbf{Definición 1.15.} Sexan $G$ e $H$ grupos entre os cales se establece un homomorfismo $f: G \longrightarrow H$. Defínense:\\

\begin{enumerate}
    \item O \textcolor{magenta}{\textbf{núcleo de $f$}}, denotado por \textcolor{magenta}{$Ker$ $f$}, como o seguinte conxunto:

    \begin{center}
    \fcolorbox{magenta}{white}{
        $Ker$ $f = \{x \in G \hspace{1MM} | \hspace{1MM} f(x) = 1_{H}\}$
        }  
    \end{center}
    
    \item A \textcolor{magenta}{\textbf{imaxe de $f$}}, denotada por \textcolor{magenta}{$Im$ $f$}, como o conxunto:
    
    \begin{center}
    \fcolorbox{magenta}{white}{
        $Im$ $f = \{y \in H \hspace{1MM} | \hspace{1MM} \exists \hspace{1MM} x \in G : f(x) = y\}$
        }  
    \end{center}
\end{enumerate}

\vspace{3mm}

\noindent O seguinte resultado é a xeneralización ó caso dos homomorfismos dunha proposición vista xa para espazos vectoriais:\\

\begin{proposition} \label{prop1.7}
Sexan $(G, \cdot)$, $(H, \circ)$ senllos grupos entre os cales se establece un homomorfismo $f: G \longrightarrow H$. Entón:
\begin{center}
    $f$ é inxectivo $\Longleftrightarrow Ker$ $f = \{1_{G}\}$
\end{center}
\end{proposition}

\vspace{2mm}

\noindent \textbf{\textit{\underline{Demostración}}}

\vspace{2mm}

\noindent \fcolorbox{magenta}{white}{$\Longrightarrow$/} Supóñase $f$ inxectivo. Sexa $a \in Ker$ $f$, e demóstrese que $a = 1_{G}$. En efecto, como $a \in Ker$ $f$, en particular $f(a) = 1_{H}$. Como a imaxe do elemento neutro de $G$ por $f$ é o elemento neutro de $H$, sendo $f$ inxectivo, necesariamente $a = 1_{H}$.\\

\noindent \fcolorbox{magenta}{white}{$\Longleftarrow$/} Reciprocamente, supóñase que $Ker$ $f = 1_{H}$.  Sexan $a,b \in G \hspace{1mm} | \hspace{1mm} f(a) = f(b)$, e próbese que $a = b$. Tense:
\begin{center}
$f(a) = f(b) \implies f(a) \circ [f(b)]^{-1} = f(b) \circ [f(b)]^{-1} \implies f(a) \circ [f(b)]^{-1} = 1_{H} \implies f(a \cdot b^{-1}) = 1_{H} \implies$
\end{center}
\begin{center}
$a \cdot b^{-1} \in Ker \hspace{1mm} f \underset{\underset{\textbf{HIP}}{\Uparrow}}{=} \{1_{G}\} \implies a \cdot b^{-1} = 1_{G} \implies a = b \hspace{3mm} \square$
\end{center}

\vspace{3mm}

\begin{proposition} \label{prop1.8}
Sexan $(G, \cdot)$, $(H, \circ)$ senllos grupos entre os cales se establece un homomorfismo $f: G \longrightarrow H$. Entón:
\begin{enumerate}
    \item $G' < G \implies f(G') < H$
    \item $H' < H \implies f^{-1}(H') < G$
    \item Se $G' \triangleleft G$ e $f$ é sobrexectivo $\implies f(G') \triangleleft H$
    \item $H' \triangleleft H \implies f^{-1}(H') \triangleleft G$
\end{enumerate}
\end{proposition}

\vspace{2mm}

\noindent \textbf{\textit{\underline{Demostración}}}

\vspace{2mm}

\noindent \textcolor{magenta}{\textbf{(1)}} Véxase en, primeiro lugar, que $f(G') \neq \varnothing$. Certamente, sendo $G'$ subgrupo de $G$, en particular $1_{G} \in G'$, logo $f(1_{G}) = 1_{H} \in f(G') \implies f(G') \neq \varnothing$ \\

\noindent Hai que comprobar a continuación que se cumpre a condición de subgrupo, i.e. $\forall \hspace{1mm} y_{1},y_{2} \in f(G')$, $y_{1} \circ y_{2}^{-1} \in f(G')$. Tense:
$$y_{1},y_{2} \in f(G') \implies \exists \hspace{1MM} x_{1},x_{2} \in G' \hspace{1MM} | \hspace{1MM} f(x_{1}) = y_{1}, f(x_{2}) = y_{2}$$

$$y_{1} \circ y_{2}^{-1} = f(x_{1}) \circ f(x_{2})^{-1} = f(x_{1}) \circ f(x_{2}^{-1}) = f(x_{1} \cdot x_{2}^{-1}) \underset{\underset{G' < G}{\Uparrow}}{\implies} y_{1} \circ y_{2}^{-1} \in f(G')$$

\noindent \textcolor{magenta}{\textbf{(2)}} Primeiramente, compróbase que $f^{-1}(H') \neq \varnothing$. Sendo $H'$ subgrupo de $H$, $1_{H} = f(1_{G}) \in H'\implies 1_{G} \in f^{-1}(H') \implies f^{-1}(H') \neq \varnothing$\\

\noindent Verifíquese agora que $f^{-1}(H')$ cumpre a condición de subgrupo, i.e. que $\forall \hspace{1mm} x,y \in f^{-1}(H'), x \cdot y \in f^{-1}(H')$. Tense:\\

 $x \cdot y^{-1} \in f^{-1}(H') \Longleftrightarrow f(x \cdot y^{-1}) \in H' \Longleftrightarrow f(x) \circ f(y^{-1}) \in H' \Longleftrightarrow f(x) \circ f(y)^{-1} \in H'$\\
 
\noindent o cal é certo, pois $H' < H$. \\

\noindent \textcolor{magenta}{\textbf{(3)}} Por \magbf{(1)} cúmprese que $f(G')$ é subgrupo de $H$. Falta comprobar que se conserva o carácter normal, isto é: 
$$\forall \hspace{1mm} h \in H \hspace{2MM} \forall \hspace{1mm} f(g') \in f(G') \hspace{2mm} h \circ f(g') \circ h^{-1} \in f(G')$$

\noindent Sendo $f$ unha aplicación sobrexectiva, garántese que $\forall \hspace{1mm} h \in H \hspace{2mm} \exists \hspace{1mm} h \in G \hspace{1mm} | \hspace{1mm} f(x) = h$. Así:\\

$h \circ f(g') \circ h^{-1} = f(x) \circ f(g') \circ f(x)^{-1} = f(x \cdot g') \circ f(x^{-1}) = f(x \cdot g' \cdot x^{-1}) \underset{G' \triangleleft G}{\in} f(G') \implies f(G') \triangleleft H $ \\

\noindent \textcolor{magenta}{\textbf{(4)}} Aplicando \magbf{(2)} tense que $f^{-1}(H') < G$. Falta probar que se conserva o carácter normal, isto é:

$$\forall \hspace{1mm} a \in G \hspace{2MM} \forall \hspace{1mm} x \in f^{-1}(H') \hspace{2mm} a \cdot x \cdot a^{-1} \in f^{-1}(H')$$

\noindent Como $f$ é homomorfismo de grupos, cúmprese que $f(a \cdot x \cdot a^{-1}) = f(a) \circ f(x) \circ f(a^{-1})$. Tense: 
\[ 
\left. \begin{array}{r} 
a \in G \implies f(a) \in f(G)\\[1ex]
x \in f^{-1}(H') \implies f(x) \in H'\\[1ex]
H' \triangleleft H
\end{array} \right\} 
\implies f(a) \circ f(x) \circ f(a)^{-1} \in H' \implies f(a \cdot x \cdot a^{-1}) \in H' \implies a \cdot x \cdot a^{-1} \in f^{-1}(H')\hspace{2MM} \square
\]

\vspace{3mm}

\noindent Como consecuencia inmediata deste resultado tense: \\

\begin{corollary} \label{cor1.2}
Sexan $(G, \cdot)$, $(H, \circ)$ senllos grupos entre os cales se establece un homomorfismo $f: G \longrightarrow H$. Entón:
\begin{enumerate}
    \item $Ker$ $f$ é un subgrupo normal de $G$
    \item $Im$ $f$ é un subgrupo de $H$
\end{enumerate}
\end{corollary}

\noindent \textbf{\textit{\underline{Demostración}}}

\vspace{2mm}

\noindent \textcolor{magenta}{\textbf{(1)}} $\{1_{H}\}$ é un subgrupo normal de $H$, polo que aplicando o apartado \textbf{(4)} da proposición obtense o resultado. \\

\noindent \textcolor{magenta}{\textbf{(2)}} Como $G < G$, $f(G) = Im(f) < H$ \hspace{2MM} $\square$

\vspace{3mm}

\noindent Todo homomorfismo é en particular unha aplicación, e como tal pode posuír carácter inxectivo, sobrexectivo ou bixectivo. Tamén pode suceder que os grupos do dominio e a imaxe coincidan. Para estes casos emprégase unha terminoloxía que se introduce a continuación, e que é análoga á empregada para as aplicacións lineares: \\

\noindent \textbf{Definición 1.16.} Sexa $f: G \longrightarrow H$ un homomorfismo de grupos.
\begin{enumerate}
    \item Se $G = H$, dise que $f$ é un \magbf{endomorfismo}.
    \item Se $f$ é inxectivo, dise que é un \magbf{monomorfismo}.
    \item Se $f$ é sobrexectivo, denominarase \magbf{epimorfismo}.
    \item Se $f$ é bixectivo, chamaráselle \magbf{isomorfismo}.
    \item Se $f$ é un isomorfismo tal que $G = H$, dise que $f$ é un \magbf{automorfismo}.
\end{enumerate}

\vspace{3mm}

\noindent \textbf{Definición 1.17.} Sexan $G$ e $H$ grupos. Dirase que $G$ e $H$ son \magbf{isomorfos}, e denotarase por \textcolor{magenta}{$G \simeq H$}, se existe un isomorfismo de grupos $f: G \longrightarrow H$.\\

\noindent A continuación preséntase un epimorfismo de grupos que será empregado con frecuencia nas demostracións do seguinte apartado: 

\begin{mdframed}[linecolor = magenta]

\noindent Sexa $G$ un grupo arbitario e $N$ un subgrupo normal de $G$, de tal xeito que se poida dotar o conxunto cociente $G/N$ con estrutura de grupo. Tense que a aplicación: 

    \begin{center}
        $p$ : $G  \longrightarrow \displaystyle \frac{G}{N} $\\
        \vspace{2mm}
        $\hspace{4mm} a \leadsto aN$
    \end{center} 
    
\noindent denominada \magbf{proxección canónica de $G$ sobre $G/N$}, é un epimorfismo de grupos. En efecto, é sobrexectiva, pois todo elemento de $G/N$ é clase de equivalencia dun elemento $a \in G$.\\

\noindent Calcúlese o núcleo de $p$:

$$Ker(p) = \{ a \in G \hspace{1mm} | \hspace{1mm} aN = 1_{G/N} = 1N\} = N$$

\end{mdframed}

\textcolor{magenta}{\subsection{Teoremas de isomorfía}}

\vspace{5mm}

\noindent Os \magbf{teoremas de isomorfía}, tamén coñecidos como os \magbf{teoremas de isomorfía de Noether}, son tres teoremas salientables da teoría de grupos. Estes describen a relación entre os homomorfismos, os grupos cociente e os subgrupos normais, e son de grande utilidade para construír isomorfismos de grupos. A súa importancia radica en que existen analoxías para outras moitas estruturas alxébricas, ademais dos grupos, coma os aneis ou os módulos.\\

\noindent Estes resultados foron formulados pola matemática alemá Emmy Noether (1882-1935) no seu artigo, \textit{Abstrakter Aufbau der Idealtheorie in algebraischen Zahl-und Funktionenkörpern}, o cal foi publicado en 1927 na revista científica \textit{Mathematische Annalen}.\\

\vspace{2mm}

\begin{theorem}[\magbf{1.º Teorema de Isomorfía}] \label{th1.5} 
Sexan $(G, \cdot)$, $(H, \circ)$ senllos grupos entre os cales se establece un homomorfismo $f: G \longrightarrow H$. Entón, cúmprese:
\begin{center}
    $\displaystyle \frac{G}{Ker \hspace{1mm} f} \simeq Im \hspace{1mm} f$
\end{center}
\end{theorem}

\vspace{2mm}

\noindent \textbf{\textit{\underline{Demostración}}}

\vspace{2mm}

\noindent En primeiro lugar, cómpre notar que, sendo $Ker \hspace{1mm} f$ un subgrupo normal de $G$, o conxunto  $\displaystyle \frac{G}{Ker \hspace{1mm} f}$  está dotado de estrutura de grupo. Ademais, recórdese que $Im \hspace{1mm} f$ é un subgrupo de $H$. \\

\noindent Defínese a seguinte correspondencia entre os grupos $\displaystyle \frac{G}{Ker \hspace{1mm} f}$ e $Im \hspace{1mm} f$: 

    \begin{center}
        $p: \displaystyle \frac{G}{Ker \hspace{1mm} f} \longrightarrow Im \hspace{1mm} f$ \\
        \vspace{2mm}
        $\hspace{22mm} a(Ker \hspace{1mm} f) \leadsto p(\hspace{1mm}a(Ker \hspace{1mm} f)\hspace{1mm}) := f(a)$
    \end{center} 
    
\noindent Vaise demostrar que esta correspondencia é, en realidade, un isomorfismo de grupos. Para iso, hai que probar os seguintes puntos: 

\begin{itemize}
    \item É $p$ unha aplicación?
    
    Tense que ver que $p$ é unha aplicación ben definida, isto é, que non depende do representante da clase de equivalencia. Noutras palabras: 
    
    $$\forall \hspace{1mm} a,b \in G \hspace{1mm} | \hspace{1mm} [a] = [b], \hspace{2mm} f(a) = f(b) ? $$
    
    Tense: 
    
    \begin{center}
    $a \cdot Ker \hspace{1mm} f = b \cdot Ker \hspace{1mm} f \implies a \sim b \implies a^{-1} \cdot b \in Ker \hspace{1mm} f \implies f(a^{-1} \cdot b) = 1_{H} \implies $
    \end{center}
    \begin{center}
        $f(a^{-1}) \circ f(b) = 1_{H} \implies f(a)^{-1} \circ f(b) = 1_{H} \implies f(b) = f(a)$
    \end{center}
    
    Así, cúmprese que $p$ é unha aplicación ben definida.\\
    
    \item É $p$ un homomorfismo?
    
    $p( \hspace{1mm} (a \cdot Ker \hspace{1mm} f) \cdot (b \cdot Ker \hspace{1mm} f) \hspace{1mm} ) = p(a \cdot b \cdot Ker \hspace{1mm} f) = f(a \cdot b) = f(a) \circ f(b) = p(a \cdot Ker \hspace{1mm} f) \circ p(b \cdot Ker \hspace{1mm} f)$
    
    Verifícase entón que, efectivamente, $p$ é un homomorfismo.\\
    
    \item É $p$ sobrexectiva?
    
    Si, pois \hspace{2mm} $\forall \hspace{1mm} f(a) \in Im \hspace{1mm} f, \exists \hspace{1mm} a \cdot Ker \hspace{1mm} f \hspace{1mm} | \hspace{1mm} p( a \cdot Ker \hspace{1mm} f) = f(a)$\\ 
    
    \item É $p$ inxectiva?
    
    Toda vez que se coñece que $p$ é un homomorfismo de grupos, probar o seu carácter inxectivo é equivalente a demostrar que $Ker \hspace{1mm} p$ = $Ker \hspace{1mm} f$. Tense: \\
    
    $Ker \hspace{1mm} p = \{a \cdot Ker \hspace{1mm} f \in \displaystyle \frac{G}{Ker \hspace{1mm} f} \hspace{1mm} | \hspace{1mm} f(a) = 1 \} = \{a \cdot Ker \hspace{1mm} f \in \displaystyle \frac{G}{Ker \hspace{1mm} f} \hspace{1mm} | \hspace{1mm} a \in Ker \hspace{1mm} f \} = Ker \hspace{1mm} f$
    
\end{itemize}

\noindent Por todo o anterior, tense que $p$ é un isomorfismo de grupos. Entón, pódese concluír que $\displaystyle \frac{G}{Ker \hspace{1mm} f} \simeq Im f$. $\square$

\vspace{3mm}

\begin{theorem}[\magbf{2.º Teorema de Isomorfía}] \label{th1.6}
Sexa $G$ un grupo. Supóñanse $H$ e $K$ subgrupos normais de $G$, cumprindo $K \subset H$. Entón, verifícase:
\begin{center}
    $\displaystyle \frac{G/K}{H/K}$ $\simeq \displaystyle \frac{G}{H}$
\end{center}
\end{theorem}

\vspace{2mm}

\noindent \textbf{\textit{\underline{Demostración}}}

\noindent Sendo $K$ un subgrupo normal, e como $K \subset H$, tense que os conxuntos $\displaystyle \frac{G}{K}$ e $\displaystyle \frac{H}{K}$ están dotados da estrutura de grupo. \\

\noindent O proceso a seguir na demostración é o seguinte: construirase un epimorfismo entre os grupos $\displaystyle \frac{G}{K}$ e $\displaystyle \frac{G}{H}$ cuxo núcleo sexa $\displaystyle \frac{H}{K}$, para despois poder aplicar o \hyperref[th1.5]{\magbf{1.º Teorema de Isomorfía}}.\\

\noindent Así pois, defínase a seguinte correspondencia entre $\displaystyle \frac{G}{K}$ e $\displaystyle \frac{G}{H}$: 

    \begin{center}
            $f: \displaystyle \frac{G}{K} \longrightarrow \displaystyle \frac{G}{H}$ \\
        \vspace{2mm}
        $\hspace{22mm} x \cdot K \leadsto f(x \cdot K) := x \cdot H$
    \end{center} 
    
\noindent Para probar que $f$ é un epimorfismo hai que ver 3 aspectos: 

\begin{itemize}
    
    \item É $f$ unha aplicación ben definida?
    
    Dados $x,y \in G$, $xK = yK \implies xH = yH$?
    
    \[ 
    \left. \begin{array}{r} 
    xK = yK \Longleftrightarrow x^{-1}y \in K \\[1ex]
    K \subset H
    \end{array} \right\} 
    \implies x^{-1}y \in H \implies xH = yH \implies f \text{ é aplicación}
    \]
    
    \item É $f$ un homomorfismo de grupos?
    
    \begin{center}
        $f(xK \cdot yK) = f(xy \cdot K) = xy \cdot H = xH \cdot yH = f(xK) \cdot f(yK) \hspace{5mm} \forall \hspace{1mm} xK, yK \in \displaystyle \frac{G}{K}$
    \end{center}
    
    \item É $f$ sobrexectiva?
    
    Si, pois \hspace{2mm} $\forall \hspace{1mm} xH \in \displaystyle \frac{G}{H} \hspace{2mm} \exists \hspace{1mm} xK \in \displaystyle \frac{G}{K} \hspace{1mm} | \hspace{1mm} f(xK) = xH \implies Im \hspace{1mm} f = \frac{G}{H}$
    
\end{itemize}

\noindent Cúmprese así que $f$ é, efectivamente, un epimorfismo de grupos. Calcúlese agora o seu núcleo: 

$$Ker \hspace{1mm} f = \{xK \in \displaystyle \frac{G}{K} \hspace{1mm} | \hspace{1mm} f(xK) = 1_{G/H}\} = \{xK \in \displaystyle \frac{G}{K} \hspace{1mm} | \hspace{1mm} xH = H \} = \{xK \in \displaystyle \frac{G}{K} \hspace{1mm} | \hspace{1mm} x \in H\} = \displaystyle \frac{H}{K}$$

\noindent Xuntando todo anterior, e aplicando o \hyperref[th1.5]{\magbf{1.º Teorema de Isomorfía}}, pódese garantir que, certamente, $\displaystyle \frac{G/K}{H/K} \simeq \displaystyle \frac{G}{H}$. $\square$

\vspace{2mm}

\begin{theorem}[\magbf{3.º Teorema de Isomorfía}] \label{th1.7}
Sexa $G$ un grupo. Supóñanse $H$ un subgrupo normal de $G$ e $K$ un subgrupo arbitrario de $G$. Entón, cúmprese:
\begin{center}
    $\displaystyle \frac{KH}{H} \simeq \displaystyle \frac{K}{K \cap H}$
\end{center}
\end{theorem}

\vspace{2mm}

\noindent \textbf{\textit{\underline{Demostración}}}

\vspace{2mm}

\begin{mdframed}[linecolor = magenta]

\noindent Antes de comezar coa demostración do teorema, cómpre facerse a seguinte pregunta: \textit{por que $KH$ é un subgrupo de $G$?}\\

\noindent Como $H$ e $K$ son subgrupos, $1 \in K \cap H $. Logo, $1 \cdot 1 = 1 \in KH \implies KH \neq \varnothing$\\

\noindent Dados $kh, k_{1}h_{1} \in KH$, aplicarase o \hyperref[th1.1]{\magbf{Teorema 1.1}} para demostrar que $KH$ é en verdade un subgrupo:

$$(kh)(k_{1}h_{1})^{-1} = khh_{1}^{-1}k_{1}^{-1} = kh_{2}k_{1}^{-1} \underset{H \triangleleft G}{=} kk_{1}^{-1}h_{3} = k_{2}h_{3} \in KH$$

\end{mdframed}

\vspace{3mm}

\noindent O esquema da proba deste teorema é o mesmo ca no caso do \hyperref[th1.6]{\magbf{2.º Teorema de Isomorfía}}: construirase un epimorfismo entre os grupos $K$ e $\displaystyle \frac{KH}{H}$, de tal xeito que o seu núcleo sexa $K \cap H$ e se poida aplicar o \hyperref[th1.5]{\magbf{1.º Teorema de Isomorfía}}.\\

\noindent Considérese, pois, a aplicación: 

    \begin{center}
            $f: \displaystyle K \longrightarrow \displaystyle \frac{KH}{H}$ \\
        \vspace{2mm}
        $\hspace{15mm} x \leadsto f(x) := xH$
    \end{center} 
    
\noindent a cal está ben definida, pois todo elemento posúe unha única clase de equivalencia asociada. Queda por probar entón: 

\begin{itemize}
    
    \item É $f$ un homomorfismo?
    
    $$\forall x,y \in K \hspace{4mm} f(xy) = xy \cdot H = xH \cdot yH = f(x) \cdot f(y) \implies f\text{ é homomorfismo}$$  
    
    \item É $f$ sobrexectiva?

    $$\forall \hspace{1mm} khH \in \displaystyle \frac{KH}{H} \hspace{2mm} \exists \hspace{1mm} k \in K \hspace{1mm} | \hspace{1mm} f(k) = kH = khH \text{ ?}$$    
    
    $$kH = khH \Longleftrightarrow k^{-1}kh = h \in H \Longleftrightarrow Im \hspace{1mm} f = \displaystyle \frac{KH}{H}$$
    
\end{itemize}

\noindent Con isto, queda demostrado que $f$ é un epimorfismo de grupos. Calcúlese a continuación o seu núcleo: 

$$Ker \hspace{1mm} f = \{ k \in K \hspace{1mm} | \hspace{1mm} kH = H \} = \{k \in K \hspace{1mm} | \hspace{1mm} k \in H \} = K \cap H$$

\noindent Aplicando entón o \hyperref[th1.5]{\magbf{1.º Teorema de Isomorfía}}, garántese que, certamente, $\displaystyle \frac{KH}{H} \simeq \displaystyle \frac{K}{K \cap H}$. $\square$ \pagebreak

\noindent A partir dos teoremas de isomorfía extráese unha enorme multitude de resultados útiles para traballar con diferentes grupos. En particular, cando o grupo é cíclico (e polo tanto, abeliano) tense o seguinte: \\

\begin{proposition} \label{prop1.9}
Sexa $(G, \cdot)$ un grupo cíclico. Entón:
\begin{enumerate}
    \item Se $G$ é finito, $\exists \hspace{1mm} n \in \mathbb{N} \hspace{1mm} | \hspace{1mm} (G, \cdot) \simeq (\mathbb{Z}_{n}, +)$
    \item Se $G$ é infinito, entón $(G, \cdot) \simeq (\mathbb{Z}, +)$
\end{enumerate}
\end{proposition}

\vspace{2mm}

\noindent \textbf{\textit{\underline{Demostración}}}

\vspace{2mm}

\noindent Recórdese que, se $G$ é un grupo cíclico, entón $\exists \hspace{1mm} a \in G \hspace{1mm} | \hspace{1mm} \langle a \rangle = G$. \\

\noindent Defínese a seguinte aplicación:

    \begin{center}
            $f: \mathbb{Z} \longrightarrow G$ \\
        \vspace{2mm}
        $\hspace{15mm} i \leadsto f(i) := a^{i}$
    \end{center} 
    
\noindent Vaise demostrar que $f$ é un epimorfismo de grupos e calcularase o seu núcleo, para posteriormente aplicar o \hyperref[th1.5]{\magbf{1.º Teorema de Isomorfía}}. Para iso, haberá que probar o seguinte:

\begin{itemize}
    
    \item É $f$ un homomorfismo de grupos?
    
    Dados $i,j \in \mathbb{Z}$, tense:
    
    $$f(i + j) = a^{i+j} = a^{i} \cdot a^{j} = f(i) \cdot f(j) \implies f \text{ é homomorfsimo}$$
    
    \item É $f$ un epimorfismo?
    
    Si, pois  $\forall a^{i} \in G, \hspace{2mm} \exists i \in \mathbb{Z} \hspace{1mm} | \hspace{1mm} f(i) = a^{i} \implies Im \hspace{1mm} f = G$ 
    
\end{itemize}

\vspace{2mm}

\noindent Calcúlese o núcleo de $f$:

$$Ker \hspace{1mm} f = \{i \in \mathbb{Z} \hspace{1mm} | \hspace{1mm} a^{i} = 1 \} = 
\begin{cases}
\{i \in \mathbb{Z} \hspace{1mm} : \hspace{1mm} n \hspace{1mm}| \hspace{1mm} i \} = n\mathbb{Z} & \text{se } |G| = n \\
\{0\} & \text{se } |G| = \infty
\end{cases}
$$

\noindent En efecto, se $G$ é infinito, $a_{i} \neq a_{j} \Longleftrightarrow i \neq j$.\\

\noindent Por todo o anterior, aplicando o \hyperref[th1.5]{\magbf{1.º Teorema de Isomorfía}}, conclúese:

\begin{itemize}
    \item Se $|G| = n$, entón $(G, \cdot) \simeq ( \displaystyle \frac{\mathbb{Z}}{n\mathbb{Z}}, +)$
    \item Se $|G| = \infty$, entón $(G, \cdot) \simeq (\mathbb{Z}, +)$
\end{itemize}

\noindent $\square$

\textcolor{magenta}{\subsection{O teorema de correspondencia. Subgrupos do grupo cociente}}

\vspace{5mm}

\noindent No estudo da estrutura dos grupos cociente, interesa coñecer os seus subgrupos. Máis aínda: cabe preguntarse se os subgrupos do grupo cociente $G/N$ se relacionan dalgún xeito cos subgrupos de $G$.\\

\noindent A resposta é afirmativa, e o resultado que recolle tal relación é consecuencia inmediata dun resultado máis xeral: o \magbf{teorema de correspondencia}, que se enuncia a continuación:\pagebreak

\vspace{5mm}

\begin{theorem}[\magbf{Teorema de Correspondencia}] \label{th1.8}
Sexan $(G, \cdot)$ e $(H, \circ)$ grupos entre os cales se establece un epimorfismo $f: G \longrightarrow H$. Considérense os seguintes conxuntos:
$$S = \{G' < G \hspace{1mm} | \hspace{1mm} Ker \hspace{1mm} f \subset G'\} \hspace{10mm} T = \{H' < H\}$$
\noindent Cúmprese que os conxuntos $S$ e $T$ están en bixección. En particular, as aplicacións: 
    \begin{center}
            $\phi: S \longrightarrow T$ \\
        \vspace{2mm}
        $\hspace{25mm} G' \leadsto \phi(G') := f(G')$
    \end{center} 
\noindent e a súa inversa:
    \begin{center}
            $\phi^{-1}: T \longrightarrow S$ \\
        \vspace{2mm}
        $\hspace{35mm} H' \leadsto \phi^{-1}(H') := f^{-1}(H')$
    \end{center} 
\noindent son bixectivas.
\end{theorem}

\vspace{2mm}

\noindent \textbf{\textit{\underline{Demostración}}}

\vspace{2mm}

\noindent Sendo $f$ un homomorfismo, cúmprese que:

\begin{itemize}
    \item $\forall \hspace{1mm} G' \in S$, \hspace{2mm} $f(G') < H \implies f(G') \in T$
    \item $\forall \hspace{1mm} H' \in T$, \hspace{2mm} $f^{-1}(H') < G$
\end{itemize}

\noindent Entón, a primeira pregunta a contestar é a seguinte: pódese garantir que  $f^{-1}(H')$ contén a $Ker \hspace{1mm} f$?\\

\noindent A resposta é afirmativa. En efecto, como $H' < H$, en particular $1_{H} \in H'$. Logo:

$$ 1_{H} \in H' \implies f^{-1}(\{1_{H}\}) = Ker \hspace{1mm} f \subset f^{-1}(H')$$

\noindent Probarase entón que $\phi$ e $\phi^{-1}$ son bixeccións entre $S$ e $T$. Para iso, tal e como recolle o seguinte esquema:

$$ G' \overset{\phi}{\longrightarrow} f(G') \overset{\phi^{-1}}{\longrightarrow} f^{-1}(f(G')) \overset{?}{=} G' $$

$$ H' \overset{\phi^{-1}}{\longrightarrow} f^{-1}(H') \overset{\phi}{\longrightarrow} f(f^{-1}(H')) \overset{?}{=} H'$$

\noindent abonda demostrar as igualdades sinaladas.\\

\noindent No caso da segunda igualdade, esta tense polo \textbf{carácter sobrexectivo} de $f$ (a modo de exercicio, pódese comprobar). Non obstante, para a primeira, a sobrexectividade non aporta nada para probala. Véxase entón por dobre inclusión:\\

\noindent \fcolorbox{magenta}{white}{$\supset/$} $x \in G' \implies f(x) \in f(G') \implies x \in f^{-1}(f(G'))$ \\

\noindent \fcolorbox{magenta}{white}{$\subset/$} $x \in f^{-1}(f(G)) \implies f(x) \in f(G) \implies \exists \hspace{1mm} g_{1} \in G' \hspace{1mm} | \hspace{1mm} f(x) = f(g_{1}) \implies f(x) \circ f(g_{1})^{-1} = 1_{H} \implies 1_{H} = f(x \cdot (g_{1})^{-1}) \implies x \cdot (g_{1})^{-1} \in Ker \hspace{1mm} f \subset G' \implies \exists \hspace{1mm} g_{2}\in G' \hspace{1mm} | \hspace{1mm} x \cdot (g_{1})^{-1} = g_{2} \implies x = g_{2} \cdot g_{1}$\\

\noindent Así, $x$ escríbese como operación de dous elementos de $G'$, polo que $x \in G'$. $\square$

\vspace{3mm}

\noindent Como consecuencia inmediata deste resultado tense:

\begin{corollary}
Sexa $G$ un grupo e $N$ un subgrupo normal de $G$. Considérense os seguintes conxuntos:
$$E = \{G' < G \hspace{1mm} | \hspace{1mm} G' \supset N \} \hspace{10mm} F = \{ H < \displaystyle \frac{G}{N} \}$$
\noindent Cúmprese que existe unha bixección entre $E$ e $F$.
\end{corollary}

\vspace{2mm}

\noindent \textbf{\textit{\underline{Demostración}}}

\vspace{2mm}

\noindent A proxección canónica de $G$ sobre $\displaystyle \frac{G}{N}$ é un epimorfismo cuxo núcleo é $N$. Aplicándolle o \hyperref[th1.8]{\magbf{Teorema de Correspondencia}}, o resultado é inmediato. $\square$

\vspace{3mm}

\noindent Gracias a este resultado pódense calcular, por exemplo, tódolos subgrupos do grupo cociente $\displaystyle \frac{\mathbb{Z}}{12\mathbb{Z}} \equiv \mathbb{Z}_{12}$:\\

\begin{mdframed}[linecolor = magenta]

\noindent En efecto, como $12\mathbb{Z} \triangleleft \mathbb{Z}$, o conxunto $\mathbb{Z}_{12}$ está dotado de estrutura de grupo. \\

\noindent Tódolos subgrupos de $\mathbb{Z}$ son da forma $m\mathbb{Z}$, con $m \in \mathbb{Z}$. Entón, aplicando o \hyperref[th1.8]{\magbf{Teorema de Correspondencia}}, pódese asegurar que os subgrupos de $\mathbb{Z}_{12}$ son da forma $\displaystyle \frac{m\mathbb{Z}}{12\mathbb{Z}}$. Agora ben, en base ao \hyperref[th1.2]{\magbf{Teorema de Lagrange}}, sendo $\mathbb{Z}_{12}$ finito, $|\displaystyle \frac{m\mathbb{Z}}{12\mathbb{Z}}|$ é divisor de $|\mathbb{Z}_{12}| = 12$.\\

\noindent Doutra banda, de acordo co \hyperref[th1.6]{\magbf{2.º Teorema de Isomorfía}}, $\displaystyle \frac{\mathbb{Z}/12\mathbb{Z}}{m\mathbb{Z}/12\mathbb{Z}} \simeq \displaystyle \frac{\mathbb{Z}}{m\mathbb{Z}}$, podendo garantir entón que $\displaystyle \frac{12}{x} = m$, onde $x = |\displaystyle \frac{m\mathbb{Z}}{12\mathbb{Z}}|$. Logo, tense que $m \in \{1,2,3,4,6,12\}$.\\

\noindent Así, conclúese que os subgrupos de $\displaystyle \frac{\mathbb{Z}}{12\mathbb{Z}}$ son: \\

$$\{\displaystyle \frac{\mathbb{Z}}{12\mathbb{Z}}, \displaystyle \frac{2\mathbb{Z}}{12\mathbb{Z}}, \displaystyle \frac{3\mathbb{Z}}{12\mathbb{Z}}, \displaystyle \frac{4\mathbb{Z}}{12\mathbb{Z}}, \displaystyle \frac{6\mathbb{Z}}{12\mathbb{Z}}, \displaystyle \frac{12\mathbb{Z}}{12\mathbb{Z}}\}$$

\end{mdframed}

\textcolor{magenta}{\subsection{O teorema de Cayley}}

\vspace{5mm}

\noindent Cando se estudan os grupos, en moitas ocasións, ante o alto nivel de abstracción esixido, interesa poder representar un grupo arbitario en termos máis concretos , de xeito que resulte máis sinxela esta tarefa ao traballar cun grupo máis familiar. \\

\noindent Con tal fin, o matemático inglés Arthur Cayley (1821-1895) demostrou no 1878 un teorema que reduce o estudo de calquera grupo ó estudo dos subgrupos de \textbf{grupos simétricos}. Cando Cayley introduciu o que hoxe se denomina \textit{grupo}, non quedaba claro de inmediato que iso era equivalente aos grupos xa coñecidos naquel entón, que hoxe en día chamamos \textit{grupos de permutación}.\\

\noindent Antes de introducir tal teorema, cómpre definir: \\

\noindent \textbf{Definición 1.18}. Sexa $G$ un grupo arbitrario. Defínese o \magbf{grupo simétrico de $G$}, denotado por \textcolor{magenta}{$(S_{G}, \circ)$}, como o conxunto de tódalas aplicacións bixectivas de $G$ en $G$ dotado da operación composición.\\

\noindent Con isto, enúnciase a continuación: \\

\begin{theorem}[\magbf{Teorema de Cayley}] \label{th1.9}
Todo grupo $G$ é isomorfo a un subgrupo dun grupo simétrico.
En particular, se $G$ é finito, con $|G| = n$, entón é isomorfo a un subgrupo de $S_{n}$.
\end{theorem}

\vspace{2mm}

\noindent \textbf{\textit{\underline{Demostración}}}

\vspace{2mm}

\noindent Fíxese $a \in G$. Defínese a \textbf{translación segundo $a$ pola esquerda} como a seguinte aplicación:\\

    \begin{center}
            $\gamma_{a}: G \longrightarrow G$ \\
        \vspace{2mm}
        $\hspace{19mm} x \leadsto \gamma_{a}(x) := ax$
    \end{center} 

\noindent Tense que $\gamma_{a}$ é bixectiva   $\forall \hspace{1mm} a \in G$. En efecto:

\begin{itemize}
    
    \item $\gamma_{a}$ é inxectiva
    
    Dados $x,y \in G$, $\gamma_{a}(x) = \gamma_{a}(y) \Longleftrightarrow ax = ay \Longleftrightarrow a^{-1}ax = a^{-1}ay \Longleftrightarrow x = y$
    
    \item $\gamma_{a}$ é sobrexectiva
    
    Dado $y \in G$, $\exists! \hspace{1mm} x \in G \hspace{1mm} | \hspace{1mm} y = ax \Longleftrightarrow x = a^{-1}y$
    
\end{itemize}

\noindent Así, a cada elemento $a \in G$ pódeselle asignar unha única bixección $\gamma_{a} \in S_{G}$. Isto permite definir, á súa vez, unha nova aplicación: \\

    \begin{center}
            $\Phi: G \longrightarrow S_{G}$ \\
        \vspace{2mm}
        $\hspace{17mm} a \leadsto \Phi(a) := \gamma_{a}$
    \end{center} 
    
\noindent A aplicación $\Phi$ resulta ser un monomorfismo de grupos. Certamente:

\begin{itemize}
    
    \item $\Phi$ é un homomorfismo de grupos
    
    Dados $a,b \in G$, $\forall x \in G$, tense: 
    
    $$[\Phi(a) \circ \Phi(b)](x) = (\gamma_{a} \circ \gamma_{b})(x) = \gamma_{a}(bx) = abx = (ab)x$$
    
    Así, $\Phi(a) \circ \Phi(b) = ab \hspace{4mm} \forall \hspace{1mm} a,b \in G$
    
    \item $\Phi$ é inxectivo
    
    Calcúlese o núcleo de $\Phi$:
    $$Ker \hspace{1mm} \Phi = \{ a \in G \hspace{1mm} | \hspace{1mm} \gamma_{a} = id_{G} \} = \{ a \in G \hspace{1mm} | \hspace{1mm} \gamma_{a}(x) = x \hspace{3mm} \forall \hspace{1mm} x \in G \} = \{ a \in G \hspace{1mm} | \hspace{1mm} ax = x \hspace{3mm} \forall \hspace{1mm} x \in G \} = \{1_{G}\}$$
    
\end{itemize}

\noindent Como $\Phi$ é un homomorfismo de grupos, o \hyperref[th1.5]{\magbf{1.º Teorema de Isomorfía}} garante que $G \simeq Im \hspace{1mm} \Phi$. Sendo a imaxe de $\Phi$ un subgrupo de $S_{G}$, obtense o resultado. $\square$

\vspace{3mm}

\textcolor{magenta}{\section{Grupos libres}}

\vspace{5mm}

\noindent Dado un conxunto arbitrario, verase neste apartado como, a partir del, se pode construír un grupo, que recibirá o nome de \magbf{grupo libre}.\\

\noindent Considérese un conxunto calquera $X$. Sexa outro conxunto, que será denotado por $X^{-1}$, o cal é disxunto con $X$ (i.e. $X \cap X^{-1} = \varnothing$) e bixectivo con el, é dicir, existe unha aplicación bixectiva: \\

    \begin{center}
            $f: X \longrightarrow X^{-1}$ \\
        \vspace{2mm}
        $\hspace{15mm} x \leadsto f(x) := x^{-1}$
    \end{center} 
    
\noindent Sexa agora un conxunto unitario $X'$ cuxo único elemento será denotado por 1. Se $x \in X$, empregarase a notación $x^{1} \equiv x$ e $x^{0} \equiv 1$.\\

\noindent Os elementos de $X$ adoitan ser denotados por letras: $a,b,c \dots$, e os de $X^{-1}$ mediante letras con expoñente -1: $a^{-1}, b^{-1}, c^{-1} \dots$.\\

\noindent \textbf{Definición 1.19}. Unha \magbf{palabra} en $X$ é unha sucesión de elementos do conxunto $X \cup X^{-1} \cup \{1\}$, de tal xeito que, a partir dun determinado elemento, tódolos restantes da secuencia son 1:\\

\fcolorbox{magenta}{white}{
$a_{1}a_{2}a_{3}a_{4} \cdots a_{n}1 \cdots 1111 \cdots \hspace{1mm} | \hspace{1mm} a_{i} \in X \cup X^{-1} \cup \{1\} \hspace{4mm} \forall \hspace{1mm} i \text{, cumprindo que } \exists n \in \mathbb{N} : a_{i} = 1 \hspace{4mm} \forall \hspace{1mm} i \geq n$
}

\vspace{2mm}

\noindent Por exemplo, a secuencia $s1txs^{-1}m^{-1}1a11 \cdots 1a^{-1}mst^{-1}1111 \cdots$ é unha palabra.\\

\noindent Suprimindo os 1s do final, toda palabra pode ser expresada da seguinte maneira:
$$\begin{pmatrix}
x_{1}^{\varepsilon_{1}} & x_{2}^{\varepsilon_{2}} & x_{3}^{\varepsilon_{3}} & \cdots & x_{n}^{\varepsilon_{n}}
\end{pmatrix}
\text{    con } x_{i} \in X ,\hspace{3mm} \varepsilon_{i} \in \{0, \pm1\} \hspace{3mm} 1 \leq i \leq n-1 ,\hspace{3mm} \varepsilon_{n} \in \{\pm1\}
$$

\noindent \textbf{Definición 1.20}. A secuencia constante $\begin{pmatrix} 1 & 1 & 1 & 1 & 1 & \cdots \end{pmatrix}$ denomínase \magbf{palabra baleira}.\\

\noindent \textbf{Definición 1.21}. Dise que unha palabra é \magbf{reducida} se coincide coa palabra baleira, ou ben se é da forma $\begin{pmatrix} x_{1}^{\varepsilon_{1}} & x_{2}^{\varepsilon_{2}} & \cdots & x_{n}^{\varepsilon_{n}}\end{pmatrix}$, con $\varepsilon_{i} \in \{\pm1\}$, e onde $x$ e $x^{-1}$ nunca son adxacentes. \\

\noindent Así, por exemplo, $xyz$ e $xyzx^{-1}$ son palabras reducidas, mentres que $xyx^{-1}xz$ non o é.\\

\noindent No conxunto $X \cup X^{-1} \cup \{1\}$, pódese definir a operación \textbf{xustaposición}, que consiste na unión de palabras como unha soa. Restrinxíndonos ao subconxunto formado polas palabras reducidas, obtense: \\

\begin{theorem} \label{th1.10}
Sexa $X$ un conxunto. Denótese por $F$ o conxunto de tódalas palabras reducidas en $X$. Cúmprese que $F$, xunto coa operación de xustaposición, posúe estrutura de grupo, que recibe o nome de \textbf{grupo libre}.
\end{theorem}

\vspace{2mm}

\noindent \textbf{\textit{\underline{Demostración}}}

\vspace{2mm}

\noindent Pódese ver en \cite{rotman}. $\square$

\vspace{3mm}

\noindent Aínda que se omite a demostración, cómpre puntualizar que o elemento neutro será a palabra baleira, e que se definirá o elemento simétrico do seguinte xeito:
$$\begin{pmatrix} x_{1}^{\varepsilon_{1}} & x_{2}^{\varepsilon_{2}} & \cdots & x_{n}^{\varepsilon_{n}}\end{pmatrix}^{-1} = \hspace{2mm} \begin{pmatrix} x_{n}^{-\varepsilon_{n}} & \cdots & x_{2}^{-\varepsilon_{2}} & x_{1}^{-\varepsilon_{1}}\end{pmatrix}$$

\noindent \textbf{Observación 1.9}. Se $X = \{x\}$, entón $F \simeq \mathbb{Z}$.\\

\noindent Todo grupo libre cumpre o seguinte: 

\begin{theorem}[\magbf{Propiedade universal do grupo libre}] \label{th1.11}
Sexa $X$ un conxunto. Denótese por $F$ o grupo libre xerado a partir de $X$. Considérese un grupo $G$ arbitrario. Sexa $i: X \longrightarrow F$ a aplicación definida por $i(x) = x$. Entón, para toda aplicación $f: X \longrightarrow G$, existe un único homomorfismo de grupos $h: F \longrightarrow G$ tal que $h \circ i = f$.
\end{theorem}

$$
     \large \xymatrix{
        X \ar[dr]_f \ar[r]^i & F \ar@{-->}[d]^h \\
          & G
    }    
$$

\vspace{3mm}

\noindent A partir do \hyperref[th1.10]{\magbf{Teorema 1.10}}, dedúcese o seguinte resultado: \\

\begin{corollary} \label{cor1.4}
Todo grupo é isomorfo a un cociente dun grupo libre.
\end{corollary}

\vspace{2mm}

\noindent \textbf{\textit{\underline{Demostración}}}

\vspace{2mm}

\noindent Sexa $G$ un grupo arbitrario. Considérese $X \subset G$ un conxunto de xeradores de $G$, i.e. $\langle X \rangle = G$ (no peor dos casos, pódese considerar $X = G$).\\

\noindent Fágase $F$ grupo libre sobre $X$. Considérense a aplicación $i: X \longrightarrow F$ da \hyperref[th1.11]{\magbf{Propiedade universal do grupo libre}} e a aplicación $inc: X \hookrightarrow G$. En base á \hyperref[th1.11]{\magbf{Propiedade universal}}, existe un único homomorfismo de grupos $h: F \longrightarrow G$ tal que $h \circ i = inc$, obtendo así o seguinte diagrama conmutativo:

$$
     \large \xymatrix{
        X \ar@{_{(}->}[dr]_{inc} \ar[r]^i & F \ar@{-->}[d]^h \\
        & G
    }    
$$

\noindent Véxase que $h$ é epimorfismo de grupos, i.e. que $h$ é sobrexectivo:\\

\noindent Segundo a \hyperref[prop1.2]{\magbf{Proposición 2}}, pódese escribir $G = \langle X \rangle = \{a_{1} \cdot \dots \cdot a_{n} \hspace{1mm} | \hspace{1mm} (a_{i} \in X)\lor(a_{i}^{-1} \in X)\}$. En particular, dado un elemento $a_{1} \cdot \dots \cdot a_{n} \in G$, este pódese interpretar como $a_{1}^{\varepsilon_{1}} \cdot \dots \cdot a_{n}^{\varepsilon_{n}}$, con $\varepsilon_{i} \in \{\pm1\}$.Tense: 
$$inc(x) = x = (h \circ i)(x) = h(i(x)) = h(x) \implies h(a_{1}^{\varepsilon_{1}} \cdot \dots \cdot a_{n}^{\varepsilon_{n}}) \underset{\underset{\text{$h$ hom.}}{\Uparrow}}{=} a_{1}^{\varepsilon_{1}} \cdot \dots \cdot a_{n}^{\varepsilon_{n}}$$

\noindent Así, obtense que $Im \hspace{1mm} h = G$. Aplicando o \hyperref[th1.5]{\magbf{1.º Teorema de Isomorfía}}, cúmprese que $G \simeq \displaystyle \frac{F}{ker \hspace{1mm} h}$. $\square$

\textcolor{magenta}{\subsection{Presentacións de grupos. Os grupos diedrais}}

\vspace{5mm}

\noindent \textbf{Definición 1.22}. Sexa $X$ un conxunto. Considérese un grupo $G$ arbitrario. Tómese o homomorfismo $h$ nas condicións da \hyperref[th1.11]{\magbf{Propiedade universal do grupo libre}} e sexa así $\Delta$ un conxunto de xeradores de $Ker \hspace{1mm} h$. Chámaselle \magbf{presentación libre de $G$} ao par \textcolor{magenta}{$\langle X \hspace{1mm} | \hspace{1mm} \Delta \rangle \equiv \langle X , \Delta \rangle$}. Os elementos de $\Delta$ denomínanse \magbf{relacións}.\\

\noindent Como exemplo desta definición estudarase un moi sinxelo exemplo de grupos finitos: os \textbf{grupos diedrais}.\\

\noindent \textbf{Definición 1.23}. Chámaselle \magbf{grupo diedral} ou \magbf{grupo diédrico}, denotado por \magbf{$D_{2n}$}, ao grupo formado polas isometrías dun polígono regular de $n$ lados.\\

\noindent Un polígono regular con $n$ lados ten $2n$ isometrías: $n$ rotacións e $n$ simetrías. Se $n$ é impar, cada eixo de simetría conecta o punto medio dun lado ao seu vértice oposto. Se $n$ é par, $n/2$ eixos conectan os puntos medios dos lados opostos e outros $n/2$ conectan vértices opostos.\\

$$
\begin{tikzpicture} %----- O triángulo e os ejes de simetría
\draw 
(0,0) coordinate (A) -- 
(3,0) coordinate (B) -- 
(1.5,2.6) coordinate (C) --
(0,0);
\draw
(1.5,0) coordinate (S_{C})
(0.75,1.3) coordinate (S_{B})
(2.25,1.3) coordinate (S_{A});
\path
(A) node[below] {A}
(B) node[below] {B}
(C) node[above right] {C};
\draw [magenta,dashed, shorten <=-0.5cm,shorten >=-0.5cm](A) -- (S_{A});
\draw [magenta,dashed, shorten <=-0.5cm,shorten >=-0.5cm](B) -- (S_{B});
\draw [magenta, dashed, shorten <=-0.5cm, shorten >=-0.5cm](C) -- (S_{C});
\end{tikzpicture}
\hspace{3cm}
\begin{tikzpicture} %-----O cuadrado e os ejes de simetría
\draw
(0,0) coordinate (A) --
(3,0) coordinate (B) --
(3,3) coordinate (C) --
(0,3) coordinate (D) --
(0,0);
\draw
(1.5,0) coordinate (AB)
(3,1.5) coordinate (BC)
(1.5,3) coordinate (CD)
(0,1.5) coordinate (DA);
\path
(A) node[below] {A}
(B) node[below] {B}
(C) node[above] {C}
(D) node[above] {D};
\draw [magenta,dashed, shorten <=-0.5cm,shorten >=-0.5cm](A) -- (C);
\draw [magenta,dashed, shorten <=-0.5cm,shorten >=-0.5cm](B) -- (D);
\draw [magenta, dashed, shorten <=-0.5cm, shorten >=-0.5cm](AB) -- (CD);
\draw [magenta,dashed, shorten <=-0.5cm,shorten >=-0.5cm](DA) -- (BC);
\end{tikzpicture}
$$

\noindent Todo grupo diedral $D_{2n}$ admite a seguinte presentación:
$$D_{2n} = \langle s, t \hspace{1mm} | \hspace{1mm} s^{n} = 1, t^{2} = 1, tsts = 1 \rangle$$

\noindent En particular, para un triángulo equilátero, tense:
$$D_{6} = \langle s, t \hspace{1mm} | \hspace{1mm} s^{3} = 1, t^{2} = 1, tsts = 1 \rangle$$

\noindent polo que os seus elementos son:
$$D_{6} = \{1,s,s^{2},t,st,s^{2}t\} = \{R_{0},R_{1},R_{2},S_{A},S_{B},S_{C}\}$$

\noindent onde $R_{i}$ é o xiro de ángulo $\displaystyle \frac{360 \cdot i}{3} = 120i$ graos $\forall \hspace{1mm} i \in \{0,1,2\}$, e $S_{A}$ denota a simetría con eixo que pasa por $A$ (analogamente para o resto de vértices).\\

\noindent Sendo $D_{6}$ un grupo finito, pódese construír a súa \magbf{táboa de Cayley}, a cal describe como é a operación definida nel (neste caso, a composición). Esta táboa foi introducida por Arthur Cayley no seu artigo de 1854, \textit{On the Theory of Groups, as depending on the symbolic equation $\theta^{n}$ = 1}.\\

\noindent Dado un grupo $G = \{g_{1}, \dots, g_{n}\}$ finito cunha operación $\cdot$ definida nel, a súa táboa de Cayley cumpre que o elemento $(i,j)$ resulta ser $g_{i} \cdot g_{j}$ $\forall \hspace{1mm} i,j \in \{1, \dots, n\}$.\\

\noindent \textbf{Observación 1.10}. Na táboa de Cayley, cada elemento do grupo aparecerá unha única vez en cada fila e cada columna. En efecto, dado un elemento $g_{i}$, a fila $i$-ésima resulta ser a imaxe da aplicación $\gamma_{g_{i}}$, definida no \hyperref[th1.9]{\magbf{Teorema de Cayley}}. Para o caso das columnas, dado un elemento $g_{j}$, a columna $j$-ésima é a imaxe da aplicación \textbf{translación segundo $g_{j}$ pola dereita}, que consiste en operar con $a$ pola dereita do elemento considerado.\\

\noindent Para o grupo $D_{6}$, a táboa de Cayley presenta o seguinte aspecto: 

$$
\begin{tabular}{|c|c|c|c|c|c|c|}
     \hline & $\magen{R_{0}}$ & $\magen{R_{1}}$ & $\magen{R_{2}}$ & $\magen{S_{A}}$ & $\magen{S_{B}}$ & $\magen{S_{C}}$\\
     \hline $\magen{R_{0}}$ &  $R_{0}$ & $R_{1}$ & $R_{2}$ & $S_{A}$ & $S_{B}$ & $S_{C}$\\
     \hline $\magen{R_{1}}$ & $R_{1}$ & $R_{2}$ & $R_{0}$ & $S_{B}$ & $S_{C}$ & $S_{A}$ \\
     \hline $\magen{R_{2}}$ & $R_{2}$ & $R_{0}$ & $R_{1}$ & $S_{C}$ & $S_{A}$ & $S_{B}$ \\
     \hline $\magen{S_{A}}$ & $S_{A}$ & $S_{C}$ & $S_{B}$ & $R_{0}$ & $R_{2}$ & $R_{1}$ \\
     \hline $\magen{S_{B}}$ & $S_{B}$ & $S_{A}$ & $S_{C}$ & $R_{1}$ & $R_{0}$ & $R_{2}$ \\
     \hline $\magen{S_{C}}$ & $S_{C}$ & $S_{B}$ & $S_{A}$ & $R_{2}$ & $R_{1}$ & $R_{0}$\\
     \hline
\end{tabular}
$$

\vspace{3mm}

\noindent \textbf{Exercicio}. Calcular os subgrupos de $D_{6}$ xunto coas súas ordes, e determinar cales deles son cíclicos. Comprobar se algún deles é subgrupo normal. Calcular o seu centro. A que grupo simétrico é isomorfo $D_{6}$ ?\\

\vspace{3mm}

\magbf{\section{Accións de grupos sobre conxuntos}}

\vspace{5mm}


\noindent As accións de grupos sobre conxuntos son un concepto unificador da teoría de grupos finitos; de feito, tamén están presentes en máis partes da álxebra abstracta. Neste apartado darase primeiramente unha base teórica das accións, para expoñer a continuación como dan pé a certos resutados con máis peso na teoría de grupos finitos; en particular, a \hyperref[prop1.11]{\magbf{Fórmula das clases}} e os \hyperref[th1.12]{\magbf{Teoremas de Sylow}}.

\vspace{3mm}

\magbf{\subsection{A noción de acción}}

\vspace{5mm}

\noindent \textbf{Definición 1.24}. Sexan $(G, \cdot)$ un grupo e $X$ un conxunto. Unha aplicación da forma:\\

\begin{center}
    $\varphi: G \times X \longrightarrow X$ \\
        \vspace{2mm}
    $\hspace{20mm} (g,x) \leadsto \varphi(g,x) := gx$
\end{center}

\noindent denomínase \magbf{acción}, \magbf{actuación} ou \magbf{operación de $G$ sobre $X$} se cumpre as seguintes condicións:

\begin{enumerate}

    \item $g(hx) = (g \cdot h)x \hspace{2mm} \forall \hspace{1mm} g,h \in G \hspace{2mm} \forall \hspace{1mm} x \in X$
    
    \item $1x = x \hspace{2mm} \forall \hspace{1mm} x \in X$
    
\end{enumerate} 

\noindent Dirase en tal caso que \magbf{$G$ actúa} ou \magbf{opera sobre $X$ mediante $\varphi$}.\\

\noindent Exemplos de accións de grupos sobre conxuntos son os seguintes:

\begin{enumerate}
    
    \item Toda operación interna nun grupo $G$ é unha acción de $G$ sobre si mesmo:
    
    \begin{center}
    $\cdot : G \times G \longrightarrow G$ \\
    \vspace{2mm}
    \hspace{6mm} $(x,y) \leadsto x \cdot y$
    \end{center}  
    
    \item A conxugación de elementos dun grupo $G$ é unha acción de $G$ sobre si mesmo:
    
    \begin{center}
    $G \times G \longrightarrow G$ \\
    \vspace{2mm}
    \hspace{8mm} $(g,x) \leadsto g \cdot x \cdot g^{-1}$
    \end{center}  
    
    \item Dado un grupo $G$ e un subgrupo $H < G$, a aplicación:
    
    \begin{center}
    $H \times G \longrightarrow G$ \\
    \vspace{2mm}
    \hspace{1mm} $(h,g) \leadsto h \cdot g$
    \end{center}  
    
    é unha acción do subgrupo $H$ sobre o conxunto $G$.
    
    \item Dado un grupo $G$, sexa $\mathscr{H}$ o conxunto de tódolos subgrupos de $G$. A conxugación de subgrupos:
    
    \begin{center}
    $G \times \mathscr{H} \longrightarrow \mathscr{H}$ \\
    \vspace{2mm}
    \hspace{8mm} $(g,H) \leadsto g \cdot H \cdot g^{-1}$
    \end{center}  
    
    é unha acción do grupo $G$ sobre o conxunto $\mathscr{H}$.
    
\end{enumerate}

\vspace{3mm}

\magbf{\subsection{Accións e relacións de equivalencia. A fórmula das clases}}

\vspace{5mm}

\noindent Toda acción dun grupo define unha relación de equivalencia no conxunto sobre o que actúa. Sexan un grupo $G$, un conxunto $X$ e unha acción de $G$ sobre $X$, $\varphi$, coma na \textbf{Definición 1.24}. Dados $x,y \in X$, defínese a seguinte relación binaria:\\ 

\begin{center}
  \fcolorbox{magenta}{white}{
$x \sim y \Longleftrightarrow \exists \hspace{1mm} g \in G \hspace{1Mm} | \hspace{1mm} gx = y$
}\\ 
\end{center}

\noindent Pódese comprobar facilmente que, efectivamente, esta relación é de equivalencia. Como tal, define unha partición de $X$, onde as clases de equivalencia son da forma:

$$[x] = \{ y \in X \hspace{1mm} | \hspace{1mm} \exists \hspace{1mm} g \in G : y = gx\} = \{ gx \hspace{1mm} | \hspace{1mm} g \in G\} = Gx$$\\

\noindent \textbf{Definición 1.25}. Sexan $G$ un grupo e $X$ un conxunto. Considérese $\varphi$ unha acción de $G$ sobre $X$. Dado un elemento $x \in X$, defínese a \magbf{órbita de $x$}, denotada por \magen{$Gx$}, como a súa clase de equivalencia segundo a relación definida por $\varphi$.\\

\noindent Para cada $x \in G$, pódese definir tamén: \\

\noindent \textbf{Definición 1.26}. Sexa un grupo $G$ que actúa sobre un conxunto $X$. Para cada $x \in X$, defínese o \magbf{subgrupo de isotropía de $x$} ou \magbf{estabilizador de $x$}, denotado por \magen{$G_{x}$}, como o conxunto de elementos da órbita de $x$ que deixan invariante a $x$.

\begin{center}
\fcolorbox{magenta}{white}{
$G_{x} = \{ g \in G \hspace{1mm} | \hspace{1mm} gx = x\}$  
}\\
\end{center}

\noindent \textbf{Exercicio}. Comprobar que $G_{x}$ é un subgrupo de $G$.\\

\noindent Se $G$ é un grupo finito tense o seguinte:\\

\begin{proposition}\label{prop1.10}
Sexa $G$ un grupo finito. Entón, para cada $x \in X$, o cardinal da súa órbita coincide co índice do seu subgrupo de isotropía.
\end{proposition}

\vspace{2mm}

\noindent \textbf{\textit{\underline{Demostración}}}

\vspace{2mm}

\noindent Recórdese que o índice dun subgrupo é o número de clases de equivalencia que determina sobre o grupo ó que pertence. Polo tanto, probar o enunciado da proposición equivale a demostrar que a órbita de $x$ e o conxunto $\displaystyle \frac{G}{G_{x}}$ son equipotentes, isto é, que posúen o mesmo cardinal. Para iso, darase unha aplicación bixectiva entre os dous conxuntos.\\

\noindent Defínase así a seguinte correspondencia entre os conxuntos considerados:\\

\begin{center}
    $f: Gx \longrightarrow \displaystyle \frac{G}{G_{x}}$ \\
        \vspace{2mm}
    $\hspace{16mm} gx \leadsto f(gx) := gG_{x}$
\end{center}

\noindent Hai que demostrar: \\

\begin{itemize}
    
    \item É $f$ unha aplicación?\\
    
    Dados $g, g' \in G \hspace{1mm} | \hspace{1mm} g(x) = g'(x)$, cúmprese que $gG_{x} = g'G_{x}$?\\
    
    Si, pois:
    
    $$gx = g'x \Longleftrightarrow g^{-1}gx = g^{-1}g'x \Longleftrightarrow x = g^{-1}g'x$$
    
    $$gGx = g'Gx \Longleftrightarrow g^{-1}g' \in G_{x} \underset{\underset{\textbf{Def } G_{x}}{\Uparrow}}{\Longleftrightarrow} g^{-1}g'x = x$$
    
    Así, $f$ é unha aplicación ben definida.\\
    
    \item É $f$ bixectiva?\\
    
    En efecto, $f$ é inxectiva:
    $$f(gx) = f(g'x) \Longleftrightarrow gG_{x} = g'G_{x} \Longleftrightarrow g^{-1}g' \in G_{x} \Longleftrightarrow g^{-1}g'x = x \Longleftrightarrow g'x = gx$$
    
    $f$ é sobrexectiva, pois $\forall \hspace{1mm} gG_{x} \in \displaystyle \frac{G}{G_{x}} \hspace{4mm} \exists \hspace{1mm} gx \in Gx \hspace{1mm} | \hspace{1mm} f(gx) = gG_{x}$
    
\end{itemize}

\noindent Achouse así unha bixección entre ámbolos dous conxuntos, o que equivale a que os seus cardinais sexan iguais. Logo, $\#Gx = (G : G_{x})$, tal e como se pretendía demostrar. $\square$

\vspace{3mm}

\begin{proposition}[\magbf{Fórmula das clases}] \label{prop1.11}
Sexa $G$ un grupo finito actuando sobre un conxunto finito $X$. Entón, verifícase a seguinte igualdade:
\begin{center}
    $\#X = \underset{x \in X}{\sum}\#Gx = \underset{x \in X}{\sum}(G:G_{x})$
\end{center}
\end{proposition}

\vspace{2mm}

\noindent \textbf{\textit{\underline{Demostración}}}

\vspace{2mm}

\noindent Pola \hyperref[prop1.10]{\magbf{Proposición 1.10}}, sendo $G$ finito, para cada $x \in X$, o cardinal da súa órbita coincide co índice do seu estabilizador.\\

\noindent Como $G$ actúa sobre $X$, pódese definir unha relación de equivalencia na que as clases son as órbitas dos elementos de $X$. Por ser esta unha relación de equivalencia, dá lugar a unha partición de $X$, cumpríndose:
$$\underset{x \in X}{\bigcup}Gx = X \text{    , con } Gx \cap Gx' = \varnothing \text{ se } Gx \neq Gx'$$

\noindent Aplicando entón o principio de inclusión-exclusión, obtense o resultado. $\square$

\magbf{\section{Teoría de Sylow}}

\vspace{2mm}

\noindent Un dos obxectivos fundamentais da teoría de grupos finitos é a clasificación, salvo isomorfismo, destes grupos. Unha ferramenta fundamental para tratar esa clasificación é a \magbf{Teoría de Sylow}, denominada así na honra do matemático noruegués Peter Ludwig Mejdell Sylow (1832-1918), un dos máis destacados contribuíntes á teoría de grupos, e que será tratada neste apartado. 

\vspace{3mm}

\magbf{\subsection{Os \textit{p}-grupos e os \textit{p}-subgrupos de Sylow}}

\vspace{5mm}

\noindent \textbf{Definición 1.27}. Sexa $G$ un grupo finito e $p$ un número primo. Dise que $G$ é \magbf{\textit{p}-grupo} se a súa orde é unha potencia de $p$.\\

\vspace{3mm}

\begin{lemma} \label{lem1.1}
Sexa $G$ un $p$-grupo actuando sobre un conxunto $X$. Considérese o seguinte subconxunto de $X$:
\begin{center}
    $X_{0} = \{ x \in X \hspace{1mm} | \hspace{1mm} Gx = \{x\} \hspace{1mm}\}$
\end{center}
\noindent Entón, cúmprese a seguinte igualdade:
\begin{center}
    $\#X = \#X_{0} + \dot{p}$
\end{center}
\end{lemma}

\vspace{2mm}

\noindent \textbf{\textit{\underline{Demostración}}}

\vspace{5mm}

\noindent Cómpre aclarar en primeiro lugar que $\dot{p}$ denota un múltiplo calquera de $p$.\\

\noindent Considérese $x \in X - X_{0}$, i.e. $x \in X \hspace{1mm} | \hspace{1mm} Gx \neq \{x\}$. Aplicando a \hyperref[prop1.10]{\magbf{Proposición 1.10}}, sábese:
$$\#Gx = (G : G_{x}) \underset{\underset{\hyperref[th1.2]{\magbf{Lagrange}}}{\Uparrow}}{=} \displaystyle \frac{|G|}{|G_{x}|} \underset{\underset{G \text{ \textit{p}-grupo}}{\Uparrow}}{=} \displaystyle \frac{p^{r}}{p^{s}} = p^{r-s} \hspace{5mm} r \geq s$$

\noindent Entón, segundo a \hyperref[prop1.11]{\magbf{Fórmula das clases}}, obtense:
$$\#X = \underset{x \in X}{\sum}\#Gx = \#X_{0} + \underset{x \notin X_{0}}{\sum}\#Gx = \#X_{0} + \dot{p} \hspace{5mm} \square$$

\vspace{3mm}

\noindent \textbf{Observación 1.11}. Polo lema anterior, tense que $\#X \equiv \#X_{0}$ (mód $p$).\\

\noindent \textbf{Definición 1.28}. Sexa $G$ un grupo finito e $H$ un subgrupo de $G$. Dirase que $H$ é \magbf{\textit{p}-subgrupo de Sylow de $G$} se a súa orde é a maior potencia de $p$ que divide á orde de $G$.\\

\noindent Por exemplo, supóñase un grupo finito de orde 693000. A factorización en números primos de 693000 é $2^{3}\cdot3^{2}\cdot5^{3}\cdot7\cdot11$. Logo:

\begin{itemize}
    
    \item Se $H$ é 2-subgrupo de Sylow, a súa orde é $2^{3} = 8$.\\
    
    \item Se $H$ é 3-subgrupo de Sylow, a súa orde é $3^{2} = 9$.\\
    
    \item Se $H$ é 5-subgrupo de Sylow, a súa orde é $5^{3} = 125$.\\
    
    \item Se $H$ é 7-subgrupo de Sylow, a súa orde é 7.\\
    
    \item Se $H$ é 11-subgrupo de Sylow, a súa orde é 11.
    
\end{itemize}

\magbf{\subsection{Os teoremas de Sylow}}

\vspace{5mm}

\noindent Os \magbf{Teoremas de Sylow} engloban tres teoremas enunciados e demostrados por Ludwig Sylow no 1872, nun artigo titulado \textit{Théorèmes sur les groupes de substitutions}, o cal foi publicado nos \textit{Mathematische Annalen}.\\

\noindent Estes resultados constitúen recíprocos parciais ao \hyperref[th1.2]{\magbf{Teorema de Lagrange}}, o cal garante que, para todo grupo finito, a súa orde é múltiplo da orde de calquera dos seus subgrupos. O recíproco cúmprese para os grupos cíclicos; é dicir, se $G$ é un grupo cíclico (e polo tanto, abeliano) de orde $n$ e $m \hspace{1mm} | \hspace{1mm} n$, entón $G$ contén exactamente un subgrupo de orde $m$. Pero isto non é certo, en xeral, para grupos non abelianos: por exemplo, $A_{4}$ non posúe subgrupos de orde 6. Os \magbf{Teoremas de Sylow} garanten, baixo certas condicións, a existencia e a unicidade (salvo conxugación) destes grupos, ademais de dar cantos existen.\\

\vspace{3mm}

\begin{theorem}[\magbf{Teoremas de Sylow}] \label{th1.12}
Sexan $G$ un grupo finito de orde $n$ e $p$ un número primo. Verifícanse as seguintes afirmacións: 
\begin{enumerate}
    \item \textbf{(Primeiro teorema de Sylow). }\textit{$G$ ten p-subgrupos de Sylow, e todo p-subgrupo está contido nun p-subgrupo de Sylow.}
    \item \textbf{(Segundo teorema de Sylow). }\textit{Tódolos p-subgrupos de Sylow de $G$ son conxugados. En particular, se $S$ e $T$ son p-subgrupos de Sylow, $\exists \hspace{1mm} g \in G \hspace{1mm} | \hspace{1mm} T = gSg^{-1}$.}
    \item \textbf{(Terceiro teorema de Sylow). }\textit{Denótese por $n_{p}$ o número de p-subgrupos de Sylow en $G$. Tense:
    \begin{enumerate}
        \item $n_{p} \hspace{1mm} | \hspace{1mm} (G:1) \equiv |G|$
        \item $n_{p} \hspace{1mm} | \hspace{1mm} (G:S)$, sendo S un p-subgrupo de Sylow
        \item $n_{p} \equiv 1$ (mód p) 
    \end{enumerate}}
\end{enumerate}
\end{theorem}

\vspace{2mm}

\noindent \textbf{\textit{\underline{Demostración}}}

\vspace{2mm}

\noindent A proba que aquí se dá destes resultados é obra do matemático alemán Helmut Wielandt (1910-2001).\\

\noindent \magbf{(1)} Comezarase probando que $G$ posúe \textit{p}-subgrupos de Sylow.\\

\noindent Nótese que se pode escribir $n = p^{\alpha} \cdot n'$, con m.c.d.($p,n'$) = 1. Supoñerase, sen perda de xeneralidade, que $\alpha > 0$, pois en caso contrario o resultado é trivial.\\

\noindent Considérese o conxunto $M = \{X_{1}, \dots, X_{k}\} \subset \mathcal{P}(G)$, onde para cada $i \in \{1, \dots, k\}$ se ten que $\#X_{i} = p^{\alpha}$. O cardinal de $M$ é o número de subconxuntos de $G$ que se poden formar con $p^{\alpha}$ elementos. Así:
$$\#M = \binom{n}{p^{\alpha}} = \displaystyle \frac{n!}{p^{\alpha}!(n-p^{\alpha})!} = \displaystyle \frac{n \cdot (n-1) \cdot \dots \cdot (n - (p^{\alpha} -1))}{p^{\alpha}!}$$

\noindent Reordenando factores, pódese escribir:
$$\#M = \displaystyle \frac{n \cdot (n-1) \cdot (n-2) \cdot \dots \cdot (n-(p^{\alpha}-1))}{p^{\alpha} \cdot 1 \cdot 2 \cdot \dots \cdot (p^{\alpha} - 1)}$$

\noindent Para cada $m \in \{0, 1, \dots, p^{\alpha} -1\}$, cúmprese que o factor $\displaystyle \frac{n-m}{m}$ é primo con $p$. Escríbase:
\begin{center}
$n - m = p^{r} \cdot x$, con mcd($x,p$) = 1\par
\vspace{3mm}
$m = p^{s} \cdot y$, con mcd($y,p$) = 1  
\end{center} 

\noindent e véxase que $r = s$. En caso contrario, un deles tería que ser menor có outro. Supóñase, por exemplo, que $r < s$. Entón:
$$p^{\alpha} \cdot n' = n = m + p^{r} \cdot x = p^{s} \cdot y + p^{r} \cdot x = p^{r} \cdot (p^{s-r}y + x) \implies p^{\alpha-r} \cdot n' = p^{s-r}y + x \Longleftrightarrow$$ 
$$ \Longleftrightarrow x = p^{\alpha - r} \cdot n' - p^{s-r}y = p^{s-r} \cdot (y + p^{\alpha - s}\cdot n') \implies p \hspace{1mm} | \hspace{1mm} x$$

\noindent Non obstante, tíñase que mcd($x,p$) = 1, chegando así a un absurdo. Supoñendo $s > r$ chegaríase igualmente a unha contradición. Así, necesariamente, $r = s$, cumpríndose que $\displaystyle \frac{n-m}{m}$ é primo con $p$.\\

\noindent Como isto se verifica $\forall \hspace{1mm} m \in \{0, 1, \dots, p^{\alpha} -1\}$, verifícase entón que mcd($\#M, p$) = 1.\\

\noindent Defínase a seguinte acción de $G$ sobre $M$: 

    \begin{center}
    $G \times M \longrightarrow M$ \\
    \vspace{2mm}
    \hspace{1mm} $(g,X_{i}) \leadsto gX_{i}$
    \end{center}  
    
\noindent Entón, en base á \hyperref[prop1.11]{\magbf{Fórmula das clases}} e á \hyperref[prop1.10]{\magbf{Proposición 1.10}}, obtense:

    \[ 
    \left. \begin{array}{r} 
    \#M = \underset{i = 1}{\overset{k}{\sum}} GX_{i} \\[1ex]
    mcd(\#M, p) = 1
    \end{array} \right\} 
    \implies \exists \hspace{1mm} X \in M \hspace{1mm} | \hspace{1mm} mcd(\#GX, p) = 1 \Longleftrightarrow mcd((G : G_{X}), p) = 1\\
    \]
    
\vspace{2mm}
    
\noindent Probarase que $S = G_{X}$ é un p-subgrupo de Sylow en $G$, con orde $p^{\alpha}$. En efecto, se $g \in G_{X}$, entón $gX = X$. Logo:
$$\forall \hspace{1mm} g \in G \hspace{2mm} \forall \hspace{1mm} u \in X \text{, } gu \in X \implies g \in Xu^{-1} \implies S \subset Xu^{-1}$$

\noindent Así, sendo $S = G_{X}$ subgrupo, tense:
$$|S| \leq \#Xu^{-1} \overset{\magbf{(?)}}{=} \#X = p^{\alpha}$$

\noindent \magbf{(?)} Esta igualdade é certa porque cada $g \in G$ establece unha bixección:

    \begin{center}
    $X_{i} \longrightarrow gX_{i}$ \\
    \vspace{2mm}
    \hspace{1mm} $x_{i} \leadsto gx_{i}$
    \end{center} 
    
\noindent para cada $i \in \{1, 2, \dots, k\}$ (pódese comprobar facilmente que é unha aplicación bixectiva).\\

\noindent Queda demostrar que $|S| = p^{\alpha}$. Dunha banda, tense que $(G:S) = \displaystyle \frac{|G|}{|S|} = \displaystyle \frac{p^{\alpha} \cdot n'}{|S|}$. Logo, $|S| = \displaystyle \frac{p^{\alpha} \cdot n'}{(G:S)}$.\\

\noindent Como mcd($(G:S),p$) = mcd($(G:G_{X}), p) = 1$, necesariamente, $(G:S)$ divide a $n'$. Facendo $\displaystyle \frac{n'}{(G:S)} = m$, véxase que $m = 1$. En efecto:

    \[ 
    \left. \begin{array}{r} 
    n', (G:S) \in \mathbb{Z}^{+} \implies m \in \mathbb{Z}^{+} \\[1ex]
    |S| \leq p^{\alpha} \\[1ex]
    |S| = p^{\alpha} \cdot m
    \end{array} \right\} 
    \implies m = 1 \implies |S| = p^{\alpha}
    \]

\noindent Así, demostrouse que $S = G_{X}$ é un \textit{p}-subgrupo de Sylow de $G$.\\

\noindent A continuación, verase que todo \textit{p}-subgrupo de $G$ está contido nun \textit{p}-subgrupo de Sylow. Considérese así $H$ un \textit{p}-subgrupo de $G$. Defínase a seguinte acción:

    \begin{center}
    $H \times \displaystyle \frac{G}{S} \longrightarrow \displaystyle \frac{G}{S}$ \\
    \vspace{2mm}
    \hspace{1mm} $(h,gS) \leadsto hgS$
    \end{center}  

\noindent Está ben definida como aplicación?
$$gS = g'S \overset{?}{\implies} hgS = hg'S$$

$$hgS = hg'S \Longleftrightarrow (hg)^{-1}hg' = g^{-1}h^{-1}hg' = g^{-1}g' \in S \Longleftrightarrow gS = g'S$$\\

\noindent Deste xeito queda probado que se ten unha aplicación ben definida.\\

\noindent Esta é unha actuación do \textit{p}-grupo $H$ sobre o  conxunto $\displaystyle \frac{G}{S}$. Recórdese que mcd($\# \displaystyle \frac{G}{S}, 1$) = 1. Así, aplicando o \hyperref[lem1.1]{\magbf{Lema 1.1}}, tense:

$$\# G/S = \# (G/S)_{0} + \dot{p} \implies \exists \hspace{1mm} gS \in \displaystyle \frac{G}{S} \hspace{1mm} | \hspace{1mm} HgS = gS \implies Hg \subset gS \implies H \subset gSg^{-1}$$

\noindent Tendo en conta que $S$ e $gSg^{-1}$ son conxuntos bixectivos, cúmprese que $|S| = |gSg^{-1}| = p^{\alpha}$. Así, $gSg^{-1}$ é tamén \textit{p}-subgrupo de Sylow de $G$, quedando demostrada así a afirmación.\\

\vspace{5mm}

\noindent \magbf{(2)} Sexa $T$ un \textit{p}-subgrupo de Sylow de $G$. Entón, en particular, $T$ é \textit{p}-subgrupo de $G$. Logo, tal e como se viu na proba de \magbf{(1)}, $\exists \hspace{1mm} g \in G \hspace{1mm} | \hspace{1mm} T \subset gSg^{-1}$, onde $S$ é \textit{p}-subgrupo de Sylow.\\

\noindent Agora ben, como $T$ é \textit{p}-subgrupo de Sylow, $|T| = p^{\alpha} = |gSg^{-1}|$. Así, tense que $T = gSg^{-1}$.\\

\vspace{5mm}

\noindent \magbf{(3)} Denótese por $n_{p}$ o número de \textit{p}-subgrupos de Sylow en $G$.\\

\noindent \magbf{(a)} e \magbf{(b)}\\

\noindent Considérese a actuación conxugación de subgrupos, vista nos exemplos de actuacións de grupos sobre conxuntos:

    \begin{center}
    $G \times \mathscr{H} \longrightarrow \mathscr{H}$ \\
    \vspace{2mm}
    \hspace{8mm} $(g,H) \leadsto g \cdot H \cdot g^{-1}$
    \end{center}  
    
\noindent Sexa $S$ o \textit{p}-subgrupo de Sylow mencionado no teorema. Entón, segundo \magbf{(2)}, pódese afirmar que $n_{p}$ é o número de subgrupos conxugados de $S$, os cales conforman a órbita de $S$, $GS$. Así, tense:
$$n_{p} = \#GS \underset{\underset{\hyperref[prop1.10]{\magbf{Prop. 1.10}}}{\Uparrow}}{=} (G:G_{S}) \underset{\underset{\hyperref[th1.2]{\magbf{Th Lagrange}}}{\Uparrow}}{=} \displaystyle \frac{|G|}{|G_{S}|}$$

\noindent Escribindo $|G| = |S|(G:S)$ e $|G_{S}| = |S|(G_{S}:S)$ (o cal ten sentido pois $S$ é subgrupo de $G_{S}$), obtense:
$$n_{p} = \#GS = (G:G_{S}) = \displaystyle \frac{|S|(G:S)}{|S|(G_{S}:S)} = \displaystyle \frac{(G:S)}{(G_{S}:S)}$$

\noindent Así, $n_{p} \hspace{1mm} | \hspace{1mm} (G:S)$. E como $(G:S)$ é divisor de $|G| \equiv (G:1)$, tense que $n_{p} \hspace{1mm} | \hspace{1mm} (G:1)$.\\

\noindent \magbf{(c)}\\

\noindent Considérese o conxunto $\sum = \{S_{1}, \dots, S_{t} \hspace{1mm} | \hspace{1mm} S_{i} \text{ é subgrupo de Sylow de } G\}$.\\

\noindent Defínase a seguinte aplicación:

    \begin{center}
    $S_{1} \times \Sigma \longrightarrow \Sigma$ \\
    \vspace{2mm}
    \hspace{12mm} $(g,S_{i}) \leadsto g \cdot S_{i} \cdot g^{-1}$
    \end{center}  
    
\noindent a cal é unha actuación do \textit{p}-grupo $S_{1}$ sobre o conxunto $\Sigma$ (poderíase ter escollido calquera outro \textit{p}-subgrupo en $\Sigma$ para definir a actuación). Entón, aplicando o \hyperref[lem1.1]{\magbf{Lema 1.1}}, cúmprese:
$$n_{p} = \# \Sigma = \# \Sigma_{0} + \dot{p}$$

\noindent Verase a continuación que en $\Sigma_{0}$ hai un único elemento, obtendo así o resultado. Ademais, probarase que ese elemento é $S_{1}$. Tense, para todo $s_{1} \in S_{1}$, tense:\\
$$s_{1}S_{i}s_{1}^{-1} = S_{i} \implies S_{1}S_{i} = S_{i}S_{1} \implies S_{1}S_{i} \text{ é grupo}$$

\noindent Como $S_{1}, S_{i} \subset S_{1}S_{i}$, tense que $S_{1} < S_{1}S_{i}$ e $S_{i} < S_{1}S_{i}$ e, ademais, sendo $S_{1}, S_{i}$ \textit{p}-subgrupos de Sylow, son conxugados en $S_{1}S_{i}$. Polo tanto:
$$\exists \hspace{1mm} s_{1}s_{i} \in S_{1}S_{i} \hspace{1mm} | \hspace{1mm} (s_{1}s_{i})S_{i}(s_{1}s_{i})^{-1} = S_{1} \implies s_{1}s_{i}S_{i}s_{i}^{-1}s_{1}^{-1} = S_{1} \overset{\overset{s_{i}S_{i}s_{i}^{-1} = S_{i}}{\downarrow}}{\implies} s_{1}S_{i}s_{1}^{-1} = S_{1} \implies$$
$$\implies S_{i} = s_{1}^{-1}S_{1}s_{1} = S_{1}$$\\

\noindent Entón, $\# \Sigma_{0} = 1$, obtendo que $n_{p} - 1 = \dot{p}$, o cal, por definición, equivale a que $n_{p} \equiv 1$ (mód \textit{p}). $\square$\\

\vspace{3mm}

\begin{corollary}\label{cor1.5}
Sexa $G$ un grupo finito e $p$ un número primo. Considérese $S$ un p-subgrupo de Sylow de $G$. Entón, verifícase:
\begin{center}
$S \text{ é o único p-subgrupo de Sylow de } G \Longleftrightarrow S \text{ é normal en } G$
\end{center}
\end{corollary}

\vspace{2mm}

\noindent \textbf{\textit{\underline{Demostración}}}

$$S \triangleleft G \Longleftrightarrow gSg^{-1} = S \hspace{3mm} \forall \hspace{1mm} g \in G \overset{\textbf{(*)}}{\Longleftrightarrow} n_{p} = 1$$

\noindent \textbf{(*)} En base ao \hyperref[th1.12]{\magbf{2.º Teorema de Sylow}}, tódolos \textit{p}-subgrupos de Sylow de $G$ son conxugados. Así, $S$ é o único \textit{p}-subgrupo de Sylow en $G$. $\square$\\

\vspace{3mm}

\begin{theorem} \label{th1.13} 
Sexa $G$ un p-grupo non trivial. Entón: 
\begin{enumerate}
    \item \textit{$Z(G) \neq \{1\}$ e, ademais, $|Z(G)| \geq p$.}
    \item \textit{Para todo divisor da orde de $G$, existe un subgrupo desa orde}.
\end{enumerate}
\end{theorem}

\vspace{2mm}

\noindent \textbf{\textit{\underline{Demostración}}}

\vspace{2mm}

\noindent \magbf{(1)} Considérese a actuación conxugación de elementos de $G$:

    \begin{center}
    $G \times G \longrightarrow G$ \\
    \vspace{2mm}
    \hspace{8mm} $(g,x) \leadsto g \cdot x \cdot g^{-1}$
    \end{center}  
    
\noindent Sendo $G$ un \textit{p}-grupo, segundo o \hyperref[lem1.1]{\magbf{Lema 1.1}}, $\#G = \#G_{0} + \dot{p}$. Agora ben:
$$G_{0} = \{x \in G \hspace{1mm} | \hspace{1mm} gxg^{-1} = x \hspace{3mm} \forall \hspace{1mm} g \in G\} = \{x \in G \hspace{1mm} | \hspace{1mm} gx = xg \hspace{3mm} \forall \hspace{1mm} g \in G\} = Z(G)$$

    \[ 
    \left. \begin{array}{r} 
    |G| \equiv 0 \text{ (mód \textit{p})} \\[1ex]
    |Z(G)| \equiv |G| \text{ (mód \textit{p})}
    \end{array} \right\} 
    \implies |Z(G)| \equiv 0 \text{ (mód \textit{p})}\\
    \]
    
\vspace{5mm}
    
    \[ 
    \left. \begin{array}{r} 
    |Z(G)| \equiv 0 \text{ (mód \textit{p})} \\[1ex]
    1 \in Z(G)
    \end{array} \right\} 
    \implies |Z(G)| \geq p\\
    \]
    
\vspace{5mm}

\noindent \magbf{(2)} Razoarase de xeito indutivo sobre a orde do grupo.\\

\noindent Para $|G| = p^{0} = 1$, o resultado é trivial.\\

\noindent Sexa $|G| = p^{n}$,  $n \in \mathbb{N}$. Supóñase certa a afirmación para \textit{p}-grupos de orde $p^{r}$, para cada $r \in \{1, \dots, n-1\}$. \\

\noindent Sendo $G$ \textit{p}-grupo, por \magbf{(1)}, $Z(G) \neq \{1\}$ e $|Z(G)| = p^{s}$.\\

\noindent Pódese garantir que $\exists \hspace{1mm} c \in Z(G)$ tal que $|c| = p$. En efecto, como $|Z(G)| = p^{s}$
, $\exists \hspace{1mm} x \neq 1$ en $Z(G)$, con $|x| = p^{t}$. Entón:
$$1 = x^{p^{t}} = (x^{p^{t-1}})^{p} \implies |x^{p^{t-1}}| = p$$

\noindent Fágase así $c = x^{p^{t-1}}$ e considérese o subgrupo $C = \langle c \rangle$. Cúmprese entón que $|C| = p$. Ademais, tense que $C \triangleleft G$, pois $\forall \hspace{1mm} g \in G$, $\forall \hspace{1mm} c^{i} \in C$, tense que $gc^{i}g^{-1} \underset{\underset{c \in Z(G)}{\Uparrow}}{=} c^{i} \in C$.\\

\noindent Con isto, pódese asegurar que o conxunto $\displaystyle \frac{G}{C}$ está dotado de estrutura de grupo, o cal cumpre:
$$|G/C| = \displaystyle \frac{p^{n}}{p} = p^{n-1}$$

\noindent Por \textbf{hipótese de indución}, $\forall \hspace{1mm} p^{m} \hspace{1mm} | \hspace{1mm} p^{n-1}$, $\exists \hspace{1mm} H/C < G/C$ tal que $|H/C| = p^{m}$. Sendo $|C| = p$, tense que $|H| = p^{m+1}$, con $m+1 \leq n$, obtendo así o resultado. $\square$\\

\vspace{3mm}

\noindent Como consecuencia deste teorema obtense:\\

\begin{corollary}[\magbf{Teorema de Cauchy}] \label{cor1.6}
Sexan $G$ un grupo finito de orde $n$ e $p$ un número primo que divida a $n$. Entón:
\begin{enumerate}
    \item \textit{Existe un elemento $x \in G$ de orde $p$}.
    \item \textit{Para toda potencia de $p$ que divide a $n$, existe un subgrupo de $G$ desa orde}.
\end{enumerate}
\end{corollary}

\vspace{2mm}

\noindent \textbf{\textit{\underline{Demostración}}}

\vspace{2mm}

\noindent \magbf{(1)} Empregando a notación do \hyperref[th1.12]{\magbf{1.º Teorema de Sylow}}, escríbase $n = p^{\alpha} \cdot n'$, con mcd($p,n'$) = 1. En base a tal teorema, $\exists \hspace{1mm} S$ \textit{p}-subgrupo de Sylow de $G$, con $|S| = p^{\alpha}$.

    \[ 
    \left. \begin{array}{r} 
    S \text{ é p-grupo de orde } p^{\alpha} \\[1ex]
    p \hspace{1mm} | \hspace{1mm} |S| 
    \end{array} \right\} 
    \underset{\underset{\hyperref[th1.13]{\magbf{Th. 1.13}}}{\Uparrow}}{\implies} \exists \hspace{1mm} H < S : |H| = p\\
    \]

\noindent Sendo $p$ primo, tense que $H$ é cíclico, polo que $\exists \hspace{1mm} h \in H \hspace{1mm} | \hspace{1mm} \langle h \rangle = H$. Así, $|h| = p$.\\

\noindent \magbf{(2)} $$p^r \hspace{1mm} | \hspace{1mm} n \implies p^{r} \hspace{1mm} | \hspace{1mm} p^{\alpha} = |S| \underset{\underset{\magbf{Th. 1.13}}{\Uparrow}}{\implies} \exists \hspace{1mm} H < G : |H| = p^{r} \hspace{5mm} \square$$

\vspace{5mm}

\noindent \textbf{Exercicio.} Calcular tódolos subgrupos de Sylow dun grupo $G$ de orde 28. %--Suprimir?


\chapter[Unidade 2. Aneis]{\textbf{Aneis}}

\thispagestyle{noheader}

\noindent Tal e como se definirá a continuación, un anel é unha estrutura abstracta formada por un conxunto no que se definen dúas operacións internas. Unha delas, a cal se adoita denominar suma, é conmutativa, mentres que a outra operación, habitualmente chamada produto, non ten por que posuír tal propiedade. Esta última distinción produce dúas teorías diferentes: a teoría de aneis conmutativos e a teoría de aneis non conmutativos, as cales foron estudadas de xeito independente ata a década de 1930. Nestes apuntamentos estúdanse maioritariamente os aneis conmutativos, aínda que tamén se introducirán algúns conceptos sobre aneis non conmutativos.\\

\noindent O termo \textit{anel} comeza a empregarse a partir do termo alemán \textit{``Zahlring''} (anel de números), acuñado por David Hilbert (1862 - 1943) no 1892 e que apareceu por primeira vez no seu informe \textit{Der Zahlbericht}, publicado no 1897.\\

\noindent Os aneis non conmutativos non foron obxecto de estudo sistemático ata mediados do século XIX, a partir dos intentos do matemático irlandés William Hamilton (1805 - 1865) por estender os números complexos a máis de dúas dimensións. Pola contra, os aneis conmutativos apareceron moito antes, a partir da teoría de números, a xeometría alxébrica e a teoría de invariantes. Os polinomios, os números enteiros e a divisibilidade foron, durante séculos, as bases do estudo dos aneis. Foi Richard Dedekind (1831 - 1916) quen deu pé ao desenvolvemento da teoría de aneis de polinomios. Ademais, a el débeselle o termo \textit{corpo} (en alemán, \textit{``Körper"}). Non foi ata 1921 cando Emmy Noether deu un paso moi importante na axiomatización dos aneis, conseguindo unificar o estudo dos aneis de polinomios e os aneis de números baixo unha única teoría de aneis conmutativos.

\magbf{\section{Aneis}}

\magbf{\subsection{Definición e propiedades}}

\vspace{5mm}

\noindent \textbf{Definición 2.1}. Un \magbf{anel} é unha terna $(A, +, \cdot)$, onde $A$ é un conxunto e \hspace{0.5mm} $+$ \hspace{1mm}, $\cdot$ \hspace{0.5mm} son dúas operacións internas definidas en A:
\begin{align*}
    + : A \times A \longrightarrow A & & \cdot : A \times A \longrightarrow A\\
    (a,b) \leadsto a + b & & (a,b) \leadsto a \cdot b
\end{align*}

\noindent as cales conviremos en chamar \textbf{suma} e \textbf{produto}, respectivamente, e que cumpren:\\

\begin{itemize}

    \item $(A, +)$ é un grupo abeliano. 
    
    \vspace{1mm}
    
    \item $(A, \cdot)$ é un \textbf{monoide}, i.e. a operación $\cdot$ é asociativa, e posúe elemento neutro, denotado por \textbf{1}.
    
    \vspace{1mm}
    
    \item O produto é \magbf{distributivo} respecto da suma, isto é, $\forall \hspace{1mm} a,b,c \in A$ :
    
    \begin{itemize}
    
        \item $a \cdot (b + c) = a \cdot b + a \cdot c$
        
        \item $(a + b) \cdot c = a \cdot c + b \cdot c$
        
    \end{itemize}
    
\end{itemize}

\noindent Se, ademais, o produto posúe a propiedade conmutativa, dirase que $(A,+,\cdot)$ é un \magbf{anel conmutativo}.\\

\noindent \textbf{Observación 2.1}. Existe outra definición de anel, recollida por certos autores, a cal non esixe a existencia de elemento neutro para o produto. Esta condición afectará á validez de resultados posteriores que se verán nesta unidade.\\

\noindent \textbf{Observación 2.2}. Sendo $(A,+)$ un grupo abeliano, empregarase a notación aditiva vista na unidade anterior: o elemento neutro denotarase por \textbf{0}; e o elemento simétrico de $a \in A$ desígnase como $-a$, denominado \textbf{elemento oposto}.\\

\noindent Ilústrese esta definición con algúns exemplos:

\begin{itemize}
    \item $(\mathbb{Z},+,\cdot)$, $(\mathbb{Q},+,\cdot)$ $(\mathbb{R},+,\cdot)$ e $(\mathbb{C},+,\cdot)$ son aneis conmutativos.
    
    \item $(\mathbb{Z}_{n},+,\cdot)$ é un anel conmutativo $\forall \hspace{1mm} n \in \mathbb{Z}$
    
    \item $(M_{n}(\mathbb{R}),+,\cdot)$ é un anel, non conmutativo se $n >$  1.
\end{itemize}

\vspace{3mm}

\noindent A partir da definición de anel extráese unha serie de \textbf{propiedades} que esta estrutura debe cumprir:

\begin{enumerate}

    \item $a \cdot 0 = 0 = 0 \cdot a \hspace{4mm} \forall \hspace{1mm} a \in A$
    
    \item $(-a) \cdot b = -(a \cdot b) = a \cdot (-b) \hspace{4mm} \forall \hspace{1mm} a,b \in A$
    
    \item $(-a) \cdot (-b) = ab \hspace{4mm} \forall \hspace{1mm} a,b \in A$

\end{enumerate}

\vspace{2mm}

\noindent \textbf{\textit{\underline{Demostración}}}

\vspace{2mm}

\noindent \magbf{(1)} Pódese aproveitar propiedade distributiva e a idempotencia do 0 respecto da operación suma (recórdese que nun grupo, o único elemento idempotente respecto da operación considerada é o elemento neutro):
$$a \cdot 0 + a \cdot 0 = a \cdot (0+0) = a \cdot 0 \implies a \cdot 0 = 0$$

\noindent \magbf{(2)} Séguese da propiedade distributiva do produto respecto da suma:
$$a \cdot b + (-a) \cdot b = (a + (-a)) \cdot b = 0 \cdot b = 0$$

\noindent \magbf{(3)} En base á propiedade anterior, tense:
$$(-a) \cdot (-b) = -((-a) \cdot b) = a \cdot b$$

\magbf{\subsection{Elementos dun anel}} \label{ElAnillo}

\vspace{5mm}

\noindent Neste apartado defínense certos elementos de especial interese no estudo dos aneis.\\

\noindent \textbf{Definición 2.2}. Sexa $(A,+,\cdot) \equiv A$ un anel e considérese un elemento $a \in A$. Dirase que $a$ é \magbf{divisor de cero} se existe un elemento $b \in A$ non nulo tal que $a \cdot b = 0$ ou $b \cdot a = 0$.\\

\noindent \textbf{Definición 2.3}. Sexa $A$ un anel. Dirase que un elemento $a \in A$ é \magbf{divisor de cero propio} se $a$ é un divisor de cero \textbf{non nulo}.\\

\noindent Volvendo aos exemplos de aneis vistos, tense:\\

\begin{itemize}

    \item $(\mathbb{Z},+,\cdot)$, $(\mathbb{Q},+,\cdot)$, $(\mathbb{R},+,\cdot)$ e $(\mathbb{C},+,\cdot)$ non conteñen divisores de cero propios.
    
    \item $(\mathbb{Z}_{n},+,\cdot)$ contén divisores de cero propios $\Longleftrightarrow$ $n$ non é un número primo. Por exemplo, en $Z_{6}$, tense que $[3] \cdot [4] = [3 \cdot 4] = [12] = [0]$.
    
    \item $(M_{n}(\mathbb{R}),+,\cdot)$ contén divisores de cero propios. En efecto, existen pares de matrices non nulas cuxo produto resulta ser unha matriz nula. Por exemplo:
    
    $$\begin{pmatrix}
    1 & 1\\
    2 & 2
    \end{pmatrix}
    \cdot
    \begin{pmatrix}
    1 & 1 \\
    -1 & -1
    \end{pmatrix}
    = \begin{pmatrix}
    0 & 0 \\
    0 & 0
    \end{pmatrix}$$
    
\end{itemize}

\noindent \textbf{Definición 2.4}. Sexa $A$ un anel. Dado un elemento $a \in A$, este recibirá o nome de \magbf{unidade} se posúe elemento simétrico respecto do produto, isto é:
$$a \text{ é \magbf{unidade}} :\Longleftrightarrow \exists \hspace{1mm} a^{-1} \in A \hspace{1mm} | \hspace{1mm} a \cdot a^{-1} = 1 = a^{-1} \cdot a$$

\noindent O conxunto de unidades dun anel $A$ adoitará ser denotado en diante por \magen{$\mathcal{U}(A)$}.\\

\noindent A continuación introdúcense algúns exemplos:

\begin{enumerate}
    \item $\mathcal{U}(\mathbb{Z}) = \{-1,1\}$
    \item $\mathcal{U}(\mathbb{Q}) = \mathbb{Q} - \{0\}$
    \item $\mathcal{U}(\mathbb{R}) = \mathbb{R} - \{0\}$
    \item $\mathcal{U}(\mathbb{Z}_{5}) = \{[1],[2],[3],[4]\}$
    \item $\mathcal{U}(\mathbb{Z}_{6}) = \{[1],[5]\}$
    \item Máis en xeral, as unidades dun anel $\mathbb{Z}_{n}$ son as clases dos enteiros coprimos con $n$.
    \item $\mathcal{U}(M_{n}(\mathbb{R})) = GL_{n}(\mathbb{R})$. Este conxunto engloba tódalas matrices cadradas de orde $n$ inversibles, e denomínase \magbf{grupo linear xeral de orde $n$}.\\
\end{enumerate}

\noindent \textbf{Exercicio}. Demostrar que o conxunto $\mathcal{U}(A)$, dotado co produto do anel $A$, é un grupo.\\

\noindent \textbf{Observación 2.3}. Se un elemento $a \in A$ é unha unidade, entón non pode ser un divisor de cero. En efecto:

\[ 
\left. \begin{array}{r} 
a \cdot b = 0\\[1ex]
\exists \hspace{1mm} a^{-1}
\end{array} \right\} 
\implies a^{-1} \cdot a \cdot b = a^{-1} \cdot (a \cdot b) = a^{-1} \cdot 0 = 0 = (a^{-1} \cdot a) \cdot b =  1 \cdot b = b \implies b = 0
\]

\magbf{\subsection{Subaneis}}

\vspace{5mm}

\noindent Viuse na unidade anterior que dentro dun grupo existen certos subconxuntos que preservan a condición de grupo, os cales reciben o nome de subgrupos. Pois ben, para os aneis tamén existen subconxuntos que conservan a estrutura de anel: os chamados \textbf{subaneis}. Con todo, soamente se introducirán certas nocións básicas sobre estes, e non serán tratados coa mesma relevancia cós subgrupos na unidade anterior.\\

\noindent \textbf{Definición 2.5}. Sexa $(A,+,\cdot)$ un anel. Un \magbf{subanel de $A$} é un subconxunto $B \subset A$ tal que a terna $(B,+,\cdot)$ posúe estrutura de anel, isto é:\\

\begin{itemize}
    \item $(B,+)$ é grupo abeliano $\Longleftrightarrow (B,+) < (A,+)$
    \item $\forall \hspace{1mm} b_{1},b_{2} \in B, b_{1} \cdot b_{2} \in B$
    \item $1 \in B$\\
\end{itemize}

\noindent Moitos resultados que se tiñan para subgrupos poden ser trasladados á teoría de aneis. Por exemplo, viuse  que a intersección arbitraria de subgrupos preservaba a condición de subgrupo. O resultado é completamente análogo para o caso dos subaneis, como recolle a seguinte proposición:

\pagebreak

\begin{proposition} \label{prop2.1}
Sexa $A$ un anel arbitrario. Considérese unha familia de subaneis de $A$, $\{B_{i}\}_{i \in I}$. Cúmprese que a súa intersección, $\underset{i \in I}{\bigcap}B_{i}$, é tamén un subanel de $A$.
\end{proposition}

\vspace{2mm}

\noindent \textbf{\textit{\underline{Demostración}}}

\vspace{2mm}

\noindent En primeiro lugar, hai que garantir que a intersección destes conxuntos é non baleira. En efecto, todo subanel posúe, cando menos, dous elementos: os respectivos elementos neutros da suma e do produto. Así, cúmprese que $\underset{i \in I}{\bigcap}B_{i} \neq \varnothing$.\\

\noindent Hai que comprobar entón que tal conxunto verifica a condición de subanel. Tense:

\begin{itemize}
    
    \item $(\underset{i \in I}{\bigcap}B_{i}, +)$ é un subgrupo de $(A, +)$
    
    Certamente, como $B_{i}$ é subanel de $A \hspace{3mm} \forall \hspace{1mm} i \in I$, cúmprese que $(B_{i},+) < (A,+) \hspace{4mm} \forall \hspace{1mm} i \in I$. Sendo a intersección de subgrupos un subgrupo, obtense o resultado.
    
    \item $\forall \hspace{1mm} b_{1},b_{2} \in \underset{i \in I}{\bigcap}B_{i}$, $b_{1} \cdot b_{2} \in \underset{i \in I}{\bigcap}B_{i}$
    $$b_{1},b_{2} \in \underset{i \in I}{\bigcap}B_{i} \implies b_{1}, b_{2} \in B_{i} \hspace{4mm} \forall \hspace{1mm} i \in I \underset{\underset{B_{i} \text{ subanel}}{\Uparrow}}{\implies} b_{1} \cdot b_{2} \in B_{i} \hspace{4mm} \forall \hspace{1mm} i \in I \implies b_{1} \cdot b_{2} \in \underset{i \in I}{\bigcap}B_{i}$$
    
    \item $1 \in \underset{i \in I}{\bigcap}B_{i}$
    
    Trivial, pois como $B_{i}$ é subanel, $1 \in B_{i} \hspace{3mm} \forall \hspace{1mm} i \in I$ \hspace{3mm} $\square$
    
\end{itemize}

\vspace{3mm}

\noindent \textbf{Definición 2.6}. Sexa $A$ un anel. Defínese o \magbf{centro de $A$}, denotado por \magen{$Z(A)$}, como o seguinte subconxunto de $A$:
\begin{center}
\fcolorbox{magenta}{white}{
$Z(A) = \{a \in A \hspace{1mm} | \hspace{1mm} a \cdot b = b \cdot a \hspace{3mm} \forall \hspace{1mm} b \in A\}$ }
\end{center}

\vspace{3mm}

\noindent \textbf{Exercicio}. Demostrar que o centro dun anel é un subanel.\\

\noindent \textbf{Definición 2.7}. Un \magbf{enteiro gaussiano} é un número complexo cuxas partes real e imaxinaria son números enteiros.\\

\noindent O conxunto de enteiros gaussianos denótase por \magen{$\mathbb{Z}[i]$}:
\begin{center}
\fcolorbox{magenta}{white}{
$\mathbb{Z}[i] = \{a + bi \in \mathbb{C} \hspace{1mm} | \hspace{1mm} a,b \in \mathbb{Z} \}$ }
\end{center}

\vspace{3mm}

\noindent \textbf{Exercicio}. Probar que o conxunto dos enteiros gaussianos é un subanel de $(\mathbb{C},+,\cdot)$.\\

\noindent \textbf{Exercicio}. Ademais do anel dos enteiros gaussianos, en $\mathbb{C}$ pódense definir un tipo semellante de subaneis. Sexa $n \in \mathbb{Z}$ e $\sqrt{n} \in \mathbb{C}$ unha raíz cadrada de $n$. Considérese o seguinte subconxunto de $\mathbb{C}$:
\begin{center}
\fcolorbox{magenta}{white}{
$\mathbb{Z}[\sqrt{n}] = \{a + b\sqrt{n} \in \mathbb{C} \hspace{1mm} | \hspace{1mm} a,b \in \mathbb{Z} \}$ }
\end{center}

\noindent que algúns autores denominan \textbf{$\mathbb{Z}$ ampliado por $\sqrt{n}$.} Demostrar que este é un subanel de $(\mathbb{C},+,\cdot) \hspace{3mm} \forall \hspace{1mm} n \in \mathbb{Z}$.

\magbf{\subsection{Dominios e corpos}} \label{DomCorp}

\vspace{5mm}

\noindent Os conxuntos con estrutura de anel pódense distinguir ou clasificar dalgún xeito segundo as propiedades dos seus elementos ou operacións. Por exemplo, a primeira distinción feita separa os aneis conmutativos dos non conmutativos, segundo se a operación produto cumpre a propiedade conmutativa ou non. A continuación distínguense dous tipos de aneis atendendo ós elementos que conteñen; en particular, os divisores de cero propios, no caso dos \textbf{dominios}, e as unidades, no caso dos \textbf{corpos}:\\ 

\noindent \textbf{Definición 2.8}. Un \magbf{dominio} (tamén chamado \magbf{dominio cero} ou \magbf{dominio de integridade}) é un anel \textbf{\textit{conmutativo}} sen divisores de cero propios.\\

\noindent \textbf{Definición 2.9}. Un \magbf{corpo} é un anel no que todo elemento non nulo (isto é, distinto do neutro respecto da operación suma) posúe simétrico respecto da operación produto.\\

\noindent \textbf{Observación 2.4}. Da definición de corpo dedúcese que, se $K$ é un corpo, $\mathcal{U}$($K$) = $K - \{0\}$.\\

\magbf{\section{Ideais}}

\magbf{\subsection{Definición. Ideais principais}}

\vspace{5mm}

\noindent \textbf{Definición 2.10}. Sexa $(A,+,\cdot)$ un anel \textbf{\textit{conmutativo}}. Considérese $\mathcal{I}$ un subconxunto de $A$. Dirase que $\mathcal{I}$ é un \magbf{ideal de $A$} se verifica as seguintes condicións:\\
\begin{itemize}
    \item $(\mathcal{I},+) < (A,+)$
    \item $\forall \hspace{1mm} x \in A \hspace{2mm} \forall \hspace{1mm} a \in \mathcal{I}, x \cdot a \in \mathcal{I} \Longleftrightarrow A\mathcal{I} \subset \mathcal{I}$\\
\end{itemize}

\noindent \textbf{Observación 2.5}. Nótese que para a definición que se acaba de dar o anel ten que ser conmutativo. Noutro caso, habería que distinguir os conceptos de \textbf{ideal pola esquerda} e \textbf{ideal pola dereita}, facendo moitos dos resultados que se van ver nesta materia inválidos. A partir deste punto, \magbf{\textit{tódolos resultados que aparezan nesta unidade (exceptuando algunhas definicións) farán referencia a aneis conmutativos}}.\\

\begin{proposition} \label{prop2.2}
Sexa $A$ un anel conmutativo. Considérese $\mathcal{I}$ un ideal de A. Se $1 \in \mathcal{I}$, entón $\mathcal{I} = A$.
\end{proposition}

\vspace{2mm}

\noindent \textbf{\textit{\underline{Demostración}}}

\vspace{2mm}

\noindent Por definición, $\mathcal{I}$ é ideal $\Longleftrightarrow [\hspace{1mm} \forall \hspace{1mm} x \in A \hspace{3mm} \forall \hspace{1mm} a \in \mathcal{I} \implies x \cdot a \in \mathcal{I} \hspace{1mm}]$. En particular, como $1 \in \mathcal{I}$, $\forall \hspace{1mm} x \in A, \hspace{1mm} x \cdot 1 = x\in \mathcal{I}$. Así, tense que $\mathcal{I} = A$. $\square$\\

\vspace{3mm}

\noindent Exemplos de ideais son os seguintes:

\begin{enumerate}
    \item Nun anel $A$ conmutativo, $\{0\}$ e $A$ son ideais triviais de $A$. Calquera outro ideal de $A$ será denominado \textbf{ideal propio}.
    \item No anel dos números enteiros, os ideais son da forma $n\mathbb{Z}$, con $n \in \mathbb{Z}$. En particular, $1\mathbb{Z} = \mathbb{Z}$.\\
\end{enumerate}

\noindent \textbf{Definición 2.11}. Sexa $A$ un anel, e considérese $a \in A$. Defínese o \magbf{ideal principal xerado polo elemento $a$}, denotado por $(a) \equiv Aa \equiv aA$, como o seguinte subconxunto de $A$:

\begin{center}
\fcolorbox{magenta}{white}{
$(a) = \{x \cdot a \hspace{1mm} | \hspace{1mm} x \in A\}$
}
\end{center}

\noindent En $\mathbb{Z}$, todo ideal é principal, pois pódese escribir da forma $(n) = n\mathbb{Z} = \{nx \hspace{1mm} | \hspace{1mm} x \in \mathbb{Z}\}$.\\

\begin{proposition} \label{prop2.3}
Sexa $A$ un anel conmutativo. Se $a \in A$ é unha unidade, entón $(a) = A$.
\end{proposition}

\vspace{2mm}

\noindent \textbf{\textit{\underline{Demostración}}}

\vspace{2mm}

\noindent Sexa $a$ unha unidade de $A$. Entón, $\exists \hspace{1mm} a^{-1} \in A$. Tense:

\[ 
\left. \begin{array}{r} 
a \in (a)\\[1ex]
a^{-1} \in A
\end{array} \right\} 
\implies a^{-1} \cdot a \in (a) \implies 1 \in (a) \underset{\underset{\hyperref[prop2.2]{\magbf{Prop 2.2}}}{\Uparrow}}{\implies} (a) = A \hspace{3mm} \square
\]

\magbf{\subsection{Anel cociente}}

\vspace{5mm}

\noindent A diferenza doutras estruturas alxébricas, non é posible definir o cociente dun anel por un subanel. O subconxunto adecuado para definir cocientes son os ideais.\\

\noindent Considérese $(A, +, \cdot)$ un anel e $\mathcal{I}$ un ideal de $A$.\\

\noindent Por definición de anel, $(A,+)$ ten estrutura de grupo abeliano; logo, todo subgrupo seu vai ser normal; en particular, por definición de ideal, tense que $(\mathcal{I},+) < (A,+)$; logo, $(\mathcal{I},+) \triangleleft (A, +)$. Isto leva a pensar en establecer unha analoxía entre subgrupos normais (dentro da teoría de grupos) e ideais (dentro da teoría de aneis).\\

\noindent En particular, tal analoxía pódese entender no sentido en que se facilita a definición do \magbf{anel cociente} como unha extensión natural do grupo cociente:\\

\noindent Sendo $(\mathcal{I}, +) \triangleleft (A, +)$, tense que o par $\left (\displaystyle \frac{A}{\mathcal{I}}, + \right )$ ten estrutura de grupo. Dado $x \in A$, a súa clase de equivalencia é $[x] = x + \mathcal{I}$.\\

\noindent Considérense a continuación $a + \mathcal{I}$, $b + \mathcal{I} \in \displaystyle \frac{A}{\mathcal{I}}$. Defínese a continuación a seguinte operación:
$$(a + \mathcal{I}) \cdot (b + \mathcal{I}) := (a \cdot b) + \mathcal{I}$$

\noindent Comprobarase que esta operación está ben definida no conxunto cociente, i.e. que non depende do representante escollido. Tendo en conta que:
$$a + \mathcal{I} = a' + \mathcal{I} \Longleftrightarrow a - a' \in \mathcal{I}$$
$$b + \mathcal{I} = b' + \mathcal{I} \Longleftrightarrow b - b' \in \mathcal{I}$$
$$a \cdot b +\mathcal{I} = a' \cdot b' + \mathcal{I} \Longleftrightarrow a \cdot b - a' \cdot b' \in \mathcal{I}$$
\noindent cúmprese:
$$
\left. \begin{array}{r} 
\left. \begin{array}{r} 
a - a' \in \mathcal{I}\\[1ex]
b \in A
\end{array} \right\} 
\implies (a - a') \cdot b \in \mathcal{I} \implies a \cdot b - a' \cdot b \in \mathcal{I} \\
\\
\left. \begin{array}{r}
b - b' \in \mathcal{I}\\[1ex]
a' \in A
\end{array} \right\}
\implies a' \cdot (b - b') \in \mathcal{I} \implies a' \cdot b - a' \cdot b' \in \mathcal{I}
\end{array} \right\} 
\underset{\underset{(\mathcal{I},+) \text{ grupo}}{\Uparrow}}{\implies} (a \cdot b) - (a' \cdot b) + (a' \cdot b) - (a' \cdot b') \in \mathcal{I} \implies
$$
\begin{flushright}
$\implies a \cdot b - a' \cdot b' \in \mathcal{I}$\\
\end{flushright}

\noindent Esta operación é asociativa, pois depende da asociatividade en $\mathcal{I}$. Ademais, cúmprese que o elemento neutro da mesma é $1 + \mathcal{I}$.\\

\noindent Unha vez definida esta operación en $\displaystyle \frac{A}{\mathcal{I}}$, cúmprese que a terna $\left (\displaystyle \frac{A}{\mathcal{I}}, +, \cdot \right )$ posúe estrutura de anel, e recibe o nome de \magbf{anel cociente de $A$ por $\mathcal{I}$}.\pagebreak
\magbf{\subsection{Operacións con ideais. Ideais coprimos}}

\vspace{5mm}

\noindent Sexa $A$ un anel (conmutativo) e $\mathcal{I}$ un ideal de $A$. Por definición, tense que $(\mathcal{I},+) < (A,+)$. Polo tanto, parece razoable pensar que resultados estudados acerca de operación con subgrupos poidan ser ampliados ó caso dos aneis. Pois ben, esta afirmación resultará ser certa; en particular, tense: \\

\noindent \textbf{Observación 2.6.} En xeral, a unión de ideais non resulta ser un ideal. Isto xorde do feito de que a unión de subgrupos, en xeral, non é un subgrupo.\\

\noindent A seguinte proposición recolle operacións con ideais que preservan a estrutura de ideal; en particular, a intersección, a suma e o produto: \\

\begin{proposition} \label{prop2.4}
Sexa $A$ un anel conmutativo. Considérense $\mathcal{I}$, $\mathcal{J}$ ideais de $A$. Verifícanse as seguintes afirmacións:
\begin{enumerate}
    \item A intersección dos ideais $\mathcal{I}$ e $\mathcal{J}$, $\mathcal{I}\cap\mathcal{J}$, é un ideal de $A$.
    \item O conxunto $\mathcal{I}\mathcal{J} = \{\underset{i = 1}{\overset{n}{\sum}}{a_{i}b_{i}} \hspace{1mm} | \hspace{1mm} a_{i} \in \mathcal{I} \text{, } b_{i} \in \mathcal{J}\}$ é un ideal de $A$, denominado \textbf{ideal produto}.
    \item O conxunto $\mathcal{I} + \mathcal{J} = \{a + b \hspace{1mm} | \hspace{1mm} a \in \mathcal{I} \text{, } b \in \mathcal{J}\}$ é un ideal de $A$, denominado \textbf{ideal suma}.
    \item $\mathcal{IJ} \subset \mathcal{I}\cap\mathcal{J}$
\end{enumerate}
\end{proposition}

\vspace{2mm}

\noindent \textbf{\textit{\underline{Demostración}}}

\vspace{2mm}

\noindent \magbf{(1)} Sendo $\mathcal{I}$ e $\mathcal{J}$ ideais, cúmprese que $(\mathcal{I}, +) \text{, } (\mathcal{J}, +)$ son subgrupos de $(A, +)$. Hai que verificar a segunda condición de ideal, é dicir, se dados $h \in \mathcal{I}\cap\mathcal{J}$ e $x \in A$, $h \cdot x \in \mathcal{I}\cap\mathcal{J}$. Tense:
\[ 
\left. \begin{array}{r} 
h \in \mathcal{I} \underset{\underset{\mathcal{I} \textbf{ ideal}}{\Uparrow}}{\implies} \forall \hspace{1mm} x \in A \text{, } hx \in \mathcal{I}\\[1ex]
h \in \mathcal{J} \underset{\underset{\mathcal{J} \textbf{ ideal}}{\Uparrow}}{\implies} \forall \hspace{1mm} x \in A \text{, } hx \in \mathcal{J}
\end{array} \right\} 
\implies \forall \hspace{1mm} x \in A \text{, } hx \in \mathcal{I \cap J}
\]

\vspace{3mm}

\noindent \magbf{(2)} Para demostrar que $\mathcal{IJ}$ sexa subgrupo hai que probar:
\begin{itemize}
    \item $(\mathcal{IJ}, +)$ é subgrupo\\
    
    Sexan $\underset{i = 1}{\overset{n}{\sum}}{a_{i}b_{i}}$, $\underset{j = 1}{\overset{m}{\sum}}{c_{j}b_{j}} \in \mathcal{IJ}$. Entón, obviamente, $\underset{i = 1}{\overset{n}{\sum}}{a_{i}b_{i}} + \underset{j = 1}{\overset{m}{\sum}}{c_{j}d_{j}} \in \mathcal{IJ}$
    
    \item $\forall \hspace{1mm} x \in A \hspace{4mm} \forall \hspace{1mm} \underset{i = 1}{\overset{n}{\sum}}{a_{i}b_{i}} \in \mathcal{IJ}$, \hspace{3mm} $x\underset{i = 1}{\overset{n}{\sum}}{a_{i}b_{i}} \in \mathcal{IJ}$\\
    
    Nótese que $x \cdot \underset{i = 1}{\overset{n}{\sum}}{a_{i}b_{i}} = \underset{i = 1}{\overset{n}{\sum}}{a_{i}xb_{i}} = \underset{i = 1}{\overset{n}{\sum}}{(a_{i}x)b_{i}} = \underset{i = 1}{\overset{n}{\sum}}{a_{i}(b_{i}x)}$\\
    
    Sendo $\mathcal{I}$ un ideal, $a_{i}x \in \mathcal{I}$. Analogamente, por ser $\mathcal{J}$ un ideal, $b_{i}x \in \mathcal{J}$. Entón, observando as igualdades anteriores, tense que $x \cdot \underset{i = 1}{\overset{n}{\sum}}{a_{i}b_{i}} \in \mathcal{IJ}$\\
\end{itemize}

\noindent \magbf{(3)} Analogamente a \magbf{(2)} verase que $\mathcal{I + J}$ verifica a definición de ideal:

\begin{itemize}
    \item $(\mathcal{I + J}, +)$ é subgrupo\\
    
    Dados $a_{1} + b_{1}, a_{2} + b_{2} \in \mathcal{I + J}$, tense:
    
    $$(a_{1} + b_{1}) + (a_{2} + b_{2}) = (a_{1} + a_{2}) + (b_{1} + b_{2})$$
    
    Por ser $(\mathcal{I}, +)$ e $(\mathcal{J}, +)$ subgrupos (pois $\mathcal{I}$ e $\mathcal{J}$ son ideais), cúmprese que $a_{1} + a_{2} \in \mathcal{I}$ e $b_{1} + b_{2} \in \mathcal{J}$, logo $(a_{1} + a_{2}) + (b_{1} + b_{2}) \in \mathcal{I + J}$\\
    
    \item $\forall \hspace{1mm} x \in A \hspace{4mm} \forall \hspace{1mm} a + b \in \mathcal{I + J}$, \hspace{2mm} $x \cdot (a + b) \in \mathcal{I + J}$\\
    
    Sendo $A$ un anel, pódese aplicar a propiedade distributiva do produto respecto da suma, obtendo así:
    $$x \cdot (a + b) = x \cdot a + x \cdot b$$
    
    Sendo $\mathcal{I}$ e $\mathcal{J}$ ideais, cúmprese que $x \cdot a \in \mathcal{I}$ e $x \cdot b \in \mathcal{J}$. Así, $x \cdot a + x \cdot b = x \cdot (a + b) \in \mathcal{I + J}$ \\
    
\end{itemize}

\noindent \magbf{(4)} Sexa $x \in \mathcal{IJ}$. Entón, pódese escribir $x = \underset{i = 1}{\overset{n}{\sum}}a_{i}b_{i}$, con $a_{i} \in \mathcal{I}$, $b_{i} \in \mathcal{J}$ para cada $i \in \{1, \dots , n\}$.\\

\noindent Para cada $i \in \{1, \dots, n\}$, tense:

$$
\left. \begin{array}{r} 
\left. \begin{array}{r} 
a_{i} \in A\\[1ex]
b_{i} \in \mathcal{J}
\end{array} \right\} 
\overset{\mathcal{J} \text{ ideal}}{\overset{\Downarrow}{\implies}} a_{i}b_{i} \in \mathcal{J} \\
\\
\left. \begin{array}{r}
a_{i} \in \mathcal{I}\\[1ex]
b_{i} \in A
\end{array} \right\} 
\underset{\mathcal{I} \text{ ideal}}{\underset{\Uparrow}{\implies}} a_{i}b_{i} \in \mathcal{I}
\end{array} \right\} 
\implies a_{i}b_{i} \in \mathcal{I \cap J} \implies x = \underset{i = 1}{\overset{n}{\sum}}a_{i}b_{i} \in \mathcal{I \cap J} \implies \mathcal{IJ} \subset \mathcal{I \cap J} \hspace{4mm} \square 
$$

\vspace{3mm}

\noindent \textbf{Exercicio.} Achar un contraexemplo que evidencie que $\mathcal{I \cap J} \not\subset \mathcal{IJ}$.\\

\noindent \textbf{Definición 2.12.} Sexa $A$ un anel. Considérense $\mathcal{I}$ e $\mathcal{J}$ ideais de $A$. Dirase que $\mathcal{I}$ e $\mathcal{J}$ son \magbf{ideais coprimos} se $\exists \hspace{1mm} a \in \mathcal{I}, \hspace{2mm} b \in \mathcal{J}$ tales que $1 = a + b$.\\

\noindent Os seguintes resultados permiten caracterizar os ideais coprimos mediante as operacións que se acaban de definir:\\

\begin{proposition} \label{prop2.5}
Sexa $A$ un anel conmutativo. Considérense $\mathcal{I}$ e $\mathcal{J}$ ideais de $A$. Entón:
\begin{center}
    $\mathcal{I}$ e $\mathcal{J}$ son coprimos $\Longleftrightarrow \mathcal{I + J} = A$  
\end{center}
\end{proposition}

\vspace{2mm}

\noindent \textbf{\textit{\underline{Demostración}}}

\vspace{2mm}

\noindent \fcolorbox{magenta}{white}{$\Longrightarrow$/} Supóñanse $\mathcal{I}$ e $\mathcal{J}$ ideais coprimos. Entón, por definición, $\exists \hspace{1mm} a \in \mathcal{I}, b \in \mathcal{J} \hspace{1mm} | \hspace{1mm} 1 = a + b$.\\

\noindent Sendo $\mathcal{I + J}$ ideal de $A$, tense que $\mathcal{I + J} \subset A$. Probarase agora a outra inclusión.\\

\noindent Dado $x \in A$, pódese escribir: 
$$x = x \cdot 1 = x \cdot (a + b) = x \cdot a + x \cdot b$$

\noindent Sendo $\mathcal{I}$ e $\mathcal{J}$ ideais, $x \cdot a \in \mathcal{I}$ e $x \cdot b \in \mathcal{J}$, logo $x \cdot a + x \cdot b \in \mathcal{I + J}$. Así, $A \subset \mathcal{I + J}$.\\

\noindent \fcolorbox{magenta}{white}{$\Longleftarrow/$} Reciprocamente, supóñase agora que $\mathcal{I + J} = A$, e véxase que $\mathcal{I}$ e $\mathcal{J}$ son coprimos. Tense:

    \[ 
    \left. \begin{array}{r} 
    1 \in A \\[1ex]
    A = \mathcal{I + J}
    \end{array} \right\}
    \implies \exists \hspace{1mm} a \in \mathcal{I}, b \in \mathcal{J} \hspace{1mm} | \hspace{1mm} 1 = a + b \implies \mathcal{I} \text{ e } \mathcal{J} \text{ son coprimos} \hspace{4mm} \square
    \]
    
\vspace{3mm}

\begin{proposition} \label{prop2.6}
Sexa $A$ un anel conmutativo. Considérense $\mathcal{I}$ e $\mathcal{J}$ ideais de $A$. Entón:
\begin{center}
    $\mathcal{I}$ e $\mathcal{J}$ son coprimos $\Longrightarrow \mathcal{IJ} = \mathcal{I \cap J}$  
\end{center}
\end{proposition}

\vspace{2mm}

\noindent \textbf{\textit{\underline{Demostración}}}

\vspace{2mm}

\noindent Supóñanse $\mathcal{I}$ e $\mathcal{J}$ ideais coprimos. Entón, por definición, $\exists \hspace{1mm} a \in \mathcal{I}, b \in \mathcal{J} \hspace{1mm} | \hspace{1mm} 1 = a + b$.\\

\noindent Pola \hyperref[prop2.4]{\magbf{Proposición 2.4}} sábese que $\mathcal{IJ} \subset \mathcal{I \cap J}$. Abondará entón probar a outra inclusión.\\

\noindent Dado $x \in \mathcal{I \cap J}$, pódese escribir:
$$x = x \cdot 1 = x \cdot (a + b) = x \cdot a + x \cdot b$$

\noindent Tense entón:

$$
\left. \begin{array}{r} 
\left. \begin{array}{r} 
x \in \mathcal{J}\\[1ex]
a \in \mathcal{I}
\end{array} \right\} 
\implies x \cdot a \in \mathcal{IJ} \\
\\
\left. \begin{array}{r}
x \in \mathcal{I}\\[1ex]
b \in \mathcal{J}
\end{array} \right\} 
\implies x \cdot b \in \mathcal{IJ}
\end{array} \right\} 
\implies x \cdot a + x \cdot b \in \mathcal{IJ} \implies \mathcal{I \cap J} \subset \mathcal{IJ} \hspace{5mm} \square
$$

\vspace{3mm}

\begin{proposition} \label{prop2.7}
Sexa $A$ un anel conmutativo. Considérense $\mathcal{I}_{1}, \dots, \mathcal{I}_{n}$ ideais de $A$ coprimos 2 a 2. Entón, cúmprese:
\begin{center}
    $\underset{i = 1}{\overset{n}{\bigcap}}\mathcal{I}_{i} = \underset{i = 1}{\overset{n}{\prod}}\mathcal{I}_{i}$  
\end{center}
\end{proposition}

\vspace{2mm}

\noindent \textbf{\textit{\underline{Demostración}}}

\vspace{2mm}

\noindent Realizarase a proba por indución no número de ideais. 
\begin{enumerate}
    \item En primeiro lugar, compróbase que a afirmación é certa para o menor caso no que teña sentido (aquí $n = 2$). En efecto, para dous ideais cúmprese o resultado, sen máis ca decatarse de que se probou na \hyperref[prop2.6]{\magbf{Proposición 2.6}}.\\
    \item Supóñase agora certa a afirmación para o natural $n-1 \geq 2$, e véxase que se cumpre tamén para $n$.
    
    Sexan $\mathcal{I}_{1}, \ldots, \mathcal{I}_{n}$ ideais coprimos 2 a 2 de $A$. Por \textbf{hipótese de indución} pódese escribir:
    $$\mathcal{I}_{1} \cdot \mathcal{I}_{2} \cdot \ldots \cdot \mathcal{I}_{n} = \mathcal{I}_{1} \cdot \underset{i = 2}{\overset{n}{\prod}}\mathcal{I}_{i} = \mathcal{I}_{1} \cdot \underset{i = 2}{\overset{n}{\bigcap}}\mathcal{I}_{i}$$
    Probarase que $\mathcal{I}_{1}$ é coprimo con $\underset{i = 2}{\overset{n}{\prod}}\mathcal{I}_{i} = \underset{i = 2}{\overset{n}{\bigcap}}\mathcal{I}_{i}$.
    
    Tense o seguinte:
    
    \vspace{2mm}
    
    $\mathcal{I}_{1}$ e $\mathcal{I}_{2}$ son coprimos $\implies 1 = x_{1} + y_{2}$, con $x_{1} \in \mathcal{I}_{1}, y_{2} \in \mathcal{I}_{2}$
    
    \noindent \vdots
    
    $\mathcal{I}_{1}$ e $\mathcal{I}_{n}$ son coprimos $\implies 1 = x_{1} + y_{n}$, con $x_{1} \in \mathcal{I}_{1}, y_{n} \in \mathcal{I}_{n}$  \\  
    
    Entón, multiplicando as sumas anteriores, obtense:
    $$1 = (x_{1} + y_{2}) \cdot \ldots \cdot (x_{1} + y_{n}) = y_{2} \cdot \ldots \cdot y_{n} + \textcolor{magenta}{\Xi}$$
    
    onde \textcolor{magenta}{$\Xi$} é o resto do desenvolvemento do produto. Cúmprese que \textcolor{magenta}{$\Xi$} $\in \mathcal{I}_{1}$, pois no resto dos sumandos resultantes dese desenvolvemento aparecerá $x_{1}$ como factor. Doutra banda, $y_{2} \cdot \ldots \cdot y_{n} \in \underset{i = 2}{\overset{n}{\prod}}\mathcal{I}_{i} = \underset{i = 2}{\overset{n}{\bigcap}}\mathcal{I}_{i}$.
    
    Así, por definición, $\mathcal{I}_{1}$ é coprimo con $\underset{i = 2}{\overset{n}{\prod}}\mathcal{I}_{i} = \underset{i = 2}{\overset{n}{\bigcap}}\mathcal{I}_{i}$, podendo garantir así que $\underset{i = 1}{\overset{n}{\bigcap}}\mathcal{I}_{i} = \underset{i = 1}{\overset{n}{\prod}}\mathcal{I}_{i}$, como se quería probar. $\square$
\end{enumerate}

\magbf{\subsection{Ideais primos e ideais maximais}}

\vspace{5mm}

\noindent \textbf{Definición 2.13.} Sexa $A$ un anel conmutativo. Considérese $\mathcal{P}$ un ideal de $A$. Dirase que $\mathcal{P}$ é \magbf{ideal primo de $A$} se verifica as seguintes condicións:\\
\begin{enumerate}
    \item $\mathcal{P}$ é un ideal propio, i.e. $\mathcal{P} \neq A$
    \item Dados $x,y \in A$, se $x \cdot y \in \mathcal{P} \implies x \in \mathcal{P}$ ou $y \in \mathcal{P}$\\
\end{enumerate}

\noindent \textbf{Definición 2.14.} Sexa $A$ un anel conmutativo. Considérese $\mathcal{M}$ un ideal de $A$. Dirase que $\mathcal{M}$ é \magbf{ideal maximal de $A$} se cumpre as seguintes condicións:
\begin{enumerate}
    \item $\mathcal{M}$ é un ideal propio, i.e. $\mathcal{M} \neq A$
    \item Se $\mathcal{I}$ é outro ideal de $A$ tal que $\mathcal{M} \subset \mathcal{I} \subset \mathcal{A}$, entón $\mathcal{I} = \mathcal{M}$ ou $\mathcal{I} = A$\\
\end{enumerate}

\noindent A seguinte proposición permite caracterizar os ideais primos e maximais dun anel conmutativo segundo a estrutura do anel cociente ó que dan lugar:\\

\begin{proposition} \label{prop2.8}
Sexa $A$ un anel conmutativo. Verifícase o seguinte:
\begin{enumerate}
    \item Un ideal $\mathcal{P}$ de $A$ é primo $\Longleftrightarrow \displaystyle \frac{A}{\mathcal{P}}$ é un dominio
    \item Un ideal $\mathcal{M}$ de $A$ é maximal $\Longleftrightarrow \displaystyle \frac{A}{\mathcal{M}}$ é un corpo
\end{enumerate}
\end{proposition}

\vspace{2mm}

\noindent \textbf{\textit{\underline{Demostración}}}

\vspace{2mm}

\noindent \magbf{Proba de (1)}\\

\noindent \fcolorbox{magenta}{white}{$\Longrightarrow$/} Supóñase $\mathcal{P}$ un ideal primo de $A$. Probarase por definición que $\displaystyle \frac{A}{\mathcal{P}}$ é un dominio, é dicir, comprobarase que non contén divisores propios de cero.\\

\noindent Considérense $a + \mathcal{P}, b + \mathcal{P} \in \displaystyle \frac{A}{\mathcal{P}} \hspace{1mm} | \hspace{1mm} (a + \mathcal{P}) \cdot (b + \mathcal{P}) = 0_{A/\mathcal{P}} = \mathcal{P}$. Tense:
$$(a + \mathcal{P}) \cdot (b + \mathcal{P}) = 0 \Longleftrightarrow (a \cdot b) + \mathcal{P} = 0_{A/\mathcal{P}} \Longleftrightarrow a \cdot b \in \mathcal{P}$$

\noindent Sendo $\mathcal{P}$ un ideal primo, por definición, isto último implica que $a \in \mathcal{P}$ ou $b \in \mathcal{P}$, o cal é equivalente a que $a + \mathcal{P} = \mathcal{P}$ ou $b + \mathcal{P} = \mathcal{P}$, obtendo así que, efectivamente, no anel cociente non hai divsores de cero propios.\\

\noindent \fcolorbox{magenta}{white}{$\Longleftarrow$/} Reciprocamente, supóñase que $A/\mathcal{P}$ ten estrutura de dominio, e véxase que $\mathcal{P}$ é un ideal primo.\\

\noindent Sendo $A/\mathcal{P}$ un dominio, garántese que $A/\mathcal{P} \neq \{0\}$, logo $A \neq \mathcal{P}$.\\

\noindent Dados $x, y \in A$, se $x \cdot y \in \mathcal{P}$, $(x \cdot y) + \mathcal{P} = \mathcal{P} \implies (x + \mathcal{P}) \cdot (y + \mathcal{P}) = \mathcal{P}$. Sendo $A/\mathcal{P}$ un dominio, non existen divisores propios de cero, polo que necesariamente $x + \mathcal{P} = \mathcal{P}$ ou $y + \mathcal{P} = \mathcal{P}$, o cal equivale a que $x \in \mathcal{P}$ ou $y \in \mathcal{P}$. Polo tanto, asegúrase así que $\mathcal{P}$ é un ideal primo de $A$.\\

\vspace{3mm}

\noindent \magbf{Proba de (2)}\\

\noindent $A/\mathcal{M}$ é corpo $\overset{\magbf{[1]}}{\Longleftrightarrow}$ Os únicos ideais de $A/\mathcal{M}$ son $A/\mathcal{M}$ e $\mathcal{M}/\mathcal{M} \overset{\magbf{[2]}}{\Longleftrightarrow}$ Os únicos ideais de $A$ que conteñen a $\mathcal{M}$ son $\mathcal{M}$ e $A \Longleftrightarrow \mathcal{M}$ é un ideal maximal \hspace{2mm} $\square$ \\

\noindent $^{\magbf{[1]}}$ \textit{A proba deste resultado aparece no epígrafe de \hyperref[corpos]{\magbf{corpos}}}.\\
\noindent $^{\magbf{[2]}}$ \textit{Isto é consecuencia do \hyperref[th2.4]{\magbf{Teorema de correspondencia para aneis}}}.\\

\vspace{3mm}

\noindent Consecuencia inmediata desta proposición é o seguinte:\\

\begin{corollary} \label{cor2.1}
Todo ideal maximal é un ideal primo.
\end{corollary}

\vspace{2mm}

\noindent \textbf{\textit{\underline{Demostración}}}

\vspace{2mm}

\noindent Tal e como se estudará e demostrará no epígrafe de \hyperref[corpos]{\magbf{corpos}}, todo corpo é un dominio. Polo tanto, se $\mathcal{I}$ é un ideal maximal dun anel $A$, $A/\mathcal{I}$ é un corpo. Logo, $A/\mathcal{I}$ é un dominio, o que equivale a que $\mathcal{I}$ é un ideal primo. $\square$\\

\vspace{3mm}

\begin{proposition}[\magbf{Existencia de ideal maximal}] \label{prop2.9}
exa $A$ un anel conmutativo. Entón, existe un ideal maximal $\mathcal{M}$ en $A$.
\end{proposition}

\vspace{2mm}

\noindent \textbf{\textit{\underline{Demostración}}}

\vspace{2mm}

\noindent Sexa $\Sigma$ o conxunto de ideais propios do anel $A$:
$$\Sigma = \{ \mathcal{I} \subset A \hspace{1MM} | \hspace{1MM} \mathcal{I} \text{ é ideal de } A, \hspace{2mm} \mathcal{I} \neq A\}$$

\noindent Cúmprese que $\Sigma \neq \varnothing$, pois en particular $\{0\} \in \Sigma$.\\

\noindent Considérese en $\Sigma$ a relación $``\subset"$, a cal é unha relación de orde parcial. Escóllase $\{I_{\lambda}\}_{\lambda \in \Lambda}$ unha cadea de $(\Sigma, \subset)$ , isto é, un subconxunto de $\Sigma$ totalmente ordenado por $`` \subset"$. \\

\noindent Véxase a continuación que $\underset{\lambda \in \Lambda}{\bigcup}\mathcal{I_{\lambda}}$ serve como cota superior da cadea. Para iso, demostrarase en primeiro lugar que é un ideal.\\

\noindent Sexan $x, y \in \underset{\lambda \in \Lambda}{\bigcup}\mathcal{I_{\lambda}}$. Entón, $\exists \hspace{1mm} \alpha, \beta \in \Lambda \hspace{1mm} | \hspace{1mm} x \in \mathcal{I}_{\alpha}, y \in \mathcal{I}_{\beta}$.\\

\noindent Como o conxunto $\{I_{\lambda}\}_{\lambda \in \Lambda}$ está totalmente ordenado, un deles está contido no outro. Supóñase por exemplo que $\mathcal{I}_{\alpha} \subset \mathcal{I}_{\beta}$.\\

\noindent Sendo $\mathcal{I}_{\beta}$ un ideal, $x + y \in \mathcal{I}_{\beta} \subset \underset{\lambda \in \Lambda}{\bigcup}\mathcal{I_{\lambda}}$.\\

\noindent Sexa agora $\alpha \in A$. Probarase que, dado $x \in \underset{\lambda \in \Lambda}{\bigcup}\mathcal{I_{\lambda}}$, $\alpha \cdot x \in \underset{\lambda \in \Lambda}{\bigcup}\mathcal{I_{\lambda}}$. Como $x \in \underset{\lambda \in \Lambda}{\bigcup}\mathcal{I_{\lambda}}$, $\exists \hspace{1mm} \lambda \in \Lambda \hspace{1mm} | \hspace{1mm} x \in \mathcal{I}_{\lambda}$. Agora ben, sendo $\mathcal{I}_{\lambda}$ un ideal de $A$, tense que $\alpha \cdot x \in \mathcal{I}_{\lambda} \subset \underset{\lambda \in \Lambda}{\bigcup}\mathcal{I_{\lambda}}$.\\

\noindent Así, conseguiuse demostrar que $\underset{\lambda \in \Lambda}{\bigcup}\mathcal{I_{\lambda}}$ é un ideal de $A$.\\

\noindent Pero... é este ideal distinto do anel $A$? A resposta é afirmativa, pois en caso contrario, $1 \in \underset{\lambda \in \Lambda}{\bigcup}\mathcal{I_{\lambda}}$, o cal implicaría que $\exists \hspace{1mm} \lambda \in \Lambda \hspace{1mm} | \hspace{1mm} 1 \in \mathcal{I}_{\lambda}$, logo $\mathcal{I}_{\lambda} = A$, contradicindo así a propia definición de $\Sigma$ como o conxunto dos ideais \textit{propios} de $A$.\\

\noindent Deste xeito, $\underset{\lambda \in \Lambda}{\bigcup}\mathcal{I_{\lambda}}$ serve como cota superior da cadea. Chégase con isto a que $\Sigma$ é un conxunto ordenado no cal cada cadea posúe unha cota superior. Polo tanto, aplicando o \magbf{lema de Zorn}, garántese a existencia dun elemento $\mathcal{M}$ maximal de $\Sigma$. $\square$\\

\vspace{3mm}

\begin{corollary} \label{cor2.2}
Sexa $A$ un anel conmutativo. Dado un ideal $\mathcal{I} \subset A$, existe un ideal $\mathcal{M}$ maximal de $A$ tal que $\mathcal{I \subset M}$.
\end{corollary}

\noindent \textbf{\textit{\underline{Demostración}}}

\vspace{2mm}

\noindent Abonda aplicar a \hyperref[prop2.9]{\magbf{Proposición 2.9}} ao anel $\displaystyle \frac{A}{\mathcal{I}}$: sendo non baleiro, garántese a existencia dun ideal maximal $\displaystyle \frac{\mathcal{M}}{\mathcal{I}}$. Entón, $\mathcal{M}$ é un ideal maximal de $A$ $^{\magbf{[3]}}$. $\square$\\

\vspace{3mm}

\noindent $^{\magbf{[3]}}$ \textit{Esta última afirmación é consecuencia do \hyperref[th2.4]{\magbf{Teorema de correspondencia para aneis}}}.\\

\vspace{3mm}

\noindent \textbf{Definición 2.15.} Sexa $A$ un anel. Defínese o \magbf{radical de Jacobson de $A$} como a intersección de tódolos ideais maximais de $A$:

\begin{center}
    \fcolorbox{magenta}{white}{$J_{A} := \bigcap \hspace{1mm} \{\mathcal{M} \subset A \hspace{1mm} | \hspace{1mm} \mathcal{M} \text{ é ideal maximal de } A\}$}\\
\end{center}

\magbf{\section{Homomorfismos de aneis}}

\magbf{\subsection{Definición e propiedades}}

\vspace{5mm}

\noindent \textbf{Definición 2.16}. Sexan $(A, +, \cdot)$ e $(B, +, \cdot)$ dous aneis \textit{\underline{arbitrarios}} entre os cales se establece unha apliación $f: A \longrightarrow B$. Tal aplicación dirase un \magbf{homomorfismo de aneis} se verifica:\\

\begin{enumerate}
    \item $f: (A, +) \longrightarrow (B, +)$ é un homomorfismo de grupos, isto é, $f(a_{1} + a_{2}) = f(a_{1}) + f(a_{2}) \hspace{2mm} \forall \hspace{1mm} a_{1}, a_{2} \in A$
    \item $f(a_{1} \cdot a_{2}) = f(a_{1}) \cdot f(a_{2}) \hspace{2mm} \forall \hspace{1mm} a_{1}, a_{2} \in A$
    \item $f(1_{A}) = 1_{B}$\\
\end{enumerate}

\noindent \textbf{Observación 2.7.} Por simplicidade, denótanse as operacións suma e produto dos aneis $A$ e $B$ de igual xeito, aínda que en xeral non teñen por que coincidir.\\

\noindent Exemplos de homomorfismos de aneis son os seguintes:\\

\begin{enumerate}
    \item Dado un anel arbitrario $(A, +, \cdot)$, a aplicación identidade:
    
        $$ id: A \longrightarrow A $$
        
        é un homomorfismo de aneis.
        
    \item Sexa $A$ un anel arbitrario e $B$ un subanel de $A$. A aplicación inclusión:
    
        $$ i: B \hookrightarrow A $$
        
        é un homomorfismo de aneis.
        
    \item Considerando un anel $A$ arbitrario e $\mathcal{I}$ un ideal de $A$, a aplicación:
    
        \begin{center}
            $A \longrightarrow \displaystyle \frac{A}{\mathcal{I}}$\\
            \vspace{2mm}
            $\hspace{2mm} a \leadsto [a]$
        \end{center}
        
        é un homomorfismo de aneis, denominado \textbf{proxección canónica de $A$ sobre $\displaystyle \frac{A}{\mathcal{I}}$}.\\
    
\end{enumerate}

\noindent A partir da definición de homomorfismo de aneis extráese unha serie de propiedades que este tipo de aplicacións debe cumprir:

\begin{enumerate}
    \item $f(0_{A}) = 0_{B}$
    \item $f(-a) = -f(a) \hspace{3mm} \forall \hspace{1mm} a \in A$
    \item Se $a \in \mathcal{U}(A)$, entón $f(a^{-1}) = (f(a))^{-1}$
\end{enumerate}

\noindent As dúas primeiras propiedades son evidentes, pois $f$ é un homomorfismo entre os grupos $(A,+)$ e $(B,+)$. A proba da terceira propiedade é un exercicio moi sinxelo e análogo á demostración vista para homomorfismos de grupos.

\magbf{\magbf{\subsection{Ideais, subaneis e homomorfismos}}}

\vspace{5mm}

\noindent De xeito análogo aos homomorfismos de grupos, tense a seguinte definición:\\

\noindent \textbf{Definición 2.17.} Sexan $A$ e $B$ dous aneis arbitrarios entre os cales se establece un homomorfismo $f: A \longrightarrow B$. Defínense:\\

\begin{enumerate}
    \item O \magbf{núcleo de $f$}, denotado por \magen{$Ker \hspace{1mm} f$}:
    \begin{center}
        \fcolorbox{magenta}{white}{
        $Ker \hspace{1mm} f := \{a \in A \hspace{1mm} | \hspace{1mm} f(a) = 0_{B}\} = f^{-1}(\{0_{B}\})$
        }
    \end{center}
    
    \item A \magbf{imaxe de $f$}, denotada por \magen{$Im \hspace{1mm} f$}:
    \begin{center}
        \fcolorbox{magenta}{white}{
        $Im \hspace{1mm} f := \{b \in B \hspace{1mm} | \hspace{1mm} \exists \hspace{1mm} a \in A : f(a) = b\} = f(A)$}
    \end{center}
\end{enumerate}

\vspace{3mm}

\noindent Tamén se ten o seguinte resultado, idéntico ó que se viu en homomorfismos de grupos:\\

\begin{proposition} \label{prop2.10}
Sexan $A$ e $B$ dous aneis arbitrarios entre os cales se establece un homomorfismo $f: A \longrightarrow B$. Entón, cúmprese:
\begin{center}
    $f$ é inxectivo $\Longleftrightarrow Ker \hspace{1mm} f = \{0_{A}\}$  
\end{center}
\end{proposition}

\vspace{2mm}

\noindent \textbf{\textit{\underline{Demostración}}}

\vspace{2mm}

\noindent É completamente análoga á vista en homomorfismos de grupos, observando que neste caso, a operación considerada é a suma. $\square$\\

\begin{proposition} \label{prop2.11}
Sexan $A$ e $B$ aneis conmutativos entre os cales se establece un homomorfismo $f: A \longrightarrow B$. Verifícase o seguinte:
\begin{enumerate}
    \item Se $A'$ é un subanel de $A$, $f(A')$ é un subanel de $B$.
    \item Se $B'$ é un subanel de $B$, $f^{-1}(B')$ é un subanel de $A$.
    \item Se $\mathcal{I}$ é un ideal de $A$, $f(\mathcal{I})$ é un ideal de $f(A)$.
    \item Se $\mathcal{J}$ é un ideal de $B$, $f^{-1}(\mathcal{J})$ é un ideal de $A$.
\end{enumerate}
\end{proposition}

\vspace{2mm}

\noindent \textbf{\textit{\underline{Demostración}}}

\vspace{2mm}

\noindent \magbf{(1)} Hai que comprobar que $f(A')$ verifica a condición de subanel:

\begin{itemize}
    \item $(f(A'), +)$ é un grupo abeliano\\
    
    Sendo $f$ homomorfismo de aneis, en particular é homomorfismo de grupos. Polo tanto, como $A'$ é subanel, $(A', +)$ ten estrutura de subgrupo, logo $(f(A'), +)$ tamén. A conmutatividade está garantida porque $(B,+)$ é grupo abeliano.\\
    
    \item $\forall \hspace{1mm} x, y \in f(A')$, $x \cdot y \in f(A')$\\
    
    Como $x,y \in f(A')$, $\exists \hspace{1mm} a,b \in A' \hspace{1mm} | \hspace{1mm} f(a) = x, f(b) = y$. Sendo $f$ un homomorfismo de aneis, tense:
    $$x \cdot y = f(a) \cdot f(b) = f(a \cdot b)$$
    Pola condición de subanel de $A'$, $a \cdot b \in A'$. Así, cúmprese que $f(a \cdot b) = x \cdot y \in f(A')$.\\
    
    \item $1_{B} \in f(A')$\\
    
    Como $A'$ é subanel, cúmprese que $1_{A} \in A'$. Sendo $f$ homomorfismo de aneis, $f(1_{A}) = 1_{B}$, obtendo así o resultado.\\
\end{itemize}

\noindent \magbf{(2)} Véxase que $f^{-1}(B')$ é subanel de $A$:

\begin{itemize}
    \item $(f^{-1}(B'), +)$ é un grupo abeliano\\
    
    Razoando de xeito análogo a \magbf{(1)}, sendo $B'$ subanel, en particular $(B', +)$ é un subgrupo abeliano. Como $f$ é homomorfismo de aneis, tamén é homomorfismo de grupos. Logo, $(f^{-1}(B'), +)$ é subgrupo de $A$, e ademais, é abeliano porque $(A, +)$ é grupo abeliano.\\
    
    \item $\forall \hspace{1mm} x, y \in f^{-1}(B')$, $x \cdot y \in f^{-1}(B')$\\
    
    Como $x,y \in f^{-1}(B')$, $\exists \hspace{1mm} a,b \in B' \hspace{1mm} | \hspace{1mm} f(x) = a, f(y) = b$. Dado que $f$ é homomorfismo de aneis, tense:
    $$f(x \cdot y) = f(x) \cdot f(y) = a \cdot b$$
    Sabendo que $B'$ é subanel, $a \cdot b \in B'$. Así, verifícase que $x \cdot y \in f^{-1}(B')$.\\
    
    \item $1_{A} \in f^{-1}(B')$\\
    
    Sendo $B'$ subanel de $B$, $1_{B} \in B'$. Así, $f^{-1}(\{1_{B}\}) \subset f^{-1}(B')$. Como $f$ é homomorfismo de aneis, débese cumprir que $f(1_{A}) = 1_{B}$. Así, tense que $1_{A} \in f^{-1}(\{1_{B}\}) \subset f^{-1}(B')$.\\
\end{itemize}

\magbf{(3)} Probarase que $f(\mathcal{I})$ é un ideal de $f(A)$:

\begin{itemize}
    \item $(f(\mathcal{I}), +) < (f(A), +)$\\
    
    Certamente, como $\mathcal{I}$ é ideal de $A$, cúmprese que $(\mathcal{I},+) < (A, +)$. Sendo $f$ homomorfismo de aneis, en particular é homomorfismo de grupos, logo $(f(\mathcal{I}), +)$ é un subgrupo de $(B, +)$. Agora ben, como $\mathcal{I} \subset A$, tense que $f(\mathcal{I}) \subset f(A) \subset B$, garantindo así que $(f(\mathcal{I}), +) < (f(A), +)$. \\
    
    \item $\forall \hspace{1mm} x\in f(\mathcal{I}) \hspace{3mm} \forall \hspace{1mm} y \in f(A)$, $x \cdot y \in f(\mathcal{I})$\\
    
    En efecto, se $x \in f(\mathcal{I})$, $\exists \hspace{1mm} a \in \mathcal{I} \hspace{1mm} | \hspace{1mm} f(a) = x$. Analogamente, como $y \in f(A)$, $\exists \hspace{1mm} b \in A \hspace{1mm} | \hspace{1mm} f(b) = y$. Sendo $f$ homomorfismo de aneis, tense:
    $$x \cdot y = f(a) \cdot f(b) = f(a \cdot b)$$
    Como $\mathcal{I}$ é un ideal de $A$, $a \cdot b \in \mathcal{I}$, verificándose así que $x \cdot y = f(a \cdot b) \in f(\mathcal{I})$.\\
\end{itemize}

\magbf{(4)} Verificarase que $f^{-1}(\mathcal{J})$ é un ideal de $A$:

\begin{itemize}
    \item $(f^{-1}(\mathcal{J}), +) < (A, +)$\\
    
    Certamente, como $\mathcal{J}$ é ideal de $B$, cúmprese que $(\mathcal{J},+) < (B, +)$. Sendo $f$ homomorfismo de aneis, en particular é homomorfismo de grupos, logo $(f^{-1}(\mathcal{J}), +)$ é un subgrupo de $(A, +)$.\\
    
    \item $\forall \hspace{1mm} x\in f^{-1}(\mathcal{J}) \hspace{3mm} \forall \hspace{1mm} y \in A$, $x \cdot y \in f^{-1}(\mathcal{J})$\\
    
    En efecto, se $x \in f^{-1}(\mathcal{J})$, $\exists \hspace{1mm} a \in \mathcal{J} \hspace{1mm} | \hspace{1mm} f(x) = a$. Analogamente, como $y \in A$, $\exists \hspace{1mm} b \in B \hspace{1mm} | \hspace{1mm} f(y) = b$. Sendo $f$ homomorfismo de aneis, tense:
    $$f(x \cdot y) = f(x) \cdot f(y) = a \cdot b$$
    Como $\mathcal{J}$ é un ideal de $B$, $f(x \cdot y) = a \cdot b \in \mathcal{J}$, verificándose así que $x \cdot y \in f^{-1}(\mathcal{J})$. $\square$\\
\end{itemize}

\noindent \textbf{Observación 2.8.} Os apartados \magbf{(1)} e \magbf{(2)} son aplicables a aneis arbitrarios. Non obstante, non sucede así con \magbf{(3)} e \magbf{(4)}, sendo preciso especificar que os aneis involucrados sexan conmutativos.\\

\noindent \textbf{Observación 2.9.} Dado un homomorfismo de aneis conmutativos $f: A \longrightarrow B$, se $\mathcal{I}$ é un ideal de $A$, en xeral $f(\mathcal{I})$ non é un ideal de $B$. Por exemplo, considérese o homomorfismo inclusión de $\mathbb{Z}$ en $\mathbb{Q}$:
$$i : \mathbb{Z} \hookrightarrow \mathbb{Q}$$
\noindent Tense que $\mathbb{Z}$ é un ideal de $\mathbb{Z}$. Non obstante, $i(\mathbb{Z}) = \mathbb{Z}$ non é un ideal de $\mathbb{Q}$. Máis aínda, calquera ideal de $\mathbb{Z}$, $n\mathbb{Z}$, non é un ideal en $\mathbb{Q}$.

\noindent Agora ben, se $f$ é un homomorfismo sobrexectivo, a imaxe de todo ideal de $A$ resulta ser un ideal de $B$.\\

\noindent Como consecuencia inmediata da proposición anterior, obtense o seguinte resultado:\\

\begin{corollary} \label{cor2.3}
Sexan $A$ e $B$ aneis conmutativos entre os cales se establece un homomorfismo $f: A \longrightarrow B$. Verifícase o seguinte:
\begin{enumerate}
    \item $Ker \hspace{1mm} f$ é un ideal de $A$.
    \item $Im \hspace{1mm} f$ é un subanel de $B$.
\end{enumerate}
\end{corollary}

\vspace{2mm}

\noindent \textbf{\textit{\underline{Demostración}}}

\vspace{2mm}

\noindent \magbf{(1)} Abonda decatarse de que $Ker \hspace{1mm} f = f^{-1}(\{0_{B}\})$. Como $\{0_{B}\}$ é un ideal de $B$, aplicando o apartado \magbf{(4)} da \hyperref[prop2.11]{\magbf{Proposición 2.11}}, obtense o resultado.\\

\noindent \magbf{(2)} Sabendo que $Im \hspace{1mm} f = f(A)$, e que $A$ é un subanel de si mesmo, basta aplicar o apartado \magbf{(1)} da \hyperref[prop2.11]{\magbf{Proposición 2.11}} para obter o resultado. $\square$\\

\noindent Os homomorfismos de aneis son en particular aplicacións, e como tales poden posuír carácter inxectivo e/ou sobrexectivo. Ademais, pode suceder que o dominio e o rango sexan o mesmo anel. Para este tipo de situacións pódese facer uso da mesma terminoloxía empregada no caso de homomorfismos de grupos:\\

\noindent \textbf{Definición 2.18.} Sexa $f: A \longrightarrow B$ un homomorfismo de aneis.
\begin{enumerate}
    \item Se $A = B$, dise que $f$ é un \magbf{endomorfismo}.
    \item Se $f$ é inxectivo, dise que é un \magbf{monomorfismo}.
    \item Se $f$ é sobrexectivo, denominarase \magbf{epimorfismo}.
    \item Se $f$ é bixectivo, chamaráselle \magbf{isomorfismo}.
    \item Se $f$ é un endomorfismo bixectivo, dise que $f$ é un \magbf{automorfismo}.
\end{enumerate}

\vspace{3mm}

\noindent \textbf{Definición 2.19.} Sexan $A$ e $B$ aneis. Dirase que $A$ e $B$ son \magbf{isomorfos}, e denotarase por \textcolor{magenta}{$A \simeq B$}, se existe un isomorfismo de aneis $f: A \longrightarrow B$.

\magbf{\subsection{Teoremas de isomorfía de aneis}}

\vspace{5mm}

\noindent Xa se adiantaba na unidade anterior que os enunciados dos \magbf{teoremas de isomorfía} son tamén aplicables a outras estruturas alxébricas, en particular os aneis. Recóllense neste apartado os enunciados e as correspondentes demostracións destes resultados:\\

\vspace{3mm}

\begin{theorem}[\textbf{1.º Teorema de Isomorfía de aneis}] \label{th2.1}
Sexan $A$ e $B$ senllos aneis entre os cales se establece un homomorfismo $f: A \longrightarrow B$. Entón, cúmprese:
\begin{center}
    $\displaystyle \frac{A}{Ker \hspace{1mm} f} \simeq Im \hspace{1mm} f$
\end{center}
\end{theorem}

\vspace{2mm}

\noindent \textbf{\textit{\underline{Demostración}}}

\vspace{2mm}

\noindent En primeiro lugar, cómpre notar que $\displaystyle \frac{A}{Ker \hspace{1mm} f}$ ten estrutura de anel porque $Ker \hspace{1mm} f$ é un ideal de $A$. \\

\noindent Defínase a seguinte aplicación:\\

\begin{center}
$ \pi: \displaystyle \frac{A}{Ker \hspace{1mm} f} \longrightarrow Im \hspace{1mm} f$\\
\vspace{3mm}
\hspace{25mm} $a + Ker \hspace{1mm} f \leadsto \pi(a + Ker \hspace{1mm} f ) := f(a)$\\
\end{center}

\noindent e véxase que é un isomorfismo de aneis.\\

\noindent Esta aplicación xa foi definida para a proba do \hyperref[th1.5]{\magbf{1.º Teorema de Isomorfía de grupos}}, e xa se ten comprobado que, efectivamente, é un isomorfismo \textbf{de grupos}. Así, a proba redúcese a comprobar que conserva o produto interno así coma o seu elemento neutro:

\begin{itemize}
    
    \item $\forall \hspace{1mm} a,b \in A$, $\pi((a + Ker \hspace{1mm} f) \cdot (b + Ker \hspace{1mm} f)) = \pi(a + Ker \hspace{1mm} f) + \pi(b + Ker \hspace{1mm} f)$\\
    
    Certamente:
    $$\pi((a + Ker \hspace{1mm} f) \cdot (b + Ker \hspace{1mm} f)) = \pi(a \cdot b + Ker \hspace{1mm} f) = f(a \cdot b) \overset{\overset{f \text{ hom.}}{\downarrow}}{=} f(a) \cdot f(b) = \pi(a + Ker \hspace{1mm} f) + \pi(b + Ker \hspace{1mm} f)$$
    
    \item $\pi(1_{A} + Ker \hspace{1mm} f) = 1_{B}$\\
    
    Cúmprese pola propia definición de $\pi$: $$\pi(1_{A} + Ker \hspace{1mm} f) = f(1_{A}) \underset{\underset{f \text{hom.}}{\uparrow}}{=} 1_{B}$$
    
\end{itemize}

\noindent Por todo o anterior, cúmprese que $\pi$ é, efectivamente, un isomorfismo de aneis, obtendo así o resultado. $\square$ \pagebreak

\vspace{3mm}

\begin{theorem}[\textbf{2.º Teorema de Isomorfía de aneis}] \label{th2.2}
Sexa $A$ un anel. Supóñanse $\mathcal{I}$ e $\mathcal{J}$ ideais de $A$, cumprindo $\mathcal{J} \subset \mathcal{I}$. Entón, verifícase:
\begin{center}
    $\displaystyle \frac{A/\mathcal{J}}{\mathcal{I}/\mathcal{J}}$ $\simeq \displaystyle \frac{A}{\mathcal{I}}$
\end{center}
\end{theorem}

\vspace{2mm}

\noindent \textbf{\textit{\underline{Demostración}}}

\vspace{2mm}

\noindent Cómpre ter en conta que, gracias ao \hyperref[th1.6]{\magbf{2.º Teorema de Isomorfía de grupos}}, sabemos que os grupos $\left ( \displaystyle \frac{A/\mathcal{J}}{\mathcal{I}/\mathcal{J}}, + \right )$ e $\left ( \displaystyle \frac{A}{\mathcal{I}}, + \right )$ son isomorfos.\\

\noindent Defínase a seguinte aplicación:

\begin{center}
$ f: \displaystyle \frac{A}{\mathcal{J}} \longrightarrow \displaystyle \frac{A}{\mathcal{I}}$\\
\vspace{3mm}
\hspace{25mm} $a + \mathcal{J} \leadsto f(a + \mathcal{J}) := a + \mathcal{I}$\\
\end{center}

\noindent Esta mesma aplicación xa foi empregada na proba do \hyperref[th1.6]{\magbf{2.º Teorema de Isomorfía de grupos}}, cumprindo ser un epimorfismo de grupos con $Ker \hspace{1mm} f = \displaystyle \frac{\mathcal{I}}{\mathcal{J}}$. Tense entón que $\displaystyle \frac{\mathcal{I}}{\mathcal{J}}$ é un ideal de $\displaystyle \frac{A}{\mathcal{J}}$.\\

\noindent Se se consegue demostrar que $f$ conserva o produto e o seu elemento neutro, poderase aplicar o \hyperref[th2.1]{\magbf{1.º Teorema de Isomorfía de aneis}} para garantir que o resultado é certo.

\begin{itemize}
    \item $\forall \hspace{1mm} a,b \in A$, tense:
    $$f((a + \mathcal{J}) \cdot (b + \mathcal{J})) = f(a \cdot b + \mathcal{J}) = a \cdot b + \mathcal{I} = (a + \mathcal{I}) \cdot (b + \mathcal{I}) = f(a + \mathcal{J}) \cdot f(b + \mathcal{J})$$
    
    \item $f(1 + \mathcal{J}) = 1 + \mathcal{I}$
\end{itemize}

\noindent Cúmprese entón que $f$ é un homomorfismo de aneis. Así, aplicando o \hyperref[th2.1]{\magbf{1.º Teorema de Isomorfía de aneis}}, obtense o resultado. $\square$\\

\vspace{3mm}

\noindent \textbf{Observación 2.10}. Existe o \magbf{3.º Teorema de Isomorfía de aneis}, e será enunciado a continuación. Non obstante, involucra resultados que non aparecen nestes apuntamentos. Por esa razón, non se vai incluír aquí a súa demostración. Do mesmo xeito, o seu interese nesta materia é nulo.\\ 

\begin{theorem}[\magbf{3.º Teorema de Isomorfía de aneis}] \label{th2.3}
Sexa $A$ un anel. Supóñanse $\mathcal{I}$ un ideal de $A$ e $B$ un subanel arbitrario de $A$. Entón, cúmprese:
\begin{center}
    $\displaystyle \frac{B + \mathcal{I}}{\mathcal{I}} \simeq \displaystyle \frac{B}{B \cap \mathcal{I}}$
\end{center}
\end{theorem}

\vspace{2mm}

\noindent \textbf{\textit{\underline{Demostración}}}

\vspace{2mm}

\noindent Pódese atopar en \cite{viray}. $\square$

\vspace{2mm}

\magbf{\subsection{Teorema de correspondencia de aneis}} \label{ThCorrAn}

\vspace{5mm}

\noindent Tal e como sucede cos teoremas de isomorfía, o \magbf{Teorema de Correspondencia} que se enunciou na unidade anterior tamén se pode adaptar á teoría de aneis do seguinte xeito:\pagebreak

\begin{theorem} [\magbf{Teorema de correspondencia de aneis}] \label{th2.4}
Sexan $A$ e $B$ aneis entre os cales se establece un epimorfismo $f: A \longrightarrow B$. Considérense os seguintes conxuntos:
$$I = \{\mathcal{I} \subset A \hspace{1mm} | \hspace{1mm} \mathcal{I} \text{ é ideal de }A \text{, }Ker \hspace{1mm} f \subset \mathcal{I}\} \hspace{10mm} J = \{\mathcal{J} \subset B \hspace{1mm} | \hspace{1mm} \mathcal{J} \text{ é ideal de } B\}$$
\noindent Cúmprese que os conxuntos $I$ e $J$ están en bixección. En particular, as aplicacións: 
    \begin{center}
            $\phi: I \longrightarrow J$ \\
        \vspace{2mm}
        $\hspace{20mm} \mathcal{I} \leadsto \phi(\mathcal{I}) := f(\mathcal{I})$
    \end{center} 
\noindent e a súa inversa:
    \begin{center}
            $\phi^{-1}: J \longrightarrow I$ \\
        \vspace{2mm}
        $\hspace{32mm} \mathcal{J} \leadsto \phi^{-1}(\mathcal{J}) := f^{-1}(\mathcal{J})$
    \end{center} 
\noindent son bixectivas.
\end{theorem}

\vspace{2mm}

\noindent \textbf{\textit{\underline{Demostración}}}

\vspace{2mm}

\noindent Sábese, pola \hyperref[prop2.11]{\magbf{Proposición 2.11}}, que a imaxe dun ideal de $A$ por $f$ é un ideal de $f(A) = B$ (esta igualdade tense polo carácter sobrexectivo de $f$). Do mesmo xeito, a imaxe recíproca dun ideal de $B$ por $f$ resulta ser un ideal de $A$.\\

\noindent Sendo $f$ un epimorfismo de aneis, en particular $f: (A, +) \longrightarrow (B,+)$ é un epimorfismo de grupos. Entón, segundo o \hyperref[th1.8]{\magbf{Teorema de Correspondencia de grupos}}, garántese que existe unha bixección entre o conxunto de subgrupos de $A$ que conteñen a $Ker \hspace{1mm} f$ e o conxunto de subgrupos de $B$. En particular, tense o seguinte esquema:

$$ \mathcal{I} \overset{\phi}{\longrightarrow} f(\mathcal{I}) \overset{\phi^{-1}}{\longrightarrow} f^{-1}(f(\mathcal{I})) = \mathcal{I}$$

$$ \mathcal{J} \overset{\phi^{-1}}{\longrightarrow} f^{-1}(\mathcal{J}) \overset{\phi}{\longrightarrow} f(f^{-1}(\mathcal{J})) = \mathcal{J}$$\\

\noindent sabendo certas as igualdades anteriores (\textbf{no sentido de conxuntos}) gracias ó teorema mencionado.\\

\noindent Así, xuntando ámbolos dous resultados, cúmprese que, efectivamente, $\phi$ e $\phi^{-1}$ son bixeccións, completando deste xeito a proba. $\square$\\

\vspace{3mm}

\noindent Razoando de xeito completamente análogo a como se fixo no caso de grupos, séguese inmediatamente do teorema anterior:\\

\begin{corollary} \label{cor2.4}
Sexa $A$ un anel e $\mathcal{I}$ un ideal de $A$. Considérense os seguintes conxuntos:
$$M = \{\mathcal{J} \subset A \hspace{1mm} | \hspace{1mm} \mathcal{J} \text{ é ideal de A, }\mathcal{I} \subset \mathcal{J}\} \hspace{10mm} N = \{ \mathcal{K} \subset \displaystyle \frac{A}{\mathcal{I}} \hspace{1mm} | \hspace{1mm} \mathcal{K} \text{ é ideal de } \displaystyle \frac{A}{\mathcal{I}}\}$$
\noindent Cúmprese que existe unha bixección entre $M$ e $N$.
\end{corollary}

\vspace{2mm}

\noindent \textbf{\textit{\underline{Demostración}}}

\vspace{2mm}

\noindent Abonda aplicar o \hyperref[th2.4]{\magbf{Teorema de Correspondencia de aneis}} á proxección canónica de $A$ en $A/\mathcal{I}$, que é un epimorfismo de núcleo $\mathcal{I}$. $\square$ \\

\magbf{\subsection{Teorema chinés dos restos}}

\vspace{5mm}

\noindent O \magbf{Teorema chinés dos restos}, en realidade, xurdiu formulado como un problema nun tratado matemático do século III, denominado \textit{Sun Zi Suanjing}. O seu enunciado reza así:\\

\begin{mdframed}[linecolor = magenta]
    \textit{``Tense un conxunto de elementos dos que non se sabe cantos hai. Se se contan de tres en tres, sobran dous; de cinco en cinco, sobran tres; e de sete en sete, sobran dous. Cantos elementos hai?"}
\end{mdframed}

\noindent Nese manuscrito non se dá ningunha proba ou algoritmo para resolver o problema. Algunhas versións deste teorema foron recollidas polos matemáticos indios Aryabhata (século VI) e Brahmagupta (século VII), así coma por Fibonacci na súa obra \textit{Liber Abaci} (1202).\\

\noindent Este teorema, formulado en termos da teoría de números (congruencias), aparece enunciado e demostrado no \textit{Shushu Jiuzhang}, obra do matemático chinés Qin Jiushao, en 1247.\\

\noindent A continuación vaise enunciar este resultado en termos da teoría de aneis conmutativos, sendo esta unha xeneralización do resultado anterior á álxebra abstrcta. Posteriormente, aplicarase ao caso particular dos números enteiros, que foi o resultado probado por Qin Jiushao.\\

\begin{theorem}[\magbf{Teorema chinés dos restos}] \label{th2.5}
Sexa $A$ un anel. Considérense $\mathcal{I}_{1}, \ldots, \mathcal{I}_{n}$ ideais de $A$.
\noindent Defínase a seguinte aplicación: 
    \begin{center}
            $\phi: A \longrightarrow \displaystyle \frac{A}{\mathcal{I}_{1}} \times \ldots \times \displaystyle \frac{A}{\mathcal{I}_{n}}$ \\
        \vspace{2mm}
        $\hspace{24mm} x \leadsto \phi(x) := (x + \mathcal{I}_{1}, \ldots, x + \mathcal{I}_{n})$
    \end{center} 
\noindent Verifícase o seguinte:
\begin{enumerate}
    \item $\phi$ é un homomorfismo de aneis
    \item $\phi$ é inxectiva $\Longleftrightarrow \overset{n}{\underset{i = 1}{\bigcap}}\mathcal{I}_{i} = \{0\}$
    \item $\phi$ é sobrexectiva $\Longleftrightarrow \mathcal{I}_{1}, \ldots, \mathcal{I}_{n}$ son coprimos 2 a 2 
\end{enumerate}
\end{theorem}

\vspace{2mm}

\noindent \textbf{\textit{\underline{Demostración}}}

\vspace{2mm}

\noindent \magbf{Proba de (1)}\\ 

\noindent Déixase como exercicio.\\

\noindent \magbf{Proba de (2)}\\ 

\noindent Sabendo que $\phi$ é un homomorfismo, de ser inxectivo, $Ker \hspace{1mm} \phi = \{0\}$. Calcúlese o núcleo de $\phi$:
$$Ker \hspace{1mm} \phi = \{x \in A \hspace{1mm} | \hspace{1mm} \phi(x) = 0\} = \{ x \in A \hspace{1mm} | \hspace{1mm} (x + \mathcal{I}_{1}, \ldots, x + \mathcal{I}_{n}) = 0\} = \{ x \in A \hspace{1mm} | \hspace{1mm} x + \mathcal{I}_{i} = \mathcal{I}_{i} \hspace{2mm} \forall \hspace{1mm} i \in \{1, \ldots, n\}\} = $$
$$= \{x \in A \hspace{1mm} | \hspace{1mm} x \in \mathcal{I}_{i} \hspace{2mm} \forall \hspace{1mm} i \in \{1, \ldots, n\}\} = \{x \in A \hspace{1mm} | \hspace{1mm} x \in \overset{n}{\underset{i = 1}{\cap}}\mathcal{I}_{i}\} = \overset{n}{\underset{i = 1}{\bigcap}}\mathcal{I}_{i}$$

\noindent \magbf{Proba de (3)}\\

\noindent \fcolorbox{magenta}{white}{$\Longrightarrow$/} Supóñase $f$ sobrexectiva. Entón, dado calquera elemento de $\displaystyle \frac{A}{\mathcal{I}_{1}} \times \ldots \times \displaystyle \frac{A}{\mathcal{I}_{n}}$, en particular $(1 + \mathcal{I}_{1}, 0 + \mathcal{I}_{2}, \ldots, 0 + \mathcal{I}_{n})$, sábese que $\exists \hspace{1mm} x_{1} \in A \hspace{1mm} | \hspace{1mm} \phi(x_{1}) = (1 + \mathcal{I}_{1}, 0 + \mathcal{I}_{2}, \ldots, 0 + \mathcal{I}_{n})$.\\

\noindent Pola propia definición de $\phi$, $\phi(x_{1}) = (x_{1} + \mathcal{I}_{1}, x_{1} + \mathcal{I}_{2}, \ldots, x_{1} + \mathcal{I}_{n})$. Logo, tense:
$$(x_{1} + \mathcal{I}_{1}, x_{1} + \mathcal{I}_{2}, \ldots, x_{1} + \mathcal{I}_{n}) = (1 + \mathcal{I}_{1}, 0 + \mathcal{I}_{2}, \ldots, 0 + \mathcal{I}_{n}) \Longleftrightarrow 
\begin{cases}
x_{1} + \mathcal{I}_{1} = 1 + \mathcal{I}_{1}\\
x_{1} + \mathcal{I}_{2} = 0 + \mathcal{I}_{2}\\
\vdots\\
x_{1} + \mathcal{I}_{n} = 0 + \mathcal{I}_{n}
\end{cases}
\Longleftrightarrow
\begin{cases}
x_{1} - 1 \in \mathcal{I}_{1}\\
x_{1} \in \mathcal{I}_{2}\\
\vdots\\
x_{1} \in \mathcal{I}_{n}
\end{cases}
$$

\noindent Por todo o anterior, e tendo en conta a definición de ideais coprimos, séguese:

\[ 
\left. \begin{array}{r} 
1 = (1 - x_{1}) + x_{1}\\[1ex]
1 - x_{1} \in \mathcal{I}_{1}\\[1ex]
x_{1} \in \mathcal{I}_{i} \hspace{2mm} \forall \hspace{1mm} i \in \{2, \ldots, n\}
\end{array} \right\} 
\implies \mathcal{I}_{1} \text{ é coprimo con } \mathcal{I}_{i} \hspace{2mm} \forall \hspace{1mm} i \in \{2, \ldots, n\}
\]

\vspace{2mm}

\noindent Razoando de xeito análogo para cada compoñente $i$-ésima, con $i \in \{2, \ldots, n\}$, obtense o resutado.\\

\noindent \fcolorbox{magenta}{white}{$\Longleftarrow$/} Reciprocamente, supóñase que os ideais $\mathcal{I}_{1}, \mathcal{I}_{2}, \ldots, \mathcal{I}_{n}$ son coprimos 2 a 2, e compróbese que $\phi$ é un epimorfismo.\\

\noindent Considérese un elemento $(x_{1} + \mathcal{I}_{1}, \ldots, x_{n} + \mathcal{I}_{n}) \in \displaystyle \frac{A}{\mathcal{I}_{1}} \times \ldots \times \displaystyle \frac{A}{\mathcal{I}_{n}}$. Este elemento pódese escribir do seguinte xeito:

\begin{center}
$(x_{1} + \mathcal{I}_{1}, \ldots, x_{n} + \mathcal{I}_{n}) = (x_{1} + \mathcal{I}_{1}, 0 + \mathcal{I}_{2}, \ldots, 0 + \mathcal{I}_{n}) + \ldots + (0 + \mathcal{I}_{1}, \ldots, 0 + \mathcal{I}_{n-1}, x_{n} + \mathcal{I}_{n}) = $ 
\end{center}
$$\hspace{-7mm} = (x_{1} + \mathcal{I}_{1}, x_{1} + \mathcal{I}_{2}, \ldots, x_{1} + \mathcal{I}_{n}) \cdot (1 + \mathcal{I}_{1}, 0 + \mathcal{I}_{2}, \ldots, 0 + \mathcal{I}_{n}) + \ldots + (x_{n} + \mathcal{I}_{1}, \ldots, x_{n} + \mathcal{I}_{n-1}, x_{n} + \mathcal{I}_{n}) \cdot (0 + \mathcal{I}_{1}, \ldots, 0 + \mathcal{I}_{n-1}, 1 + \mathcal{I}_{n}) = $$
\begin{center}
$= \phi(x_{1}) \cdot (1 + \mathcal{I}_{1}, 0 + \mathcal{I}_{2}, \ldots, 0 + \mathcal{I}_{n}) + \ldots + \phi(x_{n}) \cdot (0 + \mathcal{I}_{1}, \ldots, 0 + \mathcal{I}_{n-1}, 1 + \mathcal{I}_{n})$\\
\end{center}

\noindent Por hipótese, $\mathcal{I}_{1}$ é coprimo con $\mathcal{I}_{j} \hspace{2mm} \forall \hspace{1mm} j \in \{2, \ldots, n\}$. Entón, como se viu na \hyperref[prop2.7]{\magbf{Proposición 2.7}}, $\mathcal{I}_{1}$ é coprimo con $\overset{n}{\underset{j = 2}{\bigcap}}\mathcal{I}_{j}$. Entón, por definición, $\exists \hspace{1mm} z_{1} \in \mathcal{I}_{1}, y_{1} \in \overset{n}{\underset{j = 2}{\bigcap}}\mathcal{I}_{j} \hspace{1mm} | \hspace{1mm} 1 = z_{1} + y_{1}$.\\

\noindent Cúmprese que $\phi(y_{1}) = (y_{1} + \mathcal{I}_{1}, y_{1} + \mathcal{I}_{2}, \ldots, y_{1} + \mathcal{I}_{n}) = (1 + \mathcal{I}_{1}, 0 + \mathcal{I}_{2}, \ldots, 0 + \mathcal{I}_{n})$.\\

\noindent Facendo este mesmo razoamento con cada un dos ideais $\mathcal{I}_{j}$, garántese que os elementos $(1 + \mathcal{I}_{1}, 0 + \mathcal{I}_{2}, \ldots, 0 + \mathcal{I}_{n}), \ldots, (0 + \mathcal{I}_{1}, \ldots, 0 + \mathcal{I}_{n-1}, 1 + \mathcal{I}_{n})$ posúen antecendente en $A$ mediante $\phi$, os cales se denotan aquí por $y_{1}, \ldots, y_{n}$. \\

\noindent Así, sendo $\phi$ un homomorfismo, e recordando o desenvolvemento anterior, pódese escribir:
$$(x_{1} + \mathcal{I}_{1}, \ldots, x_{n} + \mathcal{I}_{n}) = \phi(x_{1}) \cdot \phi(y_{1}) + \ldots + \phi(x_{n}) \cdot \phi(y_{n}) = \phi(x_{1} \cdot y_{1}) + \ldots + \phi(x_{n} \cdot y_{n}) = \phi(x_{1} \cdot y_{1} + \ldots + x_{n} \cdot y_{n})$$

\noindent Deste xeito, asegúrase que $\phi$ é un epimorfismo, concluíndo así a demostración. $\square$\\

\vspace{3mm}

\noindent A continuación, verase a formulación particular deste teorema para o anel dos números enteiros, a cal xa foi vista na materia de \textit{Linguaxe matemática, conxuntos e números}:\\

\noindent Considérense $n_{1}\mathbb{Z}, \ldots, n_{r}\mathbb{Z}$ ideais de $\mathbb{Z}$. Sexa $(x_{1} + n_{1}\mathbb{Z}, \ldots, x_{r} + n_{r}\mathbb{Z}) \in \displaystyle \frac{\mathbb{Z}}{n_{1}\mathbb{Z}} \times \ldots \times \displaystyle \frac{\mathbb{Z}}{n_{r}\mathbb{Z}}$. Intentarase buscar un antecedente deste elemento mediante o homomorfismo $\phi$ descrito no teorema, i.e. un elemento $x \in \mathbb{Z} \hspace{1mm} | \hspace{1mm} (x_{1} + n_{1}\mathbb{Z}, \ldots, x_{r} + n_{r}\mathbb{Z}) = (x + n_{1}\mathbb{Z}, \ldots, x + n_{r}\mathbb{Z})$.\\

\noindent Tense que $\phi$ é un epimorfismo cando, e só cando, os ideais $n_{1}\mathbb{Z}, \ldots, n_{r}\mathbb{Z}$ son coprimos 2 a 2. Esta condición é equivalente a que exista solución en $\mathbb{Z}$ ao seguinte sistema:

$$
\begin{cases}
x \equiv x_{1} \hspace{2mm} (m\acute{o}d \hspace{1mm} n_{1})\\
x \equiv x_{2} \hspace{2mm} (m\acute{o}d \hspace{1mm} n_{2})\\
\vdots \\
x \equiv x_{r} \hspace{2mm} (m\acute{o}d \hspace{1mm} n_{r})
\end{cases}
$$

\pagebreak

\noindent Véxanse a continuación un par de exemplos:

\begin{enumerate}
    \item O problema que se vía na introdución pode ser formulado mediante o seguinte sistema:
    
    $$
    \begin{cases}
    x \equiv 2 \hspace{2mm} (m\acute{o}d \hspace{1mm} 3)\\
    x \equiv 3 \hspace{2mm} (m\acute{o}d \hspace{1mm} 5)\\
    x \equiv 2 \hspace{2mm} (m\acute{o}d \hspace{1mm} 7)
    \end{cases}
    $$
    
    \item Demostrar que $\displaystyle \frac{\mathbb{Z}}{6\mathbb{Z}} \simeq \displaystyle \frac{\mathbb{Z}}{2\mathbb{Z}} \times \displaystyle \frac{\mathbb{Z}}{3\mathbb{Z}}$.
    
    Defínase o homomorfismo $\phi: \mathbb{Z} \longrightarrow \displaystyle \frac{\mathbb{Z}}{2\mathbb{Z}} \times \displaystyle \frac{\mathbb{Z}}{3\mathbb{Z}}$.\\
    
    $\phi$ é sobrexectivo $\Longleftrightarrow 2\mathbb{Z}, 3\mathbb{Z}$ son coprimos, o cal é certo. Áchese o núcleo de $\phi$:
    $$Ker \hspace{1mm} \phi = 2\mathbb{Z} \cap 3\mathbb{Z} = 6\mathbb{Z}$$
    
    Aplicando o \hyperref[th2.1]{\magbf{1.º Teorema de Isomorfía de aneis}}, cúmprese que $\displaystyle \frac{\mathbb{Z}}{6\mathbb{Z}} \simeq \displaystyle \frac{\mathbb{Z}}{2\mathbb{Z}} \times \displaystyle \frac{\mathbb{Z}}{3\mathbb{Z}}$.
    
\end{enumerate}

\magbf{\section{Dominios}}

\vspace{3mm}

\noindent Recórdese a definición de dominio, que aparece na sección \hyperref[DomCorp]{\magbf{1.4.}}:\\

\noindent \textbf{Definición 2.8}. Un \magbf{dominio} (tamén chamado \magbf{dominio cero} ou \magbf{dominio de integridade}) é un anel \textbf{\textit{conmutativo}} sen divisores de cero propios.\\

\noindent Exemplos de dominios son os aneis de números $(\mathbb{Z},+,\cdot)$, $(\mathbb{Q},+,\cdot)$ $(\mathbb{R},+,\cdot)$ e $(\mathbb{C},+,\cdot)$.\\

\magbf{\subsection{Elementos dun dominio}}

\vspace{5mm}

\noindent Nun dominio pódense atopar os seguintes elementos distinguidos:\\

\noindent \textbf{Definición 2.20.} Sexa $A$ un dominio. Considérense dous elementos $a,b \in A$. Dirase que $b$ é \magbf{múltiplo} de $a$ (ou equivalentemente, que $a$ \magbf{divide a } $b$, e denotarase por \magen{$a | b$}, se existe un elemento $c \in A$ tal que $b = ac$.\\

\begin{center}
\fcolorbox{magenta}{white}{
    $a | b \hspace{2mm}:\Longleftrightarrow \exists \hspace{1mm} c \in A \hspace{1mm} | \hspace{1mm} b = ac$ 
}
\end{center}

\noindent \textbf{Definición 2.21.} Sexa $A$ un dominio e $a,b \in A$. Dirase que $a$ e $b$ están \magbf{asociados}, ou son \magbf{elementos asociados}, se $a|b$ e $b|a$.\\

\begin{center}
\fcolorbox{magenta}{white}{
    $a$ e $b$ están \magbf{asociados}\hspace{2mm}$:\Longleftrightarrow a|b \hspace{1mm} \wedge \hspace{1mm} b|a$ 
}
\end{center}

\noindent \textbf{Observación 2.11}. Dous elementos asociados diferéncianse nunha unidade. En efecto:

\[ 
\left. \begin{array}{r} 
a|b \implies \exists \hspace{1mm} c \in A \hspace{1mm} | \hspace{1mm} b = ac\\[1ex]
b|a \implies \exists \hspace{1mm} d \in A \hspace{1mm} | \hspace{1mm} a = bd\\
\end{array} \right\} 
\implies b = ac = bdc \Longleftrightarrow b(1 - dc) = 0
\]

\vspace{3mm}

\noindent \textbf{Definición 2.22}. Sexa $A$ un dominio e $a$ un elemento \textbf{non nulo} de $A$. $a$ dise \magbf{irreducible} se cumpre:

\begin{enumerate}
    \item $a \notin \mathcal{U}(A)$
    \item $a = bc \implies b \in \mathcal{U}(A)$ ou $c \in \mathcal{U}(A)$\\
\end{enumerate}

\noindent \textbf{Definición 2.23}. Sexa $A$ un dominio e $a$ un elemento \textbf{non nulo} de $A$. Dirase que $a$ é \magbf{primo} se o ideal $(a)$ é primo; equivalentemente, se se verifica:

\begin{enumerate}
    \item $a \notin \mathcal{U}(A)$
    \item $a | bc \implies a | b$ ou $a | c$
\end{enumerate}

\begin{proposition} \label{prop2.12}
Todo elemento primo dun dominio é irreducible.
\end{proposition}

\vspace{2mm}

\noindent \textbf{\textit{\underline{Demostración}}}

\vspace{2mm}

\noindent Supóñase $a\in A$ un elemento primo. Entón, por definición, $a \notin \mathcal{U}(A)$. Véxase a continuación que se cumpre a segunda condición da definición de elemento irreducible.\\

\noindent Considérense $b, c \in A \hspace{1mm} | \hspace{1mm} a = bc$. Entón, en particular, $a | bc$. Tense o seguinte:

\[ 
\left. \begin{array}{r} 
a | bc \implies bc \in (a) \\[1ex]
a \text{ primo } \implies (a) \text{ primo}\\
\end{array} \right\} 
\implies b \in (a) \text{ ou } c \in (a)
\]

\noindent Supóñase, por exemplo, que $b \in (a)$ (para $c$ sería análogo):

\[ 
\left. \begin{array}{r} 
b \in (a) \implies b = ar \\[1ex]
a = bc\\
\end{array} \right\} 
\implies a = arc \Longleftrightarrow a(1 - rc) = 0 \Longleftrightarrow 1 = rc \Longleftrightarrow c \in \mathcal{U}(A) \hspace{5mm} \square
\]

\vspace{3mm}

\noindent \textbf{Observación 2.12}. Cómpre salientar que, tal e como se empregou na demostración anterior, sempre é posible considerar $b,c \in A \hspace{1mm} | \hspace{1mm} a = bc$. En particular, tense que $a = a \cdot 1$.\\

\noindent \textbf{Observación 2.13}. O recíproco da proposición anterior, en xeral, non é certo. Velaquí un contraexemplo:\\

\noindent Sexa o anel $\mathbb{Z}[\sqrt{-5}] = \{a + b\sqrt{
-5} \hspace{1mm} | \hspace{1mm} a,b \in \mathbb{Z}\}$. Defínase a seguinte aplicación:

        \begin{center}
            $N: \mathbb{Z}[\sqrt{-5}] \longrightarrow \displaystyle \mathbb{N}$\\
            \vspace{2mm}
            $\hspace{15mm} a + b\sqrt{-5} \leadsto a^{2} + 5b^{2}$
        \end{center}
        
\noindent Esta aplicación cumpre as seguintes propiedades:

\begin{enumerate}
    \item $N(\alpha\beta) = N(\alpha)N(\beta) \hspace{2mm} \forall \hspace{1mm} \alpha, \beta \in \mathbb{Z}[\sqrt{-5}]$
    \item $N(\alpha) = 1 \Longleftrightarrow \alpha \in \mathcal{U}(\mathbb{Z}[\sqrt{-5}])$
\end{enumerate}

\noindent \textbf{Exercicio}. Demostrar que, efectivamente, $N$ cumpre tales propiedades.\\

\noindent Mediante esta aplicación verase que 3, sendo irreducible en $\mathbb{Z}[\sqrt{-5}]$, non é primo.\\

\noindent Supóñanse $\alpha, \beta \in \mathbb{Z}[\sqrt{-5}] \hspace{1mm} | \hspace{1mm} 3 = \alpha\beta$. Entón: 

$$N(3) = 9 = N(\alpha)N(\beta) \implies N(\alpha) \in \{1,3,9\}$$

\noindent $N(\alpha) = N(\beta) = 3$ non pode darse, porque a ecuación $3 = a^{2} + 5b^{2}$ non ten solución con $a,b \in \mathbb{Z}$. Logo, $N(\alpha) = 1$ ou $N(\beta) = 1$, polo que $\alpha \in \mathcal{U}(\mathbb{Z}[\sqrt{-5}])$ ou $\beta \in \mathcal{U}(\mathbb{Z}[\sqrt{-5}])$. Así, efectivamente, 3 é irreducible.\\

\noindent Non obstante, non é primo. Tense que 3 divide a 6 = $2 \cdot 3 =$ $(1 - \sqrt{-5})(1 + \sqrt{-5})$. Non obstante, $N(1 + \sqrt{-5}) = N(1 - \sqrt{-5}) = 6$, mentres que $N(3) = 9$. Como 9 non divide a 6, 3 non pode dividir a $1 - \sqrt{-5}$ nin a $1 + \sqrt{-5}$.\\

\noindent \textbf{Exercicio}. Demostrar que 2, $1-\sqrt{-5}$ e $1+\sqrt{-5}$ son irreducibles, pero non primos, en $\mathbb{Z}[\sqrt{-5}]$.\pagebreak

\noindent \textbf{Definición 2.24}. Sexa $A$ un dominio. Considérense $a,b \in A$. Defínese o \magbf{máximo común divisor de $a$ e $b$}, denotado por \magen{mcd$(a,b) \equiv (a,b)$ }, como o elemento $d \in A$ que cumpre:

\begin{enumerate}
    \item $d|a$ e $d|b$
    \item Se $c|a$ e $c|b \implies c|d$
\end{enumerate}

\noindent \textbf{Observación 2.14}. En xeral, nun dominio arbitrario, dous elementos non teñen por que posuír un máximo común divisor. A modo de exercicio pódese comprobar, por exemplo, que en $\mathbb{Z}[\sqrt{-5}]$ non existe mcd$(6, 2 + 2\sqrt{-5})$.

\magbf{\subsection{Dominios de factorización única}}

\vspace{5mm}

\noindent \textbf{Definición 2.25}. Sexa $A$ un dominio. Considérese $a \in A$ un elemento \textbf{non nulo} de $A$. Dise que $a$ admite unha \magbf{factorización en elementos irreducibles} se existe unha unidade $u \in \mathcal{U}(A)$ e elementos $p_{1}, \ldots, p_{r}$ irreducibles tales que $a = u \cdot p_{1} \cdot \ldots \cdot p_{r}$.\\

\noindent Tal factorización será única se dadas dúas factorizacións de $a$ en elementos irreducibles:
$$a = u \cdot p_{1} \cdot p_{r} = v \cdot q_{1} \cdot q_{s}$$

\noindent estas cumpren as seguintes condicións:\\

\begin{enumerate}
    \item $r = s$
    \item $p_{i}$ e $q_{\sigma(i)}$ son asociados $\Longleftrightarrow p_{i} = u_{i} \cdot q_{\sigma(i)}, \hspace{2mm} \sigma \in S_{r}$ e $u_{i} \in \mathcal{U}(A) \hspace{2mm} \forall \hspace{1mm} i \in \{1, \ldots, r\}$\\
\end{enumerate}

\noindent \textbf{Definición 2.26}. Sexa $A$ un dominio. Dirase que $A$ é un \magbf{dominio factorial}, ou un \magbf{dominio de factorización única}, se todo elemento non nulo de $A$ admite unha, e só unha, factorización en elementos irreducibles.\\

\noindent En diante, cando se faga mención a un dominio de factorización única, poderá aparecer a abreviatura \magbf{DFU}.\\

\noindent A modo de exemplo, recórdese o anel $\mathbb{Z}[\sqrt{-5}]$. \textbf{Este non é un DFU}: tense que 6 = $2 \cdot 3$ = $(1 + \sqrt{-5})(1 - \sqrt{-5})$, sendo tódolos factores elementos irreducibles en $\mathbb{Z}[\sqrt{-5}]$. Ningún par deses elementos están asociados, logo as factorizacións son diferentes. Así, 6 é un elemento que admite máis dunha factorización en elementos irreducibles.\\

\noindent \textbf{Observación 2.15}. Nun dominio arbitrario, un elemento calquera non ten por que posuír unha factorización en elementos irreducibles.\\

\noindent De tódolos elementos irreducibles anteriores, ningún era primo. Polo tanto, cabe preguntarse se este feito está relacionado coa caracterización dos dominios de factorización única. Pois ben, a resposta é afirmativa, e aparece recollida na seguinte proposición:\\

\begin{proposition} \label{prop2.13}
Sexa $A$ un dominio no que todo elemento non nulo admite unha factorización en elementos irreducibles. Entón, as seguintes afirmacións son equivalentes:
\begin{enumerate}
    \item $A$ é un dominio factorial.
    \item Todo elemento irreducible de $A$ é primo.
\end{enumerate}
\end{proposition}

\vspace{2mm}

\noindent \textbf{\textit{\underline{Demostración}}}

\vspace{2mm}

\noindent \fcolorbox{magenta}{white}{(2) $\implies$ (1)}\\

\noindent Supóñase que todo elemento irreducible de $A$ é primo, e véxase que $A$ é un DFU.\\

\noindent Considérese así $a \in A$ e sexan dúas factorizacións de $a$ en elementos irreducibles, $u \cdot p_{1} \cdot \ldots \cdot p_{r} = v \cdot q_{1} \cdot \ldots \cdot  q_{s}$. Tense o seguinte:

    \[ 
    \left. \begin{array}{r} 
    p_{1} \text{ é primo (por hipótese) } \\[1ex]
    p_{1} \hspace{1mm} | \hspace{1mm} v \cdot q_{1} \cdot \ldots \cdot q_{s}
    \end{array} \right\}
    \implies p_{1} \hspace{1mm} | \hspace{1mm} q_{1} \underset{\underset{q_{1} \text{ irreducible}}{\Uparrow}}{\implies} q_{1} = p_{1} \cdot v_{1}, \hspace{1mm} v_{1} \in \mathcal{U}(A)
    \]
    
\noindent Obsérvese que, en realidade, por definición de elemento primo, $p_{1}$ ten que dividir a algún $q_{i}$, $i \in \{1, \ldots, s\}$. Escolleuse $q_{1}$ unicamente por simplificación de notación.\\

\noindent Con isto, substituíndo $q_{1}$ na segunda factorización, obtense:

$$u \cdot \cancel{p_{1}} \cdot p_{2} \cdot \ldots \cdot p_{r} = v \cdot \cancel{p_{1}} \cdot v_{1} \cdot q_{2} \cdot \ldots \cdot q_{s} \Longleftrightarrow u \cdot p_{2} \cdot \ldots \cdot p_{r} = v' \cdot q_{2} \cdot \ldots \cdot q_{s}$$

\noindent onde $v' = v \cdot v_{1} \in \mathcal{U}(A)$. Repetindo este razoamento para o resto de elementos irreducibles da factorización, obtense que, efectivamente, a factorización de $a$ en elementos irreducibles é única, obtendo así que $A$ é un DFU.\\

\noindent \fcolorbox{magenta}{white}{(1) $\implies$ (2)}\\

\noindent Reciprocamente, supóñase $A$ un DFU, e véxase que todo elemento irreducible en $A$ é primo.\\

\noindent Sexa $p$ un elemento irreducible de $A$. Quérese ver que se $p|ab$, entón $p|a$ ou $p|b$. \\

\noindent Como $p|ab$, entón $\exists \hspace{1mm} c \in A \hspace{1mm} | \hspace{1mm} ab = pc$. Sabendo que $A$ é un DFU, tense:
$$ab = pc \Longleftrightarrow u \cdot p_{1} \cdot \ldots \cdot p_{r} \cdot v \cdot q_{1} \cdot \ldots \cdot q_{s} = p \cdot w \cdot t_{1} \cdot \ldots \cdot t_{l} \hspace{5mm} p_{i}, q_{j}, t_{k} \text{ irreducibles }, u,v,w \in \mathcal{U}(A)$$

\noindent Sendo $p$ irreducible e $A$ un DFU, necesariamente, tense que cumprir que $p = u_{i}p_{i}$ ou $p = v_{j}q_{j}$, con $u_{i}, v_{j} \in \mathcal{U}(A)$, para algún $i \in \{1,\ldots,r\}$ ou para algún $j \in \{1, \ldots, s\}$. Así:

    \[ 
    \left. \begin{array}{r} 
    p = u_{i}p_{i} \\[1ex]
    \text{ou} \\[1ex]
    p = v_{j}q_{j}
    \end{array} \right\}
    \implies 
    \left. \begin{array}{r} 
    p|a\\[1ex]
    \text{ou} \\[1ex]
    p|b
    \end{array} \right\}
    \implies p \text{ é primo} \hspace{2mm}  \square
    \]
    
\vspace{3mm}
    
\noindent Nun dominio de factorización única, todo par de elementos posúe máximo común divisor:\\

\noindent Considérese $A$ un DFU e $a,b \in A$. Sexan as respectivas factorizacións en elementos irreducibles de $a$ e $b$,
$$a = u \cdot p_{1}^{\alpha_{1}} \cdot \ldots \cdot p_{r}^{\alpha_{r}} \cdot q_{1}^{\gamma_{1}} \cdot \ldots \cdot q_{s}^{\gamma_{s}} \hspace{5mm} b = v \cdot p_{1}^{\beta_{1}} \cdot \ldots \cdot p_{r}^{\beta_{r}} \cdot t_{1}^{\varepsilon_{1}} \cdot \ldots \cdot t_{l}^{\varepsilon_{l}}$$

\noindent Tense que mcd$(a,b) = \displaystyle \overset{r}{\underset{i = 1}{\prod}}p_{i}^{\text{mín}\{\alpha_{i},\beta_{i}\}}$

\magbf{\subsection{Dominios de ideais principais}}

\vspace{5mm}

\noindent \textbf{Definición 2.27}. Dirase que un dominio $A$ é un \magbf{dominio de ideais principais} se todo ideal de $A$ é principal.\\

\noindent En diante, cando se faga mención a un dominio de ideais principais, poderá aparecer a abreviatura \magbf{DIP}.\\

\noindent $(\mathbb{Z}, +, \cdot)$ é un claro exemplo dun dominio de ideais principais. Todo ideal de $\mathbb{Z}$ é da forma $n\mathbb{Z} = (n)$.\\

\noindent Unha característica moi importante deste tipo de dominios é que todo DIP resulta ser un DFU. Demostrarase a continuación este resultado, pero para iso, hai que se apoiar en dous lemas previos:\pagebreak

\begin{lemma} \label{lem2.1}
Sexa $A$ un dominio de ideais prinicipais. Todo conxunto non nulo de ideais de $A$ posúe un elemento maximal.
\end{lemma}

\vspace{2mm}

\noindent \textbf{\textit{\underline{Demostración}}}

\vspace{2mm}

\noindent Esta demostración, de xeito análogo á \hyperref[prop2.9]{\magbf{Proposición 2.9}}, baséase na aplicación do \magbf{Lema de Zorn}: \\

\noindent Todo conxunto non nulo de ideais de $A$ está parcialmente ordenado pola relación de inclusión, $\subset$. Hai que demostrar que toda cadea do conxunto considerado está limitada superiormente, para así poder aplicar o \magbf{Lema de Zorn}. Esa proba pódese consultar en \cite{fraleigh}. $\square$

\vspace{3mm}

\begin{lemma}[\textbf{Teorema de Bézout}] \label{lem2.2}
Sexa $A$ un dominio de ideais principais. Sexan $a,b,d \in A$. Entón, verifícase:
\begin{center}
    $d = mcd(a,b) \Longleftrightarrow (d) = (a) + (b)$  
\end{center}
\end{lemma}

\vspace{2mm}

\noindent \textbf{\textit{\underline{Demostración}}}

\vspace{2mm}

\noindent \fcolorbox{magenta}{white}{$\Longrightarrow$/} Supóñase $d$ o máximo común divisor de $a$ e $b$, e véxase que $(d) = (a) + (b)$. Probarase isto mediante a comprobación da dobre inclusión:

\begin{itemize}
    \item $(a) + (b) \subset (d)$
    
    Sendo $d = mcd(a,b)$, en particular $d|a$ e $d|b$. Entón:
    
    \[ 
    \left. \begin{array}{r} 
    d|a \implies (a) \subset (d)  \\[1ex]
    d|b \implies (b) \subset (d)
    \end{array} \right\}
    \implies (a) + (b) \subset (d)
    \]
    
    \item $(d) \subset (a) + (b)$
    
    Sábese que $(a) + (b)$ é un ideal de $A$. Como $A$ é un dominio de ideais principais, tense que cumprir que $\exists \hspace{1mm} c \in A \hspace{1mm} | \hspace{1mm} (c) = (a) + (b)$. Véxase que $c = d$.\\
    
    Da igualdade $(c) = (a) + (b)$ obtense que $a, b \in (c)$. Así:
    
    \[ 
    \left. \begin{array}{r} 
    a \in (c) \implies c|a  \\[1ex]
    b \in (c) \implies c|b
    \end{array} \right\}
    \underset{\underset{d = mcd(a,b)}{\Uparrow}}{\implies} c|d \implies (d) \subset (c)
    \]
    
\end{itemize}

\noindent \fcolorbox{magenta}{white}{$\Longleftarrow$/} Reciprocamente, supóñase que $(d) = (a) + (b)$, e chéguese a que $d = mcd(a,b)$.\\

\noindent En realidade, a proba desta implicación é practicamente idéntica á anterior:\\

\noindent Da igualdade $(d) = (a) + (b)$ sábese:
\begin{align*} %--Úsase para permitir o uso do salto de línea
a \in (d) \implies d|a\\
b \in (d) \implies d|b
\end{align*}

\noindent Sexa agora $c$ outro divisor común de $a$ e $b$. Entón:

    \[ 
    \left. \begin{array}{r} 
    c|a \implies (a) \subset (d)  \\[1ex]
    c|b \implies (b) \subset (d)
    \end{array} \right\}
    \implies (a) + (b) \subset (c)
    \]
    
\noindent Por hipótese, $(d) = (a) + (b)$, polo que se ten $(d) \subset (c)$. Isto implica que $c|d$. Así, tense que $d = mcd(a,b)$, tal e como se pretendía demostrar. $\square$

\pagebreak

\noindent Enunciamos a continuación:\\

\begin{theorem} \label{th2.6}
Sexa $A$ un dominio de ideais prinicipais. Entón, $A$ é un dominio de factorización única.
\end{theorem}

\vspace{2mm}

\noindent \textbf{\textit{\underline{Demostración}}}

\vspace{2mm}

\noindent En primeiro lugar, probarase que todo elemento de $A$ admite unha factorización en elementos irreducibles. Considérese $\Sigma$ o seguinte conxunto:
$$\Sigma = \{(a) \hspace{1mm} | \hspace{1mm} (a \in A-\{0\})\wedge (a \text{ non admite factorización en elementos irreducibles})\}$$

\noindent e véxase que $\Sigma = \varnothing$.\\

\noindent Razoando por redución ao absurdo, supóñase que $\Sigma \neq \varnothing$ . Entón, aplicando o \hyperref[lem2.1]{\magbf{Lema 2.1}}, $\Sigma$ posúe un elemento maximal, que será un ideal $(a)$, con $a \neq 0$, que non factoriza en elementos irreducibles. En consecuencia, $a$ non é un elemento irreducible, polo que se $a = bc$, $b,c \notin \mathcal{U}(A)$.\\

\noindent Algún elemento do conxunto $\{b,c\}$ non factoriza en elementos irreducibles. Supóñase $b$ tal elemento. Entón, tense:

    \[ 
    \left. \begin{array}{r} 
    (b) \in \Sigma \\[1ex]
    b|a \implies (a) \subset (b)
    \end{array} \right\}
    \implies \text{Contradición con que }(a) \text{ é elemento maximal de }\Sigma
    \]
    
\noindent Polo tanto, necesariamente, $\Sigma = \varnothing$. Logo, todo elemento de $A$ admite unha factorización en elementos irreducibles.\\

\noindent A continuación, verase que tal factorización é única. Gracias á \hyperref[prop2.13]{\magbf{Proposición 2.13}}, sábese que isto é equivalente a probar que todo elemento irreducible de $A$ é primo.\\

\noindent Considérese así $p$ un elemento irreducible de $A$. É $p$ primo?\\

\noindent Sexan $a,b \in A \hspace{1mm} | \hspace{1mm} p | ab \Longleftrightarrow pc = ab$, cumprindo $p \nmid a$. Verase que $p|b$.\\

\noindent Sexa $d = $ mcd$(p, a)$. Tense así:

    \[ 
    \left. \begin{array}{r} 
    p \text{ irreducible} \\[1ex]
    d|p \\[1ex]
    \end{array} \right\}
    \implies d \in \mathcal{U}(A) \implies (d) = A
    \]
    
\noindent Polo \hyperref[lem2.2]{\magbf{Lema 2}}, $(d) = (p) + (a)$. Entón, tense que $(a)$ e $(p)$ son ideais coprimos, logo $\exists \hspace{1mm} \alpha, \beta \in A \hspace{1mm} | \hspace{1mm} 1 = \alpha \cdot p + \beta \cdot a$. Tense así:
$$1 = \alpha \cdot p + \beta \cdot a \implies b = \alpha \cdot p \cdot b + \beta \cdot a \cdot b = \alpha \cdot p \cdot b + \beta \cdot p \cdot c = p \cdot (\alpha \cdot b + \beta \cdot c) \implies p|b$$

\noindent Probouse así que $p$ é primo. Logo, $A$ é un dominio de factorización única, tal e como se pretendía demostrar. $\square$

\magbf{\subsection{Dominios euclídeos}}

\vspace{5mm}

\noindent \textbf{Definición 2.28}. Sexa $E$ un dominio. Dirase que $E$ é un \magbf{dominio euclídeo} se existe unha aplicación
$$\Phi : E - \{0\} \longrightarrow \mathbb{N}$$
\noindent que verifique as seguintes condicións:\\

\begin{enumerate}
    \item $a | b \implies \Phi(a) \leq \Phi(b) \hspace{3mm} \forall a,b \in E -\{0\}$
    \item Dados $a,b \in E$, con $b \neq 0$, $\exists \hspace{1mm} q, r \in E \hspace{1mm} | \hspace{1mm} a = bq + r$, cumprindo que $r = 0$ ou $\Phi(r) < \Phi(b)$\\
\end{enumerate}

\noindent En diante, cando se faga mención a un dominio de ideais principais, poderá aparecer a abreviatura \magbf{DE}.\\

\noindent Os seguintes son exemplos de dominios euclídeos:

\begin{enumerate}
    \item $(\mathbb{Z}, |\cdot|)$ é un dominio euclídeo
    \item Sexa $K$ un corpo. Considérese a aplicación:
    
    \begin{center}
        $\delta: K \longrightarrow \mathbb{N}$ \\
        \vspace{2mm}
        $\hspace{13mm} \lambda \leadsto \delta(\lambda) := 1$
    \end{center} 
    
    Cúmprese que $(K, \delta)$ é un dominio euclídeo.
    
    \item O anel dos enteiros gaussianos, $\mathbb{Z}[i]$, é un dominio euclídeo coa aplicación:
    
        \begin{center}
        $\Phi: \mathbb{Z}[i] \longrightarrow \mathbb{N}$ \\
        \vspace{2mm}
        $\hspace{30mm} a + bi \leadsto \Phi(a + bi) := a^{2} + b^{2}$
    \end{center} 
    
\end{enumerate}

\noindent Unha propiedade importante dos dominios euclídeos é a que recolle o seguinte resultado:\\

\begin{theorem} \label{th2.7}
Sexa $E$ un dominio euclídeo. Entón, $E$ é un dominio de ideais principais.
\end{theorem}

\vspace{2mm}

\noindent \textbf{\textit{\underline{Demostración}}}

\vspace{2mm}

\noindent Considérese $\mathcal{I}$ un ideal arbitrario de $E$, e véxase que é un ideal principal.\\

\noindent Se $\mathcal{I} = \{0\}$, este é trivialmente un ideal principal. Supóñase así $\mathcal{I} \neq \{0\}$. Considérese o conxunto $$\Delta = \{\Phi(a) \hspace{1mm} | \hspace{1mm} a \in \mathcal{I} - \{0\}\} \subset \mathbb{N}$$
\noindent sendo $\Phi$ a aplicación que aparece na definición de dominio euclídeo. Cúmprese que este conxunto posúe un minimal $b$, i.e. un elemento $\Phi(b) \in \mathbb{N}$, con $b \in \mathcal{I} - \{0\}$, cumprindo que $\Phi(b) \leq \Phi(a) \hspace{2mm} \forall \hspace{1mm} a \in \mathcal{I}-\{0\}$.\\

\noindent Probarase que $\mathcal{I} = (b)$, i.e. dado calquera $a \in \mathcal{I}$, $a \in (b)$.\\

\noindent Sabendo que $E$ é un dominio euclídeo e que $b \neq 0$, tense que $\exists \hspace{1mm} q,r \in E \hspace{1mm} | \hspace{1mm} a = bq + r$. Distínguense dous casos:

\begin{itemize}
    \item Se $r = 0$, $a = bq$, logo $a \in (b)$
    \item Se $r \neq 0$, tense que cumprir que $\Phi(r) < \Phi(b)$. 
    
    Doutra banda, tense que $r = a - bq$. Como $\mathcal{I}$ é un ideal, $a - bq \in \mathcal{I}$. Achouse así un elemento $r \in \mathcal{I} - \{0\}$ tal que $\Phi(r) < \Phi(b)$, o cal contradí o feito de que $\Phi(b)$ sexa un minimal de $\Delta$. Polo tanto, esta posibilidade non pode darse.
\end{itemize}

\noindent En consecuencia, $a \in (b)$, logo $\mathcal{I} = (b)$. Así, efectivamente, $E$ é un dominio de ideais principais. $\square$

\vspace{3mm}

\noindent \textbf{Observación 2.16}. O recíproco do teorema anterior, en xeral non é certo. O anel $\mathbb{Z}[\frac{1 + \sqrt{-19}}{2}]$ é un dominio de ideais principais, pero non é un dominio euclídeo.\\

\noindent A continuación describirase o cálculo do máximo común divisor de dous elementos dun dominio euclídeo. O procedemento que se emprega para tal fin denomínase \magbf{algoritmo de Euclides}. Antes da explicación do proceso, hai que ter en conta o seguinte:\\

\begin{mdframed}[linecolor = magenta]

\noindent Sexa $(E, \Phi)$ un dominio euclídeo. Considérense $a,b \in E$, con $b \neq 0$. Entón, sábese que $\exists \hspace{1mm} q,r \in E \hspace{1mm} | \hspace{1mm} a = bq + r$, onde $r = 0$ ou $\Phi(r) < \Phi(b)$. \\

\noindent Cúmprese que \magbf{$mcd(a,b) = mcd(b,r)$}. Para comprobalo, probarase que os divisores comúns de $a$ e $b$ son, precisamente, os divisores comúns de $b$ e $r$.

\begin{itemize}
    \item Se $c|a$ e $c|b$, entón $\exists \hspace{1mm} x,y \in E \hspace{1mm} | \hspace{1mm} a = cx, b = cy$. Entón, tense:
    $$r = a - bq \Longleftrightarrow r = cx - cyq = c(x - yq) \implies c|r$$
    
    \item Se $d|b$ e $d|r$, entón $\exists \hspace{1mm} m,n \in E \hspace{1mm} | \hspace{1mm} b = md, r = nd$. Logo:
    $$a = bq + r \Longleftrightarrow a = mdq + nd = d(mq + n) \implies d|a$$
\end{itemize}

\end{mdframed}

\noindent Con isto, descríbese a continuación o algoritmo:\\

\noindent \textbf{Paso 1}. Calcúlanse $q_{1}, r_{1} \hspace{1mm} | \hspace{1mm} a = bq_{1} + r_{1}$. Distínguense dous casos:

\begin{itemize}
    \item  Se $r_{1} = 0$, entón $b|a$ e polo tanto $mcd(a,b) = b$.
    \item Se $r_{1} \neq 0$, entón, pola observación anterior
    $$mcd(a,b) = mcd(b, r_{1})$$
    reducindo así o problema do cálculo do máximo común divisor de $a$ e $b$ ó cálculo do máximo común divisor de $b$ e $r_{1}$. Recórdese que se ten que $0 < \Phi(r) < \Phi(b)$.
    
\end{itemize}

\noindent Facendo $b = r_{0}$, realízanse división sucesivas coas que se constrúe unha cadea de restos como segue:\\

\noindent \textbf{Paso $i$}. Se $i \geq 2$, e $r_{i-1} \neq 0$, entón constrúese $r_{i}$ como o resto da división de $r_{i-2}$ por $r_{i-1}$, obtendo así unha cadea de restos tal que
$$0 \leq \Phi(r_{i}) < \Phi(r_{i-1}) < \ldots < \Phi(r_{1}) < \Phi(r_{0}) = \Phi(b)$$

\noindent O novo resto cumpre unha das dúas seguintes condicións:

\begin{itemize}
    \item Se $r_{i} = 0$, rematou o proceso, pois entón $r_{i-1}|r_{i-2}$ e así $mcd(a,b) = mcd(r_{i-2}, r_{i-1}) = r_{i-1}$.
    \item Se $r_{i} \neq 0$, entón repítese este paso, construíndo $r_{i+1}$ como o resto da división de $r_{i-1}$ e $r_{i}$, obtendo así unha nova peza na cadea de restos:
    $$0 \leq \Phi(r_{i+1}) < \Phi(r_{i}) < \Phi(r_{i-1}) < \ldots < \Phi(r_{1}) < \Phi(r_{0}) = \Phi(b)$$
\end{itemize}

\noindent Este proceso rematará nun número finito de pasos, porque $\mathbb{N}$ é un conxunto que posúe mínimo. De ser $r_{k+1}$ o primeiro resto nulo, teríase que $r_{k}|r_{k-1}$, logo $mcd(a,b) = mcd(r_{k-1}, r_{k}) = r_{k}$.\\

\noindent En xeral, para o desenvolvemento do algoritmo de Euclides, empregarase a seguinte táboa:

\begin{center}
    \begin{tabular}{c|c|c|c|c|c|c|c}
        &  $q_{1}$ & $q_{2}$ & $q_{3}$ & \ldots & $q_{k-1}$ & $q_{k}$ & $\leftarrow$ cocientes\\ \hline
        $a$ & $b$ & $r_{1}$ & $r_{2}$ & \ldots & $r_{k-2}$ & $r_{k-1}$ & $r_{k}$ \hspace{12mm} $\leftarrow$ dividendos/divisores\\ \hline
        $r_{1}$ & $r_{2}$ & $r_{3}$ & \ldots  & \ldots & $r_{k}$ & 0 & \hspace{18mm} $\leftarrow$ restos (con $r_{k+1} = 0$)
    \end{tabular}    
\end{center}

\vspace{7mm}

\noindent Os aspectos principais vistos neste epígrafe recóllense no seguinte cadro:\\

\begin{mdframed}[linecolor = classicrose, linewidth = 1mm]

\noindent \textbf{\underline{Relacións entre os tipos de dominios:}}

\begin{center}
    \magbf{DE $\implies$ DIP $\implies$ DFU}
\end{center}

\vspace{3mm}

\noindent \textbf{\underline{Cálculo do máximo común divisor:}}
\begin{itemize}
    \item \magbf{DFU} $\implies$ Produto dos factores irreducibles comúns elevados ó menor expoñente
    \item \magbf{DIP} $\implies$ Teorema de Bézout
    \item \magbf{DE} $\implies$ Algoritmo de Euclides
\end{itemize}

\end{mdframed}

\magbf{\section{Corpos}} \label{corpos}

\vspace{3mm}

\noindent Recórdese a definición de corpo, que aparece na sección \hyperref[DomCorp]{\magbf{1.4.}}:\\

\noindent \textbf{Definición 2.9}. Un \magbf{corpo} é un anel no que todo elemento non nulo (isto é, distinto do neutro respecto da operación suma) posúe simétrico respecto da operación produto.\\

\noindent Exemplos de corpos son os aneis de números $(\mathbb{Q},+,\cdot)$ $(\mathbb{R},+,\cdot)$ e $(\mathbb{C},+,\cdot)$.\\

\begin{proposition} \label{prop2.14}
Sexa $K$ un corpo. Entón, $K$ é un dominio.
\end{proposition}

\vspace{2mm}

\noindent \textbf{\textit{\underline{Demostración}}}

\vspace{2mm}

\noindent Sendo $K$ un corpo, $\mathcal{U}(K) = K - \{0\}$. \\

\noindent Como xa se comprobou na sección \hyperref[ElAnillo]{\magbf{1.2.}}, unha unidade non pode ser un divisor de cero. Polo tanto en $K$ non existen divisores de cero propios. Así, efectivamente, $K$ é un dominio. $\square$\\

\noindent \textbf{Observación 2.17}. O recíproco da proposición anterior, en xeral, non é certo. Como contraexemplo máis claro tense o anel dos números enteiros $(\mathbb{Z},+,\cdot)$. Este é un dominio, pero non un corpo, pois $\mathcal{U}(\mathbb{Z}) = \{-1,1\}$.\\

\noindent Non obstante, cúmprese o seguinte:\\

\begin{proposition} \label{prop2.15}
Sexa $D$ un dominio enteiro finito. Entón, $D$ é un corpo.
\end{proposition}

\vspace{2mm}

\noindent \textbf{\textit{\underline{Demostración}}}

\vspace{2mm}

\noindent Sendo $D = \{d_{1}, \ldots, d_{n}\}$ un dominio, $\forall \hspace{1mm} a,b \in D-\{0\}$, $ab \neq 0$.\\

\noindent Considérese un elemento $a \neq 0$ de $D$, e véxase que $a \in \mathcal{U}(D)$.\\

\noindent Tense que $ad_{i} = ad_{j} \Longleftrightarrow d_{i} = d_{j}$. En efecto:
$$ad_{i} - ad_{j} = 0 \Longleftrightarrow a(d_{i} - d_{j}) = 0 \underset{\underset{
\substack{
D \text{ dominio} \\ a \neq 0
}
}{\Uparrow}}{\Longleftrightarrow} d_{i} - d_{j} = 0 \Longleftrightarrow d_{i} = d_{j}$$

\noindent Así, o conxunto $A = \{ad_{1}, \ldots, ad_{n}\}$ é bixectivo con $D$. Como $A \subset D$, tense que $A = D$.\\

\noindent Como $D$ é un dominio, en particular é un anel, logo $1 \in D$. Polo tanto, $\exists i \in \{1, \ldots, n\} \hspace{1mm} | \hspace{1mm} ad_{i} = 1$. \\

\noindent Entón, $ad_{i} = 1 \Longleftrightarrow d_{i} = a^{-1}$. En consecuencia, $a \in \mathcal{U}(D)$.\\

\noindent Este razoamento serve para calquera $a \in D-\{0\}$. Polo tanto, $D$ é un corpo, tal e como se quería demostrar. $\square$ \\

\vspace{3mm}

\noindent O seguinte resultado permite caracterizar os corpos segundo os seus ideais:\\

\begin{proposition} \label{prop2.16}
Sexa $A$ un anel conmutativo non nulo. Entón, cúmprese:
\begin{center}
    $A$ é corpo $\Longleftrightarrow$ Os únicos ideais de $A$ son $\{0\}$ e $A$  
\end{center}
\end{proposition}

\vspace{2mm}

\noindent \textbf{\textit{\underline{Demostración}}}

\vspace{2mm}

\noindent \fcolorbox{magenta}{white}{$\Longrightarrow$/} Supóñase $A$ un corpo. Considérese $\mathcal{I}$ un ideal non nulo de $A$. Tense:
$$\mathcal{I} \neq \{0\} \implies \exists \hspace{1mm} x \in \mathcal{I} \hspace{1mm} | \hspace{1mm} x \neq 0 \overset{\overset{A \text{ corpo}}{\Downarrow}}{\implies} \exists \hspace{1mm} x^{-1}$$

Polo tanto:

    \[ 
    \left. \begin{array}{r} 
    \exists \hspace{1mm} x^{-1}  \\[1ex]
    x \in \mathcal{I} \\[1ex]
    \mathcal{I} \text{ ideal}
    \end{array} \right\}
    \implies x^{-1} \cdot x = 1 \in \mathcal{I} \implies A = \mathcal{I}
    \]
    
\noindent \fcolorbox{magenta}{white}{$\Longleftarrow$/} Supóñase agora que os únicos ideais de $A$ son $\{0\}$ e $A$, e véxase que $A$ é un corpo.\\

\noindent Sexa $a \in A, a \neq 0$, e compróbese que $\exists \hspace{1mm} a^{-1}$.\\

\noindent Considérese o ideal principal xerado por $a$, $(a)$, que é non nulo porque $a \neq 0$. Por hipótese, $A = (a)$, polo que $1 \in (a)$. Así, $\exists \hspace{1mm} a' \in A \hspace{1mm} | \hspace{1mm} a'a = aa' = 1$. $\square$

\magbf{\subsection{Característica dun corpo}}

\vspace{5mm}

\noindent Sexa $K$ un corpo. Entón, pódese definir unha aplicación:
    \begin{align*}
        h: \mathbb{Z} & \longrightarrow K \\
        n & \leadsto h(n) := n \cdot 1_{K}
    \end{align*} 
    
\noindent É fácil comprobar que esta aplicación é un homomorfismo de aneis. Sabendo isto, tense que $Ker \hspace{1mm} h$ é un ideal de $\mathbb{Z}$. Distínguense dous casos:\\

\begin{itemize}
    \item $Ker \hspace{1mm} h = \{0\}$ (i.e. $h$ é inxectivo)
    
    Neste caso, aplicando o \hyperref[th2.1]{\magbf{1.º Teorema de Isomorfía de aneis}}, cúmprese que $\mathbb{Z} \simeq h(\mathbb{Z}) \subset K$.\\
    Ademais, pódese estender esta aplicación ao corpo dos números racionais como segue:
    \begin{align*}
        h': \mathbb{Q} & \longrightarrow K\\
        n/m & \leadsto h'(n/m) := n \cdot 1_{K} \cdot (m - 1)^{-1}
    \end{align*} 
    
    Cúmprese que esta extensión é tamén un monomorfismo de aneis. Polo tanto, aplicando de novo o \hyperref[th2.1]{\magbf{1.º Teorema de Isomorfía de aneis}}, $\mathbb{Q} \simeq h'(\mathbb{Q}) \subset K$. Sendo $\mathbb{Q}$ un corpo, $h'(\mathbb{Q})$ tamén o é. Como $K$ é corpo, dirase que $h'(\mathbb{Q})$ é un \magbf{subcorpo} de $K$.\\
    
    Baixo estas condicións, dirase que \magbf{$K$ é un corpo de característica 0}.\\
    \item $Ker \hspace{1mm} h = n\mathbb{Z}$, $n \neq 0$
    
    Nesta situación, necesariamente, $n$ será un número primo. En caso contrario, $\exists \hspace{1mm} p,q \in \mathbb{Z} \hspace{1mm} | \hspace{1mm} n = pq$. Entón, teríase:
    $$0 = n \cdot 1_{K} = h(n) = h(pq) = p \cdot 1_{K} \cdot q \cdot 1_{K}$$
    Sendo $K$ un corpo, en particular é un dominio. Logo, teríase que cumprir que $p \cdot 1_{K} = 0$ ou que $q \cdot 1_{K} = 0$, cousa que non se pode dar porque $p,q \notin n\mathbb{Z}$.\\
    
    Así, $Ker \hspace{1mm} h = p\mathbb{Z}$, sendo $p$ un número primo. Aplicando o \hyperref[th2.1]{\magbf{1.º Teorema de Isomorfía de aneis}}, cúmprese que $\mathbb{Z}/p\mathbb{Z} \simeq h(\mathbb{Z}) \subset K$.\\
    
    Dirase entón que \magbf{$K$ é un corpo de característica $p$}.\\
\end{itemize}

\noindent \textbf{Observación 2.18}. Nótese que, se a característica dun corpo é non nula, esta cumpre ser o menor enteiro positivo tal que $1_{K} + \overset{n}{\dots} + 1_{K} = 0$. De non existir tal enteiro positivo, dise que a característica de $K$ é cero.

\magbf{\section{Aneis de polinomios}}

\magbf{\subsection{Definición e xeneralidades}}

\vspace{5mm}

\noindent \textbf{Definición 2.29}. Sexa $A$ un anel. Un \magbf{polinomio na variable $X$ con coeficientes en $A$} é unha expresión alxébrica da forma
$$a_{0} + a_{1}X + a_{2}X^{2} + \dots + a_{n-1}X^{n-1} + a_{n}X^{n} \hspace{5mm} a_{i} \in A \hspace{2mm} \forall \hspace{1mm} i \in \{0, 1, \ldots, n\}$$

\noindent Alternativamente, a expresión anterior pódese denotar de calquera das seguintes formas:

$$\displaystyle \sum_{i = 0}^{n} a_{i}X^{i} \equiv \displaystyle \sum_{i \in \mathbb{N}} a_{i}X^{i} \hspace{2mm} a_{i} = 0 \hspace{2mm} \almostall i \in \mathbb{N}$$

\noindent O conxunto de tódolos polinomios na variable $X$ con coeficientes no anel $A$ denótase por \magen{$A[X]$}.\\

\noindent Defínanse as seguintes operacións en $A[X]$:\\

 Dados $f = \displaystyle \sum_{i=0}^{m}{a_{i}X^{i}}$, $g = \displaystyle \sum_{i=0}^{n}{b_{i}X^{i}} \in A[X]$:
    $$f + g = \displaystyle \sum_{i = 0}^{\text{máx}\{m,n\}}{(a_{i} + b_{i})X^{i}}$$
    $$f \cdot g = \displaystyle \sum_{i=0}^{m}\sum_{j=0}^{n}(a_{i} \cdot b_{j})X^{i+j}$$
    
\noindent entendendo que:\\
\begin{itemize}
    \item Se máx\{$m,n$\} = $m$, entón $b_{j} = 0 \hspace{2mm} \forall \hspace{1mm} j \in \{n+1, \dots, m\}$
    \item Se máx\{$m,n$\} = $n$, entón $a_{i} = 0 \hspace{2mm} \forall \hspace{1mm} i \in \{m+1, \dots, n\}$\\
\end{itemize}

\noindent O neutro da operación suma é o \magbf{polinomio nulo}, $f = 0$, e o do produto, $g = 1$.\\

\noindent \textbf{Exercicio.} Demostrar que, con estas operacións, $A[X]$ posúe estrutura de anel. A terna $(A[X], +, ·)$ denomínase \magbf{anel de polinomios na variable $X$ con coeficientes en $A$}.\\

\noindent \textbf{Definición 2.30}. Sexa $A$ un anel. Considérese $f \in A[X]$. Defínese o \magbf{grao de $f$}, denotado por \magbf{$gr(f) \equiv \partial f$}, como unha función:\\
\begin{center}
    $\partial: A[X] \longrightarrow \mathbb{N}\cup \{-\infty\}$\\
    \vspace{5mm}
    \hspace{70mm} $f \leadsto \partial f := 
    \begin{cases}
    -\infty & \text{se } f = 0\\
    n & \text{se } f \neq 0 \text{ e } n = \text{ máx}\{i \in \mathbb{N} \hspace{1mm} | \hspace{1mm} a_{i} \neq 0\}
    \end{cases}$\\
\end{center}

\vspace{5mm}

\noindent Por convenio, defínense as seguintes propiedades do grao:
\begin{enumerate}
    \item $-\infty + i = -\infty \hspace{5mm} \forall \hspace{1mm} i \in \mathbb{N}$
    \item $-\infty < i \hspace{5mm} \forall \hspace{1mm} i \in \mathbb{N}$
    \item $-\infty + (-\infty) = -\infty$
\end{enumerate}

\pagebreak

\noindent Se $\partial f = n$:

\begin{itemize}
    \item $a_{n}$ denomínase o \magbf{coeficiente principal} de $f$.
    \item $a_{0}$ denomínase o \magbf{termo independente} de $f$.
    \item $\{a_{i}\}_{i = 0}^{n}$ son os \magbf{coeficientes} de $f$.
    \item Se $a_{n} = 1$, $f$ dise un \magbf{polinomio mónico}.\\
\end{itemize}

\noindent Se $\partial f = 0$, $f$ é un \magbf{polinomio constante}, e é da forma $a_{0} \in A$. Tense así que $A \subset A[X]$.\\

\begin{proposition} \label{prop2.17}
Sexa $A$ un anel conmutativo. Considérense $f$, $g \in A[X]$. Verifícanse as seguintes afirmacións:
\begin{enumerate}
    \item $\partial (f+g) \leq$ máx$\{\partial f, \partial g\}$ 
    \item $\partial (f \cdot g) \leq \partial f + \partial g$. A igualdade dáse no caso de que $A$ sexa un dominio.
\end{enumerate}
\end{proposition}

\vspace{2mm}

\noindent \textbf{\textit{\underline{Demostración}}}

\vspace{2mm}

\noindent Pódese consultar en \cite{hartley}. $\square$\\ 

\begin{proposition} \label{prop2.18}
Sexa $A$ un anel. Entón, cúmprese:
\begin{center}
    $A$ é dominio $ \Longleftrightarrow A[X]$ é dominio  
\end{center}
\end{proposition}

\vspace{2mm}

\noindent \textbf{\textit{\underline{Demostración}}}

\vspace{2mm}

\noindent \fcolorbox{magenta}{white}{$\Longrightarrow$/} Supóñase $A$ un dominio. Considérense $f,g \in A[X] \hspace{1mm} | \hspace{1mm} f \cdot g = 0$. Entón:
$$f \cdot g = 0 \implies \partial (f \cdot g) \underset{\underset{A \text{ dominio}}{\Uparrow}}{=} \partial f + \partial g = \partial 0 = -\infty$$

\noindent Logo, $\partial f + \partial g = -\infty$. Segundo as propiedades do grao, tense que cumprir que $\partial f = -\infty$ ou $\partial g = -\infty$, o cal equivale a que $f = 0$ ou $g = 0$. En consecuencia, $A[X]$ é un dominio.\\

\noindent \fcolorbox{magenta}{white}{$\Longleftarrow$/} Reciprocamente, supóñase que $A[X]$ é un dominio. Recórdese que $A \subset A[X]$. Logo, se $A[X]$ non contén divisores de cero propios, $A$ tampouco. Así, $A$ é un dominio. $\square$\\

\begin{proposition} \label{prop2.19}
Sexa $A$ un dominio. Entón, verifícase:
\begin{center}
    $\mathcal{U}(A) = \mathcal{U}(A[X])$  
\end{center}
\end{proposition}

\vspace{2mm}

\noindent \textbf{\textit{\underline{Demostración}}}

\vspace{2mm}

\noindent \fcolorbox{magenta}{white}{$\subset$/} Isto é evidente, pois $A \subset A[X]$.\\

\noindent \fcolorbox{magenta}{white}{$\supset$/} Sexa $f = \displaystyle \sum_{i = 0}^{n}{a_{i}X^{i}} \in \mathcal{U}(A[X])$. Entón, $\exists \hspace{1mm} g = \displaystyle \sum_{i = 0}^{n}{b_{i}X^{i}} \in A[X] \hspace{1mm} | \hspace{1mm} fg = 1$. Así:

    \[ 
    \left. \begin{array}{r} 
    \partial (fg) \overset{\overset{A \text{ dominio}}{\Downarrow}}{=} \partial f + \partial g  \\[1ex]
    \partial (fg) = \partial 1 = 0
    \end{array} \right\} 
    \implies \partial f + \partial g = 0 \underset{\underset{\partial f, \partial g \in \mathbb{N}}{\Uparrow}}{\implies} 
    \begin{cases}
    \partial f = 0 \implies a_{i} = 0 \hspace{3mm} \forall \hspace{1mm} i > 0\\
    e\\
    \partial g = 0 \implies b_{i} = 0 \hspace{3mm} \forall \hspace{1mm} i > 0
    \end{cases}
    \]
    
\noindent Por todo o anterior, tense que cumprir que $f = a_{0}$ e $g = b_{0}$, obtendo así que $f \cdot g = a_{0} \cdot b_{0} = 1$. Logo, como $a_{0} \in A$, $f = a_{0} \in \mathcal{U}(A)$. $\square$\\

\noindent Todo anel de polinomios verifica a seguinte propiedade universal:\\

\begin{theorem}[\magbf{Propiedade universal do anel de polinomios}] \label{th2.8}
Sexa $A$ un anel. Sexa $\lambda: A \longrightarrow A[X]$ a inclusión de $A$ en $A[X]$. Entón, dado un homomorfismo de aneis $\varphi: A \longrightarrow B$ e un elemento $\alpha \in B$ fixo, existe un único homomorfismo de aneis $\varphi_{\alpha}: A[X] \longrightarrow B$ tal que $\varphi_{\alpha} \circ \lambda = \varphi$.
\end{theorem}

$$
     \large \xymatrix{
        A \ar[dr]_\varphi \ar[r]^{\lambda} & A[X] \ar@{-->}[d]^{\varphi_{\alpha}} \\
          & B
    }    
$$

\vspace{3mm}

\noindent A aplicación $\varphi_{\alpha}$ está definida de tal xeito que cumpre as seguinte propiedades:

\begin{itemize}
    \item $\varphi_{\alpha}(a) = \varphi(a) \hspace{2mm} \forall \hspace{1mm} a \in A$
    \item $\varphi_{\alpha}(X) = \alpha$
    \item $\varphi_{\alpha}\left(\overset{n}{\underset{i = 0}{\sum}}a_{i}X^{i}\right) = \displaystyle \sum_{i = 0}^{n}{\varphi(a_{i})\alpha^{i}} $
\end{itemize}

\noindent \textbf{Observación 2.19}. De cara á aplicación práctica desta propiedade universal, dado un anel $A$, haberá que escoller o anel $B$ co cal se definirá un homomorfismo entre $A$ e $B$. Ademais, tense que definir explicitamente $\varphi$, e demostrar que é un homomorfismo.\\

\noindent Esta propiedade é de especial interese para a construción de isomorfismos.\\

\noindent A continuación amósanse un par de exemplos notables de aplicación desta propiedade universal:

\begin{enumerate}
    \item Sexa $B$ un anel e $A$ un subanel de $B$.\\
    
    Nesta situación, pódese considerar como o homomorfismo $\varphi$ da propiedade universal a inclusión de $A$ en $B$, $i: A \hookrightarrow B$. Tense entón o seguinte diagrama conmutativo:

$$
     \large \xymatrix{
        A \ar@{_{(}->}[dr]_i \ar[r]^{\lambda} & A[X] \ar@{-->}[d]^{\overline{\varphi}} \\
          & B
    }    
$$

\vspace{3mm}

Considérese $f = \displaystyle \sum_{j=0}^{n}{a_{j}X^{j}} \in A[X]$. Dado $\alpha \in B$, tense:
$$\overline{\varphi}(f) = \overline{\varphi}\left(\overset{n}{\underset{j=0}{\sum}}a_{j}X^{j}\right) = \displaystyle \sum_{j=0}^{n}{i(a_{j}) \cdot \alpha^{j}} = \displaystyle \sum_{j=0}^{n}{a_{j}\alpha^{j}} = f(\alpha)$$

O homomorfismo $\overline{\varphi}$ recibe o nome de \magbf{avaliación de $f$ en $\alpha$}.\\

\item Sexa $f: A \longrightarrow B$ un homomorfismo de aneis. Tense o seguinte diagrama:

$$
     \large \xymatrix{
        A \ar[d]_f \ar[r]^{\lambda} & A[X] \ar@{-->}[d]^{\overline{\varphi}} \\
        B \ar[r]_{\lambda '} & B[X]
    }    
$$

onde $\lambda$ e $\lambda '$ simbolizan as respectivas inclusións de $A$ e $B$ nos seus aneis de polinomios.\\

Cúmprese que $\lambda ' \circ f: A \longrightarrow B[X]$ é un homomorfismo de aneis. Logo, segundo a \hyperref[th2.8]{\magbf{Propiedade universal do anel de polinomios}}, existe un único homomorfismo de aneis $\overline{\varphi}: A[X] \longrightarrow B[X]$ que fai conmutativo o diagrama anterior.\\

Dado un polinomio $\displaystyle \sum_{i=0}^{n}{a_{i}X^{i}} \in A[X]$, tense:
$$\overline{\varphi}\left(\underset{i=0}{\overset{n}{\sum}}{a_{i}X^{i}}\right) = \displaystyle \sum_{i=0}^{n}f(a_{i})X^{i}$$

Un caso particular deste exemplo é aquel no que $A = \mathbb{Z}$ e $B = \mathbb{Z}_{q}$, con $q \in \mathbb{Z}$. O diagrama que se presenta neste caso é o seguinte:

$$
     \large \xymatrix{
        \mathbb{Z} \ar[d]_p \ar[r]^{\lambda} & \mathbb{Z}[X] \ar@{-->}[d]^{\overline{\varphi}} \\
     \mathbb{Z}_{q} \ar[r]_{\lambda '} & \mathbb{Z}_{q}[X]
    }    
$$

Dado un polinomio $f = \displaystyle \sum_{i=0}^{n}{a_{i}X^{i}} \in \mathbb{Z}[X]$, tense:

$$\overline{\varphi}\left(\underset{i=0}{\overset{n}{\sum}}{a_{i}X^{i}}\right) = \displaystyle \sum_{i=0}^{n}p(a_{i})X^{i} = \displaystyle \sum_{i=0}^{n}[a_{i}]X^{i} = \displaystyle \sum_{i=0}^{n}(a_{i} + q\mathbb{Z})X^{i}$$    

Neste caso particular, o homomorfismo $\overline{\varphi}$ denomínase \magbf{redución de $f$ módulo $q$}.
    
\end{enumerate}

\magbf{\subsection{Divisibilidade no anel de polinomios}}

\vspace{5mm}

\noindent Todo anel de polinomios verifica o seguinte resultado:\\

\begin{theorem} \label{th2.9}
Sexa $A$ un anel. Considérense $f,g \in A[X]$, de xeito que o coeficiente principal de $g$ sexa unha unidade de $A$. Entón, existen, e son únicos, $q, r \in A[X]$ tales que $f = g \cdot q + r$, cumprindo que $r = 0$ ou $\partial r < \partial g$.
\end{theorem}

\vspace{2mm}

\noindent \textbf{\textit{\underline{Demostración}}}

\vspace{2mm}

\noindent Supóñanse $f = \overset{n}{\underset{i = 0}{\sum}}{a_{i}X^{i}}$ e $g = \overset{d}{\underset{i = 0}{\sum}}{b_{i}X^{i}}$, con $\partial f = n$ e $\partial g = d$.\\

\noindent A proba deste resultado farase por indución no grao de $f$.\\

\noindent Para $n = 0$, distínguese:

\begin{itemize}
    \item $\partial f = \partial g \implies f = a_{0}$, $g = b_{0}$
    
    Por hipótese, o coeficiente principal de $g$ é unha unidade. Logo, sendo $g = b_{0}$, cúmprese que $\exists \hspace{1mm} b_{0}^{-1}$. Así, tense:
    $$a_{0} = b_{0} \cdot b_{0}^{-1} \cdot a_{0} + 0$$
    polo que abonda tomar $q = b_{0}^{-1} \cdot a_{0}$ e $r = 0$.
    
    \item $\partial f < \partial g \implies f = g \cdot 0 + f$
    
    Así, basta escoller $q = 0$ e $r = f$, cumpríndose que $\partial r < \partial g$
    
\end{itemize}

\noindent Supóñase agora certa a afirmación para os polinomios de grao 1, 2, $\ldots, n-1$, e véxase que tamén se cumpre para $n$.\\

\noindent Considérese o polinomio $f_{1} = f - a_{n} \cdot b_{d}^{-1} \cdot x^{n-d} \cdot g$. Obsérvase que $n-1 = \partial f_{1} < \partial f = n$.\\

\noindent Por \textbf{hipótese de indución}, sábese que $\exists \hspace{1mm} q_{1}, r_{1} \in A[X] \hspace{1mm} | \hspace{1mm} f_{1} = gq_{1} + r_{1}$, con $r_{1} = 0$ ou $\partial r_{1} < \partial g$. Tense:
$$f = f_{1} + a_{n}b_{d}^{-1}x^{n-d}g = gq_{1} + r_{1} + a_{n}b_{d}^{-1}x^{n-d}g = g(q_{1} + a_{n}b_{d}^{-1}x^{n-d}) + r_{1}$$

\noindent Así, pódese tomar $q = q_{1} + a_{n}b_{d}^{-1}x^{n-d}$ e $r = r_{1}$, cumpríndose igualmente que $r = 0$ ou $\partial r < \partial g$.\\

\noindent Con isto, demostrouse a \textbf{existencia} de tales $q$ e $r$. Por que se ten a \textbf{unicidade}?\\

\noindent Supóñanse $q_{1}, q_{2}, r_{1}, r_{2} \hspace{1mm} | \hspace{1mm} f = gq_{1} + r_{1} = gq_{2} + r_{2} \Longleftrightarrow g(q_{1} - q_{2}) = r_{2} - r_{1}$. \\

\noindent Tense así que $\partial (g(q_{1} - q_{2})) = \partial (r_{2} - r_{1})$. Sendo o coeficiente principal de $g$ unha unidade, en particular non pode ser un divisor de cero. Logo:
$$\partial (g(q_{1} - q_{2})) = \partial g + \partial (q_{1} - q_{2}) = \partial (r_{2} - r_{1}) \leq m\acute{a}x \{\partial r_{2}, \partial r_{1}\} < \partial g$$

\noindent Entón, en resumo, chegouse ao seguinte:
$$\partial g + \partial (q_{1} - q_{2}) < \partial g \implies \partial (q_{1} - q_{2}) = -\infty \implies q_{1} - q_{2} = 0 \Longleftrightarrow q_{1} = q_{2}$$

\noindent Como $q_{1} - q_{2} = 0$, obtense ademais que $r_{2} - r_{1} = 0 \Longleftrightarrow r_{2} = r_{1}$, quedando probada así a unicidade. $\square$ \\

\noindent Nótese a semellanza do enunciado deste teorema cunha das condicións da definición de dominio euclídeo. Tendo isto en conta, como consecuencia inmediata deste resultado tense:\\

\begin{corollary} \label{cor2.5}
Sexa $K$ un corpo. Entón, $(K[X], \partial)$ é un dominio euclídeo.
\end{corollary}

\vspace{2mm}

\noindent \textbf{\textit{\underline{Demostración}}}

\vspace{2mm}

\noindent Dunha banda, sábese que o grao é unha función que lle asigna un número natural a todo polinomio distinto de cero.\\

\noindent Por outra parte, sendo $K$ un corpo, todo elemento de $K-\{0\}$ é unha unidade. Polo tanto, o teorema anterior é aplicable a calquera par de polinomios non nulos, pois os seus coeficientes principais sempre serán unidades. $\square$\\

\noindent \textbf{Observación 2.20}. Aínda que o corolario anterior garante que $K[X]$ cumpre a definición de dominio euclídeo, en realidade enuncia algo máis forte: que o cociente e o resto son \textit{\underline{únicos}}. Isto, en xeral, non ten por que se cumprir nun dominio euclídeo arbitrario. \\

\noindent A modo de exemplo, tómese o dominio euclídeo dos enteiros gaussianos, $(\mathbb{Z}[i], \Phi)$, con $\Phi(a + bi) = a^{2} + b^{2}$.\\

\noindent Sexan $a = 3 - 2i$ e $b = 5 + i$. Cúmprese:

\begin{itemize}
    \item $a = bq_{1} + r_{1}$, con $q_{1} = 1 - i$ e $q_{2} = -3 + 2i$. Verifícase que $13 = \Phi(r_{1}) < \Phi(b) = 26$
    \item $a = bq_{2} + r_{2}$, con $q_{2} = -i$ e $r_{2} = 2 + 3i$. Verifícase que $13 = \Phi(r_{2}) < \Phi(b) = 26$
\end{itemize}

\vspace{3mm}

\begin{theorem}[\magbf{Teorema do resto}] \label{th2.10}
Sexa $A$ un dominio. Considérese $f \in A[X]$ non nulo e $\alpha \in A$. Cúmprese que o resto da división de $f$ entre (X - $\alpha$) é o valor numérico de $f$ en $\alpha$, $f(\alpha)$.
\end{theorem}

\vspace{2mm}

\noindent \textbf{\textit{\underline{Demostración}}}

\vspace{2mm}

\noindent Sendo o coeficiente principal de $(x-\alpha)$ unha unidade de $A$, segundo o \hyperref[th2.9]{\magbf{Teorema 9}}, $\exists \hspace{1mm} q,r \in A[X] \hspace{1mm} | \hspace{1mm} f = (x-\alpha)q + r$, con $r = 0$ ou $\partial r < 1$ (porque o grao de $(x-\alpha)$ é 1). Nestas condicións, garántese que $r$ é un polinomio constante e, polo tanto, un elemento de $A$.\\

\noindent Calcúlese $f(\alpha)$:
$$f(\alpha) = (\alpha - \alpha)q + r = 0 + r = r \hspace{6mm} \square\\$$

\vspace{3mm}

\noindent \textbf{Definición 2.31}. Sexa $A$ un anel. Considérense $f \in A[X]$ e $\alpha \in A$. Dirase que $\alpha$ é \magbf{raíz de $f$} se o valor númerico de $f$ en $\alpha$ é cero, $f(\alpha) = 0$.\\

\begin{corollary}[\magbf{Teorema do factor}] \label{cor2.6}
Sexa $A$ un dominio. Considérense $f \in A[X]$ e $\alpha \in A$. Entón, verifícase:
\begin{center}
    $\alpha$ é raíz $\Longleftrightarrow (x-\alpha)$ divide a $f$
\end{center}
\end{corollary}

\vspace{2mm}

\noindent \textbf{\textit{\underline{Demostración}}}

\vspace{2mm}

\noindent É o caso particular do \hyperref[th2.10]{\magbf{Teorema do resto}} no cal $f(\alpha) = 0$. $\square$\\

\vspace{3mm}

\noindent O seguinte resultado é unha xeneralización do \hyperref[cor2.6]{\magbf{Teorema do factor}}:\\

\begin{proposition} \label{prop2.20}
Sexa $A$ un dominio. Considérese $f$ un polinomio non nulo de $A[X]$, e sexan $\alpha_{1}, \ldots, \alpha_{r}$ raíces de $f$. Entón, verifícase que f é divisible por $\overset{r}{\underset{i = 1}{\Pi}}{(X - \alpha_{i})}$ .
\end{proposition}

\vspace{2mm}

\noindent \textbf{\textit{\underline{Demostración}}}

\vspace{2mm}

\noindent Realizarase un razoamento indutivo sobre o número de raíces, $r$.\\

\noindent O caso $r = 1$ está demostrado no \hyperref[cor2.6]{\magbf{Teorema do factor}}. Supóñase certo o resultado para $r-1$ raíces, e véxase que se cumpre tamén para $r$.\\

\noindent Por \textbf{hipótese de indución}, $\overset{r-1}{\underset{i = 1}{\Pi}}(X-\alpha_{i})$ divide a $f$, podendo escribir $f = g \cdot \overset{r-1}{\underset{i = 1}{\Pi}}(X-\alpha_{i})$, con $g \in A[X]$.\\

\noindent Sábese que $\alpha_{r}$ é raíz de $f$. Logo, $f(\alpha_{r}) = 0$. Tense entón:
$$0 = f(\alpha_{r}) = g(\alpha_{r}) \cdot \overset{r-1}{\underset{i = 1}{\Pi}}(\alpha_{r}-\alpha_{i})$$

\noindent Como $\alpha_{r} \neq \alpha_{i} \hspace{2mm} \forall \hspace{1mm} i \in \{1, \ldots, r-1\}$, $\overset{r-1}{\underset{i = 1}{\Pi}}(\alpha_{r}-\alpha_{i}) \neq 0$. Así, sendo $A$ un dominio, tense que cumprir que $g(\alpha_{r}) = 0$.\\

\noindent Aplicando o \hyperref[th2.10]{\magbf{Teorema do resto}}, obtense que $g$ é divisible por $(X - \alpha_{r})$. Así:
$$f = g \cdot \overset{r-1}{\underset{i = 1}{\Pi}}(X-\alpha_{i}) = h \cdot (X - \alpha_{r}) \cdot \overset{r-1}{\underset{i = 1}{\Pi}}(X-\alpha_{i}) = h \cdot \overset{r}{\underset{i = 1}{\Pi}}(X-\alpha_{i}) \hspace{6mm} \square$$

\begin{proposition} \label{prop2.21}
Sexa $A$ un dominio. Considérese $f \in A[X]$, con $\partial f = n$. Verifícanse as seguintes afirmacións:
\begin{enumerate}
    \item O número de raíces de $f$ é menor ou igual ca $n$
    \item Se $\alpha_{1}, \ldots, \alpha_{n}$ son raíces de $f$, entón $f = c \cdot \overset{n}{\underset{i = 1}{\Pi}}(X - \alpha_{i})$, con $c$ o coeficiente principal de $f$.
\end{enumerate}
\end{proposition}

\pagebreak

\noindent \textbf{\textit{\underline{Demostración}}}

\vspace{2mm}

\noindent \magbf{(1)} Supóñanse $\alpha_{1}, \ldots, \alpha_{r}$ raíces distintas de $f$. Entón, segundo o \hyperref[cor2.6]{\magbf{Teorema do factor}}, $f = g \cdot \overset{r}{\underset{i = 1}{\Pi}}(X-\alpha_{i})$. Aplicando a \hyperref[prop2.17]{\magbf{Proposición 17}}, obtense:
$$n = \partial f = \partial ( g \cdot \overset{r}{\underset{i = 1}{\Pi}}(X-\alpha_{i})) \underset{\underset{A \text{ dominio}}{\Uparrow}}{=} \partial g + r \implies r \leq n$$

\noindent \magbf{(2)} Sendo $\partial f = n$ e $A$ un dominio, escribindo $f = g \cdot \overset{n}{\underset{i = 1}{\Pi}}(X-\alpha_{i})$, tense que cumprir que $\partial g = 0$, polo que $g \in A$. Así, obtense que $g$ é o coeficiente prinicipal de $f$. $\square$\\

\noindent \textbf{Observación 2.21}. A hipótese de que $A$ sexa dominio é fundamental para a proposición. Noutro caso, o resultado enunciado non ten por que ser certo.\\

\noindent Como exemplo desta situación, considérese o anel de polinomios $\mathbb{Z}_{6}[X]$, o cal non é un dominio porque $\mathbb{Z}_{6}$ tampouco o é.\\

\noindent Sexa en $\mathbb{Z}_{6}[X]$ o polinomio $f = [3]X + [3]$, o cal cumpre que $\partial f = 1$. Tense que este polinomio posúe tres raíces: [1], [3] e [5].

\magbf{\subsection{Irreducibilidade de polinomios}}

\vspace{5mm}

\noindent Neste apartado comezarase estudando certos resultados sobre a irreducibilidade nos aneis de polinomios da forma $K[X]$, onde $K$ é un corpo, focalizando e particularizando os exemplos para os corpos máis coñecidos: $\mathbb{R}$ e $\mathbb{C}$. Posteriormente enunciaranse algúns criterios de irreducibilidade para polinomios con coeficientes no anel dos números enteiros.\\

\noindent Sábese que, sendo $K$ un corpo, $(K[X], \partial)$ é un dominio euclídeo (en particular, un dominio). Entón, $\mathcal{U}(K[X]) = \mathcal{U}(K) = K - \{0\}$. Polo tanto, dado $f \in K[X]$, cúmprese que $f \in \mathcal{U}(K[X]) \Longleftrightarrow \partial f = 0$.\\

\noindent Dado $f \in K[X]$, este será \magbf{irreducible sobre $K[X]$} se cumpre:\\

\begin{enumerate}
    \item $f \notin \mathcal{U}(K[X]) \Longleftrightarrow \partial f \neq 0$
    \item Se $f = gh \implies 
    \begin{cases}
    g \in \mathcal{U}(K[X]) \Longleftrightarrow \partial g = 0\\
    ou\\
    h \in \mathcal{U}(K[X]) \Longleftrightarrow \partial h = 0
    \end{cases}$
\end{enumerate}

\vspace{2mm}

\begin{proposition} \label{prop2.22}
Sexa $K$ un corpo. Considérese $f \in K[X]$, con $f \neq 0$. Verifícanse as seguintes afirmacións:
\begin{enumerate}
    \item Se $f$ é linear (i.e. $\partial f = 1)$, entón $f$ é irreducible e, ademais, posúe unha raíz en $K$.
    \item Se $\partial f \in \{2, 3\}$, tense:
    \begin{center}
        $f$ é irreducible $\Longleftrightarrow f$ non posúe raíces en $K$
    \end{center}
\end{enumerate}
\end{proposition}

\vspace{2mm}

\noindent \textbf{\textit{\underline{Demostración}}}

\vspace{2mm}

\noindent \magbf{(1)} Se $f$ é linear, entón é da forma $f = aX + b$, con $a \neq 0$. Logo, $\exists \hspace{1mm} a^{-1}$. Así, tense:
$$aX + b = 0 \Longleftrightarrow X = -a^{-1}b$$

\noindent Demostrouse así que $\alpha = -a^{-1}b \in K$ é raíz de $f$. Véxase agora que é irreducible:
    \[ 
    \left. \begin{array}{r} 
    f = gh  \\[1ex]
    \partial f = 1
    \end{array} \right\} 
    \underset{\underset{K[X] \text{ dominio}}{\Uparrow}}{\implies} 1 = \partial f + \partial g \implies
    \begin{cases}
    \partial g = 0 \Longleftrightarrow g \in \mathcal{U}(K[X])\\
    ou\\
    \partial h = 0 \Longleftrightarrow h \in \mathcal{U}(K[X])
    \end{cases}
    \Longleftrightarrow f \text{ é irreducible}
    \]
    
\vspace{2mm}

\noindent \magbf{(2)} As probas de ámbalas dúas implicacións realizaranse mediante un razoamento por paso ao contrarrecíproco.\\

\noindent \fcolorbox{magenta}{white}{$\Longrightarrow$/} Supóñase $\alpha \in K$ unha raíz de $f$. Entón, aplicando o \hyperref[cor2.6]{\magbf{Teorema do factor}}, $f = (X - \alpha) \cdot g $.\\

\noindent Sendo $K[X]$ un dominio, cúmprese que $\partial f = \partial ((X-\alpha) \cdot g) = 1 + \partial g$. Por hipótese, $\partial f \in \{2,3\}$; polo tanto, tense que $\partial g \in \{1,2\}$.\\

\noindent Así, en calquera caso, $g \notin \mathcal{U}(K[X])$, porque o seu grao sempre será maior ca cero. Como $(X - \alpha) \notin \mathcal{U}(K[X])$, $f$ ten que ser reducible.\\

\noindent \fcolorbox{magenta}{white}{$\Longleftarrow$/} Sexa $f \in K[X]$ un polinomio reducible. Entón, se $f = gh$, cúmprese que $g,h \notin \mathcal{U}(K[X]) \Longleftrightarrow \partial g, \partial h \geq 1$. Sabendo que $K[X]$ é un dominio, e que $\partial f \in \{2,3\}$, existen dúas posibilidades:
\begin{itemize}
    \item $\partial g = 1, \partial h = 2 \overset{\overset{\magbf{(1)}}{\Downarrow}}{\implies} g$ ten raíz en $K$
    \item $\partial g = 2, \partial h = 1 \underset{\underset{\magbf{(1)}}{\Uparrow}}{\implies} h$ ten raíz en $K$
\end{itemize}

\noindent Como $h|f$ e $g|f$, tódalas raíces de $g$ e $h$ son tamén raíces de $f$. Así, en calquera caso, $f$ posúe raíces en $K$. $\square$ \\

\vspace{2mm}

\begin{proposition} \label{prop2.23}
Sexa $K$ un corpo. Os seguintes enunciados son equivalentes:
\begin{enumerate}
    \item Todo polinomio irreducible de $K[X]$ é linear.
    \item Todo polinomio de $K[X]$ con grao maior ou igual ca 1 posúe unha raíz en $K$. Se $K$ cumpre esta condición, dirase que $K$ é un corpo \textbf{alxebricamente pechado}.
\end{enumerate}
\end{proposition}

\vspace{2mm}

\noindent \textbf{\textit{\underline{Demostración}}}

\vspace{2mm}

\noindent \fcolorbox{magenta}{white}{(1) $\implies$ (2)}\\

\noindent Sexa $f \in K[X]$ un polinomio de grao $n \geq 1$.\\

\noindent Sendo $K$ un corpo, $(K[X], \partial)$ é un dominio euclídeo; en particular, é un dominio de factorización única. Entón, pódese escribir $f = \overset{n}{\underset{i = 1}{\prod}}f_{i}$. Como $f_{i}$ é irreducible $\forall \hspace{1mm} i \in \{1, \ldots, n\}$, por hipótese, $f_{i}$ é linear. Así, segundo a \hyperref[prop2.22]{\magbf{Proposición 2.22}}, $f_{i}$ admite unha raíz $\alpha_{i} \in K$. En consecuencia, $f$ posúe raíces en $K$.\\

\noindent \fcolorbox{magenta}{white}{(2) $\implies$ (1)}\\

\noindent O caso $\partial f = 1$ xa foi demostrado na \hyperref[prop2.22]{\magbf{Proposición 2.22}}.\\

\noindent Razoando por paso ao contrarrecíproco, supóñase $f \in K[X]$, con $\partial f > 1$. Entón, por hipótese, $f$ posúe unha raíz $\alpha \in K$. Aplicando o \hyperref[cor2.6]{\magbf{Teorema do factor}}, pódese escribir $f = (X - \alpha) \cdot g$.\\ 

\noindent Sendo $\partial f > 1$ e $\partial (X - \alpha) = 1$, cúmprese que $\partial g \geq 1$. Así, $g \notin \mathcal{U}(K[X])$. Como $(X-\alpha)$ tampouco é unha unidade, $f$ é un polinomio reducible. $\square$\\

\pagebreak

\noindent O exemplo máis claro de corpo alxebricamente pechado é o dos números complexos, $\mathbb{C}$. \\

\noindent Considérese $f \in \mathbb{C}[X]$, con $\partial f = n \geq 1$. Entón, pódese escribir:
$$f = c \cdot \overset{n}{\underset{i = 1}{\prod}}(X - \alpha_{i})$$

\noindent sendo $c$ o coeficiente principal de $f$ e $\alpha_{1}, \ldots, \alpha_{n} \in \mathbb{C}$ as raíces de $f$. Verifícase que $f$ é irreducible $\Longleftrightarrow \partial f = 1$.\\

\noindent Pénsese agora no corpo dos números reais, $\mathbb{R}$. Este corpo \textbf{non é alxebricamente pechado}. Por exemplo, os polinomios $f = X^{2} + 1$ e $g = 3X^{2} + X + 7$ son de grao 2, pero son irreducibles sobre $\mathbb{R}$ porque non posúen raíces reais.\\

\noindent Nótese que, como $\mathbb{R} \subset \mathbb{C}$, entón $\mathbb{R}[X] \subset \mathbb{C}[X]$. En particular, calquera polinomio $f \in \mathbb{R}[X]$ pertence a $\mathbb{C}[X]$. Logo, pódese escribir 
$$f = c \cdot \overset{n}{\underset{i = 1}{\prod}}(X - \alpha_{i})$$

\noindent con $\alpha_{i} \in \mathbb{C} \hspace{2mm} \forall \hspace{1mm} i \in \{1, \ldots, n\}$, sendo esta a súa factorización en elementos irreducibles \magen{sobre $\mathbb{C}[X]$}. O noso seguinte obxectivo é achar a factorización en elementos irreducibles de $f$ sobre $\mathbb{R}[X]$ (que sabemos que existe, porque ao ser $\mathbb{R}$ un corpo, $(\mathbb{R}[X], \partial)$ é un dominio euclídeo e, en particular, un dominio de factorización única).\\ 

\noindent Sexa así $f = \overset{n}{\underset{i = 0}{\sum}}a_{i}X^{i} \in \mathbb{R}[X]$. Dado $\alpha \in \mathbb{C}$, se $\alpha$ é raíz de $f$, entón o seu conxugado, $\Bar{\alpha}$, tamén o é. En efecto, considérese o seguinte isomorfismo de aneis:
\begin{center}
    $\sigma: \mathbb{C} \longrightarrow \mathbb{C}$\\
    $\hspace{15mm} z \leadsto \sigma(z) := \Bar{z}$
\end{center}

\noindent Este isomorfismo deixa fixos os números reais, pois o conxugado dun número real é el mesmo.\\

\noindent Sendo $\alpha$ raíz de $f$, cúmprese:
$$f(\alpha) = \sum_{i=0}^{n}{a_{i}\alpha^{i}} = 0 $$

\noindent Aplicando $\sigma$ sobre $f(\alpha)$, obtense:
$$\sigma(f(\alpha)) = \sigma\left(\sum_{i=0}^{n}{a_{i}\alpha^{i}}\right) = \sum_{i=0}^{n}{a_{i}\bar{\alpha}^{i}}$$

\noindent Sendo $\sigma$ un homomorfismo de aneis, verifica que $\sigma(0) = 0$. Así, como $\displaystyle \sum_{i=0}^{n}{a_{i}\alpha^{i}} = 0$, tense que $\sigma\left(\displaystyle \sum_{i=0}^{n}{a_{i}\alpha^{i}}\right) = \displaystyle \sum_{i=0}^{n}{a_{i}\bar{\alpha}^{i}} = 0$, obtendo que $\bar{\alpha}$ é tamén raíz de $f$.\\

\noindent Sexan así $\beta_{1}, \ldots, \beta_{s},\gamma_{1}, \ldots, \gamma_{r}$ raíces de $f$, con $\beta_{i} \in \mathbb{R} \hspace{2mm} \forall \hspace{1mm} i \in \{1, \ldots, s\}$, $\gamma_{j} \in \mathbb{C} \hspace{2mm} \forall \hspace{1mm} j \in \{1, \ldots, r\}$.\\

\noindent A factorización en elementos irreducibles de $f$ sobre $\mathbb{C}[X]$ é:
$$f = c \cdot (X - \beta_{1}) \cdot \ldots \cdot (X - \beta_{s}) \cdot (X - \gamma_{1}) \cdot \ldots \cdot (X - \gamma_{r})$$

\noindent con $c \in \mathbb{R}$ o coeficiente principal de $f$.\\

\noindent Agora ben, sábese que, como $\gamma_{j} \in \mathbb{C}$, $\exists \hspace{1mm} i \in \{1, \ldots, j-1, j+1, \ldots, r\} \hspace{1mm} | \hspace{1mm} \gamma_{i} = \bar{\gamma_{j}}$. Entón, se $\gamma_{j} = a + bi$, cúmprese:
\begin{itemize}
    \item $\gamma_{j} + \bar{\gamma_{j}} = (a + bi) + (a - bi) = 2a \implies \gamma_{j} + \bar{\gamma_{j}} \in \mathbb{R}$
    \item $\gamma_{j} \cdot \bar{\gamma_{j}} = (a + bi) \cdot (a - bi) = a^{2} + b^{2} \implies \gamma_{j} \cdot \bar{\gamma_{j}} \in \mathbb{R}$
\end{itemize}

\pagebreak

\noindent Polo tanto, $(X - \gamma_{j}) \cdot (X - \bar{\gamma_{j}}) = X^{2} + (\gamma_{j} + \bar{\gamma_{j}})X + \gamma_{j} \cdot \bar{\gamma_{j}} \in \mathbb{R}[X]$, o cal é un polinomio irreducible porque non posúe raíces reais.\\

\noindent Así, sobre $\mathbb{R}[X]$, $f$ factoriza en elementos irreducibles como segue:
$$f = c \cdot \prod_{i = 1}^{s}(X - \beta_{s}) \cdot \prod_{j = 1}^{r/2}(X^{2} + (\gamma_{j} + \bar{\gamma_{j}})X + \gamma_{j} \cdot \bar{\gamma_{j}})$$

\noindent A modo de resumo, dado $f \in \mathbb{R}[X]$, sendo as súas raíces reais $\beta_{1}, \ldots, \beta_{s}$ e as súas raíces complexas $\gamma_{1}, \ldots, \gamma_{r}$, cúmprese:

\begin{mdframed}[linecolor = classicrose, linewidth = 1mm]
\noindent \magbf{Factorización sobre $\mathbb{C}[X]$}:
$$f = c \cdot (X - \beta_{1}) \cdot \ldots \cdot (X - \beta_{s}) \cdot (X - \gamma_{1}) \cdot \ldots \cdot (X - \gamma_{r})$$
\noindent \magbf{Factorización sobre $\mathbb{R}[X]$}:
$$f = c \cdot \prod_{i = 1}^{s}(X - \beta_{s}) \cdot \prod_{j = 1}^{r/2}(X^{2} + (\gamma_{j} + \bar{\gamma_{j}})X + \gamma_{j} \cdot \bar{\gamma_{j}})$$
\end{mdframed}

\noindent Ademais, isto permítenos establecer unha caracterización dos polinomios irreducibles sobre $\mathbb{R}[X]$:
\begin{center}
    $f$ é irreducible sobre $\mathbb{R}[X] \Longleftrightarrow \begin{cases}
    \partial f = 1\\
    ou\\
    (\partial f = 2)\wedge(f \text{ non ten raíces en } \mathbb{R})
    \end{cases}$
\end{center}

\vspace{5mm}

\noindent A continuación, centrarémonos nos polinomios de $\mathbb{Z}[X]$, e na súa irreducibilidade sobre $\mathbb{Z}[X]$ e $\mathbb{Q[X]}$.\\

\vspace{2mm}

\begin{proposition}\label{prop2.24}
Sexa $f \in \mathbb{Z}[X]$, con $\partial f \geq 1$. Verifícanse as seguintes afirmacións:
\begin{enumerate}
    \item Se $f$ é irreducible sobre $\mathbb{Z}[X]$, entón tamén o é sobre $\mathbb{Q}[X]$.
    \item Se $f$ é irreducible sobre $\mathbb{Q}[X]$ e, ademais, $f$ é \textbf{primitivo}, entón $f$ é irreducible sobre $\mathbb{Z}[X]$.
\end{enumerate} 
\end{proposition}

\vspace{2mm}

\noindent \textbf{\textit{\underline{Demostración}}}

\vspace{2mm}

\noindent A proba deste resultado queda como exercicio proposto. $\square$\\

\vspace{3mm}

\noindent \textbf{Definición 2.32}. Sexa $f \in \mathbb{Z}[X]$. Dise que $f$ é \magbf{primitivo} se o máximo común divisor dos seus coeficientes é igual a 1.\\

\begin{proposition} \label{prop2.25}
Sexa $f = \overset{n}{\underset{i = 0}{\sum}}a_{i}X^{i} \in \mathbb{Z}[X]$. Considérese $\alpha = r/s \in \mathbb{Q}$ unha raíz de $f$, con mcd(r,s) = 1. Entón, cúmprese que $r|a_{0}$ e $s|a_{n}$.
\end{proposition}

\vspace{2mm}

\noindent \textbf{\textit{\underline{Demostración}}}

\vspace{2mm}

\noindent Déixase como exercicio. $\square$\\

\pagebreak

\noindent A continuación, enúncianse algúns \magbf{criterios de irreducibilidade} aplicables a polinomios con coeficientes en $\mathbb{Z}$.\\

\begin{proposition}[\textbf{Criterio de Eisenstein}] \label{prop2.26}
Sexa $f = \overset{n}{\underset{i = 0}{\sum}}a_{i}X^{i} \in \mathbb{Z}[X]$, con $\partial f = n \geq 1$. Se existe un número primo $p \in \mathbb{Z}$ tal que:
\begin{itemize}
    \item $p|a_{i} \hspace{2mm} \forall i \in \{0, \ldots, n-1\}$
    \item $p \nmid a_{n}$
    \item $p^{2} \nmid a_{0}$
\end{itemize}
\noindent Entón, $f$ é irreducible en $\mathbb{Q}[X]$.
\end{proposition}

\vspace{2mm}

\noindent \textbf{\textit{\underline{Demostración}}}

\vspace{2mm}

\noindent Pódese consultar en \cite{fraleigh}. $\square$\\

\vspace{3mm}

\begin{proposition}[\magbf{Criterio de Eisenstein invertido}] \label{prop2.27}
Sexa $f = \overset{n}{\underset{i = 0}{\sum}}a_{i}X^{i} \in \mathbb{Z}[X]$, con $\partial f = n \geq 1$. Se existe un número primo $p \in \mathbb{Z}$ tal que:
\begin{itemize}
    \item $p|a_{i} \hspace{2mm} \forall i \in \{1, \ldots, n\}$
    \item $p \nmid a_{0}$
    \item $p^{2} \nmid a_{n}$
\end{itemize}
\noindent Entón, $f$ é irreducible en $\mathbb{Q}[X]$.
\end{proposition}

\vspace{2mm}

\noindent \textbf{\textit{\underline{Demostración}}}

\vspace{2mm}

\noindent Pódese realizar adaptando a proba do \hyperref[prop2.26]{\magbf{Criterio de Eisenstein}}. $\square$\\

\vspace{3mm}

\begin{proposition}[\magbf{Criterio de redución}] \label{prop2.28}
Sexa $f = \overset{n}{\underset{i = 0}{\sum}}a_{i}X^{i} \in \mathbb{Z}[X]$, con $\partial f = n \geq 1$. Se existe un número primo $p \in \mathbb{Z}$ tal que:
\begin{itemize}
    \item $p \nmid a_{n}$
    \item A redución de $f$ módulo $p$ é irreducible en $\mathbb{Z}_{p}[X]$
\end{itemize}
\noindent Entón, $f$ é irreducible en $\mathbb{Q}[X]$.
\end{proposition}

\vspace{2mm}

\noindent \textbf{\textit{\underline{Demostración}}}

\vspace{2mm}

\noindent Pódese consultar en \cite{gamboa}. $\square$\\

\vspace{3mm}

\noindent \textbf{Definición 2.33}. Sexa $p \in \mathbb{Z}$. Defínese o \magbf{polinomio ciclotómico de orde $p$} (ou \magbf{$p$-ésimo}) como o polinomio $f \in \mathbb{Z}[X]$ da forma:
$$f = X^{p-1} + X^{p-2} + \ldots + X + 1$$

\noindent \textbf{Exercicio}. Demostrar que este polinomio é irreducible en $\mathbb{Z}[X]$ e $ \mathbb{Q}[X]$.\\

\noindent \textbf{Observación 2.22}. En $\mathbb{Z}[X]$ e $\mathbb{Q}[X]$ existen polinomios irreducibles \textit{de calquera grao}. Como exemplo, o polinomio $f = X^{n} + p$, com $p \in \mathbb{Z}$ primo, é irreducible en $\mathbb{Q}[X]$ (abonda aplicarlle o \hyperref[prop2.26]{\magbf{Criterio de Eisenstein}}). Sendo ademais un polinomio primitivo, tense que é irreducible en $\mathbb{Z}[X]$.\\

\noindent \textbf{Exercicio}. Considérese o polinomio de coeficientes enteiros $f = X^{4} + 3X^{3} + 4X^{2} + X -3$. Achar se é un polinomio irreducible sobre $\mathbb{Z}[X]$.\\

\noindent \textbf{Exercicio}. Utilizando os criterios de irreducibilidade, estudar se os seguintes polinomios son irreducibles en $\mathbb{Q}[X]$:
\begin{enumerate}
    \item $X^{2} - p$, con $p$ primo
    \item $X^{4} - 10X + 5$
    \item $25X^{5} - 9X^{4} + 3X^{2} - 12$
    \item $15X^{5} - 9X^{3} + 6X^{2} - 2$
    \item $4X^{3} - X^{2} + 7$
\end{enumerate}

\noindent \textbf{Exercicio}. Considérese o polinomio de coeficientes enteiros $f = X^{4} - 1$. Estudar se é irreducible sobre $\mathbb{Q}[X]$.\\

\noindent \textbf{Exercicio}. Atopar a factorización en elementos irreducibles dos seguintes polinomios nos aneis pedidos:
\begin{enumerate}
    \item $7X^{4} - 28$ en $\mathbb{Q}[X]$, $\mathbb{R}[X]$ e $\mathbb{C}[X]$.
    \item $X^{4} - X^{3} + X^{2} + 2$ en $\mathbb{R}[X]$ e $\mathbb{C}[X]$
\end{enumerate}

\newpage

\chapter[Unidade 3. Módulos]{\textbf{Módulos}}

\thispagestyle{noheader}

\noindent Un módulo non é máis cá xeneralización dun espazo vectorial: esta estrutura defínese sobre un corpo, que como acabamos de estudar é un tipo particular de anel; un módulo está definido sobre un anel arbitrario. Como comprobaremos inmediatamente, a definición é completamente análoga.\\

\noindent Foi Richard Dedekind quen empregou por primeira vez este termo no 1871, nos seus traballos dentro da teoría de números. Cómpre puntualizar que o definiu como un subgrupo do grupo aditivo dos números complexos (o que nós chamaremos $\mathbb{Z}$-módulo). Máis tarde, no 1894, desenvolveu unha teoría de tales módulos.\\

\noindent Emmy Noether foi a primeira en usar a noción de módulo de xeito abstracto, cun anel como dominio de operadores, e tamén foi a primeira en descubrir o seu potencial. De feito, gracias ao seu traballo, o concepto de módulo converteuse no concepto central da álxebra que hoxe coñecemos.\\

\vspace{3mm}

\magbf{\section{Módulos e submódulos}}

\magbf{\subsection{Módulos: xeneralidades}}

\vspace{5mm}

\noindent \textbf{Definición 3.1}. Sexa $(A, +, \cdot)$ un anel conmutativo. Un \magbf{$A$-módulo} é un grupo abeliano $(M, +)$ no cal está definida unha operación externa, habitualmente denominada \magbf{produto por un escalar}:\\

\begin{center}
    $A \times M \longrightarrow M$\\
    \vspace{2mm}
    $\hspace{2mm} (a, x) \leadsto ax$
\end{center}

\noindent a cal verifica as seguintes condicións:

\renewcommand{\theenumi}{\roman{enumi})}
\renewcommand{\labelenumi}{\theenumi}

\begin{enumerate}
    \item $(a + b)x = ax + bx \hspace{3mm} \forall \hspace{1mm} a,b \in A \hspace{2mm} \forall \hspace{1mm} x \in M$
    \item $a(x + y) = ax + ay \hspace{3mm} \forall \hspace{1mm} a \in A \hspace{2mm} \forall \hspace{1mm} x,y \in M$
    \item $a(bx) = (ab)x \hspace{3mm} \forall \hspace{1mm} a,b \in A \hspace{2mm} \forall \hspace{1mm} x \in M$
    \item $1x = x \hspace{3mm} \forall \hspace{1mm} x \in M$
\end{enumerate}

\vspace{3mm}

\noindent Exemplos moi sinxelos de módulos son os seguintes:

\renewcommand{\theenumi}{\arabic{enumi}}
\renewcommand{\labelenumi}{\theenumi.}

\begin{enumerate}
    \item Se $A$ é un corpo, todo $A$-módulo é un \magbf{espazo vectorial sobre $A$}.
    \item O anel $A$ é un $A$-módulo co seu produto:
    \begin{center}
        $\cdot: A \times A \longrightarrow A$\\
        \vspace{2mm}
        $\hspace{5mm} (a,b) \leadsto a \cdot b$
    \end{center}
\end{enumerate}

\pagebreak

\noindent Da definición de módulo extráese unha serie de \textbf{propiedades elementais} que esta estrutura debe cumprir:

\renewcommand{\theenumi}{\roman{enumi})}
\renewcommand{\labelenumi}{\theenumi}

\begin{enumerate}
    \item $0x = 0 \hspace{2mm} \forall \hspace{1mm} x \in M$
    \item $a0 = 0 \hspace{2mm} \forall \hspace{1mm} a \in A$
    \item $(-a)x = a(-x) = -(ax) \hspace{2mm} \forall \hspace{1mm} a \in A \hspace{2mm} x \in M$
\end{enumerate}

\vspace{2mm}

\noindent \textbf{\textit{\underline{Demostración}}}

\vspace{2mm}

\begin{enumerate}
    \item  $0x = (0 + 0)x = 0x + 0x \implies 0x = 0$
    \item $a0 = a(0 + 0) = a0 + a0 \implies a0 = 0$
    \item $(-a)x + ax = (-a + a)x = 0x = 0 \implies -(ax) = (-a)x$ \par
    \vspace{1mm}
    $a(-x) + ax = a(-x + x) = a0 = 0 \implies -(ax) = a(-x)$
\end{enumerate}

\renewcommand{\theenumi}{\arabic{enumi}}
\renewcommand{\labelenumi}{\theenumi.}

\vspace{3mm}

\noindent \textbf{Observación 3.1}. Nun $A$-módulo $M$ pode ocorrer que $ax = 0$ para certo $a \in A, a \neq 0$ e certo $x \in M, x \neq 0$. En particular, esta situación pódese dar cando $M = A$ é un anel que non sexa dominio. Esta é unha notable diferenza con respecto aos espazos vectoriais. En efecto, un espazo vectorial está definido sobre un corpo, e todo corpo é un dominio.\\

\noindent Amósase a continuación outro exemplo de módulo:

\begin{enumerate}
    \setcounter{enumi}{2}
    \item Sexa $(M, +)$ un grupo abeliano. Entón, $(M,+)$ é un $\mathbb{Z}$-módulo coa seguinte operación externa:\\
    \begin{center}
        $\mathbb{Z} \times M \longrightarrow M$\\
        \vspace{2mm}
        $\hspace{55mm} (n,x) \leadsto nx : = 
        \begin{cases}
        x + \overset{n}{\dots} + x \hspace{16mm} n > 0\\
        0 \hspace{31mm} n = 0\\
        (-x) + \overset{-n}{\dots} + (-x) \hspace{5mm} n < 0
        \end{cases}$
    \end{center}
\end{enumerate}

\noindent Esta é a única estrutura de $\mathbb{Z}$-módulo posible para $M$. En efecto:\\

\noindent Supóñase $M$ un $\mathbb{Z}$-módulo arbitrario. Dados $n \in \mathbb{Z}$ e $x \in M$, distínguense os seguintes casos:
\begin{itemize}
    \item $n > 0 \implies nx = (1 + \overset{n}{\dots} + 1)x = 1x + \overset{n}{\dots} + 1x = x + \overset{n}{\dots} + x$
    \item $n = 0 \implies nx = 0x = 0$
    \item $n < 0 \implies nx = (-n)(-x) = 1 + \overset{-n}{\dots} + 1)(-x) = (1(-x) + \overset{-n}{\dots} + 1(-x) = (-x) + \overset{-n}{\dots} + (-x)$
\end{itemize}

\magbf{\subsection{Submódulos}}

\vspace{5mm}

\noindent \textbf{Definición 3.2}. Sexa $A$ un anel, $M$ un $A$-módulo e $N \subset M$. Dise que $N$ é un \magbf{$A$-submódulo de $M$} Se $N$ posúe estrutura de $A$-módulo coa operación externa definida en $M$.\\

\noindent Equivalentemente, dise que $N$ é $A$-submódulo de $M$ se $N$ é un subgrupo de $M$ coa suma e é pechado para o produto por escalares.\\

\noindent Véxanse algúns exemplos para ilustrar esta definición:\\

\begin{enumerate}
    \item Dado un $A$-módulo $M$ arbitrario, $\{0\}$ e $M$ son $A$-submódulos de $M$.
    \item Considérese $A$ como $A$-módulo. Dado $\mathcal{I} \subset A$, cúmprese:
    \begin{center}
        $\mathcal{I}$ é $A$-submódulo de $A \Longleftrightarrow \mathcal{I}$ é un ideal de $A$ 
    \end{center}
    \item Sexa $M$ un $A$-módulo e $\mathcal{I}$ un ideal de $A$. Considérese o seguinte subconxunto de $M$:
    $$\mathcal{I}M := \{\displaystyle \sum_{i = 1}^{n}a_{i}x_{i} \hspace{1mm} | \hspace{1mm} a_{i} \in \mathcal{I}, x_{i} \in M, n \in \mathbb{N}\}$$
    Tense que $\mathcal{I}M$ é un $A$-submódulo de $M$.
\end{enumerate}

\vspace{3mm}

\noindent \textbf{Exercicio}. Demostrar os exemplos 2 e 3.\\

\begin{proposition} \label{prop3.1}
Sexan $A$ un anel e $M$ un $A$-módulo. Considérese unha familia de $A$-submódulos de $M$, $\{M_{i}\}_{i \in I}$. Cúmprese que a súa intersección, $\underset{i \in I}{\bigcap}M_{i}$, é tamén un $A$-submódulo de $M$.
\end{proposition}

\vspace{2mm}

\noindent \textbf{\textit{\underline{Demostración}}}

\vspace{2mm}

\noindent Déixase como exercicio. $\square$\\

\vspace{3mm}

\noindent \textbf{Definición 3.3}. Sexa $M$ un $A$-módulo e $X$ un subconxunto de $M$. Chámaselle \magbf{$A$-submódulo de $M$ xerado por $X$} á intersección de tódolos $A$-submódulos de $M$ que conteñen a $X$. Denótase por \magbf{$\langle X \rangle \equiv \langle X \rangle_{A}$}.
\begin{center}
    \fcolorbox{magenta}{white}{$\langle X \rangle := \displaystyle \underset{N \supset X}{\bigcap} N$ \hspace{4mm} con $N$ $A$-submódulo de $M$}
\end{center}

\noindent Dirase, ademais, que $X$ é un \magbf{conxunto de xeradores} de $\langle X \rangle$.\\

\begin{proposition} \label{prop3.2}
Sexan $A$ un anel e $M$ un $A$-módulo. Considérese un subconxunto $X \subset M$. Verifícase que $\langle X \rangle$ é o menor $A$-submódulo de $M$ que contén a $X$.
\end{proposition}

\vspace{2mm}

\noindent \textbf{\textit{\underline{Demostración}}}

\vspace{2mm}

\noindent É inmediata a partir da definición de submódulo xerado por $X$. Dado $N$ un submódulo arbitrario de $M$ tal que $X \subset N$, tense:

    \[ 
    \left. \begin{array}{r} 
    X \subset N  \\[1ex]
    N \text{ submódulo de } M
    \end{array} \right\} 
    \underset{\underset{\text{Def } \langle X \rangle}{\Uparrow}}{\implies} \langle X \rangle \subset N \hspace{5mm} \square
    \]
  
\begin{proposition} \label{prop3.3}
Sexan $A$ un anel e $M$ un $A$-módulo. Considérese un subconxunto $X \subset M$. Verifícase que $\langle X \rangle$ é o conxunto das combinacións lineares de elementos de $X$, i.e. 
\begin{center}
$\langle X \rangle = \{\displaystyle \sum_{i = 1}^{n}a_{i}x_{i} \hspace{1mm} | \hspace{1mm} a_{i} \in A, x_{i} \in X, n \in \mathbb{N}\}$
\end{center}
\end{proposition} 

\vspace{2mm}

\noindent \textbf{\textit{\underline{Demostración}}}

\vspace{2mm}

\noindent Asúmase a notación $\Omega = \{\displaystyle \sum_{i = 1}^{n}a_{i}x_{i} \hspace{1mm} | \hspace{1mm} a_{i} \in A, x_{i} \in X, n \in \mathbb{N}\}$. \\

\noindent En primeiro lugar, probarase que $\Omega$ é un $A$-submódulo de $M$. Pola súa propia definición, é un conxunto pechado para a suma. Ademais, contén ao seu neutro, pois calquera combinación linear de elementos de $X$ cuxos coeficientes sexan 0 será igual a 0. Así, tense que $(\Omega, +)$ é un grupo abeliano, como subgrupo de $(M,+)$.\\

\noindent Véxase a continuación que $\Omega$ é pechado para o produto por escalares. En efecto, dados $\lambda \in A$, $\overset{n}{\underset{i = 1}{\sum}}a_{i}x_{i} \in \Omega$, tense:
$$\lambda \cdot \overset{n}{\underset{i = 1}{\sum}}a_{i}x_{i} \underset{\underset{M \text{ módulo}}{\Uparrow}}{=} \overset{n}{\underset{i = 1}{\sum}}\lambda a_{i}x_{i} \underset{\underset{M \text{ módulo}}{\Uparrow}}{=} \overset{n}{\underset{i = 1}{\sum}}(\lambda \cdot a_{i})x_{i}$$

\noindent Así, chegouse a que, efectivamente, $\Omega$ é un submódulo de $M$. Pero... é este o menor submódulo que contén a $X$?\\

\noindent Cómpre decatarse de que, trivialmente, $X \subset \Omega$, pois todo elemento de $X$ pode ser considerado como combinación linear de elementos de $X$, sen máis ca tomar $x = 1x \hspace{2mm} \forall \hspace{1mm} x \in X$.\\

\noindent Sexa agora $\overset{n}{\underset{i = 1}{\sum}}a_{i}x_{i} \in \Omega$. Para cada $i \in \{1, \ldots, n\}$, tense:

    \[ 
    \left. \begin{array}{r} 
    x_{i} \in X \\[1ex]
    a_{i} \in A
    \end{array} \right\} 
    \underset{\underset{\langle X \rangle \text{ submódulo}}{\Uparrow}}{\implies} a_{i}x_{i} \in \langle X \rangle \underset{\underset{\langle X \rangle \text{ submódulo}}{\Uparrow}}{\implies} \sum_{i = 1}^{n}a_{i}x_{i} \in \langle X \rangle \implies \Omega \subset \langle X \rangle
    \]

\noindent Así, sendo $\langle X \rangle$ o menor submódulo que contén a $X$, necesariamente, $\Omega = \langle X \rangle$. $\square$\\

\vspace{3mm}

\begin{proposition} \label{prop3.4}
Sexan $A$ un anel e $M$ un $A$-módulo. Considérese $\mathcal{I}$ un ideal de $A$. Tense:
\begin{center}
 $\mathcal{I}M = \langle ax \hspace{1mm} | \hspace{1mm} a \in \mathcal{I}, x \in M \rangle$   
\end{center}
\end{proposition}

\vspace{2mm}

\noindent \textbf{\textit{\underline{Demostración}}}

\vspace{2mm}

\noindent Asúmase a notación $\Delta = \langle ax \hspace{1mm} | \hspace{1mm} a \in \mathcal{I}, x \in M \rangle$. Probarase a igualdade por dobre inclusión:\\

\noindent \fcolorbox{magenta}{white}{$\subset$/} Sexa $\delta \in \mathcal{I}M$. Entón,  sábese que $\delta = \overset{n}{\underset{i = 1}{\sum}}a_{i}x_{i}$, con $a_{i} \in \mathcal{I}, x_{i} \in M \hspace{2mm} \forall \hspace{1mm} i \in \{1, \ldots, n\}$. Logo, pola \hyperref[prop3.3]{\magbf{Proposición 3}}, $\delta \in \Delta$.\\

\noindent \fcolorbox{magenta}{white}{$\supset$/} É inmediato: dado calquera xerador $ax \in \Delta$, por definición, este pertence a $\mathcal{I}M$, porque $a \in \mathcal{I}$ e $x \in M$. $\square$\\

\vspace{3mm}

\noindent \textbf{Observación 3.2}. Se $X$ é un conxunto unitario (i.e. que se reduce a un só elemento, $X = \{x\}$), cúmprese:
$$\langle X \rangle = Ax = \{ax \hspace{1mm} | \hspace{1mm} a \in A\}$$
Nótese que, en xeral:
$$\langle X \rangle = \displaystyle \sum_{x \in X} \langle x \rangle$$

\magbf{\subsection{Módulo cociente}}

\vspace{5mm}

\noindent Sexa $M$ un $A$-módulo e $N$ un $A$-submódulo de $M$. Sendo $M$ $A$-módulo, en particular está dotado da estrutura de grupo abeliano coa suma; polo tanto, todo subgrupo seu é normal (en particular, $N$). Así, sábese que o conxunto cociente $M/N$ está dotado de estrutura de grupo (abeliano, por ser $M$ un grupo abeliano).\\

\noindent Recórdese que:
$$(x + N) + (y + N) = (x + y) + N \hspace{5mm} \forall \hspace{1mm} x,y \in M$$

\vspace{3mm}

\noindent Defínase a seguinte operación externa:

\begin{center}
    $A \times \displaystyle \frac{M}{N} \longrightarrow \displaystyle \frac{M}{N}$\\
    \vspace{2mm}
    $\hspace{25mm} (a, x + N) \leadsto a(x + N) := ax + N$
\end{center}

\noindent Esta operación está ben definida. En efecto:
$$x + N = x' + N \implies x - x' \in N \underset{\underset{N \text{ submódulo}}{\Uparrow}}{\implies} a(x - x') = ax - ax' \in N \implies ax + N = ax' + N$$

\noindent Esta operación verifica os axiomas da definición de módulo:\\

\renewcommand{\theenumi}{\roman{enumi})}
\renewcommand{\labelenumi}{\theenumi}

\begin{enumerate}
    \item $(a + b)(x + N) = a(x + N) + b(x + N) \hspace{4mm} \forall \hspace{1mm} a,b \in A \hspace{2mm} \forall \hspace{1mm} x + N \in M/N$
    $$(a + b)(x + N) = [(a + b)x] + N \overset{\overset{M \text{ módulo}}{\Downarrow}}{=} (ax + bx) + N = (ax + N) + (bx + N) = a(x + N) + b(x + N)$$
    \item $a[(x + N) + (y + N)] = a(x + N) + a(y + N) \hspace{4mm} \forall \hspace{1mm} a \in A \hspace{2mm} \forall \hspace{1mm} x + N, y + N \in M/N$
    $$a[(x + N) + (y + N)] = a[(x + y) + N] = [a(x + y)] + N \overset{\overset{M \text{ módulo}}{\Downarrow}}{=} (ax + ay) + N = (ax + N) + (ay + N) = a(x + N) + a(y + N)$$
    \item $a[b(x + N)] = (ab)(x + N) \hspace{4mm} \forall \hspace{1mm} a,b \in A \hspace{2mm} \forall \hspace{1mm} x + N \in M/N$
    $$a[b(x + N)] = a(bx + N) = a(bx) + N \overset{\overset{M \text{ módulo}}{\Downarrow}}{=} (ab)x + N = (ab)(x + N)$$
    \item $1(x + N) = x + N \hspace{4mm} \forall \hspace{1mm} x + N \in M/N$
    $$1(x + N) = 1x + N \overset{\overset{M \text{ módulo}}{\Downarrow}}{=} x + N$$
\end{enumerate}

\noindent Deste xeito demostrouse que, dotado con esta operación externa, o grupo abeliano $\left( \displaystyle \frac{M}{N}, + \right)$ posúe estrutura de $A$-módulo, denominado \magbf{módulo cociente}.\\

\noindent \textbf{Observación 3.3}. Tense que $\displaystyle \frac{M}{N} = \{0\} \Longleftrightarrow M = N$.\\

\magbf{\subsection{Suma de submódulos}}

\vspace{5mm} 

\noindent \textbf{Definición 3.4}. Sexa $A$ un anel. Considérese $M$ un $A$-módulo e $M_{1}$, $M_{2}$ $A$-submódulos de $M$. Defínese a \magbf{suma dos submódulos $M_{1}$ e $M_{2}$}, que será denotada por \magbf{$M_{1} + M_{2}$} como o seguinte subconxunto de $M$:

\begin{center}
    \fcolorbox{magenta}{white}{
    $M_{1} + M_{2} := \{x_{1} + x_{2} \hspace{1mm} | \hspace{1mm} x_{1} \in M_{1}, x_{2} \in M_{2}\}$
    }
\end{center}

\begin{proposition} \label{prop3.5}
Sexan $A$ un anel, $M$ un $A$-módulo e $M_{1}, M_{2}$ $A$-submódulos de $M$. Tense que a suma de $M_{1}$ e $M_{2}$, $M_{1} + M_{2}$, é un $A$-submódulo de $M$.
\end{proposition}

\vspace{2mm}

\noindent \textbf{\textit{\underline{Demostración}}}

\vspace{2mm}

\noindent Sendo $M_{1}$ e $M_{2}$ $A$-submódulos, $0 \in M_{1}, M_{2}$, logo $0 = 0 + 0 \in M_{1} + M_{2}$. Ademais, pola súa propia definición, $M_{1} + M_{2}$ é pechado para a suma. Entón, $(M_{1} + M_{2}, +)$ é un subgrupo de $(M, +)$.\\

\noindent Verase a continuación que é pechado para o produto por escalares. En efecto, por ser $M_{1}$ e $M_{2}$ $A$-submódulos de $M$, son pechados para o produto por escalares. Así, dados $\lambda \in A$, $x_{1} \in M_{1}$, $x_{2} \in M_{2}$, tense:
$$\lambda(x_{1} +  x_{2}) = \underbrace{\lambda x_{1}}_{\in M_{1}} + \underbrace{\lambda x_{2}}_{\in M_{2}} \in M_{1} + M_{2} \hspace{4mm} \square$$

\vspace{3mm}

\noindent \textbf{Definición 3.5}. Sexa $A$ un anel e $M$ un $A$-módulo. Considérese unha familia arbitraria de submódulos de $M$, $\{M_{i}\}_{i \in I}$. Defínese a \magbf{suma dos submódulos $M_{i}$} como o seguinte conxunto:

\begin{center}
    \fcolorbox{magenta}{white}{
    $\displaystyle \sum_{i \in I} M_{i} := \{\sum_{i \in I} x_{i} \hspace{1mm} | \hspace{1mm} x_{i} \in M_{i} \hspace{2mm} \forall \hspace{1mm} i \in I, \hspace{2mm} x_{i} = 0 \hspace{2mm} \almostall i \in I\}$
    }
\end{center}

\pagebreak

\begin{proposition} \label{prop3.6}
Sexan $A$ un anel, $M$ un $A$-módulo e $\{M_{i}\}_{i \in I}$ unha familia arbitraria de $A$-submódulos de $M$. Tense que a suma dos submódulos $M_{i}$, $\underset{i \in I}{\sum}M_{i}$, é un $A$-submódulo de $M$.
\end{proposition}

\vspace{2mm}

\noindent \textbf{\textit{\underline{Demostración}}}

\vspace{2mm}

\noindent Déixase como exercicio. $\square$\\

\vspace{3mm}

\noindent \textbf{Observación 3.4}. Dado un módulo $M$ e unha familia arbitraria de submódulos de $M$, $\{M_{i}\}_{i \in I}$, a suma dos submódulos desa familia cumpre ser o menor submódulo que contén a tódolos $M_{i}$.\\

\noindent \textbf{Observación 3.5}. Dado un módulo $M$ e unha familia arbitraria de submódulos de $M$, $\{M_{i}\}_{i \in I}$, cúmprese o seguinte:
$$\sum_{i \in I}M_{i} = \langle \bigcup_{i \in I}M_{i} \rangle$$

\magbf{\section{Homomorfismos de módulos}}

\magbf{\subsection{Definición e exemplos}}

\vspace{5mm}

\noindent \textbf{Definición 3.6}. Sexa $A$ un anel. Considérense $M$ e $N$ dous $A$-módulos entre os cales se establece unha aplicación $f: M \longrightarrow N$. Tal aplicación $f$ dirase un \magbf{homomorfismo de $A$-módulos} se verifica as seguintes condicións:\\

\renewcommand{\theenumi}{\roman{enumi})}
\renewcommand{\labelenumi}{\theenumi}

\begin{enumerate}
    \item $f(x + y) = f(x) + f(y) \hspace{3mm} \forall \hspace{1mm} x,y \in M$
    \item $f(ax) = af(x) \hspace{3mm} \forall \hspace{1mm} a \in A \hspace{2mm} \forall x \in M$
\end{enumerate}

\vspace{3mm}

\noindent Nótese que, polo feito de ser $f$, en particular, un homomorfismo de grupos (abelianos), tense:

\begin{enumerate}
    \item $f(0_{M}) = 0_{N}$
    \item $f(-x) = -f(x) \hspace{2mm} \forall \hspace{1mm} x \in M$
\end{enumerate}

\renewcommand{\theenumi}{\arabic{enumi}}
\renewcommand{\labelenumi}{\theenumi.}

\vspace{3mm}

\noindent Exemplos de homomorfismos son os seguintes:

\begin{enumerate}
    \item Dados dous módulos $M$ e $N$ sobre o mesmo anel, a aplicación
    \begin{center}
        $f: M \longrightarrow N$\\
        \vspace{2mm}
        $\hspace{15mm} x \leadsto f(x) := 0_{N}$
    \end{center}
    é un homomorfismo de módulos, denominado \magbf{homomorfismo nulo} ou \magbf{homomorfismo cero}.
    \item Dado un módulo $M$, a aplicación identidade:
    \begin{center}
        $id_{M} \equiv 1_{M}: M \longrightarrow M$\\
        \vspace{2mm}
        $\hspace{30mm} x \leadsto id_{M}(x) := x$
    \end{center}
    é un homomorfismo de módulos.
    \item Dado un módulo $M$ e un submódulo seu $N$, a inclusión natural de $N$ en $M$:
    \begin{center}
        $i: N \hookrightarrow M$\\
        \vspace{2mm}
        $\hspace{15mm} x \leadsto i(x) := x$
    \end{center}
    é un homomorfismo de módulos.
    \item Dado un módulo $M$ e un submódulo seu $N$, a aplicación:
    \begin{center}
        $p: M \longrightarrow \displaystyle \frac{M}{N}$\\
        \vspace{2mm}
        $\hspace{20mm} x \longrightarrow p(x) := x + N$
    \end{center}
    é un homomorfismo de módulos, denominado \magbf{proxección canónica de $M$ sobre $\displaystyle \frac{M}{N}$}.
    \item A composición de homomorfismos de módulos é un homomorfismo de módulos. En particular, dados dous homomorfismos de módulos $f:M \longrightarrow N$ e $g: N \longrightarrow P$, a aplicación $g \circ f: M \longrightarrow P$ é un homomorfismo de módulos.
    \end{enumerate}

\vspace{3mm}

\noindent Do mesmo xeito ca nas teorías de grupos e de aneis, en función do carácter inxectivo ou sobrexectivo dun homomorfismo de módulos, pódese empregar a seguinte terminoloxía:\\

\noindent \textbf{Definición 3.7.} Sexa $A$ un anel e $f: M \longrightarrow N$ un homomorfismo de $A$-módulos.
\begin{enumerate}
    \item Se $M = N$, dise que $f$ é un \magbf{endomorfismo}.
    \item Se $f$ é inxectivo, dise que é un \magbf{monomorfismo}.
    \item Se $f$ é sobrexectivo, denominarase \magbf{epimorfismo}.
\end{enumerate}

\magbf{\subsection{Homomorfismos e submódulos}}

\vspace{5mm}

\noindent Tal e como se fixo para homomorfismos de grupos e de aneis, para os homomorfismos de módulos pódese definir:\\

\noindent \textbf{Definición 3.8}. Sexa $A$ un anel e $M$ e $N$ dous $A$-módulos entre os cales se establece un homomorfismo $f: M \longrightarrow N$. Defínense:\\
\begin{enumerate}
    \item O \magbf{núcleo de $f$}, denotado por \magen{$Ker \hspace{1mm} f$}:
    \begin{center}
        \fcolorbox{magenta}{white}{
        $Ker \hspace{1mm} f := \{x \in M \hspace{1mm} | \hspace{1mm} f(x) = 0_{N}\} = f^{-1}(\{0_{N}\})$
        }
    \end{center}
    
    \item A \magbf{imaxe de $f$}, denotada por \magen{$Im \hspace{1mm} f$}:
    \begin{center}
        \fcolorbox{magenta}{white}{
        $Im \hspace{1mm} f := \{y \in N \hspace{1mm} | \hspace{1mm} \exists \hspace{1mm} x \in M : f(x) = y\} = f(M)$}
    \end{center}
\end{enumerate}

\begin{proposition} \label{prop3.7}
Sexa $A$ un anel e $M$ e $N$ $A$-módulos entre os cales se establece un homomorfismo $f: M \longrightarrow N$. Verifícase o seguinte:
\begin{enumerate}
    \item Se $M'$ é un submódulo de $M$, $f(M')$ é un submódulo de $N$.
    \item Se $N'$ é un submódulo de $N$, $f^{-1}(N')$ é un submódulo de $M$.
\end{enumerate}
\end{proposition}

\vspace{2mm}

\noindent \textbf{\textit{\underline{Demostración}}}

\vspace{2mm}

\noindent Esta proba é idéntica á que se fixo no caso particular das aplicacións lineares, na materia de \textit{Espazos Vectoriais e Cálculo Matricial}. $\square$\\

\pagebreak

\begin{corollary} \label{cor3.1}
exa $A$ un anel e $M$ e $N$ $A$-módulos entre os cales se establece un homomorfismo $f: M \longrightarrow N$. Verifícase o seguinte:
\begin{enumerate}
    \item $Ker \hspace{1mm} f$ é un submódulo de $M$.
    \item $Im \hspace{1mm} f$ é un submódulo de $N$.
\end{enumerate}
\end{corollary}

\vspace{3mm}

\noindent \textbf{Definición 3.9}. Sexa $A$ un anel e $f: M \longrightarrow N$ un homomorfismo de $A$-módulos. Chámaselle \magbf{conúcleo de $f$}, denotado por \magbf{$CoKer \hspace{1mm} f$}, ao seguinte módulo cociente:

\begin{center}
    \fcolorbox{magenta}{white}{
    $CoKer \hspace{1mm} f := \displaystyle \frac{N}{Im \hspace{1mm} f}$
    }
\end{center}

\begin{proposition} \label{prop3.8}
Sexa $A$ un anel e $M$ e $N$ $A$-módulos entre os cales se establece un homomorfismo $f: M \longrightarrow N$. Entón, cúmprese:
\begin{enumerate}
    \item $f$ é inxectivo $\Longleftrightarrow Ker \hspace{1mm} f = \{0_{M}\}$
    \item $f$ é sobrexectivo $\Longleftrightarrow CoKer \hspace{1mm} f = \{0_{N}\}$
\end{enumerate}
\end{proposition}

\vspace{2mm}

\noindent \textbf{\textit{\underline{Demostración}}}

\vspace{2mm}

\noindent \magbf{(1)} É trivial, pois $f$ é, en particular, un homomorfismo de grupos.\\

\noindent \magbf{(2)} $f$ é sobrexectivo $\Longleftrightarrow Im \hspace{1mm} f = N \Longleftrightarrow N/Im \hspace{1mm} f = \{0_{N}\}$ \hspace{3mm} $\square$\\

\magbf{\subsection{Teoremas de isomorfía de módulos}}

\vspace{5mm}

\noindent \textbf{Definición 3.10}. Sexa $A$ un anel e $f: M \longrightarrow N$ un homomorfismo de $A$-módulos. Dirase que $f$ é un \magbf{isomorfismo} se existe un homomorfismo de $A$-módulos $g: N \longrightarrow M$ tal que $f \circ g = id_{N}$, $g \circ f = id_{M}$.\\

\noindent Tal homomorfismo $g$ é único. En efecto, se $g'$ é outro homomorfismo nas mesmas condicións ca $g$, cumpre:\\
\begin{itemize}
    \item $(g' \circ f \circ g) = g' \circ (f \circ g) = g' \circ id_{N} = g'$
    \item $(g' \circ f \circ g) = (g' \circ f) \circ g = id_{M} \circ g = g$
\end{itemize}

\noindent O homomorfismo $g$ será denotado por $f^{-1}$.\\

\noindent Se $f: M \longrightarrow N$ é isomorfismo, pódese escribir $f: M \overset{\sim}{\longrightarrow} N$.\\

\noindent \textbf{Definición 3.11}. Sexa $A$ un anel e $M$, $N$ senllos $A$-módulos. Dirase que $M$ e $N$ son \magbf{isomorfos}, e escribirase $\magen{M \simeq N}$, se existe un isomorfismo $f: M \longrightarrow N$.\\

\begin{proposition} \label{prop3.9}
Sexa $A$ un anel e $M$, $N$ senllos $A$-módulos entre os cales se establece un homomorfismo $f: M \longrightarrow N$. Entón, cúmprese:
\begin{center}
    $f$ é isomorfismo $\Longleftrightarrow f$ é unha aplicación bixectiva  
\end{center}
\end{proposition}

\vspace{2mm}

\noindent \textbf{\textit{\underline{Demostración}}}

\vspace{2mm}

\noindent \fcolorbox{magenta}{white}{$\Longrightarrow$/} É trivial por definición de isomorfismo.\\

\noindent \fcolorbox{magenta}{white}{$\Longleftarrow$/} Sendo $f$ bixectiva, existe aplicación inversa $f^{-1}: N \longrightarrow M$. Véxase que $f^{-1}$ é homomorfismo:

\begin{itemize}
    \item $\forall \hspace{1mm} a \in A \hspace{2mm} y \in N$, $f^{-1}(ay) = af^{-1}(y)$\\
    
    Sendo $f$ bixectiva, $\exists! \hspace{1mm} x \in M \hspace{1mm} | \hspace{1mm} f(x) = y$. Tense:
    $$f^{-1}(ay) = f^{-1}(af(x)) \overset{\overset{f \text{ homomorfismo}}{\Downarrow}}{=} f^{-1}(f(ax)) = ax = af^{-1}(y)$$
    \item $\forall \hspace{1mm} x,y \in N, f^{-1}(x + y) = f^{-1}(x) + f^{-1}(y)$\\
    
    De novo, por ser $f$ bixectiva, existen, e son únicos, $a,b \in M \hspace{1mm} | \hspace{1mm} f(a) = x$, $f(b) = y$. Así:
    $$f^{-1}(x + y) = f^{-1}(f(a) + f(b)) \overset{\overset{f \text{ homomorfismo}}{\Downarrow}}{=} f^{-1}(f(a+b)) = a + b = f^{-1}(x) + f^{-1}(y)$$
\end{itemize}

\noindent Deste xeito, queda probado que $f^{-1}$ é un homomorfismo. Logo, $f$ é un isomorfismo. $\square$\\

\vspace{3mm}

\noindent \textbf{Observación 3.6}. Dous módulos isomorfos son iguais como estruturas alxébricas. A única diferenza entre eles é o nome dos elementos.\\

\noindent Como xa se fixo cos aneis, os \magbf{teoremas de isomorfía} tamén poden ser adaptados á teoría de módulos.\\

\begin{theorem}[\magbf{1.º Teorema de Isomorfía de módulos}] \label{th3.1}
Sexa $A$ un anel e $M$ e $N$ $A$-módulos entre os cales se establece un homomorfismo $f: M \longrightarrow N$. Entón, cúmprese:
\begin{center}
    $\displaystyle \frac{M}{Ker \hspace{1mm} f} \simeq Im \hspace{1mm} f$
\end{center}
\end{theorem}

\vspace{2mm}

\noindent \textbf{\textit{\underline{Demostración}}}

\vspace{2mm}

\noindent Como $f$ é, en particular, unha aplicación, pódese considerar a súa factorización canónica:

$$
     \large \xymatrix{
        M \ar[d]_p \ar[r]^{f} & N \\
        \frac{M}{Ker \hspace{1mm} f} \ar[r]_{\pi} & Im \hspace{1mm} f \ar@{_{(}->}[u]_{i}
    }  
$$

onde $\pi$ é a seguinte aplicación:

\begin{center}
    $\pi: \displaystyle \frac{M}{Ker \hspace{1mm} f} \longrightarrow Im \hspace{1mm} f$\\
    \vspace{2mm}
    $\hspace{25mm} x + Ker \hspace{1mm} f \leadsto \pi(x + Ker \hspace{1mm} f) := f(x)$ 
\end{center}

\noindent Segundo o \magbf{Teorema da factorización canónica de aplicacións}, cúmprese que a aplicación $\pi$ é bixectiva. Se se consegue demostrar que é un homomorfismo, obterase o resultado desexado.\\

\noindent Tal e como se viu no \hyperref[th1.5]{\magbf{1.º Teorema de Isomorfía de grupos}}, esta aplicación é un isomorfismo de grupos (neste caso, o grupo abeliano formado coa suma). Logo, o problema redúcese a demostrar que $\pi$ conserva o produto por escalares.\\

\noindent Dados $a \in A$ e $x + Ker \hspace{1mm} f \in \displaystyle \frac{M}{Ker \hspace{1mm} f}$, tense:
$$\pi(a(x + Ker \hspace{1mm}f)) = \pi(ax + Ker \hspace{1mm} f) = f(ax) \overset{\overset{f \text{ homomorfismo}}{\Downarrow}}{=} af(x) = a\pi(x + Ker \hspace{1mm} f)$$

\noindent Así, efectivamente, $f$ é un homomorfismo bixectivo e, polo tanto, un isomorfismo. $\square$\pagebreak

\begin{theorem}[\magbf{2.º Teorema de Isomorfía de módulos}] \label{th3.2}
Sexa $A$ un anel e $M$ un $A$-módulo. Considérense $M_{1},M_{2}$ $A$-submódulos de $M$, con $M_{2} \subset M_{1}$. Entón, cúmprese:
\begin{center}
    $\displaystyle \frac{M/M_{2}}{M_{1}/M_{2}} \simeq \displaystyle \frac{M}{M_{1}}$
\end{center}
\end{theorem}

\vspace{2mm}

\noindent \textbf{\textit{\underline{Demostración}}}

\vspace{2mm}

\noindent Defínase a seguinte aplicación:

\begin{center}
    $\phi: \displaystyle \frac{M}{M_{2}} \longrightarrow \displaystyle \frac{M}{M_{1}}$\\
    \vspace{2mm}
    $\hspace{30mm} x + M_{2} \leadsto \phi(x + M_{2}) := x + M_{1}$
\end{center}

\noindent Esta aplicación xa foi definida no \hyperref[th1.6]{\magbf{2.º Teorema de Isomorfía de grupos}}, e sábese que está ben definida e que é un epimorfismo \textbf{de grupos} cuxo núcleo é $M_{1}/M_{2}$:
$$Ker \hspace{1mm} \phi = \{x + M_{2} \hspace{1mm} | \hspace{1mm} \phi(x+ M_{2}) = x + M_{1} = 0 + M_{1} \} = \{x + M_{2} \hspace{1mm} | \hspace{1mm} x \in M_{1} \} = \displaystyle \frac{M_{1}}{M_{2}}$$

\noindent Véxase que $\phi$ conserva a operación externa. Dado $x + M_{2} \in \displaystyle \frac{M}{M_{2}}$ e $a \in A$, tense:
$$\phi(a(x + M_{2})) = \phi(ax + M_{2}) = ax + M_{1} = a(x + M_{1}) = a \hspace{1mm} \phi(x + M_{2})$$

\noindent Así, efectivamente, $\phi$ é un homomorfismo de $A$-módulos. Aplicando o \hyperref[th3.1]{\magbf{1.º Teorema de Isomorfía de módulos}}, garántese que $\displaystyle \frac{M/M_{2}}{M_{1}/M_{2}} \simeq \displaystyle \frac{M}{M_{1}}$. $\square$\\

\vspace{3mm}

\begin{theorem}[\magbf{3.º Teorema de Isomorfía de módulos}] \label{th3.3}
Sexa $A$ un anel e $M$ un $A$-módulo. Considérense $M_{1},M_{2}$ $A$-submódulos de $M$. Entón, cúmprese:
\begin{center}
    $\displaystyle \frac{M_{1} + M_{2}}{M_{1}} \simeq \displaystyle \frac{M_{2}}{M_{1}\cap M_{2}}$
\end{center}
\end{theorem}

\vspace{2mm}

\noindent \textbf{\textit{\underline{Demostración}}}

\vspace{2mm}

$$ M_{2} \overset{i}{\longrightarrow} M_{1} + M_{2} \overset{p}{\longrightarrow}\displaystyle \frac{M_{1} + M_{2}}{M_{1}}$$

\noindent Cúmprese que, sendo $i$ e $\pi$ homomorfismos de módulos, a aplicación $\phi = p \circ i$ é un homomorfismo de módulos, estando definida como $\phi(x_{2}) = x_{2} + M_{1} \hspace{2mm} \forall \hspace{1mm} x_{2} \in M_{2} $.\\

\noindent Esta mesma aplicación xa foi definida para a proba do \hyperref[th1.7]{\magbf{3.º Teorema de Isomorfía de grupos}}, e cumpre ser un epimorfismo \textbf{de grupos} cuxo núcleo é $M_{1}\cap M_{2}$.\\

\noindent Véxase que $\phi$ preserva a operación externa. Dados $a \in A$ e $x_{2} \in M_{2}$:
$$\phi(ax_{2}) = ax_{2} + M_{1} = a(x_{2} + M_{1}) = a \hspace{1mm} \phi(x_{2} + M_{1})$$

\noindent Así, $\phi$ é un homomorfismo de módulos. Aplicando o \hyperref[th3.1]{\magbf{1.º Teorema de Isomorfía de módulos}}, obtense que $\displaystyle \frac{M_{1} + M_{2}}{M_{1}} \simeq \displaystyle \frac{M_{2}}{M_{1}\cap M_{2}}$. $\square$
\pagebreak
\magbf{\subsection{Teorema de correspondencia de módulos}}

\vspace{5mm}

\noindent Este resultado, que xa foi enunciado para grupos e aneis, tamén ten a súa versión para os módulos:\\

\begin{theorem}[\magbf{Teorema de Correspondencia de módulos}] \label{th3.4}
Sexa $A$ un anel e $M$ e $N$ senllos $A$-módulos entre os cales se establece un epimorfismo $f: M \longrightarrow N$. Considérense os seguintes conxuntos:
$$\mathcal{M} = \{M' \text{ submódulo de }M \hspace{1mm} | \hspace{1mm} Ker \hspace{1mm} f \subset M'\} \hspace{10mm} \mathcal{N} = \{N' \text{ submódulo de }N\}$$
\noindent Cúmprese que os conxuntos $\mathcal{M}$ e $\mathcal{N}$ están en bixección. En particular, as aplicacións: 
    \begin{center}
        $\phi: \mathcal{M} \longrightarrow \mathcal{N}$ \\
        \vspace{2mm}
        $\hspace{25mm} M' \leadsto \phi(M') := f(M')$
    \end{center} 
\noindent e a súa inversa:
    \begin{center}
        $\phi^{-1}: \mathcal{N} \longrightarrow \mathcal{M}$ \\
        \vspace{2mm}
        $\hspace{35mm} N' \leadsto \phi^{-1}(N') := f^{-1}(N')$
    \end{center} 
\noindent son bixectivas.
\end{theorem}

\vspace{2mm}

\noindent \textbf{\textit{\underline{Demostración}}}

\vspace{2mm}

\noindent Sábese, pola \hyperref[prop3.7]{\magbf{Proposición 3.7}}, que se $M'$ é un submódulo de $M$, $f(M')$ é un submódulo de $N$. Do mesmo xeito, se $N'$ é un submódulo de $N$, $f^{-1}(N')$ é un submódulo de $M$.\\

\noindent Ademais, gracias ao \hyperref[th1.8]{\magbf{Teorema de Correspondencia de grupos}}, as igualdades sinaladas no seguinte esquema:

$$ M' \overset{\phi}{\longrightarrow} f(M') \overset{\phi^{-1}}{\longrightarrow} f^{-1}(f(M')) \overset{?}{=} M' $$

$$ N' \overset{\phi^{-1}}{\longrightarrow} f^{-1}(N') \overset{\phi}{\longrightarrow} f(f^{-1}(N')) \overset{?}{=} N'$$

\noindent sábense certas \textbf{no sentido de conxuntos}.\\

\noindent Xuntando ámbolos dous resultados anteriores, queda probado o teorema. $\square$\\

\vspace{3mm}

\noindent Séguese inmediatamente disto:

\begin{corollary} \label{cor3.2}
Sexa $A$ un anel, $M$ un $A$-módulo e $N$ un submódulo de $M$. Considérense os seguintes conxuntos:
$$\mathcal{X} = \{M' \subset M \hspace{1mm} | \hspace{1mm} M' \text{ é submódulo de M, } N \subset M'\} \hspace{10mm} \mathcal{Y}= \{ N' \subset \displaystyle \frac{M}{N} \hspace{1mm} | \hspace{1mm} N' \text{ é submódulo de } \displaystyle \frac{M}{N}\}$$
\noindent Cúmprese que existe unha bixección entre $\mathcal{X}$ e $\mathcal{Y}$.
\end{corollary}

\vspace{2mm}

\noindent \textbf{\textit{\underline{Demostración}}}

\vspace{2mm}

\noindent Abonda aplicar o \hyperref[th3.4]{\magbf{Teorema de Correspondencia de módulos}} á proxección canónica de $M$ en $M/N$, que é un epimorfismo de núcleo $N$. $\square$ 
\pagebreak
\magbf{\section{Módulos cíclicos}}

\vspace{5mm}

\noindent \textbf{Definición 3.12}. Sexa $A$ un anel e $M$ un $A$-módulo. Dirase que $M$ é un \magbf{módulo cíclico} se pode ser xerado por un único elemento, i.e. se cumpre:
$$\exists \hspace{1mm} x \in M \hspace{1mm} | \hspace{1mm} \langle x \rangle = M = Ax$$

\noindent O exemplo máis claro de módulo cíclico é o propio anel $A$:
$$A = \langle 1 \rangle = A1$$

\begin{proposition}[\magbf{Definición equivalente de módulo cíclico}] \label{prop3.10}
Sexa $A$ un anel e $M$ un $A$-módulo. Equivalen:
\begin{enumerate}
    \item $M$ é cíclico
    \item Existe un epimorfismo de $A$-módulos $\phi: A \longrightarrow M$
    \item Existe un ideal $\mathcal{I}$ de $A$ e un isomorfismo de $A$-módulos \hspace{1mm} $\displaystyle \frac{A}{\mathcal{I}} \simeq M$
\end{enumerate}
\end{proposition}

\vspace{2mm}

\noindent \textbf{\textit{\underline{Demostración}}}

\vspace{2mm}

\noindent \fcolorbox{magenta}{white}{$(1) \implies (2)$}\\

\noindent Supóñase $M$ cíclico. Entón, $\exists \hspace{1mm} x \in M \hspace{1mm} | \hspace{1mm} M = \langle x \rangle = Ax$. Fixando tal $x$, defínase a seguinte aplicación:\\

\begin{center}
    $\phi: A \longrightarrow M$\\
    \vspace{2mm}
    $\hspace{20mm} a \leadsto \phi(a) := ax$\\
\end{center}

\noindent Cúmprese que $\phi$ é un homomorfismo de $A$-módulos:\\

\begin{itemize}
    \item $\phi(a + b) = (a + b)x \overset{\overset{M \text{ módulo}}{\Downarrow}}{=} ax + bx = \phi(a) + \phi(b) \hspace{3mm} \forall \hspace{1mm} a,b \in A$
    \item $\phi(ba) = (ba)x \overset{\overset{M \text{ módulo}}{\Downarrow}}{=} b(ax) = b \hspace{1mm} \phi(a) \hspace{3mm} \forall \hspace{1mm} a,b \in A$
\end{itemize}

\noindent Ademais, $\phi$ resulta ser un epimorfismo: dado $\lambda \in M$, $\exists \hspace{1mm} a \in A \hspace{1mm} | \hspace{1mm} \lambda = ax = \phi(a)$ (por ser $M = Ax$ un módulo cíclico).\\

\noindent \fcolorbox{magenta}{white}{$(2) \implies (3)$}\\

\noindent Por hipótese, tense un epimorfismo de módulos $\phi: A \longrightarrow M$. \\

\noindent Aplicando o \hyperref[th3.1]{\magbf{1.º Teorema de Isomorfía de módulos}}, sábese que $\displaystyle \frac{A}{Ker \hspace{1mm} \phi} \simeq Im \hspace{1mm} \phi = M$.\\

\noindent Sendo $A$ un anel, cúmprese que $Ker \hspace{1mm} \phi$ é un ideal de $A$. Así, abonda tomar $\mathcal{I} = Ker \hspace{1mm} \phi$ para obter o resultado.\\

\noindent \fcolorbox{magenta}{white}{$(3) \implies (1)$}\\

\noindent Por hipótese, tense un isomorfismo de módulos $\theta: \displaystyle \frac{A}{\mathcal{I}} \longrightarrow M$.\\

\noindent Doutra banda, o $A$-módulo $\displaystyle \frac{A}{\mathcal{I}}$ é cíclico, e está xerado por $1 + \mathcal{I}$:
$$a + \mathcal{I} \overset{\overset{A \text{ cíclico}}{\Downarrow}}{=} a \cdot 1 + \mathcal{I} = a(1 + \mathcal{I})$$

\noindent Así, $M$ tamén é cíclico, sendo xerado por $\theta(1 + \mathcal{I})$. $\square$ \pagebreak

\noindent \textbf{Definición 3.13}. Sexa $A$ un anel. Considérese $M$ un $A$-módulo. Defínese o \magbf{anulador} ou \magbf{aniquilador de $M$}, denotado por \magbf{$(0:M)$}, como o seguinte ideal de $A$:
\begin{center}
    \fcolorbox{magenta}{white}{
    $(0:M) := \{a \in A \hspace{1mm} | \hspace{1mm} ax = 0 \hspace{3mm} \forall \hspace{1mm} x \in M\}$
    }
\end{center}

\vspace{3mm}

\noindent \textbf{Definición 3.14}. Sexa $A$ un anel e $M$ un $A$-módulo. Dado un elemento $x \in M$, defínese o \magbf{anulador} ou \magbf{aniquilador de $x$}, denotado por \magbf{$(0 : x)$}, como o anulador do módulo cíclico xerado por $x$:
\begin{center}
    \fcolorbox{magenta}{white}{
    $(0 : x) := \{a \in A \hspace{1mm} | \hspace{1mm} ax = 0\} = (0 : \langle x \rangle)$
    }
\end{center}

\vspace{3mm}

\noindent Na demostración da proposición anterior viuse que, fixado $x \in M$, a seguinte aplicación:
\begin{center}
    $\phi: A \longrightarrow M$\\
    \vspace{2mm}
    $\hspace{20mm} a \leadsto \phi(a) := ax$\\
\end{center}

\noindent é un homomorfismo de $A$-módulos. Este cumpre:

\begin{itemize}
    \item $Ker \hspace{1mm} \phi = \{a \in A \hspace{1mm} | \hspace{1mm} \phi(a) = 0\} = \{a \in A \hspace{1mm} | \hspace{1mm} ax = 0\} = (0 : x)$
    \item $Im \hspace{1mm} \phi = \langle x \rangle = Ax$
\end{itemize}

\noindent Aplicando o \hyperref[th3.1]{\magbf{1.º Teorema de Isomorfía de módulos}}, obtense que $\displaystyle \frac{A}{(0 : x)} \simeq Ax$.\\

\noindent \textbf{Observación 3.7}. Se $M$ é un módulo cíclico, en particular tamén vai ser un grupo cíclico.\\

\noindent Considérese agora o caso particular do anel $A = \mathbb{Z}$. Gracias á \hyperref[prop3.10]{\magbf{Proposición 3.10}} sábese que os $\mathbb{Z}$-módulos cíclicos (en particular, os grupos cíclicos) son, salvo isomorfismos, os grupos $\displaystyle \frac{\mathbb{Z}}{n\mathbb{Z}}$, con $n \in \mathbb{Z}$.
\begin{itemize}
    \item Se $n = 0$, $\displaystyle \frac{\mathbb{Z}}{0\mathbb{Z}} = \mathbb{Z}$ é un grupo cíclico infinito.
    \item Se $n \neq 0$, $\displaystyle \frac{\mathbb{Z}}{n\mathbb{Z}}$ é un grupo cíclico finito de orde $n$.
\end{itemize}

\noindent Tal e como se viu na teoría de grupos, tense:

\begin{itemize}
    \item Todo grupo cíclico infinito é isomorfo a $\mathbb{Z}$.
    \item Cada grupo cíclico infinito é isomorfo a $\displaystyle \frac{\mathbb{Z}}{n\mathbb{Z}}$.
    \item A isomorfía entre grupos cíclicos da mesma orde (finitos ou infinitos) é unha relación de equivalencia.
\end{itemize}

\magbf{\section{Operacións con módulos}}

\magbf{\subsection{Produto directo}}

\vspace{5mm}

\noindent Sexa $A$ un anel e $\{M_{i}\}_{i \in I}$ unha familia arbitraria de $A$-módulos. Considérese o produto cartesiano da familia:
\begin{center}
 \fcolorbox{magenta}{white}{
 $\displaystyle \prod_{i \in I} M_{i} := \{f : I \longrightarrow \displaystyle \bigcup_{i \in I} M_{i} \hspace{1mm} | \hspace{1mm} f(i) \in M_{i} \hspace{3mm} \forall \hspace{1mm} i \in I\} = \{(x_{i})_{i \in I} \hspace{1mm} | \hspace{1mm} x_{i} \in M_{i} \hspace{3mm} \forall \hspace{1mm} i \in I\}$
 }  
\end{center}


\noindent Emprégase a notación $f \equiv (f(i))_{i \in I} = (x_{i})_{i \in I}$.\\

\noindent Cúmprese que $\displaystyle \prod_{i \in I} M_{i}$ posúe estrutura de $A$-módulo mediante as seguintes operacións:
\begin{itemize}
    \item $(x_{i})_{i \in I} + (y_{i})_{i \in I} := (x_{i} + y_{i})_{i \in I}$
    \item $a(x_{i})_{i \in I} := (ax_{i})_{i \in I}$
\end{itemize}


\noindent O elemento nulo é aquel $(x_{i})_{i \in I} \hspace{1mm} | \hspace{1mm} x_{i} = 0 \hspace{3mm} \forall \hspace{1mm} i \in I$.\\

\noindent \textbf{Exercicio}. Demostrar que, efectivamente, $\displaystyle \prod_{i \in I} M_{i}$ é un $A$-módulo coas operacións dadas.\\

\noindent Ao $A$-módulo $\displaystyle \prod_{i \in I} M_{i}$ denomínaselle o \magbf{produto directo da familia $\{M_{i}\}_{i \in I}$}.\\

\noindent As proxeccións:
\begin{center}
    $\pi_{j}: \displaystyle \prod_{i \in I} M_{i} \longrightarrow M_{j}$\\
    \vspace{2mm}
    $\hspace{20mm} (x_{i})_{i \in I} \leadsto \pi_{j}((x_{i})_{i \in I}) := x_{j}$
\end{center}

\noindent cumpren ser epimorfismos de $A$-módulos.\\

\magbf{\subsection{Suma directa}}

\vspace{5mm}

\noindent Sexa $A$ un anel e $\{M_{i}\}_{i \in I}$ unha familia de $A$-módulos. Considérese o subconxunto de $\displaystyle \prod_{i \in I} M_{i}$ definido como segue:
\begin{center}
    \fcolorbox{magenta}{white}{
    $\displaystyle \bigoplus_{i \in I} M_{i} := \{(x_{i})_{i \in I} \in \displaystyle \prod_{i \in I} M_{i} \hspace{1mm} | \hspace{1mm} x_{i} = 0 \hspace{3mm} \almostall i \in I\}$
    }
\end{center}

\vspace{3mm}

\noindent \textbf{Observación 3.8}. Se $I$ é un conxunto finito, $\displaystyle \prod_{i \in I} M_{i} = \displaystyle \bigoplus_{i \in I} M_{i}$.\\

\noindent Verifícase que $\displaystyle \bigoplus_{i \in I} M_{i}$ resulta ser un submódulo de $\displaystyle \prod_{i \in I} M_{i}$ (a modo de exercicio, pódese demostrar), e recibe o nome de \magbf{suma directa da familia $\{M_{i}\}_{i \in I}$}.\\

\noindent As aplicacións:

\begin{center}
    $\mu_{j}: M_{j} \longrightarrow \displaystyle \bigoplus_{i \in I} M_{i}$\\
    \vspace{2mm}
    $\hspace{45mm} x \leadsto \mu_{j}(x) := (x_{i})_{i \in I}$, $x_{i} := 
    \begin{cases}
    0 & i \neq j\\
    x & i = j
    \end{cases}$
\end{center}

\noindent cumpren ser homomorfismos de $A$-módulos.\\

\noindent Nótese que, para cada $(x_{i})_{i \in I} \in \displaystyle \bigoplus_{i \in I} M_{i}$, tense:
$$(x_{i})_{i \in I} = \displaystyle \sum_{i \in I} \mu_{i}(x_{i})$$

\noindent onde a suma ten sentido, pois é finita, xa que como $(x_{i})_{i \in I} \in \displaystyle \bigoplus_{i \in I} M_{i}$, hai unha infinidade de sumandos nulos.\\

\noindent Considérese o caso particular no cal $I = \{1, 2\}$. Téñense as aplicacións:
\begin{center}
    $M_{1} \overset{\mu_{1}}{\longrightarrow} M_{1} \oplus M_{2} \overset{\mu_{2}}{\longleftarrow} M_{2}$\\
    \vspace{2mm}
    $\hspace{-15mm} x_{1} \longmapsto (x_{1}, 0)$\\
    \vspace{2mm}
    $\hspace{15mm}(0, x_{2})$ \reflectbox{$\longmapsto$} $x_{2}$
\end{center}

\vspace{3mm}

\noindent Dados $x_{1} \in M_{1}, x_{2} \in M_{2}$, cúmprese:

$$\mu_{1}(x_{1}) + \mu_{2}(x_{2}) = (x_{1}, 0) + (0, x_{2}) = (x_{1} + 0, 0 + x_{2}) = (x_{1}, x_{2})$$

\noindent Véxase a continuación a relación entre a suma de submódulos e a suma directa:\\

\noindent Sexa $A$ un anel e $M$ un $A$-módulo. Considérese $\{M_{i}\}_{i \in I}$ unha familia arbitraria de submódulos de $M$. Tense un homomorfismo de $A$-módulos:
\begin{center}
    $\phi: \displaystyle \bigoplus_{i \in I} M_{i} \longrightarrow \displaystyle \sum_{i \in I} M_{i}$\\
    \vspace{2mm}
    $\hspace{5mm} (x_{i})_{i \in I} \leadsto \displaystyle \sum_{i \in I} x_{i}$
\end{center}

\noindent \textbf{Exercicio}. Demostrar que $\phi$ é un epimorfismo de $A$-módulos.\\

\noindent Cómpre decatarse de que ten sentido o sumatorio dos $x_{i}$, pois como $(x_{i})_{i \in I} \in \displaystyle \bigoplus_{i \in I} M_{i}$, $x_{i} = 0 \hspace{3mm} \almostall i \in I$.\\

\begin{proposition} \label{prop3.11}
Sexa $A$ un anel e $M$ un $A$-módulo. Considérese unha familia $\{M_{i}\}_{i \in I}$ arbitraria de submódulos de $M$. As seguintes condicións son equivalentes:
\begin{enumerate}
    \item $\phi$ é isomorfismo
    \item $\displaystyle \sum_{i \in I} x_{i} = 0 \hspace{5mm} x_{i} \in M_{i} \hspace{2mm} \forall \hspace{1mm} i \in I \implies x_{i} = 0 \hspace{2mm} \forall i \in I$
    \item $M_{j} \cap \left(\displaystyle \sum_{i \neq j} M_{i}\right) = \{0\} \hspace{4mm} \forall \hspace{1mm} j \in I$
\end{enumerate}
\end{proposition}

\vspace{2mm}

\noindent \textbf{\textit{\underline{Demostración}}}

\vspace{2mm}

\noindent \fcolorbox{magenta}{white}{$(1) \implies (2)$}\\

\noindent Supóñase $\phi$ un isomorfismo. Considérese $\displaystyle \sum_{i \in I} x_{i} = 0$. Tense:
$$\displaystyle \sum_{i \in I} x_{i} = \phi[(x_{i})_{i \in I}] = 0 \implies (x_{i})_{i \in I} \in Ker \hspace{1mm} \phi \overset{\overset{\magbf{(1)}}{\Downarrow}}{=} \{0\} \implies (x_{i})_{i \in I} = (0)_{i \in I} \implies x_{i} = 0 \hspace{3mm} \forall \hspace{1mm} i \in I$$

\noindent \fcolorbox{magenta}{white}{$(2) \implies (3)$}\\

\noindent Sexa $j \in I$. Véxase que $M_{j} \cap \left(\displaystyle \sum_{i \neq j} M_{i}\right) = \{0\}$.\\

\noindent Sexa $x \in M_{j} \cap \left(\displaystyle \sum_{i \neq j} M_{i}\right)$. Entón:

$$
x \in M_{j} \cap \left(\displaystyle \sum_{i \neq j} M_{i}\right) \implies 
\begin{cases}
x \in M_{j}\\
x \in \underset{i \neq j}{\sum} M_{i} \implies x = \underset{i \neq j}{\sum}x_{i} \hspace{4mm} x_{i} \in M_{i}, \hspace{2mm} x_{i} = 0 \hspace{2mm} \almostall i \in I
\end{cases}
$$
\vspace{2mm}
$$
-x + \underset{i \neq j}{\sum}x_{i} = 0 \underset{\underset{\magbf{(2)}}{\Uparrow}}{\implies} -x = 0, x_{i} = 0 \hspace{2mm} \forall \hspace{1mm} i \neq j \implies x = 0
$$

\noindent \fcolorbox{magenta}{white}{$(3) \implies (1)$}\\

\noindent Sexa $(x_{i})_{i \in I} \in Ker \hspace{1mm} \phi$. Entón, por definición, $\phi[(x_{i})_{i \in I}] = \displaystyle \sum_{i \in I} x_{i} = 0$.\\

\noindent Sexa $j \in I$. Próbese que $x_{j} = 0$:
    \[ 
    \left. \begin{array}{r} 
    \displaystyle \sum_{i \in I} x_{i} = 0 \\[1ex]
    \displaystyle \sum_{i \in I} x_{i} = x_{j} + \displaystyle \sum_{i \neq j} x_{i}
    \end{array} \right\} 
    \implies -x_{j} = \displaystyle \sum_{i \neq j} x_{i} \in M_{j} \cap \left(\displaystyle \sum_{i \neq j} M_{i}\right) \overset{\overset{\magbf{(3)}}{\Downarrow}}{=} \{0\} \implies -x_{j} = 0 \implies x_{j} = 0 \hspace{4mm} \square
    \]
    
\vspace{3mm}

\noindent Se $\phi$ é un isomorfismo, dise que a suma $\displaystyle \sum_{i \in I} M_{i}$ é unha \magbf{suma directa}. Neste caso, cada elemento $x \in \underset{i \in I}{\sum}M_{i}$ exprésase de xeito único na forma:
$$x = \displaystyle \sum_{i \in I} x_{i}, \hspace{4mm} x_{i} \in M_{i} \hspace{2mm} \forall \hspace{1mm} i \in I, \hspace{3mm} x_{i} = 0 \hspace{2mm} \almostall i \in I$$

\vspace{3mm}

\noindent En efecto, se $x = \underset{i \in I}{\sum} x_{i} = \underset{i \in I}{\sum} y_{i}$, tense:
$$\displaystyle \sum_{i \in I} (\underbrace{x_{i} - y_{i}}_{\in M_{i}}) = 0 \underset{\underset{\phi \text{ isomorfismo}}{\Uparrow}}{\implies} x_{i} - y_{i} = 0 \Longleftrightarrow x_{i} = y_{i} \hspace{3mm} \forall \hspace{1mm} i \in I$$

\vspace{3mm}

\noindent Vólvase ao caso particular onde $I = \{1,2\}$. Sexa $A$ un anel, $M$ un $A$-módulo e $M_{1}, M_{2}$ submódulos de $M$. Tense:

\begin{center}
    $\phi: M_{1} \oplus M_{2} \longrightarrow M_{1} + M_{2}$\\
    \vspace{2mm}
    $\hspace{5mm} (x_{1}, x_{2}) \leadsto x_{1} + x_{2}$
\end{center}

\noindent Verifícase que $M_{1} + M_{2}$ é suma directa $\Longleftrightarrow M_{1} \cap M_{2} = \{0\}$.\\

\noindent Do mesmo xeito, pódese considerar o caso finito para calquera $n \in \mathbb{N}$, sendo entón $I = \{1, \ldots, n\}$:

\begin{center}
    $\phi: \displaystyle \bigoplus_{i = 1}^{n} M_{i} \longrightarrow \displaystyle \sum_{i = 1}^{n} M_{i}$\\
    \vspace{2mm}
    $(x_{1}, \ldots, x_{n}) \leadsto \displaystyle \sum_{i  = 1}^{n} x_{i}$
\end{center}

\magbf{\section{Módulos libres}}

\magbf{\subsection{Independencia linear. Bases}} \label{sec:5.1}

\vspace{5mm}

\noindent \textbf{Definición 3.15}. Sexa $A$ un anel e $M$ un $A$-módulo. Dise que unha familia $\{x_{i}\}_{i \in I}$ de elementos de $M$ é \magbf{linearmente independente} se verifica a seguinte condición:\\
\begin{center}
\fcolorbox{magenta}{white}{
    Dado $\displaystyle \sum_{i \in I} a_{i}x_{i}$, \hspace{2mm} $a_{i} \in A$, $a_{i} = 0 \hspace{2mm} \almostall i \in I \implies a_{i} = 0 \hspace{2mm} \forall \hspace{1mm} i \in I$
    }
\end{center}

\noindent É obvio que $\{x_{i}\}_{i \in I}$ é linearmente independente $\Longleftrightarrow$ cada subfamilia $\{x_{i}\}_{i \in J}$, $J \subset I$ finito, é linearmente independente.\\

\noindent \textbf{Observación 3.9}. Nótese que se $\{x_{i}\}_{i \in I}$ é linearmente independente, sucede que $\underset{i \in I}{\sum} Ax_{i}$ é unha suma directa. En efecto, a aplicación:\\
\begin{center}
    $\phi: \displaystyle \bigoplus_{i \in I} Ax_{i} \longrightarrow \displaystyle \sum_{i \in I} Ax_{i}$\\
    \vspace{2mm}
    $\hspace{5mm} (a_{i}x_{i})_{i \in I} \leadsto \displaystyle \sum_{i \in I} a_{i}x_{i}$
\end{center}

\noindent é un isomorfismo. Dado $(a_{i}x_{i})_{i \in I} \in Ker \hspace{1mm} \phi$, tense:
$$\phi((a_{i}x_{i})_{i \in I}) = \displaystyle \sum_{i \in I} a_{i}x_{i} = 0 \overset{\overset{\{x_{i}\} \text{ lin indep}}{\Downarrow}}{\implies} a_{i} = 0 \hspace{3mm} \forall \hspace{1mm} i \in I \implies a_{i}x_{i} = 0 \hspace{3mm} \forall \hspace{1mm} i \in I \implies Ker \hspace{1mm} \phi = \{0\}$$

\vspace{3mm}

\noindent \textbf{Observación 3.10}. Se $\{x_{i}\}_{i \in I}$ é unha familia linearmente independente, cúmprese que $(0 : x_{i}) = \{0\} \hspace{3mm} \forall i \in I$. En efecto, para cada $i \in I$ dado $a \in (0: x_{i})$, tense que $ax_{i} = 0$, e como $\{x_{i}\}_{i \in I}$ é linearmente independente, en particular éo tamén $\{x_{i}\}$, logo $a = 0$.\\

\noindent Con isto, obtense que $Ax_{i} \simeq \displaystyle \frac{A}{(0:x_{i})} = \frac{A}{\{0\}} = A$.\\

\noindent Nótese tamén que se a familia é un conxunto unitario, $\{x\}$, a súa independencia linear equivale a que o seu anulador sexa \{0\}, $(0 : x) = \{0\}$.\\

\noindent \textbf{Definición 3.16}. Sexa $A$ un anel e $M$ un $A$-módulo. Unha \magbf{base de $M$} é un conxunto de xeradores de $M$ linearmente independentes:
\begin{center}
    $S = \{x_{i}\}_{i \in I}$ é \magbf{base de $M$} $: \Longleftrightarrow [( \langle S \rangle = M) \wedge (S $ linearmente independente)]
\end{center}

\vspace{3mm}

\noindent \textbf{Observación 3.11}. Se $M$ posúe unha base $\{x_{i}\}_{i \in I}$, cada elemento $x \in M$ exprésase de xeito único como combinación linear de elementos desa base:
$$x = \displaystyle \sum_{i \in I} a_{i}x_{i}, \hspace{4mm} a_{i} \in A \hspace{3mm} a_{i} = 0 \hspace{2mm} \almostall i \in I$$

\magbf{\subsection{Módulo libre: definición e exemplos}}

\vspace{5mm}

\noindent \textbf{Definición 3.17}. Sexa $A$ un anel e $M$ un $A$-módulo. Dirase que $M$ é \magbf{libre} se posúe algunha base.\\

\noindent Exemplos de módulos libres son os seguintes:\\
\begin{enumerate}
    \item Se $A$ é un corpo, tódolos $A$-módulos son libres. Recórdese que neste caso todo $A$-módulo é un $A$-espazo vectorial, e tense demostrado na materia de \textit{Espazos Vectoriais e Cálculo Matricial} que todo espazo vectorial posúe unha base.
    \item Considerando un anel $A$ como $A$-módulo, tense que $A$ é libre. Abonda considerar $\{1\}$ como base de $A$. Máis aínda, calquera unidade de $A$, $u$, serve como base:
    \begin{itemize}
        \item $\{u\}$ é linearmente independente
        $$au = 0 \implies (au)u^{-1} = 0 \implies a = 0$$
        \item $\langle u \rangle$ = A
        Isto cúmprese porque, considerando $A$ como $A$-módulo, cada submódulo de $A$ é un ideal de $A$. Sendo $u$ unha unidade, tense que $\langle u \rangle = (u) = A$
    \end{itemize}
    \item \{0\} é un módulo libre, con base $\{\varnothing\}$.
\end{enumerate}

\noindent Véxase a continuación un exemplo de $A$-módulo non libre:\\

\noindent Sexa $A$ un anel \textbf{que non sexa corpo} e considérese en $A$ un ideal $\mathcal{I}$ propio non trivial. Probarase a continuación que o $A$-módulo $\displaystyle \frac{A}{\mathcal{I}}$ non é libre.\\

\noindent En efecto, calquera elemento non nulo de $\mathcal{I}$ anula ao módulo $\displaystyle \frac{A}{\mathcal{I}}$. Dados $a \in \mathcal{I}$ e $\alpha + \mathcal{I} \in \displaystyle \frac{A}{\mathcal{I}}$, tense:
$$a(\alpha + \mathcal{I}) = \underbrace{a\alpha}_{\in \mathcal{I}} + \mathcal{I} = 0 + \mathcal{I}$$

\noindent Deste xeito, cúmprese que non existen familias non baleiras de elementos de $\displaystyle \frac{A}{\mathcal{I}}$ linearmente independentes.\\

\noindent Agora estudarase un exemplo de módulo libre que aparecerá ao longo deste apartado. Pero antes, introducirase unha notación adicional:\\

\noindent Sexan $A$ un anel, $I$ un conxunto e $M$ un $A$-módulo. Defínense:\\
\begin{itemize}
    \item $M^{I} := \displaystyle \prod_{i \in I} M_{i}$, con $M_{i} := M \hspace{4mm} \forall \hspace{1mm} i \in I$
    \item $M^{(I)} := \displaystyle \bigoplus_{i \in I} M_{i}$, con $M_{i} := M \hspace{4mm} \forall \hspace{1mm} i \in I$
\end{itemize}

\noindent Se $I = \varnothing$, entón $M^{I} = A$.\\

\noindent Se $I = \{1, \ldots, n\}$, emprégase a notación $M^{I} \equiv M^{n}$.\\

\noindent Preséntase o seguinte exemplo de módulo libre:

\begin{enumerate}
    \setcounter{enumi}{3}
    \item Sexa $A$ un anel e $I$ un conxunto. Considérese o $A$-módulo $A$. Tense que o conxunto:
    $$A^{(I)} = \displaystyle \bigoplus_{i \in I} A_{i}, \hspace{4mm} A_{i} := A \hspace{2mm} \forall \hspace{1mm} i \in I$$
    é un $A$-módulo libre.\\
    
    Para cada $i \in I$, sexa $e_{i} = (e_{ij})_{j \in I}$ o elemento de $A^{(I)}$ tal que:
    $$e_{ij} = \delta_{ij} = 
    \begin{cases}
    1 & i = j\\
    0 & i \neq j
    \end{cases}$$
    
    i.e. $e_{i} = \mu_{i}(1)$, recordando que $\mu_{i}: A_{i} \longrightarrow \displaystyle \bigoplus_{i \in I} A_{i}$ é o homomorfismo inclusión de $A_{i}$ en $\displaystyle \bigoplus_{i \in I} A_{i}$.\\
    
    Tense que $\{e_{i}\}_{i \in I}$ é unha base de $A^{(I)}$. En efecto:
    
    \begin{itemize}
        \item É conxunto de xeradores de $A^{(I)}$\\
        
        Sexa $x \in A^{(I)}$. Entón, $x = (x_{i})_{i \in I}$. Tense:
        $$x = (x_{i})_{i \in I} = \displaystyle \sum_{i \in I} \mu_{i}(x_{i}) = \displaystyle \sum_{i \in I} \mu_{i}(x_{i} \cdot 1) = \displaystyle \sum_{i \in I} x_{i} \mu_{i}(1) = \displaystyle \sum_{i \in I} x_{i}e_{i}$$
        
        \item É unha familia linearmente independente\\
        
        Sexa $\displaystyle \sum_{i \in I} a_{i}e_{i} = 0$, con $a_{i} \in A$, $a_{i} = 0 \hspace{3mm} \almostall i \in I$. Tense:
        $$\displaystyle \sum_{i \in I} a_{i}e_{i} = \displaystyle \sum_{i \in I} a_{i} \mu_{i}(1) = \displaystyle \sum_{i \in I} \mu_{i}(a_{i} \cdot 1) = \displaystyle \sum_{i \in I} \mu_{i}(a_{i}) = (a_{i})_{i \in I} = 0 \implies a_{i} = 0 \hspace{3mm} \forall \hspace{1mm} i \in I$$
    \end{itemize}
    
    Polo tanto, o $A$-módulo $A^{(I)}$ é, pois, un $A$-módulo libre con base $\{e_{i}\}_{i \in I}$, denominada \magbf{base canónica de $A^{(I)}$}.\\
\end{enumerate}

\vspace{3mm}

\noindent Os módulos deste último exemplo son, salvo isomorfismos, os únicos módulos libres, como se verá a continuación.\\

\begin{proposition} \label{prop3.12}
Sexa $A$ un anel e $F$ un $A$-módulo libre. Sexa $S = \{x_{i}\}_{i \in I}$ unha base de $F$. Se $M$ é un $A$-módulo calquera, considerando unha familia $\{y_{i}\}_{i \in I}$ de elementos de $M$, entón existe un único homomorfismo de módulos $f: F \longrightarrow M$ tal que $f(x_{i}) = y_{i} \hspace{2mm} \forall \hspace{1mm} i \in I$. \newline
\newline
\noindent Ademais, verifícanse os seguintes resultados:
\begin{enumerate}
    \item $f$ é inxectivo $\Longleftrightarrow \{f(x_{i})\}_{i \in I} = \{y_{i}\}_{i \in I}$ é linearmente independente
    \item $f$ é sobrexectivo $\Longleftrightarrow \langle \{f(x_{i})\}_{i \in I} \rangle = \langle \{y_{i}\}_{i \in I} \rangle = M$
    \item $f$ é bixectivo $\Longleftrightarrow \{f(x_{i})\}_{i \in I} = \{y_{i}\}_{i \in I}$ é base de $M$
\end{enumerate}
\end{proposition}

\vspace{2mm}

\noindent \textbf{\textit{\underline{Demostración}}}

\vspace{2mm}

\noindent Sexa $x \in F$. Sendo $\{x_{i}\}_{i \in I}$ base de $F$, existen elementos únicos $a_{i} \in A$, con $a_{i} = 0 \hspace{2mm} \almostall i \in I$, tales que $x = \displaystyle \sum_{i \in I} a_{i}x_{i}$.\\

\noindent Así, a imaxe de $x$ queda determinada de forma unívoca polas imaxes dos elementos da base:
$$f(x) = f \left(\displaystyle \sum_{i \in I} a_{i}x_{i}\right) = \displaystyle \sum_{i \in I} a_{i}f(x_{i}) = \displaystyle \sum_{i \in I} a_{i}y_{i}$$

\noindent Os apartados \magbf{(1)}, \magbf{(2)} e \magbf{(3)} demóstranse de xeito semellante a como se fixo na materia de \textit{Espazos Vectoriais e Cálculo Matricial}. $\square$\\

\vspace{3mm}

\begin{proposition} \label{prop3.13}
exa $A$ un anel e $F$ un $A$-módulo. As seguintes afirmacións son equivalentes:
\begin{enumerate}
    \item $F$ é libre
    \item Existe un conxunto $I$ tal que $F \simeq A^{(I)}$
\end{enumerate}
\end{proposition}

\vspace{2mm}

\noindent \textbf{\textit{\underline{Demostración}}}

\vspace{2mm}

\noindent \fcolorbox{magenta}{white}{$(2) \implies (1)$}\\

\noindent Sendo $F$ e $A^{(I)}$ isomorfos, existe un isomorfismo $\theta :  A^{(I)} \longrightarrow F$. Sabendo que $A^{(I)}$ é un módulo libre, existe unha base de $A^{(I)}$, $\{e_{i}\}_{i \in I}$. Como $\theta$ é un isomorfismo, en base á \hyperref[prop3.12]{\magbf{Proposición 3.12}}, garántese que $\{\theta(e_{i})\}_{i \in I}$ é unha base de $F$, probando así que $F$ é un módulo libre.\\

\noindent \fcolorbox{magenta}{white}{$(1) \implies (2)$}\\

\noindent Supóñase $F$ un módulo libre. Entón, pódese considerar unha base $S = \{x_{i}\}_{i \in I}$ de $F$.\\

\noindent Por ser $S$ base de $F$, en particular é un conxunto linearmente independente. Entón, como se viu no epígrafe anterior, \hyperref[sec:5.1]{\magbf{5.1.}}, $\underset{i \in I}{\sum} Ax_{i}$ é unha suma directa e, ademais, $Ax_{i} \simeq A \hspace{2mm} \forall \hspace{1mm} i \in I$.\\

\noindent Doutra banda, por ser $S$ base, é un conxunto de xeradores de $F$, verificándose entón que $F = \displaystyle \sum_{i \in I} Ax_{i}$.\\

\noindent Xuntando todo o anterior, tense:
$$ A^{(I)} = \displaystyle \bigoplus_{i \in I} A \overset{\theta}{\simeq} \displaystyle \bigoplus_{i \in I} Ax_{i} \overset{\phi}{\longrightarrow} \displaystyle \sum_{i \in I} Ax_{i} = F$$

\noindent (como $A \simeq Ax_{i}$, tamén o son as sumas directas $\displaystyle \bigoplus_{i \in I} A$ e $\displaystyle \bigoplus_{i \in I} Ax_{i}$, existindo en particular un isomorfismo $\theta$).\\

\noindent Como a suma $\displaystyle \sum_{i \in I} Ax_{i}$ é directa, sábese que $\phi$ é un isomorfismo. Así, $\phi \circ \theta$ é un isomorfismo, obtendo o resultado. $\square$\\

\magbf{\subsection{Xeradores e relacións}}

\vspace{5mm}

\noindent \textbf{Definición 3.18}. Sexa $A$ un anel e $M$ un $A$-módulo. Chámaselle \magbf{presentación libre de $M$} a un epimorfismo de módulos
$$\varepsilon: F \longrightarrow M$$
\noindent onde $F$ é un $A$-módulo libre.\\

\begin{proposition} \label{prop3.14}
Sexa $A$ un anel. Todo $A$-módulo $M$ posúe unha presentación libre. De feito, cada familia de xeradores de $M$ dá lugar a unha presentación libre.
\end{proposition}

\vspace{2mm}

\noindent \textbf{\textit{\underline{Demostración}}}

\vspace{2mm}

\noindent Sexa $\{x_{i}\}$ unha familia de xeradores de $M$.\\

\noindent Sexa $F: = A^{(I)}$ e considérese $\{e_{i}\}_{i \in I}$ a base canónica de $F$.\\

\noindent Defínase o homomorfismo de $A$-módulos:
\begin{center}
    $\varepsilon: F \longrightarrow M$\\
    \vspace{2mm}
    $\hspace{30mm} e_{i} \leadsto \varepsilon(e_{i}) := x_{i} \hspace{5mm} \forall \hspace{1mm} i \in I$ 
\end{center}

\noindent Como $\{x_{i}\}$ é un conxunto de xeradores, segundo a \hyperref[prop3.12]{\magbf{Proposición 3.12}}, cúmprese que $\varepsilon$ é un epimorfismo. $\square$\\

\noindent Como consecuencia deste resultado tense:\\

\begin{corollary} \label{cor3.3}
Todo módulo é isomorfo a un cociente dun módulo libre.
\end{corollary}

\vspace{2mm}

\noindent \textbf{\textit{\underline{Demostración}}}

\vspace{2mm}

\noindent Dado un anel $A$ e $M$ un $A$-módulo arbitrario, a proposición anterior garante que se pode atopar unha presentación libre de $M$, $\varepsilon: F \longrightarrow M$.\\

\noindent Aplicando o \hyperref[th3.1]{\magbf{1.º Teorema de Isomorfía de módulos}}, tense que $\displaystyle \frac{F}{Ker \hspace{1mm} \varepsilon} \simeq M$. $\square$\\

\vspace{3mm}

\noindent \textbf{Definición 3.19}. Sexa $M$ un $A$-módulo. Considérese unha familia de xeradores de $M$, $\{x_{i}\}_{i \in I}$, e a seguinte presentación libre de $M$:
\begin{center}
    $\varepsilon: A^{(I)} \longrightarrow M$\\
    \vspace{2mm}
    $\hspace{30mm} e_{i} \leadsto \varepsilon(e_{i}) := x_{i} \hspace{5mm} \forall \hspace{1mm} i \in I$
\end{center}

\noindent O $A$-módulo $N := Ker \hspace{1mm} \varepsilon$ denomínase \magbf{módulo de relacións da presentación libre $\varepsilon$}. \\

\noindent Sexa $\{f_{j}\}_{j \in J} \subset A^{(I)}$ unha familia de xeradores de $N$. Como $\{e_{i}\}_{i \in I}$ é unha base de $A^{(I)}$, pódese escribir:
$$f_{j} = \displaystyle \sum_{i \in I} a_{ji}e_{i} \hspace{5mm} a_{ji} \in A, a_{ji} = 0 \hspace{2mm} \almostall i \in I$$

\noindent Como $f_{j} \in Ker \hspace{1mm} \varepsilon \hspace{3mm} \forall \hspace{1mm} j \in J$, ocorre que $\varepsilon(f_{j}) = 0$, o cal dá lugar ás seguintes relacións:
$$\displaystyle \sum_{i \in I} a_{ji}x_{i} = 0 \hspace{4mm} \forall \hspace{1mm} j \in J$$

\noindent Nótese que o $A$-módulo $M$ está determinado, salvo isomorfismos, por $A^{(I)}$ e por $N$:
$$M \simeq \displaystyle \frac{A^{(I)}}{N} \hspace{10mm} x_{i} \leadsto e_{i} + N$$

\noindent Deste xeito, $M$ está determinado, salvo isomorfismos, polos conxuntos $I$, $J$ e os elementos $a_{ji}$, $j \in J, i \in I$ (aínda que distintos conxuntos $I$, $J$, $a_{ji}$ poden determinar o mesmo módulo).\\

\noindent Reciprocamente, dados dous conxuntos $I$, $J$ e dados $a_{ji} \in A$, con $i \in I, j \in J$, $a_{ji} = 0 \hspace{2mm} \almostall i \in I$, existe un $A$-módulo $M$ cunha familia de xeradores $\{x_{i}\}_{i \in I}$, suxeitos ás seguintes relacións:
$$\displaystyle \sum_{i \in I} a_{ji}x_{i} = 0 \hspace{4mm} \forall \hspace{1mm} j \in J$$

\noindent En efecto, defínase $M := \displaystyle \frac{A^{(I)}}{N}$, con $N = \langle \{f_{j}\}_{j \in J} \rangle$, onde $f_{j} = \underset{i \in I}{\sum} a_{ji}e_{i}$.\\

\noindent Cada xerador de $M$ defínese como $x_{i} := \pi(e_{i})$, con $\pi: A^{(I)} \longrightarrow M$ a presentación libre de $M$. Tense ademais:
$$\pi(f_{j}) = \pi \left( \displaystyle \sum_{i \in I} a_{ji}e_{i}\right) = \displaystyle \sum_{i \in I} a_{ji} \pi(e_{i}) = \displaystyle \sum_{i \in I} a_{ji}x_{i}$$

\noindent Falarase do $A$-módulo determinado, salvo isomorfismos, polos conxuntos $I,J$ e os elementos $a_{ji}$, como o $A$-módulo con xeradores $\{x_{i}\}_{i \in I}$ e relacións
$$\displaystyle \sum_{i \in I} a_{ji}x_{i} = 0 \hspace{6mm} \forall \hspace{1mm} j \in J$$

\noindent Así pois, cada familia de xeradores e cada familia de relacións entre eles determina, salvo isomorfismos, un único módulo. Non obstante, diferentes xeradores e diferentes relacións pode determinar o mesmo módulo.\\

\noindent Nótese que un módulo é libre cando, e só cando, posúe unha familia de xeradores para a cal a familia de relacións é baleira, i.e. posúe unha familia de xeradores sen relacións entre eles (\textbf{unha base}).\\

\noindent Como exemplo, sexa o anel $A = \mathbb{Z}$ e $M$ o $A$-módulo (i.e. grupo abeliano) con xeradores $\{x_{1}, x_{2}\}$ e relacións $x_{1} + x_{2} = 0$.\\

\noindent Considérese a seguinte presentación libre de $M$:

\begin{center}
    $\varepsilon: \mathbb{Z} \oplus \mathbb{Z} \longrightarrow M$\\
    \vspace{2mm}
    $\hspace{20mm} e_{1} \leadsto \varepsilon(e_{1}) := x_{1}$\\
    $\hspace{20mm} e_{2} \leadsto \varepsilon(e_{2}) := x_{2}$
\end{center}

\noindent Empregando a notación $N = Ker \hspace{1mm} \varepsilon$, tense:
$$N = \langle f \rangle \hspace{5mm} \text{con } f = e_{1} + e_{2}$$

\noindent Tense que $M$ é un grupo cíclico infinito. En efecto, defínase o homomorfismo de módulos:
\begin{center}
    $\phi: \mathbb{Z} \longrightarrow M$\\
    \vspace{2mm}
    $a \leadsto \phi(a) := ax_{1}$
\end{center}

\noindent Cúmprese:

\begin{itemize}
    \item $\phi$ é sobrexectivo
    \[ 
    \left. \begin{array}{r} 
    x_{1} = \phi(1)\\[1ex]
    x_{2} = \phi(-1)
    \end{array} \right\} 
    \implies \text{Os xeradores están na imaxe } \implies Im \hspace{1mm} \phi = M
    \]
    \item $\phi$ é inxectivo\\
    
    Sexa $a \in Ker \hspace{1mm} \phi$. Entón:
    $$\phi(a) = 0 \implies ax_{1} = 0$$
    \[ 
    \left. \begin{array}{r} 
    ax_{1} = 0 \\[1ex]
    ax_{1} = a\varepsilon(e_{1}) = \varepsilon(ae_{1})
    \end{array} \right\} 
    \implies ae_{1} \in Ker \hspace{1mm} \varepsilon = N = \langle f \rangle \\
    \]
    \vspace{5mm}
    Sendo $f = e_{1} + e_{2}$, $\exists \hspace{1mm} b \in \mathbb{Z} \hspace{1mm} | \hspace{1mm} ae_{1} = b(e_{1} + e_{2})$. Así:
    $$ae_{1} = be_{1} + be_{2} \Longleftrightarrow 
    \begin{cases}
    a = b\\
    b = 0
    \end{cases}
    \implies a = 0$$
\end{itemize}

\magbf{\subsection{Rango dun módulo libre}}

\vspace{5mm}

\noindent Sexa $A$ un anel e $\mathcal{I}$ un ideal de $A$. Sexa $N$ un $A$-módulo.\\

\noindent Se $\mathcal{I}N = \{0\}$, entón $N$ é un $\displaystyle \frac{A}{\mathcal{I}}$-módulo coa operación externa:

\begin{center}
    $\displaystyle \frac{A}{\mathcal{I}} \times N \longrightarrow N$\\
    \vspace{2mm}
    $\hspace{15mm} (a + \mathcal{I}, x) \leadsto (a + \mathcal{I})x := ax$
\end{center}

\noindent Está a operación ben definida? Dados $a,a' \in A$ e $x \in N$:

$$a + \mathcal{I} = a' + \mathcal{I} \Longleftrightarrow a - a' \in \mathcal{I} \implies (a - a')x = 0 \Longleftrightarrow ax - a'x = 0 \Longleftrightarrow ax = a'x$$

\noindent Séguese disto que, para cada $A$-módulo $M$, resulta que $\displaystyle \frac{M}{\mathcal{I}M}$ é un $\displaystyle \frac{A}{\mathcal{I}}$-módulo coa operación externa:

\begin{center}
    $\displaystyle \frac{A}{\mathcal{I}} \times \displaystyle \frac{M}{\mathcal{I}M} \longrightarrow \displaystyle \frac{M}{\mathcal{I}M}$\\
    \vspace{2mm}
    $\hspace{30mm} (a + \mathcal{I}, x + \mathcal{I}M) \leadsto (a + \mathcal{I})(x + \mathcal{I}M) := ax + \mathcal{I}M$
\end{center}

\noindent Tense que $\mathcal{I} \cdot \displaystyle \frac{M}{\mathcal{I}M} = 0 + \mathcal{I}M$. Dados $\lambda \in \mathcal{I}$ e $x + \mathcal{I}M \in \displaystyle \frac{M}{\mathcal{I}M}$:
$$\lambda(x + \mathcal{I}M) = \underbrace{\lambda x}_{\in \mathcal{I}M} + \mathcal{I}M = 0 + \mathcal{I}M$$

\noindent Sexa $f: M \longrightarrow N$ un homomorfismo de $A$-módulos. Considérense $M'$ submódulo de $M$ e $N'$ submódulo de $N$. Se $f(M') \subset N'$, entón tense un homomorfismo inducido por $f$:

\begin{center}
    $\overline{f}: \displaystyle \frac{M}{M'} \longrightarrow \displaystyle \frac{N}{N'}$\\
    \vspace{2mm}
    $\hspace{15mm} x + M' \leadsto f(x) + N'$
\end{center}

\noindent En particular, tense un homomorfismo de $A$-módulos:

\begin{center}
    $\overline{f}: \displaystyle \frac{M}{\mathcal{I}M} \longrightarrow \displaystyle \frac{N}{\mathcal{I}N}$\\
    \vspace{2mm}
    $\hspace{15mm} x + \mathcal{I}M \leadsto f(x) + \mathcal{I}N$
\end{center}

\noindent En efecto, $f(\mathcal{I}M) \subset \mathcal{I}N$. Dados $a \in \mathcal{I}$ e $x \in M$, $ax \in \mathcal{I}M$. Logo:
$$f(ax) = \underbrace{a}_{\in \mathcal{I}} \cdot \underbrace{f(x)}_{\in N}$$

\noindent Ademais, $\overline{f}$ verifica ser tamén un homomorfismo de $\displaystyle \frac{A}{\mathcal{I}}$-módulos.\\

\noindent \textbf{Exercicio}. Demostrar que se $f$ é un isomorfismo de $A$-módulos, entón $\overline{f}$ é un isomorfismo de $\displaystyle \frac{A}{\mathcal{I}}$-módulos.\\

\begin{theorem} \label{th3.5}
Sexa $A$ un anel e $\mathcal{I}$ un ideal de $A$. Para calquera conxunto $I$ tense un isomorfismo de $\displaystyle \frac{A}{\mathcal{I}}$-módulos:
\begin{center}
    $\displaystyle \frac{A^{(I)}}{\mathcal{I}A^{(I)}} \simeq \displaystyle \left( \frac{A}{\mathcal{I}} \right)^{(I)}$
\end{center}
\end{theorem}

\vspace{2mm}

\noindent \textbf{\textit{\underline{Demostración}}}

\vspace{2mm}

\noindent A proxección canónica $p: A \longrightarrow \displaystyle \frac{A}{\mathcal{I}}$ induce un homomorfismo:

\begin{center}
    $\theta: A^{(I)} \longrightarrow \displaystyle \left( \frac{A}{\mathcal{I}} \right)^{(I)}$\\
    \vspace{2mm}
    $\hspace{15mm} (x_{i})_{i \in I} \leadsto (p(x_{i}))_{i \in I} = (x_{i} + \mathcal{I})_{i \in I}$
\end{center}

\noindent Este cumpre ser un epimorfismo (a modo de exercicio, pódese demostrar). Determínese o seu núcleo:
$$Ker \hspace{1mm} \theta = K := \{(x_{i})_{i \in I} \in A^{(I)} \hspace{1mm} | \hspace{1mm} x_{i} \in \mathcal{I} \hspace{2mm} \forall \hspace{1mm} i \in I\}$$

\noindent Demóstrese que $K = \mathcal{I}A^{(I)}$:\\

\noindent \fcolorbox{magenta}{white}{$\subset$/} Sexa $(x_{i})_{i \in I} \in A^{(I)}$, con $x_{i} \in \mathcal{I} \hspace{2mm} x_{i} = 0 \hspace{2mm} \almostall i \in I $. Tense:
$$(x_{i})_{i \in I} = \displaystyle \sum_{i \in I} \mu_{i}(x_{i}) = \displaystyle \sum_{i \in I} \mu_{i}(x_{i} \cdot 1) = \displaystyle \sum_{i \in I} x_{i} \cdot \mu_{i}(1) = \displaystyle \sum_{i \in I} \underbrace{x_{i}}_{\in \mathcal{I}} \cdot e_{i} \in \mathcal{I}A^{(I)}$$

\noindent \fcolorbox{magenta}{white}{$\supset$/} Abonda probar que, dados $a \in \mathcal{I}$ e $x \in A^{(I)}$, $ax \in K$.\\

\noindent Tense que $x = (x_{i})_{i \in I}$, con $x_{i} \in A$, $x_{i} = 0 \hspace{2mm} \almostall i \in I$. Cúmprese:
$$ax = a(x_{i})_{i \in I} = (\underbrace{ax_{i}}_{\in \mathcal{I}})_{I \in I} \in K$$

\noindent Así, aplicando o \hyperref[th3.1]{\magbf{1.º Teorema de Isomorfía de módulos}}, obtense o resultado. $\square$\\

\vspace{3mm}

\begin{theorem} \label{th3.6}
Tódalas bases dun módulo libre posúen o mesmo cardinal.
\end{theorem}

\vspace{2mm}

\noindent \textbf{\textit{\underline{Demostración}}}

\vspace{2mm}

\noindent Sexa $A$ un anel e $F$ un $A$-módulo libre. Considérense $\{x_{i}\}_{i \in I}$ e $\{y_{j}\}_{j \in J}$ senllas bases de $F$. Tense:

    \[ 
    \left. \begin{array}{r} 
    \{x_{i}\}_{i \in I} \text{ base} \implies F \simeq A^{(I)} \\[1ex]
    \{y_{j}\}_{j \in J} \text{ base} \implies F \simeq A^{(J)} 
    \end{array} \right\} 
    \implies A^{(I)} \overset{f}{\simeq} A^{(J)}
    \]
    
\vspace{2mm}

\noindent En base a unha observación anterior, $\overline{f}$ induce un isomorfismo $\displaystyle \frac{A^{(I)}}{\mathcal{M}A^{(I)}} \overset{\overline{f}}{\simeq} \displaystyle \frac{A^{(J)}}{\mathcal{M}A^{(J)}}$.\\
    
\noindent Sendo $A$ un anel, existe $\mathcal{M}$ un ideal maximal de $A$. Entón, en base ao \hyperref[th3.5]{\magen{teorema anterior}}, tense:
$$\displaystyle \left( \frac{A}{\mathcal{M}} \right)^{(I)} \overset{\lambda_{1}}{\simeq} \displaystyle \frac{A^{(I)}}{\mathcal{M}A^{(I)}} \hspace{15mm} \frac{A^{(J)}}{\mathcal{M}A^{(J)}} \overset{\lambda_{2}}{\simeq} \displaystyle \left( \frac{A}{\mathcal{M}} \right)^{(J)}$$

\noindent Sendo $\mathcal{M}$ un ideal maximal, $\displaystyle \frac{A}{\mathcal{M}}$ é un corpo. Polo tanto, $\overline{f}, \lambda_{1}, \lambda_{2}$ son isomorfismos de $K$-espazos vectoriais. \\

\noindent Tense entón o seguinte esquema:
$$\displaystyle \left( \frac{A}{\mathcal{M}} \right)^{(I)} \overset{\lambda_{1}}{\longrightarrow} \displaystyle \frac{A^{(I)}}{\mathcal{M}A^{(I)}} \overset{\overline{f}}{\longrightarrow} \frac{A^{(J)}}{\mathcal{M}A^{(J)}} \overset{\lambda_{2}}{\longrightarrow} \displaystyle \left( \frac{A}{\mathcal{M}} \right)^{(J)} $$

\noindent Verifícase que $\lambda_{2} \circ \overline{f} \circ \lambda_{1}$ é un isomorfimo de $K$-espazos vectoriais como composición de isomorfismos de $K$-espazos vectoriais. Así, en particular, $\displaystyle \left( \frac{A}{\mathcal{M}} \right)^{(I)} = K^{(I)}$ e $\displaystyle \left( \frac{A}{\mathcal{M}} \right)^{(J)} = K^{(J)}$ son isomorfos. Logo, as súas dimensións (i.e. o número de elementos das súas bases) coinciden. Logo, $\#I = \#J$. $\square$\\

\vspace{3mm}

\noindent Este último teorema permite introducir a seguinte definición:\\

\noindent \textbf{Definición 3.20}. Sexa $A$ un anel e $F$ un $A$-módulo libre. Defínese o \magbf{rango de $F$} como o cardinal dunha base de $F$.\\

\noindent \textbf{Observación 3.12}. No caso dos espazos vectoriais, os conceptos de rango e dimensión coinciden.\\

\newpage

\chapter[Unidade 4. Módulos de tipo finito sobre un dominio de ideais principais]{\textbf{Módulos de tipo finito sobre un dominio de ideais principais}}

\thispagestyle{noheader}

\noindent Os módulos de tipo finito sobre un dominio de ideais principais constitúen o equivalente aos espazos vectoriais de dimensión finita sobre un corpo. Ao longo deste tema farase un estudo dos mesmos mantendo analoxías co que se ten feito nas materias de \textit{Espazos Vectoriais e Cálculo Matricial} e \textit{Álxebra Linear e Multilinear}. \\

\noindent En primeiro lugar, traballarase con matrices cuxos coeficientes están sobre un dominio de ideais principais e as operacións elementais que se poden realizar sobre elas, co obxectivo de demostrar que calquera matriz deste tipo admite unha forma diagonal, nun sentido máis amplo có das matrices cadradas.\\

\noindent A continuación, procederase á demostración do resultado fundamental desta unidade: o teorema de estrutura, que non é máis ca un resultado equivalente ao teorema de existencia da forma canónica de Jordan. Posteriormente, centrarémonos en profundizar nesta descomposición, coa finalidade de achar elementos que permitan clasificar, salvo isomorfismos, os módulos de tipo finito sobre un DIP.\\

\magbf{\section{Equivalencia de matrices. Diagonalización}}

\vspace{5mm}

\noindent Nesta sección, $R$ denotará un anel conmutativo.\\

\begin{lemma} \label{lem4.1}
Sexa $E_{ij}$ a matriz cadrada sobre $R$ que cumpre $e_{ij} = 1$, sendo os restantes elementos nulos. Verifícase:
\begin{enumerate}
    \item $$E_{ij} \cdot E_{rs} = 
    \begin{cases}
    0 \hspace{8mm} j \neq r\\
    E_{is} \hspace{5mm} j = r
    \end{cases}$$ 
    \item Para unha matriz $A$, tense:
        \begin{itemize}
            \item $E_{ij} \cdot A$ é a matriz que ten tódalas filas nulas, excepto a $i$-ésima, que coincide coa $j$-ésima de $A$.
            \item $A \cdot E_{ij}$ é a matriz que ten tódalas columnas nulas, excepto a $j$-ésima, que coincide coa $i$-ésima de $A$.
        \end{itemize}
\end{enumerate}
\end{lemma}

\vspace{5mm}

\noindent \textbf{Definición 4.1}. Denomínanse \magbf{matrices elementais} ás de cada un dos seguintes tipos:

\renewcommand{\theenumi}{\roman{enumi})}
\renewcommand{\labelenumi}{\theenumi}

\begin{enumerate}
    \item Dado un elemento $b \in R$ e índices $i,j$, con $i \neq j$:
    
    $$T_{ij}(b):= I + b \cdot E_{ij}$$
    
    \begin{center}
    
    \begin{tikzpicture}[>=stealth,thick,baseline]
    
    \matrix [matrix of math nodes,left delimiter=(,right delimiter=)](A){ 
    1 & 0 & \cdots & 0 & \cdots & 0 & \cdots & 0 \\
    0 & 1 & \cdots & 0 & \cdots & 0 & \cdots & 0 \\
    \vdots & \vdots & \ddots & \vdots & \ddots & \vdots & \ddots & \vdots \\
    0 & 0 & \cdots & 1 & \cdots & b & \cdots & 0 \\
    \vdots & \vdots & \ddots & \vdots & \ddots & \vdots & \ddots & \vdots \\
    0 & 0 & \cdots & 0 & \cdots & 0 & \cdots & 0 \\
   };
   
    \node[left =20pt of A-4-1.west](L)  {Fila $i$};
   
    \node[above =10pt of A-1-6.north](D)  {Columna $j$};

    \draw[->, shorten > =12pt](L.east)-- (A-4-1.west);
    \draw[->](D.south)-- (A-1-6.north);
    
    \end{tikzpicture}
        
    \end{center}
    
    \vspace{5mm}

    Esta matriz é inversible, con $[T_{ij}(b)]^{-1} = T_{ij}(-b)$. En efecto:
    
    $$(I + b \cdot E_{ij}) \cdot (I - b \cdot E_{ij}) = I - b \cdot E_{ij} + b \cdot E_{ij} + b^{2} \cdot E_{ij} \cdot E_{ij} \underset{\underset{\hyperref[lem4.1]{\magbf{Lema 4.1}}}{\Uparrow}}{\overset{\overset{i \neq j}{\Downarrow}}{=}} I$$

    \item Sexa $u \in \mathcal{U}(R)$:
    
    $$D_{i}(u) := I + (u-1) \cdot E_{ii} = I + u \cdot E_{ii} - E_{ii}$$
    
    \begin{center}
    
    \begin{tikzpicture}[>=stealth,thick,baseline]
    
    \matrix [matrix of math nodes,left delimiter=(,right delimiter=)](A){ 
    1 & 0 & \cdots & 0 & \cdots & 0 \\
    0 & 1 & \cdots & 0 & \cdots & 0 \\
    \vdots & \vdots & \ddots & \vdots & \ddots & \vdots \\
    0 & 0 & \cdots & u & \cdots & 0 \\
    \vdots & \vdots & \ddots & \vdots & \ddots & \vdots \\
    0 & 0 & \cdots & 0 & \cdots & 0 \\
   };
   
    \node[left =20pt of A-4-1.west](L)  {Fila $i$};
   
    \node[above =10pt of A-1-4.north](D)  {Columna $i$};

    \draw[->, shorten > =12pt](L.east)-- (A-4-1.west);
    \draw[->](D.south)-- (A-1-4.north);
    
    \end{tikzpicture}
        
    \end{center}
    
    Esta matriz é tamén inversible, con $[D_{i}(u)]^{-1} = D_{i}(u^{-1})$. Certamente:
    
    $$[I + (u-1) \cdot E_{ii}] \cdot [I + (u^{-1} -1) \cdot E_{ii}] = $$
    $$ = I + (u^{-1} -1) \cdot E_{ii} + (u-1) \cdot E_{ii} + (u -1) \cdot (u^{-1} -1 )\cdot E_{ii} \cdot E_{ii} =$$
    $$ = I + u^{-1}E_{ii} - E_{ii} + uE_{ii} - E_{ii} + E_{ii}  - uE_{ii} - u^{-1}E_{ii} + E_{ii} = I  $$
    
    \item Dados índices $i,j$:
    
    $$P_{ij} := I - E_{ii} - E_{jj} + E_{ij} + E_{ji}$$
    
    \begin{center}
        
    \begin{tikzpicture}[>=stealth,thick,baseline]
    
    \matrix [matrix of math nodes,left delimiter=(,right delimiter=)](A){ 
    1 & 0 & \cdots & 0 & \cdots & 0 & \cdots & 0 \\
    0 & 1 & \cdots & 0 & \cdots & 0 & \cdots & 0 \\
    \vdots & \vdots & \ddots & \vdots & \ddots & \vdots & \ddots & \vdots \\
    0 & 0 & \cdots & 0 & \cdots & 1 & \cdots & 0 \\
    \vdots & \vdots & \ddots & \vdots & \ddots & \vdots & \ddots & \vdots \\
    0 & 0 & \cdots & 1 & \cdots & 0 & \cdots & 0 \\
    \vdots & \vdots & \ddots & \vdots & \ddots & \vdots & \ddots & \vdots \\
    0 & 0 & \cdots & 0 & \cdots & 0 & \cdots & 0 \\
   };

   \node[left =30pt of A-4-1.west](L)  {Fila $i$};

   \node[below =20pt of A-8-4.south](C) {Columna $i$};

   \node[right =30pt of A-6-8.east](K)  {Fila $j$};
   
    \node[above =20pt of A-1-6.north](D)  {Columna $j$};

    \draw[->, shorten > =12pt](L.east)-- (A-4-1.west);
    \draw[->](C.north)-- (A-8-4.south);
    \draw[->, shorten > =12pt](K.west)-- (A-6-8.east);
    \draw[->](D.south)-- (A-1-6.north);

    \end{tikzpicture}
        
    \end{center}
    
    \vspace{5mm}
    
    Esta matriz é inversible, con $P_{ij}^{-1} = P_{ij}$. En efecto:
    
    $$P_{ij} \cdot P_{ij} = (I - E_{ii} - E_{jj} + E_{ij} + E_{ji}) \cdot (I - E_{ii} - E_{jj} + E_{ij} + E_{ji}) = $$
    $$ = (I - E_{ii} - E_{jj} + E_{ij} + E_{ji}) + (-\cancel{E_{ii}} + \cancel{E_{ii}} + 0 - E_{ij} - 0) + (-\cancel{E_{jj}} + 0 + \cancel{E_{jj}} - 0 - E_{ji}) + $$
    $$ + (\cancel{E_{ij}} - 0 - \cancel{E_{ij}} + 0 + E_{ii}) + (\cancel{E_{ji}} - \cancel{E_{ji}} -0 + E_{jj} + 0) = $$
    $$ = I - \cancel{E_{ii}} - \cancel{E_{jj}} + \cancel{E_{ij}} + \cancel{E_{ji}} - \cancel{E_{ij}} - \cancel{E_{ji}} + \cancel{E_{ii}} + \cancel{E_{jj}} = I$$
    
    \vspace{5mm}
    
    Obsérvese que esta demostración se fixo para o caso $i \neq j$. Se $i = j, P_{ij} = I$, sendo entón un caso trivial.\\
    
\end{enumerate}

\noindent \textbf{Definición 4.2}. Sexa $A \in M_{m \times n}(R)$. Denomínase \magbf{operacións elementais en $A$} á multiplicación de $A$ (pola esquerda ou pola dereita) por matrices.\\

\noindent As operacións elementais en $A$ son, polo tanto, as seguintes:

\begin{enumerate}
    \item $T_{ij}(b) \cdot A$ é a matriz obtida a partir de $A$ sumándolle a fila $i$-ésima de $A$ a fila $j$-ésima multiplicada por $b$
    
    $$T_{ij}(b) \cdot A = A + b \cdot E_{ij}A$$
    
    $A \cdot T_{ij}(b)$ é a matriz obtida a partir de $A$ sumándolle á columna $j$-ésima a columna $i$-ésima multiplicada por $b$
    
    $$A \cdot T_{ij}(b) = A + b \cdot AE_{ij}$$
    
    \item $D_{i}(u) \cdot A$ é a matriz obtida a partir de $A$ multiplicando por $u$ a súa fila $i$-ésima
    
    $$D_{i}(u) \cdot A = [I + (u-1)E_{ii}]A = A + uE_{ii}A - E_{ii}A$$
    
    $A \cdot D_{i}(u)$ é a matriz obtida a partir de $A$ multiplicando por $u$ a súa columna $i$-ésima
    
    $$A \cdot D_{i}(u) = A[I + (u-1)E_{ii}] = A + uAE_{ii} - AE_{ii}$$
    
    \item $P_{ij} \cdot A$ é a matriz obtida de $A$ intercambiando as filas $i$ e $j$
    
    $$P_{ij} \cdot A = A - E_{ii}A -E_{jj}A + E_{ij}A + E_{ji}A$$
    
    $A \cdot P_{ij}$ é a matriz obtida a partir de $A$ intercambiando as columnas $i$ e $j$
    
    $$A \cdot P_{ij} = A - AE_{ii} - AE_{jj} + AE_{ij} + AE_{ji}$$

\end{enumerate}

\vspace{5mm}

\noindent \textbf{Definición 4.3}. Sexan $A,B \in M_{m \times n}(R)$. $A$ e $B$ dinse \magbf{matrices equivalentes} se existen matrices inversibles $Q \in M_{m}(R)$, $P \in M_{n}(R)$ tales que $B = QAP$.\\

\noindent \textbf{Definición 4.4}. Sexa $D$ un dominio de ideais principais e $a \in D$, con $a \neq 0$. Denomínase a \magbf{lonxitude de $a$}, denotada por \magen{$l(a)$} ao número de elementos irreducibles da factorización de $a$:

$$a = u \cdot p_{1} \cdot p_{2} \cdot \dots \cdot p_{t} \implies l(a) := t$$

\noindent con $u \in \mathcal{U}(D)$ e $p_{1}, p_{2}, \dots, p_{t}$ elementos irreducibles non innecesariamente distintos.\\

\noindent Obsérvese que, se $a \in \mathcal{U}(D)$, $l(a) = 0$.\\

\begin{theorem}[\magbf{Teorema de diagonalización sobre un DIP}] \label{th4.1}
Sexa $D$ un dominio de ideais principais. Se $A \in M_{m \times n}(D)$, entón $A$ é equivalente a unha matriz diagonal da forma:
\begin{center}
    $diag(d_{1}, \dots, d_{r}, 0, \dots, 0)$ 
\end{center} 
con $0 \leq r \leq$ mín$\{m,n\}$, $d_{i} \neq 0$ e $d_{i} | d_{j}$ se $i \leq j$.
\end{theorem}

\vspace{2mm}

\noindent Cómpre precisar que se fai referencia á diagonal para matrices en xeral, non necesariamente cadradas. Neste sentido, chámaselle diagonal dunha matriz aos elementos da forma $a_{ii}$, con $1 \leq i \leq m,n$.\\

\noindent \textbf{\textit{\underline{Demostración}}}

\vspace{2mm}

\noindent Se $A = 0$, non hai nada que demostrar.\\

\noindent Supóñase $A \neq 0$. Tratarase, en primeiro lugar, de obter matrices equivalentes a $A$ cun elemento no lugar (1,1) distinto de 0 e da menor lonxitude posible.\\

\noindent Sexa $a_{ij}$ un elemento non nulo de $A$ con lonxitude mínima. Intercambiando filas e columnas, lévase este elemento á posición (1,1). Tense entón unha matriz equivalente a $A$ (a cal seguirá sendo denotada por $A$), con $a_{11} \neq 0$ e $l(a_{11})$ mínima. Pode suceder:\\

\renewcommand{\theenumi}{\arabic{enumi}}
\renewcommand{\labelenumi}{\theenumi.}

\begin{enumerate}
    \item Se existe na primeira fila un elemento $a_{1k} \neq 0$ con $a_{11} | a_{1k}$, lévase $a_{1k}$ á posición (1,2) por intercambio de columnas. Denotarase tamén por $A$ a nova matriz.\\
    
    Sexan $a := a_{11}$, $b := a_{12}$. Defínase $d := mcd(a,b)$. Sendo $D$ un DIP, aplicando o \hyperref[lem2.2]{\magbf{Teorema de Bézout}}, obtense:
    
    $$(d) = (a) + (b) \implies \exists \hspace{1mm} \alpha, \beta \in D \hspace{1mm} | \hspace{1mm} d = \alpha a + \beta b$$
    
    Ademais:
    
    \[ 
    \left. \begin{array}{r} 
    a \in (d) \implies \exists \hspace{1mm} s \in D \hspace{1mm} | \hspace{1mm} a = sd \\[1ex]
    b \in (d) \implies \exists \hspace{1mm} t \in D \hspace{1mm} | \hspace{1mm} b = td \\[1ex]
    \end{array} \right\} 
    \implies d = \alpha sd + \beta td \implies \alpha s + \beta t = 1
    \]
    
    Como $d | a$, $d | b$ e $a \nmid b$, resulta que os factores irreducibles de $d$ están estritamente contidos nos factores irreducibles de $a$, e así $l(d) < l(a)$. \\
    
    Tense:
    
    $$
    \begin{pmatrix}
    s & t \\
    -\beta & \alpha
    \end{pmatrix}
    \begin{pmatrix}
    \alpha & -t\\
    \beta & s
    \end{pmatrix}
    = 
    \begin{pmatrix}
    1 & 0\\
    0 & 1
    \end{pmatrix}
    =
    \begin{pmatrix}
    \alpha & -t \\
    \beta & s
    \end{pmatrix}
    \begin{pmatrix}
    s & t\\
    -\beta & \alpha
    \end{pmatrix}
    $$
    
    logo a matriz 
    $
    \begin{pmatrix}
    \alpha & -t\\
    \beta & s
    \end{pmatrix}
    $
    é inversible. Tamén o é a seguinte matriz:
    
    $$
    U = \begin{pmatrix}
    \alpha & -t & 0 & \cdots & 0\\
    \beta & s & 0 & \cdots & 0\\
    0 & 0 & 1 & \cdots & 0\\
    \vdots & \vdots & \vdots & \ddots & \vdots\\
    0 & 0 & 0 & \cdots & 1
    \end{pmatrix}
    \in M_{n}(D)
    $$
    
    \vspace{3mm}
    
    Así, a matriz $AU$ é equivalente a $A$, e a súa primeira fila é $(d, 0, a_{13}, \dots, a_{1n})$. Nótese que a equivalencia entre $A$ e $AU$ vén dada pola \textbf{Definición 3.3}, pois $AU = I \cdot A \cdot U$. Non obstante, o paso de $A$ a $AU$ non se dá mediante operacións elementais.\\
    
    Pódese seguir diminuíndo a lonxitude do elemento do lugar (1,1) mentres exista na 1.ª fila un elemento $a_{1k} \neq 0$ tal que $a_{11} \nmid a_{1k}$ ou, analogamente, mentres exista na 1.ª columna un elemento $a_{k1} \neq 0$ tal que $a_{11} | a_{k1}$ (neste caso multiplícase pola esquerda por unha matriz inversible). Repetirase o procedemento que se acaba de ver para a primeira fila aplicado esta vez á primeira columna:\\
    
    ,Fáganse $a := a_{11}$ e $b := a_{21}$, e defínase $d := mcd(a,b)$. Aplicando o \hyperref[lem2.2]{\magbf{Teorema de Bézout}}:
    
    $$(d) = (a) + (b) \implies \exists \hspace{1mm} \alpha, \beta \in D \hspace{1mm} | \hspace{1mm} d = \alpha a + \beta b$$
    
    \[ 
    \left. \begin{array}{r} 
    a \in (d) \implies \exists \hspace{1mm} s \in D \hspace{1mm} | \hspace{1mm} a = sd \\[1ex]
    b \in (d) \implies \exists \hspace{1mm} t \in D \hspace{1mm} | \hspace{1mm} b = td \\[1ex]
    \end{array} \right\} 
    \implies d = \alpha sd + \beta td \implies \alpha s + \beta t = 1
    \]
    
    Tense neste caso:
    
    $$
    \begin{pmatrix}
    \alpha & \beta \\
    t & -s
    \end{pmatrix}
    \begin{pmatrix}
    s & \beta \\
    t & -\alpha
    \end{pmatrix}
    =
    \begin{pmatrix}
    1 & 0\\
    0 & 1
    \end{pmatrix}
    =
    \begin{pmatrix}
    s & \beta \\
    t & -\alpha
    \end{pmatrix}
    \begin{pmatrix}
    \alpha & \beta \\
    t & -s
    \end{pmatrix}
    $$
    
    Así, a matriz $\begin{pmatrix} \alpha & \beta \\ t & -s \end{pmatrix}$ é inversible, logo tamén o é a matriz:
    
    $$
    V = \begin{pmatrix} \alpha & \beta & 0 & \cdots & 0 \\ 
                    t & -s & 0 & \cdots & 0\\
                    0 & 0 & 1 & \cdots & 0 \\
                    \vdots & \vdots & \vdots & \ddots & \vdots\\
                    0 & 0 & 0 & \cdots & 1
    \end{pmatrix}
    \in M_{m}(D)
    $$
    
    Deste xeito, a matriz \textit{VA} é equivalente a $A$, e a súa primeira columna é $\begin{pmatrix} d \\ 0 \\ a_{31} \\ \vdots \\ a_{m1} \end{pmatrix}$.\\
    
    
    \item Se $a_{11} | a_{1k}$, $1 \leq k \leq n$ e $a_{11} | a_{l1}$, $1 \leq l \leq m$, entón tense:
    
    $$
    a_{1k} = a_{11} \cdot r_{1k} \hspace{5mm} r_{1k} \in D
    $$
    $$
    a_{l1} = a_{11} \cdot r_{l1} \hspace{5mm} r_{l1} \in D
    $$
    
    Restándolle á columna $k$-ésima (resp. fila $l$-ésima) a 1.ª columna (resp. 1.ª fila) multiplicada por $r_{1k}$ (resp. $r_{l1}$) obtense unha matriz equivalente a $A$ da forma:
    
    $$
    \begin{pmatrix}
    a_{11} & 0 & 0 & \cdots & 0 \\
    0 & b_{22} & b_{23} & \cdots & b_{2n} \\
    0 & b_{32} & b_{33} & \cdots & b_{3n} \\
    \vdots & \vdots & \vdots & \ddots & \vdots \\
    0 & b_{m2} & b_{m3} & \cdots & b_{mn}
    \end{pmatrix}_{m \times n}
    $$
    
    Se na matriz anterior $\exists \hspace{1mm} b_{kl}$ tal que $a_{11} \nmid b_{kl}$, entón sumándolle á 1.ª fila a fila $k$-ésima, obtense unha nova matriz equivalente cuxa 1.ª fila é $\begin{pmatrix} a_{11} & b_{k2} & \cdots & b_{kl} & \cdots & b_{kn} \end{pmatrix}$, podendo aínda diminuír a lonxitude do elemento na posición (1,1).\\
    
    Repetindo este proceso, chégase a unha matriz do tipo anterior con $a_{11} | b_{kl}$, con $2 \leq k \leq m$, $2 \leq l \leq n$.\\
    
\end{enumerate}

\noindent O mesmo procedemento, i.e. todo o feito ata aquí, aplícase agora á submatriz

$$
B = \begin{pmatrix}
b_{22} & \cdots & b_{2n}\\
\vdots & \ddots & \vdots \\
b_{m2} & \cdots & b_{mn}
\end{pmatrix}
$$

\noindent As transformacións necesarias na submatriz non alterarán nin a 1.ª fila nin a 1.ª columna da matriz $A$ de partida.\\

\noindent Así, repetindo o proceso para a submatriz $B$, obtense unha matriz equivalente á de partida, da forma

$$
\begin{pmatrix}
a_{11} & 0 & 0 & \cdots & 0\\
0 & b_{22} & 0 & \cdots & 0 \\
0 & 0 & c_{33} & \cdots & c_{3n}\\
\vdots & \vdots & \vdots & \ddots & \vdots \\
0 & 0 & c_{m3} & \cdots & c_{mn} 
\end{pmatrix}_{m \times n}
\hspace{10mm}
\text{con } b_{22} | c_{kl} \hspace{3mm} \forall \hspace{1mm} k \in \{3, \dots, m\} \hspace{2mm} \forall \hspace{1mm} l \in \{3, \dots, n\} 
$$

\vspace{3mm}

\noindent Agora ben, cómpre preguntarse: por que $a_{11} | b_{22}$?\\

\noindent Por ser $a_{11} | b_{kl}$ \hspace{2mm} $\forall k \in \{2, \dots, m\} \hspace{2mm} \forall \hspace{1mm} l \in \{2, \dots, n\}$, a divisibilidade segue verificándose ao substituír $b_{kl}$ polo seu mcd e ao efectuar operacións elementais.\\

\noindent A continuación do proceso conduce á matriz diagonal indicada con $d_{1} = a_{11}, d_{2} = b_{22}, \dots \hspace{5mm} \square$\\

\vspace{2mm}

\noindent \textbf{Observación 4.1}. Se $D$ é un dominio euclídeo, pódese aplicar o mesmo procedemento da demostración, pero substituíndo a lonxitude polo valor $\Phi(a)$, sendo $\Phi$ a función euclídea.\par
\noindent Entón, a diagonalización conséguese mediante \textit{unicamente} operacións elementais.\\

\noindent En efecto, se existe na 1.ª fila un elemento $a_{1k} \neq 0$ con $a_{11} \nmid a_{1k}$, lévase $a_{1k}$ á posición (1,2). Facendo $a := a_{11}$ e $b := a_{12}$, sendo $D$ un dominio euclídeo, $\exists \hspace{1mm} q,r \in D \hspace{1mm} | \hspace{1mm} b = aq +r$, con $r \neq 0$ (pois $a \nmid b$) e $\Phi(r) < \Phi(a)$.\\

\noindent Restándolle á segunda columna a primeira multiplicada por $q$, obtense unha matriz equivalente da forma

$$
\begin{pmatrix}
a & r & \cdots \\
a_{21} & a_{22} - a_{21}q & \cdots \\
\vdots & \vdots & \ddots\\
a_{m1} & a_{m2} - a_{m1}q & \cdots
\end{pmatrix}
$$

\vspace{5mm}

\noindent Intercambiando as dúas primeiras columnas:

$$
\begin{pmatrix}
r & a & \cdots\\
a_{22} - a_{21}q & a_{21} & \cdots\\
\vdots & \vdots & \ddots \\
a_{m2}- a_{m1}q & a_{m1} & \cdots 
\end{pmatrix}
$$

\vspace{5mm}

\noindent Véxase a continuación un exemplo de aplicación deste teorema. Considérese a seguinte matriz:

$$
A = \begin{pmatrix}
2 & 3 & 1 \\
-3 & 2 & 4
\end{pmatrix}
\in M_{2 \times 3}(\mathbb{Z})
$$

\noindent Obsérvese que esta matriz ten os seus coeficientes en $\mathbb{Z}$, que é un dominio euclídeo xunto coa función valor absoluto.\\

\noindent Procederase á diagonalización da matriz aplicando o teorema anterior:

$$
A = \begin{pmatrix}
2 & 3 & 1 \\
-3 & 2 & 4
\end{pmatrix}
\overset{\textbf{[1]}}{\underset{C_{1} \leftrightarrow C_{3}}{\xrightarrow{\hspace{15mm}}}}
\begin{pmatrix}
1 & 3 & 2 \\
4 & 2 & -3
\end{pmatrix}
\overset{\textbf{[2]}}{\underset{\substack{C_{2} - 3C_{1} \\ C_{3} - 2C_{1}}}{\xrightarrow{\hspace{15mm}}}}
\begin{pmatrix}
1 & 0 & 0 \\
4 & -10 & -11
\end{pmatrix}
\longrightarrow
$$

$$
\overset{\textbf{[3]}}{\underset{F_{2} - 4F_{1}}{\xrightarrow{\hspace{15mm}}}}
\begin{pmatrix}
1 & 0 & 0 \\
0 & -10 & -11
\end{pmatrix}
\overset{\textbf{[4]}}{\underset{(-1)F_{2}}{\xrightarrow{\hspace{15mm}}}}
\begin{pmatrix}
1 & 0 & 0 \\
0 & 10 & 11
\end{pmatrix}
\overset{\textbf{[5]}}{\underset{C_{3} - C_{2}}{\xrightarrow{\hspace{15mm}}}}
\begin{pmatrix}
1 & 0 & 0 \\
0 & 10 & 1
\end{pmatrix}
\longrightarrow
$$

$$
\overset{\textbf{[6]}}{\underset{C_{2} \leftrightarrow C_{3}}{\xrightarrow{\hspace{15mm}}}}
\begin{pmatrix}
1 & 0 & 0 \\
0 & 1 & 10
\end{pmatrix}
\overset{\textbf{[7]}}{\underset{C_{3} - 10C_{2}}{\xrightarrow{\hspace{15mm}}}}
\begin{pmatrix}
1 & 0 & 0 \\
0 & 1 & 0
\end{pmatrix}
=B
$$

\vspace{5mm}

\noindent \textbf{[1]} Débese levar á posición (1,1), mediante intercambio de filas e columnas, o elemento da matriz con menor valor da función euclídea (neste caso, menor valor absoluto).

\noindent \textbf{[2]} Realízase o proceso de eliminación na fila 1 (isto é posible porque o elemento na posición (1,1) divide ao resto de elementos da fila).

\noindent \textbf{[3]} Realízase o proceso de eliminación na columna 1 (isto é posible porque o elemento na posición (1,1) divide ao resto de elementos da columna).

\noindent \textbf{[4]} Este paso realmente non sería necesario porque se chegaría igualmente a unha matriz diagonal sen realizalo. Non obstante, leva tamén a unha matriz diagonal, distinta, pero que cumpre igualmente as condicións expostas no teorema. (De feito, de non se facer este paso neste momento, senón unha vez se obtén a matriz diagonal, a matriz obtida tamén é diagonal e cumpre as condicións do teorema, porque se mantén a divisibilidade entre os elementos da diagonal e a equivalencia coa matriz de partida).

\noindent \textbf{[5]} Como $10 \nmid 11$, non se pode facer a eliminación. Polo tanto, réstaselle á terceira columna a segunda multiplicada polo cociente obtido na división enteira 11/10 = 1, obtendo así o resto da mesma.

\noindent \textbf{[6]} Débese levar á posición (2,2), mediante intercambio de filas e columnas, o elemento da matriz con menor valor absoluto.

\noindent \textbf{[7]} Realízase o proceso de eliminación na fila 2.\\

\noindent Na diagonal da matriz $B$ téñense os elementos $d_{1} = 1$ e $d_{2} = 1$, cumpríndose que $d_{1} | d_{2}$.\\

\noindent Cúmprese que $B$ é equivalente á matriz $A$, i.e. existen matrices inversibles $Q$ e $P$ tales que $B = QAP$.\\

\noindent Doutra banda, a matriz $B$ obtívose a partir de operacións elementais sobre a matriz $A$. Lémbrese que:

\begin{itemize}
    \item Unha operación elemental por columnas é unha multiplicación pola dereita de $A$ e unha matriz elemental.
    \item Unha operación elemental por filas é unha multiplicación pola esquerda de $A$ e unha matriz elemental.
\end{itemize}

\noindent Así, pódese escribir:

$$B = \overbrace{U_{q} \cdot \dots \cdot U_{1}}^{Q} \cdot A \cdot \underbrace{V_{1} \cdot \dots \cdot V_{p}}_{P}$$

\noindent Deste xeito, $P$ é o resultado de facer na matriz identidade $I_{n}$ as mesmas operacións elementais por columnas que se fixeron en $A$, e $Q$ é o resultado de facer en $I_{m}$ as mesmas operacións elementais por filas que se realizaron en $A$. (Neste caso particular, $m = 2$ e $n = 3$).\\

\noindent Cálculo de $P$:

$$
I_{3} = \begin{pmatrix}
1 & 0 & 0\\
0 & 1 & 0\\
0 & 0 & 1
\end{pmatrix}
\underset{C_{1} \leftrightarrow C_{3}}{\xrightarrow{\hspace{15mm}}}
\begin{pmatrix}
0 & 0 & 1\\
0 & 1 & 0\\
1 & 0 & 0
\end{pmatrix}
\underset{\substack{C_{2} - 3C_{1} \\ C_{3} - 2C_{1}}}{\xrightarrow{\hspace{15mm}}}
\begin{pmatrix}
0 & 0 & 1\\
0 & 1 & 0\\
1 & -3 & -2
\end{pmatrix}
\longrightarrow
$$

\vspace{3mm}

$$
\underset{C_{3} - C_{2}}{\xrightarrow{\hspace{15mm}}}
\begin{pmatrix}
0 & 0 & 1\\
0 & 1 & -1\\
1 & -3 & 1
\end{pmatrix}
\underset{C_{2} \leftrightarrow C_{3}}{\xrightarrow{\hspace{15mm}}}
\begin{pmatrix}
0 & 1 & 0\\
0 & -1 & 1\\
1 & 1 & -3
\end{pmatrix}
\underset{C_{3} - 10C_{2}}{\xrightarrow{\hspace{15mm}}}
\begin{pmatrix}
0 & 1 & -10\\
0 & -1 & 11\\
1 & 1 & -13
\end{pmatrix}
= P
$$

\vspace{5mm}

\noindent Cálculo de $Q$:

$$
I_{2} = \begin{pmatrix}
1 & 0\\
0 & 1
\end{pmatrix}
\underset{F_{2} - 4F_{1}}{\xrightarrow{\hspace{15mm}}}
\begin{pmatrix}
1 & 0\\
-4 & 1
\end{pmatrix}
\underset{(-1)F_{2}}{\xrightarrow{\hspace{15mm}}}
\begin{pmatrix}
1 & 0\\
4 & -1
\end{pmatrix}
 = Q
$$

\vspace{5mm}

\noindent Obsérvese que $P$ e $Q$ teñen que ser matrices inversibles, pois obtéñense como produto de matrices elementais, as cales son, en particular, inversibles.\\

\magbf{\section{O teorema de estrutura}}

\vspace{5mm}

\noindent O obxectivo desta sección é demostrar o resultado central desta unidade: o teorema de estrutura de módulos de tipo finito sobre un dominio de ideais principais.\\

\noindent Este é unha xeneralización do teorema fundamental dos grupos abelianos finitamente xerados, o cal, á súa vez, xeneraliza o teorema fundamental dos grupos abelianos finitos. A historia e a autoría destes resultados é incerta, posto que foron demostrados antes de que a teoría de grupos fose establecida. \\

\noindent Unha primeira forma do caso finito foi demostrada por Carl Friedrich Gauss (1777 - 1855), na que se clasificaban as formas cuadráticas. Posteriormente, Leopold Kronecker (1823 - 1891) demostrou o teorema fundamental para o caso finito, pero sen empregar termos da teoría de grupos; os primeiros en facelo foron Ferdinand Georg Frobenius (1849 - 1917) e Ludwig Stickelberger (1850 - 1936). Henri Poincaré (1854 - 1900) probou o teorema fundamental para grupos abelianos finitamente xerados, e ademais empregando matrices.\\

\noindent Antes de presentar o teorema, tense que definir o concepto principal que involucra:\\

\noindent \textbf{Definición 4.5}. Sexa $R$ un anel e $M$ un $R$-módulo. Dise que $M$ é un \magbf{módulo de tipo finito} ou un \magbf{módulo finitamente xerado} se existe un conxunto finito de elementos $\{x_{1}, \dots, x_{m}\} \subset M$ tal que todo elemento de $M$ se pode escribir como combinación linear deses elementos con coeficientes no anel $R$; isto é, que $M$ admite unha familia de xeradores finita. \\ 

\noindent Para a demostración do teorema hai que se apoiar nos seguintes resultados previos:\\

\begin{proposition} \label{prop4.1}
Sexa $D$ un dominio de ideais principais, $F$ un $D$-módulo libre de rango $n$ e $N$ un submódulo de $F$. Entón, $N$ é libre de rango menor ou igual ca $n$.
\end{proposition}

\vspace{2mm}

\noindent \textbf{\textit{\underline{Demostración}}}

\vspace{2mm}

\noindent Obsérvese que o módulo $\{0\}$ é libre con base \{$\varnothing$\} e rango 0.\\

\noindent Razóese por indución no rango de $F$.\\

\noindent Para o caso $n = 1$, sendo $F$ un $D$-módulo libre, garántese que $F \simeq D$. Logo, como $N$ é submódulo de $F$, tense que $N$ é isomorfo a un ideal de $D$ (pois os submódulos de $D$ son os seus ideais). Por hipótese, $D$ é un DIP, logo $\exists \hspace{1mm} a \in D \hspace{1mm} | \hspace{1mm} N \simeq (a)$. Distínguense dúas posibilidades:

\begin{itemize}
    \item $a = 0 \implies N \simeq \{0\}$, que como se apuntou anteriormente é un módulo libre de rango 0 $\leq 1 = rg(D)$
    \item $a \neq 0 \implies N \simeq (a) \simeq \displaystyle \frac{D}{(0 : a)}$
    
    Agora ben, nun dominio non existen divisores propios de cero, polo que necesariamente $\displaystyle \frac{D}{(0:a)} = \frac{D}{\{0\}} = D$. Así, $N$ é un módulo libre de rango 1.\\
\end{itemize}

\noindent  Supóñase agora certo o resultado para $D$-módulos libres de rango menor ca un certo $n \in \mathbb{N}$ dado, con $n > 1$, e véxase que tamén se cumpre para $D$-módulos libres de rango $n$.\\

\noindent Sendo $F$ libre, pódese considerar unha base de $F$, $\{e_{1}, \dots, e_{n}\}$. Considérese o seguinte submódulo de $F$:

$$F' = \langle e_{2}, \dots, e_{n} \rangle$$

\noindent que é un submódulo libre de rango $n-1$. Tense entón que $\displaystyle \frac{F}{F'}$ é un $D$-módulo libre de rango 1, con base $\{e_{1} + F'\}$. En efecto:

\begin{itemize}
    \item $\langle e_{1} + F' \rangle = \displaystyle \frac{F}{F'}$
    
    Dado $x \in F \implies x = \displaystyle \sum_{i = 1}^{n}{\lambda_{i}e_{i}} \implies x + F' \overset{\substack{x - \lambda_{1}e_{1} \in F'\\ \Downarrow}}{=} \lambda_{1}e_{1} + F' = \lambda_{1}(e_{1} + F')$\\
    
    \item $\{e_{1} + F'\}$ é linearmente independente\\
    
    $\lambda(e_{1} + F') = 0 + F' \overset{?}{\implies} \lambda = 0$\\
    
    $\lambda(e_{1} + F') = \lambda e_{1} + F' = 0 + F' \Leftrightarrow \lambda e_{1} \in F' \implies \lambda e_{1} = \displaystyle \sum_{i=2}^{n}{\mu_{i}e_{i}}$
    
    Entón, tense:
    
    \[ 
    \left. \begin{array}{r} 
    \lambda e_{1} -  \displaystyle \sum_{i=2}^{n}{\mu_{i}e_{i}} = 0\\[1ex]
    \{e_{1}, \dots, e_{n}\} \text{ linearmente independente}
    \end{array} \right\}
    \implies 
    \lambda = 0 \text{ (e } \mu_{i} = 0 \hspace{2mm} \forall \hspace{1mm} i \in \{2, \dots, n\})
    \]   
    
\end{itemize}

\vspace{3mm}

\noindent Defínase $\overline{N}:= \displaystyle \frac{N + F'}{F'} \subset \displaystyle \frac{F}{F'}$. Pode suceder:

\begin{itemize}
    \item $\overline{N} = \{0\} \implies N + F' = F' \implies N \subset F' \overset{\magbf{[1]}}{\implies} N$ é libre, con $rg(N) \leq n -1 < n$
    \item $\overline{N} \neq \{0\}$
    
    Neste caso, tense:
    
    \[ 
    \left. \begin{array}{r} 
    \overline{N} \text{ é submódulo de } \displaystyle \frac{F}{F'}\\[1ex]
    \displaystyle \frac{F}{F'} \text{ é libre de rango } 1
    \end{array} \right\}
    \underset{\substack{\Uparrow \\ \overline{N} \neq \{0\}}}{\implies} 
    \overline{N} \text{ é libre de rango 1}
    \]   
    
    \vspace{3mm}
    
    Sexa $z \in N$ tal que $\{z + F'\}$ é base de $\overline{N}$ (como $\overline{N} = \displaystyle \frac{N + F'}{F'}$, un elemento de $\overline{N}$ é da forma $n + f' + F' = n + F'$, porque $f' \in F')$.\\
    
    Tense que $N \cap F'$ é un submódulo de $F'$. Como $F'$ é un módulo libre de rango $n-1$, por \textbf{hipótese de indución}, $N \cap F'$ é un submódulo libre de rango $m \leq n-1$.\\
    
    Sexa así $\{y_{1}, \dots, y_{m}\}$ base de $N \cap F'$. Resta probar que o conxunto $\{y_{1}, \dots, y_{m}, z\}$ é unha base de $N$:
    
    \begin{enumerate}
        \item $\langle y_{1}, \dots, y_{m}, z \rangle = N$
        
        $x \in N \implies x + F' \in \overline{N} \overset{\magbf{[2]}}{\implies} \exists \hspace{1mm} \lambda \in D \hspace{1mm} | \hspace{1mm} x + F' = \lambda(z + F') \implies x - \lambda z \in F'$
        
            \[ 
            \left. \begin{array}{r} 
            x - \lambda z \in F'\\[1ex]
            x - \lambda z \in N
            \end{array} \right\}
            \implies x - \lambda z \in N \cap F' \implies x - \lambda z = \displaystyle \sum_{i = 1}^{m}{\alpha_{i}y_{i}} \implies x = \displaystyle \sum_{i = 1}^{m}{\alpha_{i}y_{i}} + \lambda z
            \]   
        
        \item $\{y_{1}, \dots, y_{m}, z\}$ é linearmente independente
        
        $\alpha_{z} + \displaystyle \sum_{i=1}^{m}{\alpha_{i}y_{i}} = 0 \overset{\magbf{[3]}}{\implies} \alpha_{z} + \displaystyle \sum_{i = 1}^{m}{\alpha_{i}y_{i}} = 0 + F' \implies \alpha_{z} + \displaystyle \sum_{i = 1}^{m}\alpha_{i}y_{i} + F' = \alpha z + F' = \alpha (z + F') = 0 + F' \overset{\magbf{[4]}}{\implies} \alpha = 0 \implies \displaystyle \sum_{i = 1}^{m}{\alpha_{i}y_{i}} = 0 \overset{\magbf{[5]}}{\implies} \alpha_{i} = 0 \hspace{2mm} \forall \hspace{1mm} i \in \{1, \dots, m\} \hspace{5mm} \square$
        
    \end{enumerate}
\end{itemize}

\vspace{3mm}

\noindent $^{\magbf{[1]}}$ \textit{Por \textbf{hipótese de indución}, $F'$ é un módulo libre de rango $n-1$} \\

\noindent $^{\magbf{[2]}}$ \textit{$\{z + F'\}$ é base de $\overline{N}$}\\

\noindent $^{\magbf{[3]}}$ \textit{$ \displaystyle \sum_{i = 1}^{m}{\alpha_{i}y_{i}} \in N \cap F' \subset F'$}\\

\noindent $^{\magbf{[4]}}$ \textit{$\{z + F'\}$ é un conxunto linearmente independente}\\

\noindent $^{\magbf{[5]}}$ \textit{$\{y_{1}, \dots, y_{m}\}$ é un conxunto linearmente independente}\\

\begin{lemma} \label{lem4.2}
Sexa $R$ un dominio de ideais principais, $M$ un $R$-módulo e $\{x_{1}, \dots, x_{n}\} \subset M$. Sexa $A = (a_{ij}) \in M_{n}(R)$ unha matriz inversible. Sexan:
\begin{center}
    $y_{i} = \displaystyle \sum_{j = 1}^{n}a_{ij}x_{j}$ \hspace{4mm} $1 \leq i \leq n$ 
\end{center}
Entón, cúmprese:
\begin{enumerate}
    \item Se $\{x_{1}, \dots, x_{n}\}$ é conxunto de xeradores de $M$, entón $\{y_{1}, \dots, y_{n}\}$ tamén o é.
    \item Se $\{x_{1}, \dots, x_{n}\}$ é linearmente independente, entón $\{y_{1}, \dots, y_{n}\}$ tamén o é.
\end{enumerate}
\end{lemma}

\pagebreak

\noindent Con estes resultados, estamos en condicións de enunciar e demostrar o seguinte:\\

\begin{theorem}[\magbf{Teorema de estrutura de módulos de tipo finito sobre un DIP}] \label{th4.2}
Sexa $D$ un dominio de ideais principais e $M$ un $D$-módulo de tipo finito. Entón, M é suma directa de $D$-módulos cíclicos:
\begin{center}
    $M = Dz_{1} \oplus \dots \oplus Dz_{s}$ 
\end{center} 
cumpríndose que 
\begin{center}
    $(0 : z_{1}) \supset \dots \supset (0 : z_{s})$ \hspace{8mm} con $(0 : z_{k}) \neq D \hspace{3mm} 1 \leq k \leq s$ 
\end{center}
\end{theorem}

\vspace{2mm}

\noindent \textbf{\textit{\underline{Demostración}}}

\vspace{2mm}

\noindent Sexa $\{x_{1}, \dots, x_{n}\}$ unha familia finita de xeradores de $M$. Sexa $F$ un $D$-módulo libre de rango $n$, $\{e_{1}, \dots, e_{n}\}$ unha base de $F$ e $\varepsilon: F \longrightarrow M$ o epimorfismo tal que $\varepsilon(e_{i}) = x_{i} \hspace{2mm} \forall \hspace{1mm} i \in \{1, \dots, n\}$.\\

\noindent Sexa $N := Ker \hspace{1mm} \varepsilon$, o cal é un submódulo de $F$. Sendo $F$ libre, pódese aplicar a \hyperref[prop4.1]{\magbf{Proposición 4.1}} para garantir que $N$ é tamén libre, de rango menor ou igual ca $n$. Ademais, en particular, $N$ tamén será un módulo de tipo finito.\\

\noindent Sexa $\{f_{1}, \dots, f_{m}\}$ unha familia finita de xeradores de $N$ (como caso particular, pode ser unha base, pero na proba non é precisa tal hipótese; tampouco se necesita que $m < n$). Escríbase:

$$f_{j} = \displaystyle \sum_{j = 1}^{n}{a_{ij}e_{j}} \hspace{8mm} 1 \leq i \leq m, \hspace{4mm} a_{ij} \in D$$

\noindent A expresión anterior pódese expresar en notación matricial:\\

$$
A = \begin{pmatrix}
a_{11} & \cdots & a_{1n}\\
\vdots & \ddots & \vdots\\
a_{m1} & \cdots & a_{mn}
\end{pmatrix} \in M_{m \times n}(D)
\hspace{5mm}
e = \begin{pmatrix}
e_{1} \\ \vdots \\ e_{n}
\end{pmatrix}
\hspace{3mm}
f = \begin{pmatrix}
f_{1} \\ \vdots \\ f_{m}
\end{pmatrix}
$$

\vspace{5mm}

\noindent Así, pódese ecribir:\\

$$
f = Ae \hspace{4mm} \Longleftrightarrow \hspace{4mm}
\begin{pmatrix}
f_{1} \\ \vdots \\ f_{m}
\end{pmatrix}
=
\begin{pmatrix}
a_{11} & \cdots & a_{1n}\\
\vdots & \ddots & \vdots\\
a_{m1} & \cdots & a_{mn}
\end{pmatrix} 
\begin{pmatrix}
e_{1} \\ \vdots \\ e_{n}
\end{pmatrix}
$$

\vspace{3mm}

\noindent Cómpre ter en conta que a expresión anterior \textbf{non é un produto de matrices} en sentido propio, pois $e,f$ non son matrices sobre $D$; non obstante, a notación resultará igualmente útil. Ademais, pódese probar facilmente a asociatividade usual: $B(Cf) = (BC)f$ para matrices ordinarias $B,C$ sobre $D$.\\

\noindent Polo \hyperref[th4.1]{\magbf{Teorema de diagonalización}}, $A$ é equivalente a unha matriz diagonal

$$B = diag(d_{1}, \dots, d_{r}, 0, \dots, 0) \hspace{8mm} d_{i} \neq 0 \hspace{1mm} \forall \hspace{1mm} i \in \{1, \dots, r\}, \hspace{3mm} d_{i} | d_{i+1} \hspace{2mm} \forall \hspace{1mm} i \in \{1, \dots, r-1\}$$

\noindent logo existen matrices inversibles $Q \in M_{m}(D)$, $P \in M_{n}(D)$ tales que $B = QAP$.\\

\noindent Sexan $f' := Qf$, $e' := P^{-1}e$. Como $Q$ e $P^{-1}$ son inversibles, segundo o \hyperref[lem4.2]{\magbf{Lema 4.2}}, $f'$ é un conxunto de xeradores de $N$ e $e'$ é unha base de $F$. Tense:

$$f' = Qf = Q(Ae) = (QA)e = (QA)(Pe') = (QAP)e' = Be'$$

\pagebreak

\noindent É dicir:
\begin{align*}
 f'_{i} = d_{i} \hspace{3mm} \forall \hspace{1mm} i \in \{1, \dots, r\} & & f'_{i} = 0 \hspace{3mm} \forall \hspace{1mm} i \in \{r+1, \dots, m\} 
\end{align*}

\vspace{3mm}

\noindent Nótese que, como os $e'_{i}$ son linearmente independentes e os $d_{i}$ son non nulos, resulta que os $f'_{1}, \dots, f'_{r}$ son linearmente independentes e, polo tanto, unha base. Deste xeito, $N$ é libre de rango $r$ e os $f'_{i}$ non nulos, que resultan ser os $r$ primeiros, forman unha base de $N$. Isto proporciona un método para calcular unha base dun submódulo dun $D$-módulo libre a partir dunha familia de xeradores.\\

\noindent Sexan $y_{i} := \varepsilon(e'_{i}) \in M$, con $1 \leq i \leq n$. Como $\varepsilon$ é un epimorfismo, en particular é sobrexectivo, logo $\{y_{1}, \dots, y_{n}\}$ é unha familia finita de xeradores de $M$:

\begin{equation}\label{ec4.1}
    M = \displaystyle \sum_{i = 1}^{n}Dy_{i}
\end{equation}

\noindent Obsérvese ademais que $Y = P^{-1}X$, sendo $X$ a familia finita de xeradores de $M$ indicada ao comezo da demostración.\\

\noindent Véxase que a suma \eqref{ec4.1} é directa, i.e. 

$$\displaystyle \sum_{i = 1}^{n}b_{i}y_{i} = 0 \implies b_{i}y_{i} = 0 \hspace{8mm} \forall \hspace{1mm} i \in \{1, \dots, n\}$$

\vspace{3mm}

$$
0 = \displaystyle \sum_{i = 1}^{n}b_{i}y_{i} = \sum_{i = 1}^{n}b_{i}\varepsilon (e'_{i}) = \varepsilon (\sum_{i = 1}^{n}b_{i}e'_{i}) 
\implies \sum_{i=1}^{n}b_{i}e'_{i} \in Ker \hspace{1mm} \varepsilon = N \implies
$$

$$
\implies \displaystyle \sum_{i = 1}^{n}b_{i}e'_{i} = \sum_{i = 1}^{r}c_{i}f'_{i} = \sum_{i = 1}^{r}c_{i}d_{i}e'_{i} \implies \begin{cases}
b_{i} = 0 \hspace{6mm} r+1 \leq i \leq n\\
b_{i} = c_{i}d_{i} \hspace{3mm} 1 \leq i \leq r
\end{cases}
$$

\vspace{3mm}

\noindent En base ao anterior, xa está garantido que $b_{i}y_{i} = 0 \hspace{2mm} \forall \hspace{1mm} i \in \{r+1, \dots, n\}$. Para o resto de elementos, tense:

$$d_{i}y_{i} = d_{i} \varepsilon(e'_{i}) = \varepsilon(d_{i}e'_{i}) = \varepsilon(f'_{i}) \underset{\substack{\Uparrow \\ f'_{i} \in Ker \hspace{1mm} \varepsilon}}{=} 0 \implies b_{i}y_{i} = 0 \hspace{3mm} 1 \leq i \leq r
$$

\vspace{3mm}

\noindent Así pois:

\begin{equation}\label{ec4.2}
    M = Dy_{1} \oplus \dots \oplus Dy_{n}
\end{equation}

\vspace{3mm}

\noindent Véxase agora que:

$$
\begin{aligned}
    (0:y_{i}) = (d_{i}) & \hspace{5mm} 1 \leq i \leq r\\
    (0 : y_{i}) = \{0\} & \hspace{5mm} r + 1 \leq i \leq n
\end{aligned}
$$

\vspace{3mm}

\noindent \fcolorbox{magenta}{white}{$\supset$/} Acábase de demostrar implicitamente na proba de que a suma \eqref{ec4.2} é directa.\\

\noindent \fcolorbox{magenta}{white}{$\subset$/} Sexa $\lambda \in (0:y_{i})$. Entón, $\lambda y_{i} = 0$. Tense:

            \[ 
            \left. \begin{array}{r} 
            \lambda y_{i} = 0\\[1ex]
            \lambda y_{i} = \lambda \varepsilon(e'_{i}) = \varepsilon(\lambda e'_{i})
            \end{array} \right\}
            \implies \lambda e'_{i} \in Ker \hspace{1mm} \varepsilon = N \implies \lambda e'_{i} = \displaystyle \sum_{j = 1}^{r}\mu_{j}f'_{j} = \sum_{j = 1}^{r}\mu_{j}d_{j}e'_{j} \implies
            \]  
            $$
            \implies 
            \begin{cases}
            \lambda = 0 \hspace{5mm} r+1 \leq i \leq n\\
            \lambda = \mu_{i}d_{i} \hspace{5mm} 1 \leq i \leq r
            \end{cases}
            $$
            
\pagebreak
            
\noindent Doutra banda, as relacións $d_{1} | d_{2} | \dots | d_{r}$ implican: 

$$\equalto{(0 : y_{1})}{(d_{1})} \supset \equalto{(0:y_{2})}{(d_{2})} \supset \cdots \supset \equalto{(0:y_{r})}{(d_{r})}$$

\noindent e polo tanto

$$(0 : y_{1}) \supset \cdots \supset (0 : y_{r}) \supset (0 : y_{r+1}) \supset \cdots \supset (0 : y_{n})$$

\vspace{3mm}

\noindent Sexa $s \in \{0, \dots, n\}$ tal que $(0 : y_{1}) = (0 : y_{2}) = \dots = (0 : y_{n-s}) = D$, con $(0 : y_{n-s+1}) \neq D$. Tomando:

\begin{align*}
    z_{1} & := y_{n-s+1}\\
    z_{2} & := y_{n-s+2} \\
    \vdots & \\
    z_{s} & := y_{n}
\end{align*}

\vspace{3mm}

\noindent Así, $M = Dy_{1} \oplus \cdots \oplus Dy_{n} = Dz_{1} \oplus \cdots \oplus Dz_{s} \hspace{8mm} \square$

\vspace{3mm}

\noindent \textbf{Observación 4.2}. Esta demostración proporciona un método para calcular unha base dun submódulo dun $D$-módulo libre de rango finito coñecendo unha familia de xeradores:\\

\begin{mdframed}[linecolor = classicrose, linewidth = 1mm]

\noindent Se $F$ é un $D$-módulo libre de rango $n$ con base $\{e_{1}, \dots, e_{n}\}$ e $\{f_{1}, \dots, f_{m}\}$ é unha familia de xeradores dun submódulo $N$ de $F$, escríbase:

$$f_{i} = \displaystyle \sum_{j = 1}^{n}a_{ij}e_{j} \hspace{5mm} 1 \leq i \leq m$$

\vspace{3mm}

\noindent Tómase $A = (a_{ij}) \in M_{m \times n}(D)$. Diagonalizando $A$, obtense:

$$QAP = B = diag(d_{1}, \dots, d{r}, 0, \dots, 0), \hspace{5mm} d_{i} \neq 0 \hspace{2mm} \forall i, \hspace{2mm} d_{i} | d_{i+1}$$

\vspace{2mm}

\noindent Tómase $f' = Qf$, e entón $\{f'_{1}, \dots, f'_{r}\}$ é unha base de $N$ e, ademais, $f'_{r+1} = \dots = f'_{n} = 0$.\\

\end{mdframed}

\vspace{3mm}

\noindent \textbf{Observación 4.3}. Sexa $M$ un $D$-módulo con xeradores $x_{1}, \dots, x_{n}$ e relacións

$$\displaystyle \sum_{j = 1}^{n}a_{ij}x_{j} = 0 \hspace{5mm} 1 \leq i \leq m$$

\vspace{3mm}

\noindent Cúmprese que $D^{n}$ é un $D$-módulo libre. Entón, considerando $\{e_{1}, \dots, e_{n}\}$ base de $D^{n}$, existe un único epimorfismo $\varepsilon: D^{n} \longrightarrow M$ tal que $\varepsilon(e_{i}) = x_{i}$. Se o seu núcleo, $N = Ker \hspace{1mm} \varepsilon$, está xerado por $\{f_{1}, \dots, f_{r}\}$, cúmprese:

$$f_{i} = \displaystyle \sum_{j = 1}^{n}a_{ij}e_{j} \hspace{5mm} 1 \leq i \leq r$$

\begin{mdframed}[linecolor = classicrose, linewidth = 1mm]
\noindent A descomposición de $M$ dada polo \hyperref[th4.2]{\magbf{Teorema de estrutura}} calcúlase, resumindo, do seguinte xeito:
\begin{enumerate}
    \item Tómase $A = (a_{ij}) \in M_{m \times n}(D)$
    \item Diagonalízase:
    $$QAP = B = diag(d_{1}, \dots, d_{r}, 0, \dots, 0) \hspace{7mm} d_{i} \neq 0 \hspace{2mm} \forall i, \hspace{3mm} d_{i} | d_{i+1}$$
    \item Tómase $Y = P^{-1}X$. Entón:
    $$M = Dy_{1} \oplus \cdots \oplus Dy_{n}$$
    cumpríndose:
    $$
    \begin{aligned}
        (0 : y_{i}) = (d_{i}) & \hspace{5mm} 1 \leq i \leq r\\
        (0 : y_{i}) = \{0\} & \hspace{5mm} r+1 \leq i \leq n
    \end{aligned}
    $$
\end{enumerate}
\end{mdframed}

\vspace{3mm}

\noindent \textbf{Observación 4.4}. Lémbrese que $Dy_{i} \simeq \displaystyle \frac{D}{(0 : y_{i})}$. Logo:\\

$$M \simeq \displaystyle \frac{D}{(0: y_{1})} \oplus \cdots \oplus \frac{D}{(0:y_{n})}$$

\vspace{5mm}

\noindent A continuación amósanse algúns exemplos de aplicación do teorema:\\

\noindent \textbf{Exemplo 1}. Sexa $M$ o grupo abeliano con xeradores $a,b,c$ e relacións
    \begin{align*}
        2a - 2c = 0 & & b-c = 0
    \end{align*}
    
\noindent Achar a descomposición de $M$ dada polo teorema de estrutura.\\

\noindent \textit{\underline{Solución}}:\\

\noindent As relacións dadas permiten obter a seguinte expresión matricial:

$$
\begin{pmatrix}
0 \\ 0 
\end{pmatrix}
=
\begin{pmatrix}
2 & 0 & -2 \\
0 & 1 & -1
\end{pmatrix}
\begin{pmatrix}
a \\ b \\ c
\end{pmatrix}
$$

\noindent A matriz dos coeficientes das relacións é a que interesa diagonalizar:

$$
A = \begin{pmatrix}
2 & 0 & -2 \\
0 & 1 & -1
\end{pmatrix}
\in M_{2 \times 3}(\mathbb{Z})
$$

\noindent Así pois, diagonalízase $A$:

$$
A = \begin{pmatrix}
2 & 0 & -2 \\
0 & 1 & -1
\end{pmatrix}
\underset{F_{1} \leftrightarrow F_{2}}{\xrightarrow{\hspace{15mm}}}
\begin{pmatrix}
0 & 1 & -1 \\
2 & 0 & -2
\end{pmatrix}
\underset{C_{1} \leftrightarrow C_{2}}{\xrightarrow{\hspace{15mm}}}
\begin{pmatrix}
1 & 0 & -1 \\
0 & 2 & -2
\end{pmatrix}
\longrightarrow
$$

\vspace{3mm}

$$
\underset{C_{3} + C_{1}}{\xrightarrow{\hspace{15mm}}}
\begin{pmatrix}
1 & 0 & 0 \\
0 & 2 & -2
\end{pmatrix}
\underset{C_{3} + C_{2}}{\xrightarrow{\hspace{15mm}}}
\begin{pmatrix}
1 & 0 & 0 \\
0 & 2 & 0
\end{pmatrix}
= B
$$

\vspace{3mm}

\noindent Tense así:

\begin{multicols}{3}
\begin{itemize}
    \item $d_{1} = 1$
    \item $d_{2} = 2$
    \item $m = 2$
    \item $n = 3$
    \item $r = 2$
    \item[\vspace{\fill}]
\end{itemize}
\end{multicols}

\vspace{3mm}

\noindent Deste xeito, como $n = 3$, $M$ descomponse da seguinte forma:

$$M = \mathbb{Z}y_{1} \oplus \mathbb{Z}y_{2} \oplus \mathbb{Z}y_{3}$$

\vspace{3mm}

\noindent cumpríndose:

\begin{align*}
    (0 : y_{1}) = (d_{1}) = (1) = \mathbb{Z} & \qquad &  (0:y_{2}) = (d_{2}) = (2) = 2\mathbb{Z} & \qquad & (0 : y_{3}) = \{0\}
\end{align*}

\pagebreak

\noindent Así, como $(0 : y_{1}) = \mathbb{Z}$, obtense: 
$$M \hspace{2mm} = \hspace{2mm} \mathbb{Z}y_{2} \oplus \mathbb{Z}y_{3} \hspace{2mm} \simeq \hspace{2mm} \displaystyle \frac{\mathbb{Z}}{2\mathbb{Z}} \oplus \mathbb{Z}$$

\vspace{3mm}

\noindent Agora ben, sábese a que é isomorfo $M$. Pero aínda hai que calcular $y_{2}$ e $y_{3}$. Isto faise, tal e como se apuntou, mediante a expresión:

\begin{equation}\label{ec4.3}
Y = P^{-1}X
\end{equation}

\vspace{3mm}

\noindent onde

\begin{align*}
    Y = 
    \begin{pmatrix}
    y_{1} \\ y_{2} \\ y_{3}
    \end{pmatrix}
    &
    \qquad
    &
    X = 
    \begin{pmatrix}
    a \\ b \\ c
    \end{pmatrix}
\end{align*}

\vspace{3mm}

\noindent Lémbrese que a matriz $P$ é a que aparece na diagonalización de $A$, $B = QAP$. Pódese escribir:

$$B = \overbrace{U_{q} \cdot \dots \cdot U_{1} \cdot I_{m}}^{Q} \cdot A \cdot \underbrace{I_{n} \cdot V_{1} \cdot \dots \cdot V_{p}}_{P}$$

\noindent onde $U_{1}, \dots, U_{q}$ son matrices elementais correspondentes ás operacións elementais por filas realizadas sobre $A$, e $V_{1}, \dots, V_{p}$ son matrices elementais asociadas ás operacións elementais por columnas. A partir da expresión de $P$, é fácil determinar, e calcular, $P^{-1}$:

$$P = I_{n} \cdot V_{1} \cdot \dots \cdot V_{p} \implies P^{-1} = V_{p}^{-1} \cdot \dots \cdot V_{1}^{-1} \cdot I_{n} = I_{n} \cdot V_{p}^{-1} \cdot \dots \cdot V_{1}^{-1}
$$

\vspace{3mm}

\noindent Deste xeito, $P^{-1}$ calcúlase aplicando sobre a matriz identidade $I_{n}$ as inversas das operacións elementais por columnas que se realizaron sobre $A$, e en orde contraria:\\

$$
I_{3} = 
\begin{pmatrix}
1 & 0 & 0\\
0 & 1 & 0\\
0 & 0 & 1
\end{pmatrix}
\underset{C_{3} - C_{2}}{\xrightarrow{\hspace{15mm}}}
\begin{pmatrix}
1 & 0 & 0 \\
0 & 1 & -1 \\
0 & 0 & 1
\end{pmatrix}
\underset{C_{3} - C_{1}}{\xrightarrow{\hspace{15mm}}}
\begin{pmatrix}
1 & 0 & -1\\
0 & 1 & -1\\
0 & 0 & 1
\end{pmatrix}
\underset{C_{1} \leftrightarrow C_{2}}{\xrightarrow{\hspace{15mm}}}
\begin{pmatrix}
0 & 1 & -1\\
1 & 0 & -1\\
0 & 0 & 1
\end{pmatrix}
= P^{-1}
$$

\vspace{3mm}

\noindent Logo, substituíndo na expresión \eqref{ec4.3}:\\

$$
\begin{pmatrix}
y_{1} \\ y_{2} \\ y_{3}
\end{pmatrix}
=
\begin{pmatrix}
0 & 1 & -1\\
1 & 0 & -1\\
0 & 0 & 1
\end{pmatrix}
\begin{pmatrix}
a \\ b \\ c
\end{pmatrix}
=
\begin{pmatrix}
b - c \\ a -c \\ c
\end{pmatrix}
$$

\vspace{3mm}

\noindent Así, tense:

$$
\begin{cases}
y_{1} = b - c = 0\\
y_{2} = a - c\\
y_{3} = c
\end{cases}
$$

\vspace{3mm}

\noindent Entón, conclúese:

\begin{center}
    \fcolorbox{magenta}{white}{
    $M = \mathbb{Z}(a-c) \oplus \mathbb{Z}c$
    }
\end{center}

\pagebreak

\noindent \textbf{Exemplo 2}. Sexa $M$ o grupo abeliano con xeradores $a,b,c,d$ e relacións:

\begin{align*}
    a + 2b + c = 0 & & -2a + b + 3d = 0 & & 4a - b + d = 0
\end{align*}

\vspace{3mm}

\noindent Achar a descomposición de $M$ dada polo teorema de estrutura.\\

\noindent \textit{\underline{Solución:}}\\

\noindent As relacións dadas pódense expresar matricialmente:

$$
\begin{pmatrix}
0 \\ 0 \\ 0
\end{pmatrix}
= 
\begin{pmatrix}
1 & 2 & -1 & 0\\
-2 & 1 & 0 & 3\\
4 & -1 & 0 & 1 
\end{pmatrix}
\begin{pmatrix}
a \\ b \\ c \\ d
\end{pmatrix}
$$

\noindent Tómese a matriz de coeficientes das relacións:

$$
A = 
\begin{pmatrix}
1 & 2 & -1 & 0\\
-2 & 1 & 0 & 3\\
4 & -1 & 0 & 1 
\end{pmatrix}
\in M_{3 \times 4}(\mathbb{Z})
$$

\noindent Diagonalizando $A$, obtense:

$$
A = 
\begin{pmatrix}
1 & 2 & -1 & 0\\
-2 & 1 & 0 & 3\\
4 & -1 & 0 & 1 
\end{pmatrix}
\underset{\substack{C_{2} - 2C_{1} \\ C_{3} + C_{1}}}{\xrightarrow{\hspace{15mm}}}
\begin{pmatrix}
1 & 0 & 0 & 0\\
-2 & 5 & -2 & 3\\
4 & -9 & 4 & 1 
\end{pmatrix}
\underset{\substack{F_{2} + 2F_{1} \\ F_{3} -4 F_{1}}}{\xrightarrow{\hspace{15mm}}}
\begin{pmatrix}
1 & 0 & 0 & 0\\
0 & 5 & -2 & 3\\
0 & -9 & 4 & 1 
\end{pmatrix}
\longrightarrow
$$

\vspace{3mm}

$$
\underset{F_{2} \leftrightarrow F_{3}}{\xrightarrow{\hspace{15mm}}}
\begin{pmatrix}
1 & 0 & 0 & 0\\
0 & -9 & 4 & 1\\
0 & 5 & -2 & 3 
\end{pmatrix}
\underset{C_{2} \leftrightarrow C_{4}}{\xrightarrow{\hspace{15mm}}}
\begin{pmatrix}
1 & 0 & 0 & 0\\
0 & 1 & 4 & -9\\
0 & 3 & -2 & 5 
\end{pmatrix}
\underset{\substack{C_{3} - 4C_{2} \\ C_{4} + 9C_{2}}}{\xrightarrow{\hspace{15mm}}}
\begin{pmatrix}
1 & 0 & 0 & 0\\
0 & 1 & 0 & 0\\
0 & 3 & -14 & 32 
\end{pmatrix}
\longrightarrow
$$

\vspace{3mm}

$$
\underset{F_{3} - F_{2}}{\xrightarrow{\hspace{15mm}}}
\begin{pmatrix}
1 & 0 & 0 & 0\\
0 & 1 & 0 & 0\\
0 & 0 & -14 & 32 
\end{pmatrix}
\underset{C_{4} + 2C_{3}}{\xrightarrow{\hspace{15mm}}}
\begin{pmatrix}
1 & 0 & 0 & 0\\
0 & 1 & 0 & 0\\
0 & 0 & -14 & 4
\end{pmatrix}
\underset{C_{3} \leftrightarrow C_{4}}{\xrightarrow{\hspace{15mm}}}
\begin{pmatrix}
1 & 0 & 0 & 0\\
0 & 1 & 0 & 0\\
0 & 0 & 4 & -14
\end{pmatrix}
\longrightarrow
$$

\vspace{3mm}

$$
\underset{C_{4} + 3C_{3}}{\xrightarrow{\hspace{15mm}}}
\begin{pmatrix}
1 & 0 & 0 & 0\\
0 & 1 & 0 & 0\\
0 & 0 & 4 & -2
\end{pmatrix}
\underset{C_{3} \leftrightarrow C_{4}}{\xrightarrow{\hspace{15mm}}}
\begin{pmatrix}
1 & 0 & 0 & 0\\
0 & 1 & 0 & 0\\
0 & 0 & -2 & 4
\end{pmatrix}
\underset{C_{4} + 2C_{3}}{\xrightarrow{\hspace{15mm}}}
\begin{pmatrix}
1 & 0 & 0 & 0\\
0 & 1 & 0 & 0\\
0 & 0 & -2 & 0
\end{pmatrix}
\longrightarrow
$$

\vspace{3mm}

$$
\underset{-C_{3}}{\xrightarrow{\hspace{15mm}}}
\begin{pmatrix}
1 & 0 & 0 & 0\\
0 & 1 & 0 & 0\\
0 & 0 & 2 & 0
\end{pmatrix}
= B
$$

\vspace{3mm}

\noindent Tense así:

\begin{multicols}{3}
\begin{itemize}
    \item $d_{1} = 1$
    \item $d_{2} = 1$
    \item $d_{3} = 2$
    \item $m = 2$
    \item $n = 3$
    \item $r = 2$
\end{itemize}
\end{multicols}

\vspace{3mm}

\noindent Logo, sendo $n = 4$, $M$ descomponse da seguinte forma:

$$M = \mathbb{Z}y_{1} \oplus \mathbb{Z}y_{2} \oplus \mathbb{Z}y_{3} \oplus \mathbb{Z}y_{4}$$

\pagebreak

\noindent cumpríndose:
\begin{align*}
    (0 : y_{1}) = (1) = (0 : y_{2}) & & (0 : y_{3}) = (2) & & (0 : y_{4}) = \{0\}
\end{align*}

\vspace{3mm}

\noindent Entón:

$$M \hspace{2mm} = \hspace{2mm} \mathbb{Z}y_{3} \oplus \mathbb{Z}y_{4} \hspace{2mm} \simeq \hspace{2mm} \displaystyle \frac{\mathbb{Z}}{2\mathbb{Z}} \oplus \mathbb{Z}$$

\vspace{3mm}

\noindent A continuación, hai que calcular $y_{3}$ e $y_{4}$. Isto faise empregando a expresión \eqref{ec4.3}, sendo neste caso:

\begin{align*}
    Y = 
    \begin{pmatrix}
    y_{1}\\ y_{2} \\ y_{3} \\ y_{4}
    \end{pmatrix}
    & & 
    X = 
    \begin{pmatrix}
    a \\ b \\ c \\ d
    \end{pmatrix}
\end{align*}

\vspace{3mm}

\noindent Como xa se indicou no exemplo anterior, $P^{-1}$ áchase aplicando sobre a matriz identidade $I_{n}$, en orde contraria, as operacións elementais por columnas inversas ás que se realizaron sobre a matriz $A$:

$$
I_{4} = 
\begin{pmatrix}
1 & 0 & 0 & 0\\
0 & 1 & 0 & 0\\
0 & 0 & 1 & 0\\
0 & 0 & 0 & 1
\end{pmatrix}
\underset{C_{4} - 2C_{3}}{\xrightarrow{\hspace{15mm}}}
\begin{pmatrix}
1 & 0 & 0 & 0\\
0 & 1 & 0 & 0\\
0 & 0 & 1 & -2\\
0 & 0 & 0 & 1
\end{pmatrix}
\underset{C_{3} \leftrightarrow C_{4}}{\xrightarrow{\hspace{15mm}}}
\begin{pmatrix}
1 & 0 & 0 & 0\\
0 & 1 & 0 & 0\\
0 & 0 & -2 & 1\\
0 & 0 & 1 & 0
\end{pmatrix}
\longrightarrow
$$

$$
\underset{C_{4} - 3C_{3}}{\xrightarrow{\hspace{15mm}}}
\begin{pmatrix}
1 & 0 & 0 & 0\\
0 & 1 & 0 & 0\\
0 & 0 & -2 & 7\\
0 & 0 & 1 & -3
\end{pmatrix}
\underset{C_{3} \leftrightarrow C_{4}}{\xrightarrow{\hspace{15mm}}}
\begin{pmatrix}
1 & 0 & 0 & 0\\
0 & 1 & 0 & 0\\
0 & 0 & 7 & -2\\
0 & 0 & -3 & 1
\end{pmatrix}
\underset{C_{4} - 2C_{3}}{\xrightarrow{\hspace{15mm}}}
\begin{pmatrix}
1 & 0 & 0 & 0\\
0 & 1 & 0 & 0\\
0 & 0 & 7 & -16\\
0 & 0 & -3 & 7
\end{pmatrix}
\longrightarrow
$$

$$
\underset{\substack{C_{3} + 4C_{2} \\ C_{4} - 9C_{2}}}{\xrightarrow{\hspace{15mm}}}
\begin{pmatrix}
1 & 0 & 0 & 0\\
0 & 1 & 4 & -9\\
0 & 0 & 7 & -16\\
0 & 0 & -3 & 7
\end{pmatrix}
\underset{C_{4} \leftrightarrow C_{2}}{\xrightarrow{\hspace{15mm}}}
\begin{pmatrix}
1 & 0 & 0 & 0\\
0 & -9 & 4 & 1\\
0 & -16 & 7 & 0\\
0 & 7 & -3 & 0
\end{pmatrix}
\underset{\substack{C_{2} + 2C_{1} \\ C_{3} - C_{1}}}{\xrightarrow{\hspace{15mm}}}
\begin{pmatrix}
1 & 2 & -1 & 0\\
0 & -9 & 4 & 1\\
0 & -16 & 7 & 0\\
0 & 7 & -3 & 0
\end{pmatrix}
= P^{-1}
$$

\vspace{3mm}

\noindent Substituíndo así en \eqref{ec4.3}:

\begin{align*}
    \begin{pmatrix}
    y_{1}\\ y_{2} \\ y_{3} \\ y_{4}
    \end{pmatrix}
    =
    \begin{pmatrix}
    1 & 2 & -1 & 0\\
    0 & -9 & 4 & 1\\
    0 & -16 & 7 & 0\\
    0 & 7 & -3 & 0
    \end{pmatrix}
    \begin{pmatrix}
    a \\ b \\ c \\ d
    \end{pmatrix}
    =
    \begin{pmatrix}
    a + 2b -c \\ -9b + 4c + d \\ -16b + 7c \\ 7b -3c
    \end{pmatrix}
\end{align*}

\vspace{3mm}

\noindent Tense deste xeito:

$$
\begin{cases}
y_{1} = a + 2b -c = 0\\
y_{2} = -9b + 4c + d = 0\\
y_{3} = -16b + 7c\\
y_{4} = 7b -3c
\end{cases}
$$

\vspace{3mm}

\noindent Entón, conclúese:

\begin{center}
    \fcolorbox{magenta}{white}{$M = \mathbb{Z}(-16b + 7c) \oplus \mathbb{Z}(7b - 3c)$}
\end{center}

\pagebreak

\noindent \textbf{Exemplo 3}. Considérese o seguinte subgrupo de $\mathbb{Z}^{4}:$
$$N = \langle 2e_{1} - 3e_{2} - e_{3}, e_{1} + e_{4}, 3e_{2} + e_{3} + 2e_{4}, 5e_{4} \rangle$$
\noindent onde $\{e_{1},e_{2},e_{3},e_{4}\}$ é a base canónica de $\mathbb{Z}^{4}$. Achar unha base para $N$.\\

\noindent \textit{\underline{Solución}}:\\

\noindent Escribindo $N = \langle f_{1}, f_{2}, f_{3}, f_{4} \rangle$, pódense expresar os xeradores de $N$ como segue:

\[
\begin{pmatrix}
f_{1}\\
f_{2}\\
f_{3}\\
f_{4}
\end{pmatrix}
=
\begin{pmatrix}
2 & -3 & -1 & 0\\
1 & 0 & 0 & 1\\
0 & 3 & 1 & 2\\
0 & 0 & 0 & 5
\end{pmatrix}
\begin{pmatrix}
e_{1}\\
e_{2}\\
e_{3}\\
e_{4}
\end{pmatrix}
\]

\vspace{3mm}

\noindent Escóllase a matriz de coeficientes:
\[
A =
\begin{pmatrix}
2 & -3 & -1 & 0\\
1 & 0 & 0 & 1\\
0 & 3 & 1 & 2\\
0 & 0 & 0 & 5
\end{pmatrix}
\in M_{4}(\mathbb{Z})
\]

\noindent Diagonalizando $A$, obtense:

\[
A = 
\begin{pmatrix}
2 & -3 & -1 & 0\\
1 & 0 & 0 & 1\\
0 & 3 & 1 & 2\\
0 & 0 & 0 & 5
\end{pmatrix}
\underset{F_{2} \leftrightarrow F_{1}}{\xrightarrow{\hspace{15mm}}}
\begin{pmatrix}
1 & 0 & 0 & 1\\
2 & -3 & -1 & 0\\
0 & 3 & 1 & 2\\
0 & 0 & 0 & 5
\end{pmatrix}
\underset{C_{4} - C_{1}}{\xrightarrow{\hspace{15mm}}}
\begin{pmatrix}
1 & 0 & 0 & 0\\
2 & -3 & -1 & -2\\
0 & 3 & 1 & 2\\
0 & 0 & 0 & 5
\end{pmatrix}
\longrightarrow
\]

\vspace{3mm}

\[
\underset{F_{2} - 2F_{1}}{\xrightarrow{\hspace{15mm}}}
\begin{pmatrix}
1 & 0 & 0 & 0\\
0 & -3 & -1 & -2\\
0 & 3 & 1 & 2\\
0 & 0 & 0 & 5
\end{pmatrix}
\underset{(-1)F_{2}}{\xrightarrow{\hspace{15mm}}}
\begin{pmatrix}
1 & 0 & 0 & 0\\
0 & 3 & 1 & 2\\
0 & 3 & 1 & 2\\
0 & 0 & 0 & 5
\end{pmatrix}
\underset{C_{3} \leftrightarrow C_{2}}{\xrightarrow{\hspace{15mm}}}
\begin{pmatrix}
1 & 0 & 0 & 0\\
0 & 1 & 3 & 2\\
0 & 1 & 3 & 2\\
0 & 0 & 0 & 5
\end{pmatrix}
\longrightarrow
\]

\vspace{3mm}

\[
\underset{F_{3} - F_{2}}{\xrightarrow{\hspace{15mm}}}
\begin{pmatrix}
1 & 0 & 0 & 0\\
0 & 1 & 3 & 2\\
0 & 0 & 0 & 0\\
0 & 0 & 0 & 5
\end{pmatrix}
\underset{\substack{C_{3} - 3C_{2} \\ C_{4} - 2C_{2}}}{\xrightarrow{\hspace{15mm}}}
\begin{pmatrix}
1 & 0 & 0 & 0\\
0 & 1 & 0 & 0\\
0 & 0 & 0 & 0\\
0 & 0 & 0 & 5
\end{pmatrix}
\underset{C_{3} \leftrightarrow C_{4}}{\xrightarrow{\hspace{15mm}}}
\begin{pmatrix}
1 & 0 & 0 & 0\\
0 & 1 & 0 & 0\\
0 & 0 & 0 & 0\\
0 & 0 & 5 & 0
\end{pmatrix}
\longrightarrow
\]

\vspace{3mm}

\[
\underset{F_{3} \leftrightarrow F_{4}}{\xrightarrow{\hspace{15mm}}}
\begin{pmatrix}
1 & 0 & 0 & 0\\
0 & 1 & 0 & 0\\
0 & 0 & 5 & 0\\
0 & 0 & 0 & 0
\end{pmatrix}
= B
\]

\vspace{3mm}

\noindent Tense así que $r = 3$. \\

\noindent Tal e como se viu no teorema de estrutura, unha base para $N$ obtense da seguinte maneira:

\[
f' = Qf 
\Longleftrightarrow 
\begin{pmatrix}
f'_{1}\\
f'_{2}\\
f'_{3}\\
f'_{4}
\end{pmatrix}
=
Q
\begin{pmatrix}
f_{1}\\
f_{2}\\
f_{3}\\
f_{4}
\end{pmatrix}
\]

\vspace{3mm}

\noindent onde $Q$ é a matriz tal que $B = QAP$. Ademais, sendo $r = 3$, sábese que $f'_{4} = 0$. Logo, $\{f'_{1}, f'_{2}, f'_{3}\}$ é unha base para $N$.\\

\noindent Para calcular $Q$, aplícanse sobre a identidade $I_{m}$ (onde neste caso $m = 4$) as mesmas operacións por elementais por filas que se fixeron sobre $A$ para diagonalizala, en orde inversa:

\[
I_{4} = 
\begin{pmatrix}
1 & 0 & 0 & 0\\
0 & 1 & 0 & 0\\
0 & 0 & 1 & 0\\
0 & 0 & 0 & 1
\end{pmatrix}
\underset{F_{3} \leftrightarrow F_{4}}{\xrightarrow{\hspace{15mm}}}
\begin{pmatrix}
1 & 0 & 0 & 0\\
0 & 1 & 0 & 0\\
0 & 0 & 0 & 1\\
0 & 0 & 1 & 0
\end{pmatrix}
\underset{F_{3} + F_{2}}{\xrightarrow{\hspace{15mm}}}
\begin{pmatrix}
1 & 0 & 0 & 0\\
0 & 1 & 0 & 0\\
0 & 1 & 0 & 1\\
0 & 0 & 1 & 0
\end{pmatrix}
\underset{(-1)F_{2}}{\xrightarrow{\hspace{15mm}}}
\begin{pmatrix}
1 & 0 & 0 & 0\\
0 & -1 & 0 & 0\\
0 & 1 & 0 & 1\\
0 & 0 & 1 & 0
\end{pmatrix}
\longrightarrow
\]

\vspace{3mm}

\[
\underset{F_{2} + 2F_{1}}{\xrightarrow{\hspace{15mm}}}
\begin{pmatrix}
1 & 0 & 0 & 0\\
2 & -1 & 0 & 0\\
0 & 1 & 0 & 1\\
0 & 0 & 1 & 0
\end{pmatrix}
\underset{F_{2} \leftrightarrow F_{1}}{\xrightarrow{\hspace{15mm}}}
\begin{pmatrix}
2 & -1 & 0 & 0\\
1 & 0 & 0 & 0\\
0 & 1 & 0 & 1\\
0 & 0 & 1 & 0
\end{pmatrix}
= Q
\]

\vspace{3mm}

\noindent Con isto, obtense:

\[
\begin{pmatrix}
f'_{1}\\
f'_{2}\\
f'_{3}\\
f'_{4}
\end{pmatrix}
=
\begin{pmatrix}
2 & -1 & 0 & 0\\
1 & 0 & 0 & 0\\
0 & 1 & 0 & 1\\
0 & 0 & 1 & 0
\end{pmatrix}
\begin{pmatrix}
f_{1}\\
f_{2}\\
f_{3}\\
f_{4}
\end{pmatrix}
=
\begin{pmatrix}
f_{1} - f_{2}\\
f_{1}\\
f_{2} + f_{4}\\
f_{3}
\end{pmatrix}
\]

\vspace{3mm}

\noindent Así, unha base para $N$ é:

\begin{center}
\fcolorbox{magenta}{white}{$\{f_{1} - f_{2},f_{1},f_{2}+f_{4}\}$}
\end{center}

\vspace{10mm}

\noindent A partir do \hyperref[th4.2]{\magbf{Teorema de estrutura}} extráense os seguintes resultados:\\

\begin{corollary} \label{cor4.1}
Sexa $D$ un dominio de ideais principais e $M$ un $D$-módulo de tipo finito. Entón, M é suma directa dun submódulo de torsión e un submódulo libre. Máis concretamente:
\begin{center}
    $M = T(M) \oplus L$
\end{center}
onde L é libre.
\end{corollary}

\vspace{2mm}

\noindent \textbf{\textit{\underline{Demostración}}}

\vspace{2mm}

\noindent Segundo o \hyperref[th4.2]{\magbf{Teorema de estrutura}}:\\

$$M = Dz_{1} \oplus \dots \oplus Dz_{s}$$

\vspace{3mm}

\noindent con $(0 : z_{1}) \supset \dots \supset (0 : z_{s})$, e $(0 : z_{k}) \neq D \hspace{2mm} \forall \hspace{1mm} k \in \{1, \dots, s\}$.\\

\noindent Sexa $r \in \{0, \dots, s\}$ tal que

\begin{align*}
    (0 : z_{i}) \neq 0 & \hspace{5mm} 1 \leq i \leq r \\
    (0 : z_{i}) = \{0\} & \hspace{5mm} r + 1 \leq i \leq s
\end{align*}

\vspace{3mm}

\noindent Entón:

\begin{subequations}
\begin{align}
    Dz_{1} \oplus \dots \oplus Dz_{r} & = T(M) \label{ec4.4a}\\
    Dz_{r+1} \oplus \dots \oplus Dz_{s} & \hspace{1mm} \text{ é libre} \label{ec4.4b}
\end{align}
\end{subequations}

\vspace{4mm}

\noindent Demóstrese a igualdade \eqref{ec4.4a} por dobre inclusión:

\pagebreak

\noindent \fcolorbox{magenta}{white}{$\subset$/} Fíxese $i \in \{1, \dots, r\}$ de xeito arbitrario. Entón, tense:

$$(0 : z_{i}) \neq 0 \implies z_{i} \in T(M) \implies Dz_{i} \subset T(M) \implies \displaystyle \bigoplus_{i = 1}^{r}Dz_{i} \subset T(M)$$

\vspace{3mm}

\noindent \fcolorbox{magenta}{white}{$\supset$/} Sexa $x \in T(M)$. Entón, tense:
\begin{itemize}
    \item $ax = 0$, con $a \in D-\{0\}$
    \item $x \in M \implies x = \displaystyle \sum_{i = 1}^{s}\lambda_{i}z_{i}$, con $\lambda_{i} \in D$
\end{itemize}

\noindent Séguese do anterior:

$$0 = ax = a(\displaystyle \sum_{i = 1}^{s} \lambda_{i}z_{i}) = \underbrace{\sum_{i=1}^{s}(a\lambda_{i}z_{i})}_{\in Dz_{i} \hspace{2mm} \forall i} \implies (a\lambda_{i})z_{i} = 0 \hspace{2mm} \forall \hspace{1mm} i \in \{1, \dots, s\} \implies a\lambda_{i} \in (0 : z_{i}) \hspace{2mm} \forall \hspace{1mm} i \in \{1, \dots, s\} \implies$$

$$
\implies a\lambda_{i} = 0 \hspace{2mm} \forall \hspace{1mm} i \in \{r+1, \dots, s\} \underset{\substack{\Uparrow \\ D \text{ dominio} \\ a \neq 0}}{\implies} \lambda_{i} = 0 \hspace{2mm} \forall \hspace{1mm} i \in \{r+1, \dots, s\}
$$

\vspace{3mm}

\noindent Deste xeito, tense que $x = \displaystyle \sum_{i = 1}^{r}\lambda_{i}z_{i} \in \bigoplus_{i = 1}^{r}Dz_{i}$.\\

\vspace{3mm}

\noindent Próbese agora a afirmación \eqref{ec4.4b}. Fixado $i \in \{r+1, \dots, s\}$ de xeito arbitrario, tense:

$$(0 : z_{i}) = \{0\} \implies Dz_{i} \simeq \displaystyle \frac{D}{(0:z_{i})} = D$$

\vspace{3mm}

\noindent Así, obtense:

$$\displaystyle Dz_{r+1} \oplus \dots \oplus Dz_{s} \simeq D \oplus \overset{\substack{s - r\\ \qquad}}{\dots} \oplus D = D^{s-r}$$

\vspace{3mm}

\noindent Logo, sendo isomorfo ao $D$-módulo libre $D^{s-r}$, tense que $L = Dz_{r+1} \oplus \dots \oplus Dz_{s}$ é libre. $\square$\\

\vspace{3mm}

\begin{corollary} \label{cor4.2}
Sexa $D$ un dominio de ideais principais e $M$ un $D$-módulo de tipo finito. Entón, cúmprese:
\begin{center}
    M é libre $\Longleftrightarrow$ M é libre de torsión
\end{center}
\end{corollary}

\vspace{2mm}

\noindent \textbf{\textit{\underline{Demostración}}}

\vspace{2mm}

\noindent \fcolorbox{magenta}{white}{$\Longrightarrow$/} Certo sempre (é un exercicio do boletín do tema 3).\\

\noindent \fcolorbox{magenta}{white}{$\Longleftarrow$/} Segundo o \hyperref[cor4.1]{\magbf{Corolario 4.1}}, $M = T(M) \oplus L$, sendo $L$ un submódulo libre.\\

\noindent Por hipótese, $M$ é libre de torsión, o cal por definición implica que $T(M) = \{0\}$. Deste xeito, $M = L$, logo $M$ é libre. $\square$

\magbf{\section{Módulos de torsión e compoñentes primarias}}

\vspace{5mm}

\noindent \textbf{Definición 4.6}. Sexa $D$ un $DIP$ e $M$ un $D$-módulo de tipo finito. Sexa $p$ un elemento primo de $D$ (i.e. $(p)$ é un ideal primo non trivial, ou equivalentemente, $p$ é irreducible). Denomínase \magbf{compoñente $p$-primaria de $M$} ao seguinte subconxunto:\\
\begin{center}
    \fcolorbox{magenta}{white}{$ M_{p} := \{y \in M \hspace{1mm} | \hspace{1mm} \exists \hspace{1mm} k \in \mathbb{N} \text{ con } p^{k}y = 0\}$}
\end{center}
\noindent isto é, os elementos anulados por algunha potencia de $p$.\\

\noindent \textbf{Exercicio}. Demostrar que $M_{p}$ é un $D$-submódulo de $M$, e que $M_{p} \subset T(M)$.\\

\noindent \textbf{Definición 4.7}. Se $M_{p} = M$, dise que $M$ é un \magbf{módulo $p$-primario}.\\

\noindent Como exemplo, cada $D$-módulo cíclico da forma $\displaystyle \frac{D}{(p^{e})}$ é $p$-primario.\\

\vspace{3mm}

\begin{lemma} \label{lem4.3}
Sexan $p_{1}, \dots, p_{h}$ primos distintos dun dominio de ideais principais $D$. Entón, dado un $D$-módulo $M$, a suma das compoñentes $p_{i}$-primarias $\hspace{2mm} \forall i \in \{1, \dots, h\}$ é unha suma directa:
\begin{center}
    $\displaystyle \sum_{i = 1}^{h}M_{p_{i}} = \bigoplus_{i = 1}^{h}M_{p_{i}}$ 
\end{center}
\end{lemma}

\vspace{2mm}

\noindent \textbf{\textit{\underline{Demostración}}}

\vspace{2mm}

\noindent Para demostrar que a suma é directa, véxase que $M_{p_{k}} \cap \left(\displaystyle \sum_{i \neq k}M_{p_{i}}\right) = \{0\}$ \hspace{2mm} $\forall \hspace{1mm} k \in \{1, \dots, h\}$.

\noindent Sexa así $y \in M_{p_{1}} \cap \left(\displaystyle \sum_{i = 2}^{h}M_{p_{i}}\right)$. Tense:

\begin{align*}
\exists & \hspace{1mm} k_{1} \in \mathbb{N} \hspace{1mm} | \hspace{1mm} p_{1}^{k_{1}} \cdot y = 0 \\
\exists & \hspace{1mm} k_{2}, \dots, k_{h} \in \mathbb{N} \hspace{1mm} | \hspace{1mm} p_{2}^{k_{2}} \cdot \ldots \cdot p_{h}^{k_{h}} \cdot y = 0
\end{align*}

\vspace{3mm}

\noindent Sendo $p_{1}, \dots, p_{h}$ primos, tense que $mcd(p_{1}^{k_{1}}, p_{2}^{k_{2}} \cdot \ldots \cdot p_{h}^{k_{h}}) = 1$. Sendo $D$ un dominio de ideais principais, polo \hyperref[lem2.2]{\magbf{Teorema de Bézout}}:
$$(1) = (p_{1}^{k_{1}}) + (p_{2}^{k_{2}} \cdot \ldots \cdot p_{h}^{k_{h}})$$

\vspace{3mm}

\noindent Deste xeito, pódese escribir:
$$1 = \lambda \cdot p_{1}^{k_{1}} + \mu \cdot p_{2}^{k_{2}} \cdot \ldots \cdot p_{h}^{k_{h}} \text{ con } \lambda, \mu \in D$$

\vspace{3mm}

\noindent Entón, obtense:
$$y = 1 \cdot y  = (\lambda \cdot p_{1}^{k_{1}} + \mu \cdot p_{2}^{k_{2}} \cdot \ldots \cdot p_{h}^{k_{h}}) \cdot y = \lambda \cdot p_{1}^{k_{1}}y + \mu \cdot p_{2}^{k_{2}} \cdot \ldots \cdot p_{h}^{k_{h}} \cdot y = 0 + 0 = 0 \implies y = 0 \hspace{5mm}
$$

\vspace{3mm}

\noindent Repetindo este razoamento para cada $k \in \{2, \dots, h\}$, tense probado que a suma é directa. $\square$\\

\pagebreak

\begin{lemma} \label{lem4.4}
Sexa $D$ un dominio de ideais principais e $M$ un $D$-módulo arbitrario. Considérense $x \in M$ e $d \in D$ tales que $(0 : x) = (d)$. Se $d = u \cdot p_{1}^{e_{1}} \cdot \ldots \cdot p_{t}^{e_{t}}$, con $p_{i} \neq p_{j}$ se $i \neq j$, é a descomposición de $d$ en factores primos, verifícase que existen $x_{1}, \dots, x_{t} \in M$ tales que:
\begin{center}
    $Dx = Dx_{1} \oplus \ldots \oplus Dx_{t}$ 
\end{center}
con $(0 : x_{i}) = (p_{i}^{e_{i}})$ para cada $i \in \{1, \dots, t\}$. En particular, cada submódulo $Dx_{i}$ é $p_{i}$-primario.
\end{lemma}

\vspace{2mm}

\noindent \textbf{\textit{\underline{Demostración}}}

\vspace{2mm}

\noindent Para cada $i \in \{1, \dots, t\}$, sexa $d_{i} := u \cdot p_{1}^{e_{1}} \cdot \ldots \cdot p_{i-1}^{e_{i-1}} \cdot p_{i+1}^{e_{i+1}} \cdot \ldots \cdot p_{t}^{e_{t}}$, de xeito que $d = d_{i}p_{i}^{e_{i}}$.\\

\noindent Defínanse os elementos $x_{i} := d_{i}x \in M$. Probarase a continuación que $(0 : x_{i}) = (p_{i}^{e_{i}})$:\\

\noindent \fcolorbox{magenta}{white}{$\supset$/} É inmediato, pois $p_{i}^{e_{i}}x_{i} = p_{i}^{e_{i}}d_{i}x = dx \overset{\overset{(0:x) = (d)}{\Downarrow}}{=} 0$\\

\noindent \fcolorbox{magenta}{white}{$\subset$/} Sexa $\lambda \in (0 : x_{i})$, de xeito que $\lambda x_{i} = 0$. Como $x_{i} = d_{i}x$, obtense que $\lambda d_{i} = (0 : x) = (d)$. Logo, existe $\mu \in D$ tal que $\lambda d_{i} = \mu d$. Así:

$$\lambda d_{i} = \mu d = \mu d_{i} p_{i}^{e_{i}} \implies \lambda = \mu p_{i}^{e_{i}} \in (p_{i}^{e_{i}})$$

\vspace{2mm}

\noindent Véxase a continuación que $Dx = Dx_{1} \oplus \dots \oplus Dx_{t}$. Agora ben, cómpre observar, en primeiro lugar, que $x_{i} \in M_{p_{i}}$, en consecuencia, $Dx_{i} \subset M_{p_{i}}$. Sábese, segundo o \hyperref[lem4.3]{\magbf{Lema 4.3}}, que a suma das compoñentes $p_{i}$ primarias de $M$ é directa. Polo tanto, $\overset{t}{\underset{i = 1}{Dx_{i}}}$ é tamén unha suma directa.\\

\noindent Polo tanto, a demostración redúcese a comprobar que $Dx = \displaystyle \sum_{i = 1}^{t}Dx_{t}$.\\

\noindent \fcolorbox{magenta}{white}{$\supset$/} $x_{i} \in d_{i}x \implies x_{i} \in Dx \implies Dx_{i} \subset Dx \hspace{2mm} \forall i \in \{1, \dots, t\} \implies Dx_{1} + \dots + Dx_{t} \subset Dx$\\

\noindent \fcolorbox{magenta}{white}{$\subset$/} Tense que $mcd(d_{1}, \dots, d_{t}) = 1$, por definición destes elementos. Sendo $D$ un dominio de ideais principais, pódese aplicar o \hyperref[lem2.2]{\magbf{Teorema de Bézout}} para obter:

$$1 = \displaystyle \sum_{i = 1}^{t}\alpha_{i}d_{i} \implies x = 1 \cdot x = \sum_{i = 1}^{t}\alpha_{i}d_{i}x = \sum_{i = 1}^{t}\alpha_{i}x_{i} \in \sum_{i = 1}^{t}Dx_{i} \implies Dx \subset Dx_{1} + \dots Dx_{t} \hspace{5mm} \square$$

\vspace{3mm}

\begin{theorem} \label{th4.3}
Sexa $D$ un dominio de ideais principais e $M$ un $D$-módulo de torsión de tipo finito. Entón, tense:
\begin{enumerate}
    \item $M = \displaystyle \bigoplus_{p}M_{p}$, onde a suma se estende a tódolos primos $p$ tales que $M_{p} \neq \{0\}$. Só existe un número finito de tales primos.
    \item Cada compoñente $p$-primaria $M_{p}$ pódese expresar da seguinte maneira:
        \begin{center}
            $M_{p} = \displaystyle \bigoplus_{i = 1}^{r}Dz_{i}$
        \end{center}
    onde $(0 : z_{i}) = (p^{e_{i}})$. Os expoñentes $e_{i}$ forman unha sucesión crecente, cumprindo:
        \begin{center}
            $1 \leq e_{1} \leq \ldots \leq e_{r}$
        \end{center}
    \item A sucesión dos expoñentes $e_{i}$ está determinada de xeito único
\end{enumerate}
\end{theorem}

\vspace{2mm}

\noindent \textbf{\textit{\underline{Demostración}}}

\vspace{2mm}

\noindent \magbf{(1)} Segundo o \hyperref[th4.2]{\magbf{Teorema de estrutura}}:

$$M = Dx_{1} \oplus \ldots \oplus Dx_{n}$$

\noindent onde os $x_{k}$ cumpren $(0 : x_{1}) \supset \dots \supset (0 : x_{n})$, con $(0 : x_{k}) \neq D$. Sendo ademais $M$ un $D$-módulo de torsión, tense tamén que $(0 : x_{k}) \neq \{0\}$.\\

\noindent Como $D$ é un DIP, para cada $k \in \{1, \dots, n\}$, existe un elemento $d_{k} \in D$ tal que $(0 : x_{k}) = (d_{k})$. Entón, cúmprese que $d_{1} | d_{2} | \dots | d_{n}$.\\

\noindent Sexan $p_{1}, \dots, p_{h}$ os factores primos, distintos 2 a 2, que aparecen nas factorizacións dos elementos $d_{k}$, con $1 \leq k \leq n$. Tense:

$$d_{k} = u_{k} \cdot p_{1}^{e_{1k}} \cdot \ldots \cdot p_{h}^{e_{hk}}$$

\vspace{2mm}

\noindent  con $u_{k} \in \mathcal{U}(D)$ e $e_{ik} \geq 0 \hspace{2mm} \forall i \in \{1, \ldots, h\}$ (será $e_{ik} = 0$ cando $p_{i}$ non apareza na factorización de $d_{k}$).\\

\noindent En base ao \hyperref[lem4.4]{\magbf{Lema 4.4}}, pódese escribir:

$$Dx_{k} = Dx_{1k} \oplus \dots \oplus Dx_{hk}$$

\vspace{2mm}

\noindent cumpríndose que $(0 : x_{ik}) = (p_{i}^{e_{ik}})$. Obsérvese que se $p_{i}$ non aparece na factorización de $d_{k}$, i.e. se $e_{ik} = 0$, entón $(0 : x_{ik}) = D$, logo necesariamente
$x_{ik} = 0$.\\

\noindent Así, resulta que $Dx_{k} \subset  M_{p_{1}} + \dots + M_{p_{h}}$; polo tanto, $M = M_{p_{1}} + \ldots + M_{p_{h}}$, e segundo o \hyperref[lem4.3]{\magbf{Lema 4.3}}, queda garantido que esta suma é directa:

$$M = \displaystyle \bigoplus_{i = 1}^{h}M_{p_{i}}$$

\vspace{2mm}

\noindent Se $p$ é outro primo tal que $p \notin \{p_{i}\}_{i = 1}^{h}$, séguese, aplicando de novo o \hyperref[lem4.3]{\magbf{Lema 4.3}}:

$$M_{p}  = M_{p} \cap M = M_{p} \cap \displaystyle \left(\sum_{i = 1}^{h}M_{p_{i}}\right) = \{0\}$$

\vspace{2mm}

\noindent Con isto, queda demostrado que existe un número finito de compoñentes primarias na suma directa.

\noindent \magbf{(2)} As igualdades seguintes:
\begin{align*}
    d_{1} & = u_{1} \cdot p_{1}^{e_{11}} \cdot \ldots \cdot p_{h}^{e_{h1}} \\
    d_{2} & = u_{2} \cdot p_{1}^{e_{12}} \cdot \ldots \cdot p_{h}^{e_{h2}} \\
    \vdots \\
    d_{n} & = u_{n} \cdot p_{1}^{e_{1n}} \cdot \ldots \cdot p_{h}^{e_{hn}}
\end{align*}

\vspace{2mm}

\noindent xunto co feito de que $d_{1} | d_{2} | \ldots | d_{n}$ dán lugar ao seguinte: 

$$e_{i1} \leq e_{i2}  \leq \ldots \leq e_{in} \hspace{7mm} 1 \leq i \leq h$$

\vspace{2mm}

\noindent A segunda parte da proposición é consecuencia disto último, así coma da seguinte igualdade, que se probará a continuación:

$$M_{p_{i}} = Dx_{i1} \oplus \dots \oplus Dx_{in} \hspace{5mm} 1 \leq i \leq h$$

\vspace{2mm}

\noindent \fcolorbox{magenta}{white}{$\supset$/} Tense que $p_{i}^{e_{ik}}x_{ik} = 0$. Logo, $x_{ik} \in M_{p_{i}}$, podendo afirmar entón que $Dx_{ik} \subset M_{p_{i}}$ e así, en consecuencia, garántese que $Dx_{i1} \oplus \dots \oplus Dx_{in} \subset M_{p_{i}}$ \\

\noindent \fcolorbox{magenta}{white}{$\subset$/} Fixado $i \in \{1, \ldots, h\}$ de xeito arbitrario, sexa $x \in M_{p_{i}} \subset M$. Lémbrese que $M$ se pode escribir:
$$M = Dx_{1} \oplus \dots \oplus Dx_{n} = (Dx_{11} \oplus \dots \oplus Dx_{h1}) \oplus \dots \oplus (Dx_{1n} \oplus \dots \oplus Dx_{hn}) $$

\vspace{2mm}

\noindent Entón, pódese expresar $x$ do seguinte xeito:

\begin{equation}\label{ec4.5}
x = \displaystyle \sum_{r = 1}^{n} \sum_{l = 1}^{h} \lambda_{lr}x_{lr} = \sum_{l = 1}^{h} \underbrace{\left (\sum_{r = 1}^{n}\lambda_{lr}x_{lr} \right )}_{\in M_{p_{l}}}
\end{equation}

\vspace{2mm}

\noindent con $x_{lr} \in Dx_{lr} \subset M_{p_{l}}$.\\

\noindent Recórdese que a suma $M_{p_{1}} + \dots + M_{p_{h}}$ é directa. Logo, como $x \in M_{p_{i}}$, tódolos sumandos da expresión \eqref{ec4.5} tales que $l \neq i$ son nulos. Así, a expresión anterior redúcese a:

$$x = \displaystyle \sum_{r = 1}^{n} \lambda_{ir}x_{ir}$$

\vspace{2mm}

\noindent e así, $x \in Dx_{i1} \oplus \dots \oplus Dx_{in}$. \\

\noindent \magbf{(3)} Falta por demostrar a unicidade dos expoñentes $e_{1}, \dots, e_{r}$.\\

\noindent Denótese, para abreviar, $N = M_{p}$, de xeito que $N = \displaystyle \bigoplus_{i = 1}^{r}Dz_{i}$,  con $(0 : z_{i}) = (p^{e_{i}})$, $1 \leq e_{1} \leq \dots \leq e_{r}$.\\

\noindent Entón, cúmprese que $N \simeq \displaystyle \frac{D}{(p^{e_{1}})} \oplus \dots \oplus \frac{D}{(p^{e_{r}})}$.\\

\noindent En primeiro lugar, probarase que $e_{r}$ é o menor enteiro $n$ tal que $p^{n}N = \{0\}$:\\

\noindent \fcolorbox{magenta}{white}{$\geq$} $p^{e_{r}}N = \{0\}$, pois $p^{e_{r}}\displaystyle \frac{D}{(p^{e_{i}})} = \{0\}$ para cada $i \in \{1, \dots, r\}$. Véxase isto:

$$p^{e_{r}} \cdot [a + (p^{e_{i}})] = p^{e_{r}} \cdot a + (p^{e_{i}}) \underset{\magbf{[1]}}{=} 0$$

\vspace{2mm}

\noindent Con isto, garántese que $p^{e_{r}} \geq n$. \\

\noindent \fcolorbox{magenta}{white}{$\leq$} Como $p^{n}N = \{0\}$, en particular tense que $p^{n} \displaystyle \frac{D}{(p^{e_{r}})}$. Séguese disto:

$$p^{n} \cdot [1 + (p^{e_{r}})] = 0 \implies p^{n} + (p^{e_{r}}) = 0 \implies p^{n} \in (p^{e_{r}}) \implies p^{e_{r}} | p^{n} \implies e_{r} \leq n$$

\vspace{2mm}

\noindent Deste xeito, $e_{r}$ está determinado de modo único por $N$. Chamaráselle \magbf{$p$-expoñente de $N$}, e razoarase por indución sobre el para demostrar a unicidade da sucesión de expoñentes.\\

\noindent Supóñanse dúas sucesións de expoñentes $e_{1}, \dots, e_{r}$ e $f_{1}, \dots, f_{s}$, de xeito que cumpren:
\begin{align*}
    N & \simeq \displaystyle \frac{D}{(p^{e_{1}})} \oplus \dots \oplus \frac{D}{(p^{e_{r}})} & & 1 \leq e_{1} \leq \dots \leq e_{r} \\
    \vspace{3mm}\\
    N & \simeq \displaystyle \frac{D}{(p^{f_{1}})} \oplus \dots \oplus \frac{D}{(p^{f_{s}})} & & 1 \leq f_{1} \leq \dots \leq f_{s}
\end{align*}

\noindent Probarase o seguinte: 

\begin{itemize}
    \item $r = s$
    \item $e_{i} = f_{i} \hspace{2mm} \forall i$
\end{itemize}

\noindent Cómpre ter en conta que xa se sabe que $e_{r} = f_{s}$, porque ámbolos dous son $p$-expoñentes de $N$, logo necesariamente son iguais.\\

\noindent Procédase entón:

\begin{enumerate}
    \item  Se $e_{r} = f_{s} = 1$, entón obtense que $e_{i} = f_{j} = 1 \hspace{2mm} \forall i \in \{1, \dots, r\} \hspace{2mm} \forall j \in \{1, \dots, s\}$. Deste xeito, $N$ é un $\displaystyle \frac{D}{(p)}$ - espazo vectorial$^{\magbf{[2]}}$, e en base á igualdade de dimensións, $r = s$.\\
    
    \item Sexa agora algún $e_{r} = f_{s} > 1$, e supóñase certo o resultado para módulos de $p$-expoñente estritamente menor ca $e_{r} = f_{s}$.\\
    
    Considérense $m,n$ tales que $e_{i} \geq 2 \Longleftrightarrow i \geq m$ e $f_{j} \geq 2 \Longleftrightarrow j \geq n$. Entón:
    
    \begin{align} \label{ec4.6}
        pN \simeq \displaystyle \bigoplus_{i = m}^{r} \frac{D}{(p^{e_{i} - 1})} & & pN \simeq \bigoplus_{j = n}^{s} \frac{D}{(p^{f_{j} - 1})}
    \end{align}
    
    En efecto, tense:
    
    $$ N \simeq \left ( \displaystyle \bigoplus_{i = 1}^{m-1} \frac{D}{(p)} \right ) \oplus \left ( \bigoplus_{i = m}^{r} \frac{D}{(p^{e_{i}})} \right ) \hspace{5mm} e_{i} \geq 2$$
    
    \vspace{2mm}
    
    e polo tanto
    
    $$pN \simeq \equalto{\underbrace{\left ( \displaystyle \bigoplus_{i = 1}^{m-1} \frac{pD}{(p)}\right )}}{\{0\}} \oplus \left ( \bigoplus_{i = m}^{r} \frac{pD}{(p^{e_{i}})}\right )$$
    
    \vspace{2mm}
    
    Ademais, $\displaystyle \frac{pD}{(p^{e_{i}})} \simeq \frac{D}{(p^{e_{i} - 1})}$, pois tense:\\
    
    $$
     \large \xymatrix{
        (p^{e_{i} - 1}) \ar@{_{(}->}[d]_{i} \ar[r] & (p^{e_{i}}) \ar@{_{(}->}[d]^{i} \\
        D \ar@{->}[d]_{p_{1}} \ar@{->}[r]^{\phi} & pD \ar@{->}[d]^{p_{2}}\\
        \displaystyle \frac{D}{(p^{e_{i} -1})} \ar@{->}[r]^{\theta} & \displaystyle \frac{pD}{(p^{e_{i}})}
    }    
    $$
    
    onde os homomorfismos
    \begin{align*}
        p_{1}(x) & = x + (p^{e_{i} -1}) \\
        p_{2}(px) & = px + (p^{e_{i}}) \\
        \phi(x) & = px
    \end{align*}
    son sobrexectivos: os dous primeiros como proxeccións canónicas dos dominios sobre os seus respectivos cocientes, e o terceiro, como consecuencia da \hyperref[prop3.10]{\magbf{definición equivalente de módulo cíclico}}.\\
    
    A sobrexectividade das aplicacións anteriores garante, á súa vez, a sobrexectividade do homomorfismo
    \begin{align*}
        \theta: \displaystyle \frac{D}{(p^{e_{i} -1})} & \longrightarrow \frac{pD}{(p^{e_{i}})}\\
         x + (p^{e_{i} -1}) & \leadsto  px + (p^{e_{i}})
    \end{align*}
    
    Verase a continuación que $\theta$ tamén é inxectivo:
    $$px + (p^{e_{1}}) = 0 \implies px \in (p^{e_{i}}) \implies p^{e_{i}} | px \implies p^{e_{i} -1} | x \implies x \in (p^{e_{i} -1}) \implies x + (p^{e_{i} -1}) = 0$$
    
    \vspace{2mm}
    
    Isto proba que $pN \simeq \displaystyle \bigoplus_{i = m}^{r} \frac{D}{(p^{e_{i} -1})}$, e analogamente tense o outro isomorfismo de \eqref{ec4.6}.\\
    
    Estes isomorfismos, xunto coa \textbf{hipótese de indución}, dán lugar ás igualdades:
    \begin{align*}
        r - m & = s - n  \\
        e_{m} & = f_{n} \\
        \vdots \\
        e_{r} & = f_{s} 
    \end{align*}
    
    \vspace{2mm}
    
    Doutra banda, sucede que:
    
    \begin{align} \label{ec4.7}
        \displaystyle \frac{N}{pN} \simeq \bigoplus_{i = 1}^{r} \frac{D}{(p)} & & \frac{N}{pN} \simeq \bigoplus_{i = 1}^{s} \frac{D}{(p)}
    \end{align}
    
    En efecto:
    
    \[
    N \simeq \displaystyle \frac{D}{(p)} \oplus \overset{^{m-1}}{\dots} \oplus \frac{D}{(p)} \oplus \frac{D}{(p^{e_{m}})} \oplus \dots \oplus \frac{D}{(p^{e_{r}})}
    \]
    
    \vspace{3mm}
    
    Tense o seguinte esquema:
    $$
     \large \xymatrix@C-1pc{ %@C - modificar espazo entre columnas; @R - modificar espazo entre filas
        pN \ar@{_{(}->}[d]_{i} & \simeq & \{0\} \ar@{_{(}->}[d] & \oplus &  \overset{^{m-1}}{\dots} & \oplus & \{0\} \ar@{_{(}->}[d] & \oplus & \displaystyle \frac{pD}{(p^{e_{m}})} \ar@{_{(}->}[d] & \oplus & \dots & \oplus & \displaystyle \frac{pD}{(p^{e_{r}})} \ar@{_{(}->}[d]\\
        N \ar[d]_{p} & \simeq & \displaystyle \frac{D}{(p)} \ar[d] & \oplus & \overset{^{m-1}}{\dots} & \oplus &  \displaystyle \frac{D}{(p)} \ar[d] & \oplus & \displaystyle \frac{D}{(p^{e_{m}})} \ar[d]_{\pi_{m}} & \oplus & \dots & \oplus & \displaystyle \frac{D}{(p^{e_{r}})} \ar[d]_{\pi_{r}}\\
        \displaystyle \frac{N}{pN} & \simeq & \displaystyle \frac{D}{(p)} & \oplus & \overset{^{m-1}}{\dots} & \oplus & \displaystyle \frac{D}{(p)} & \oplus & \displaystyle \frac{D}{(p)} & \oplus & \dots & \oplus & \displaystyle \frac{D}{(p)}
    }    
    $$    
    
    Agora ben, para cada $i \in \{m, \dots, r\}$, cúmprese que o homomorfismo $\pi_{i}$ é sobrexectivo, pois a proxección canónica sobre o cociente é un epimorfismo:
    
    \[
    p_{i} : \displaystyle \frac{D}{(p^{e_{i}})} \longrightarrow \frac{D/(p^{e_{i}})}{pD/(p^{e_{i}})}
    \]
    
    \vspace{2mm}
    
    e aplicando o \hyperref[th3.2]{\magbf{2.º Teorema de Isomorfía de módulos}}, obtense que $\displaystyle \frac{D/(p^{e^{i}})}{pD/(p^{e_{i}})} \simeq \frac{D}{pD}$\\
    
    Deste xeito, $\pi_{i}$ é a composición de $p_{i}$ cun isomorfismo, logo é un epimorfismo.\\
    
    Este mesmo procedemento demostra o outro isomorfismo en \eqref{ec4.7}.\\
    
    Así, pódese concluír que se ten un isomorfismo de $\displaystyle \frac{D}{(p)}$ - espazos vectoriais
    
    \[
    \displaystyle \bigoplus_{i = 1}^{r} \frac{D}{(p)} \simeq \bigoplus_{i = 1}^{s} \frac{D}{(p)}
    \]
    
    \vspace{2mm}
    
    e entón, segundo a igualdade de dimensións, $r = s$. Ademais:
    
    \[ 
        \left. \begin{array}{r} 
        r = s\\[1ex]
        r - m = s - n
        \end{array} \right\}
        \implies m = n
    \] 
    
    \vspace{2mm}
    
    garantindo logo:
    
    \begin{align*}
        e_{1} & = f_{1} = 1 \\
        & \vdots\\
        e_{m-1} & = f_{n-1} = 1
    \end{align*}
    
    \vspace{2mm}
    
    quedando así probada a unicidade da sucesión de expoñentes. $\square$
    
\end{enumerate}

\vspace{2mm}

\noindent $^{\magbf{[1]}}$ \textit{Isto cúmprese porque $a \cdot p^{e_{r}} \in (p^{e_{i}})$}\\

\noindent $^{\magbf{[2]}}$ \textit{Como $p$ é primo, $(p)$ é un ideal primo. Sendo $D$ un $DIP$, tense que $(p)$ é maximal, logo o cociente $\frac{D}{(p)}$ é un corpo}.

\vspace{5mm}

\magbf{\section{Invariantes}}

\vspace{5mm}

\noindent Sexa $D$ un dominio de ideais principais e $M$ un $D$-módulo de torsión de tipo finito. Ao longo deste tema víronse dúas descomposicións posibles para $M$, tal e como se recolleu na demostración do teorema anterior:\\

\renewcommand{\theenumi}{\roman{enumi})}
\renewcommand{\labelenumi}{\theenumi}

\begin{enumerate}

    \item Segundo o \hyperref[th4.2]{\magbf{Teorema de estrutura}}, pódese escribir:
    
    $$M = Dx_{1} \oplus \cdots \oplus Dx_{n}$$
    
    \vspace{2mm}
    
    con $(0 : x_{1}) \supset (0 : x_{2}) \supset \dots \supset (0 : x_{n})$, e $(0 : x_{k}) \neq D \hspace{2mm} \forall k$. Sendo ademais $M$ módulo de torsión, tamén se verifica que $(0 : x_{k}) \neq \{0\}$ para cada $k$.\\
    
    Se $d_{k}$ é tal que $(0 : x_{k}) = (d_{k})$, tense que $d_{1} \hspace{1mm} | \hspace{1mm} d_{2} \hspace{1mm} | \hspace{1mm} \dots \hspace{1mm} | \hspace{1mm} d_{n}$, e así:
    $$M \simeq \displaystyle \frac{D}{(d_{1})} \oplus \dots \oplus \frac{D}{(d_{n})}$$
\end{enumerate}

\noindent \textbf{Definición 4.8}. Os ideais $(d_{k})$ denomínanse \magbf{factores invariantes de $M$}.\\

\begin{enumerate}
    \setcounter{enumi}{1}
    \item Doutra banda, tal e como se demostrou no \hyperref[th4.3]{\magbf{Teorema 4.3}}, $M$ admite a seguinte descomposición:
    \begin{align*}
        M = & (Dx_{11} \oplus \dots \oplus Dx_{1n}) \oplus \\
        & \vdots \\
        \oplus & (Dx_{i1} \oplus \dots \oplus Dx_{in}) \oplus \\
        & \vdots \\
        \oplus & (Dx_{h1} \oplus \dots \oplus Dx_{hn}) 
    \end{align*}
    
    con $(0 : x_{ik}) = (p_{i}^{e_{ik}})$, $1 \leq k \leq n$, cumpríndose ademais que $e_{i1} \leq e_{i2} \leq \dots \leq e_{in}$, $1 \leq i \leq h$. Logo, verifícase o seguinte:
    \begin{align*}
        M \simeq & \displaystyle \left (\frac{D}{(p_{1}^{e_{11}})} \oplus \dots \oplus \frac{D}{(p_{1}^{e_{1n}})} \right ) \oplus \\
        & \vdots \\
        \oplus & \left ( \frac{D}{(p_{i}^{e_{i1}})} \oplus \dots \oplus \frac{D}{(p_{i}^{e_{in}})} \right ) \oplus \\
        & \vdots \\
        \oplus & \left (\frac{D}{(p_{h}^{e_{h1}})} \oplus \dots \oplus \frac{D}{(p_{h}^{e_{hn}})} \right )
    \end{align*}
\end{enumerate}

\noindent \textbf{Definición 4.9}. Os ideais $(p_{i}^{e_{ik}})$ denomínanse \magbf{divisores elementais de $M$}.\\

\noindent Nunha descomposición de tipo (ii) tense:
$$M_{p_{i}} = Dx_{i1} \oplus \dots \oplus Dx_{in} \simeq \displaystyle \frac{D}{(p_{i}^{e_{i1}})} \oplus \dots \oplus \frac{D}{(p_{i}^{e_{in}})} \hspace{5mm} 1 \leq i \leq h$$

\noindent Na demostración previa viuse como obter unha descomposición de tipo (ii) a partir dunha descomposición de tipo (i). \\

\noindent Do teorema anterior séguese que os divisores elementais son \textbf{invariantes} de $M$ (i.e. só dependen do módulo $M$, e non da descomposición).\\

\noindent A relación entre os elementos $d_{k}$ e $p_{i}^{e_{ik}}$ é a seguinte:
$$d_{k} = u_{k} \cdot p_{1}^{e_{1k}} \cdot \dots \cdot p_{h}^{e_{hk}} \hspace{7mm} u_{k} \in \mathcal{U}(D)$$

\noindent Se os elementos $p_{i}^{e_{ik}}$ son coñecidos, pódense reconstruír, salvo unidades, os $d_{k}$. Polo tanto, os factores invariantes están determinados polos divisores elementais.\\

\noindent En consecuencia, os factores invariantes son tamén \textbf{invariantes} de $M$.\\

\noindent Por todo o anterior, cúmprese o seguinte:\\

\renewcommand{\theenumi}{\arabic{enumi}}
\renewcommand{\labelenumi}{\theenumi.}

\begin{proposition} \label{prop4.2}
Sexa $D$ un dominio de ideais principais. Considérense $M$ e $N$ dous $D$-módulos de torsión de tipo finito. Entón, verifícase:
\begin{enumerate}
    \item $M$ e $N$ son isomorfos cando, e só cando, posúen os mesmos factores invariantes.
    \item $M$ e $N$ son isomorfos cando, e só cando, posúen os mesmos divisores elementais.
\end{enumerate}
\end{proposition}

\vspace{2mm}

\noindent \textbf{\textit{\underline{Demostración}}}

\vspace{2mm}

\noindent \magbf{Proba de (1)} \\

\noindent \fcolorbox{magenta}{white}{$\Longrightarrow$/} Supóñase que existe un isomorfismo $\phi: M \longrightarrow N$.\\

\noindent Considérese $M = Dx_{1} \oplus \dots \oplus Dx_{n}$ unha descomposición de tipo (i), cumpríndose $(0 : x_{1}) \supset \dots \supset (0 : x_{n})$, con $(0 : x_{k}) \neq D$ e $(0 : x_{k}) \neq \{0\}$, isto último por ser $M$ módulo de torsión.\\

\noindent Como $\phi$ é un isomorfismo, tense que $N = \phi(M) = D\phi(x_{1}) \oplus \dots \oplus D\phi(x_{n})$, a cal é unha descomposición de tipo (i) para $N$.\\

\noindent Tense, para cada $k \in \{1, \dots, n\}$, que $(0 : \phi(x_{k})) = (0 : x_{k})$, pois:
$$a \in (0 : x_{k}) \implies 0 = \phi(ax_{k}) = a \phi(x_{k}) \implies a \in (0 : \phi(x_{k}))$$

\noindent Así, os factores invariantes de $M$ e $N$ coinciden.\\

\noindent \fcolorbox{magenta}{white}{$\Longleftarrow$/} Reciprocamente, supóñase agora que $M$ e $N$ posúen factores invariantes idénticos, $(d_{1}), \dots, (d_{n})$. Entón, tense:
\begin{align*}
    M \simeq & \displaystyle \frac{D}{(d_{1})} \oplus \dots \oplus \frac{D}{(d_{n})} \\
    N \simeq & \displaystyle \frac{D}{(d_{1})} \oplus \dots \oplus \frac{D}{(d_{n})}  
\end{align*}

\noindent Así, efectivamente, $M \simeq N$.\\

\noindent \magbf{Proba de (2)}\\

\noindent É análoga á de \magbf{(1)}, facendo uso neste caso da descomposición de tipo (ii). $\square$\\

\vspace{3mm}

\noindent O \hyperref[th4.2]{\magbf{Teorema de estrutura}}, tal e como se enunciou nestes apuntamentos, garante a existencia dunha descomposición en submódulos cíclicos dun módulo de tipo finito sobre un DIP, sen facer mención á súa unicidade. A continuación vaise demostrar que, en efecto, tal descomposición si é única, facendo uso da proposición anterior:\\

\begin{theorem} \label{th4.4}
Sexa $D$ un dominio de ideais principais e $M$ un $D$-módulo de tipo finito. Supóñanse dúas descomposicións de $M$ nas condicións do \magbf{Teorema de estrutura}:
\begin{center}
    $M = Dz_{1} \oplus \dots \oplus Dz_{s} = Dz'_{1} \oplus \dots \oplus Dz'_{s'}$ 
\end{center} 
Entón, verifícase:
\begin{enumerate}
    \item $s = s'$
    \item $(0 : z_{i}) = (0 : z'_{i})$ para cada $i \in \{1, \dots, s\}$
\end{enumerate}
\end{theorem}

\vspace{2mm}

\noindent \textbf{\textit{\underline{Demostración}}}

\vspace{2mm}

\noindent Sexan $r$ e $r'$ tales que:

\begin{itemize}
    \item $(0 : z_{i}) \neq \{0\}$ \hspace{2mm} $\forall \hspace{1mm} i \in \{1, \dots, r\}$ e $(0 : z_{i}) = \{0\} \hspace{2mm} r < i \leq s$
    \item $(0 : z'_{i}) \neq \{0\}$ \hspace{2mm} $\forall \hspace{1mm} i \in \{1, \dots, r'\}$ e $(0 : z'_{i}) = \{0\} \hspace{2mm} r' < i \leq s'$
\end{itemize}

\noindent Entón, cúmprese:
\begin{align*}
    T(M) = & Dz_{1} \oplus \dots \oplus Dz_{r} \\
    T(M) = & Dz'_{1} \oplus \dots \oplus Dz'_{r'}
\end{align*}

\noindent Como $T(M)$ é un $D$-módulo de torsión de tipo finito sobre un DIP, os seus factores invariantes están determinados de xeito único. Así, cúmprese que $r = r'$ e, ademais, $(0 : z_{k}) = (0 : z'_{k}) \hspace{2mm} \forall \hspace{1mm} k \in \{1, \dots, r\}$.\\

\noindent Doutra banda, tense:
\begin{align*}
    \displaystyle \frac{M}{T(M)} & & \overset{\magbf{[1]}}{\simeq} & & Dz_{r+1} \oplus \dots \oplus Dz_{s} & & \simeq & & D \oplus \overset{^{s-r}}{\dots} \oplus D & & = & & D^{s-r}\\
    \displaystyle \frac{M}{T(M)} & & \overset{\magbf{[1]}}{\simeq} & & Dz'_{r'+1} \oplus \dots \oplus Dz'_{s'} & & \simeq & & D \oplus \overset{^{s'-r'}}{\dots} \oplus D & & = & & D^{s'-r'}\\
\end{align*}

\noindent Deste xeito, $D^{s-r} \simeq D^{s' - r'}$. Sendo ámbolos dous libres, os seus rangos deben coincidir, logo $s - r = s' - r'$. Sabendo ademais que $r = r'$, queda garantido que $s = s'$, probando así $\magbf{(1)}$ (e á súa vez, tamén \magbf{(2)}). $\square$\\

\noindent $^{\magbf{[1]}}$ \textit{Segundo o \hyperref[th3.3]{\magbf{3.º Teorema de Isomorfía de módulos}}, $\displaystyle \frac{M}{T(M)} = \frac{T(M) \oplus L}{T(M)} \simeq \frac{L}{T(M) \cap L}$. Como a suma $T(M) \oplus L$ é directa, $T(M) \cap L = \{0\}$, logo $\displaystyle \frac{T(M) \oplus L}{T(M)} \simeq L$ e así $\displaystyle \frac{M}{T(M)}$ é libre}.\\

\vspace{3mm}

\noindent Para un DIP $D$ e un $D$-módulo $M$, pódese escribir $M = T(M) \oplus L$, con $L$ un submódulo libre de $M$. Tal e como se viu na demostración anterior, $\displaystyle \frac{M}{T(M)} \simeq L$. Cúmprese que o rango de $L$ é un invariante de $M$, o que dá lugar á seguinte definición:\\

\noindent \textbf{Definición 4.10}. Sexa $D$ un dominio de ideais principais e $M$ un $D$-módulo de tipo finito, de xeito que $M = T(M) \oplus L$, con $L$ un submódulo libre de $M$. Chámaselle \magbf{rango de $M$} ao rango do submódulo $L$.\\

\begin{mdframed}[linecolor = classicrose, linewidth = 1mm]
\noindent Así, un $D$-módulo $M$ de tipo finito (e que polo tanto admite a descomposición $M = T(M) \oplus L$) posúe os seguintes \textbf{invariantes}:
\begin{itemize}
    \item Os factores invariantes de $T(M)$
    \item Os divisores elementais de $T(M)$
    \item O rango de $L$
\end{itemize}
\end{mdframed}

\vspace{2mm}

\noindent Deste xeito, un módulo queda caracterizado, salvo isomorfismos, polos invariantes anteriores, tal e como recolle o seguinte resultado:\\

\begin{proposition} \label{prop4.3}
Sexa $D$ un dominio de ideais principais. Considérense $M$ e $N$ dous $D$-módulos de tipo finito. Entón, verifícase:
\begin{enumerate}
    \item $M$ e $N$ son isomorfos cando, e só cando, posúen os mesmos factores invariantes e o mesmo rango.
    \item $M$ e $N$ son isomorfos cando, e só cando, posúen os mesmos divisores elementais e o mesmo rango.
\end{enumerate}
\end{proposition}

\vspace{2mm}

\noindent \textbf{\textit{\underline{Demostración}}}

\vspace{2mm}

\noindent \magbf{Proba de (1)}\\

\noindent \fcolorbox{magenta}{white}{$\Longrightarrow$/} Supóñase que existe un isomorfismo $\phi: M \longrightarrow N$.\\

\noindent Tense o seguinte diagrama:

\[
     \large \xymatrix{
        T(M) \ar@{_{(}->}[d]_{i} \ar[r]^{\phi\restriction_{T(M)}} & T(N) \ar@{_{(}->}[d]^{i} \\
        M \ar@{->}[d]_{p_{M}} \ar@{->}[r]^{\phi} & N \ar@{->}[d]^{p_{N}}\\
        \displaystyle \frac{M}{T(M)} \ar@{->}[r]^{\overline{\phi}} & \displaystyle \frac{N}{T(N)}
    }   
\]

\vspace{2mm}

\noindent cumpríndose que $\phi\restriction_{T(M)}$ e $\overline{\phi}$ son isomorfismos. Así, obtense o seguinte:

\begin{itemize}
    \item $T(M) \simeq T(N) \implies$ $M$ e $N$ posúen os mesmos factores invariantes
    \item $\displaystyle \frac{M}{T(M)} \simeq \frac{N}{T(N)} \implies rg(M) = rg(N)$\\
\end{itemize}

\vspace{3mm}

\noindent \fcolorbox{magenta}{white}{$\Longleftarrow$/}\\
\[
rg(M) = rg(N) \implies \displaystyle rg \left ( \displaystyle \frac{M}{T(M)} \right ) = rg \left ( \displaystyle \frac{N}{T(N)} \right ) \implies \frac{M}{T(M)} \overset{\phi}{\simeq} \frac{N}{T(N)}
\]

\vspace{2mm}

\noindent Como os factores invariantes de $M$ e $N$ coinciden, tense outro isomorfismo $\psi: T(M) \longrightarrow T(N)$.\\

\noindent Así, obtense o seguinte diagrama:

\[
     \large \xymatrix{
        M \ar[d]^{\theta} & \simeq & T(M) \ar[d]^{\psi} & \oplus & \displaystyle \frac{M}{T(M)} \ar[d]^{\phi} \\
        N & \simeq & T(N) & \oplus & \displaystyle \frac{N}{T(N)} 
    }   
\]

\noindent Lémbrese que $M = T(M) \oplus L_{1}$ e que $N = T(N) \oplus L_{2}$, cumpríndose que $L_{1} \simeq \displaystyle \frac{M}{T(M)}$ e $L_{2} \simeq \displaystyle \frac{N}{T(N)}$. Por iso se teñen os isomorfismos de $M$ e $N$ coas sumas directas anteriores.\\

\noindent Logo, sendo $\psi$ e $\phi$ isomorfismos, cúmprese que $\theta$ é tamén un isomorfismo, logo $M \simeq N$, tal e como se pretendía demostrar.\\

\noindent \magbf{Proba de (2)}\\

\noindent É practicamente idéntica á de \magbf{(1)}: o isomorfismo $T(M) \simeq T(N)$ garante que $M$ e $N$ posúen os mesmos divisores elementais, e viceversa. $\square$ \\

\vspace{3mm}

\noindent \textbf{Observación 4.5}. O rango de $M$ coincide co valor $n -r$ do \hyperref[th4.2]{\magbf{Teorema de Estrutura}}. En efecto, $M$ admite unha descomposición como suma directa de $D$-módulos cíclicos:
$$M = Dy_{1} \oplus \dots \oplus Dy_{n}$$

\noindent cumpríndose

\[
\begin{cases}
(0 : y_{i}) = (d_{i}) &  1 \leq i \leq r \\
(0 : y_{i}) = \{0\} &  r + 1 \leq i \leq n
\end{cases}
\]

\vspace{2mm}

\noindent con $d_{i} \neq 0$ e verificando $d_{1} \hspace{1mm} | \hspace{1mm} d_{2} \hspace{1mm} | \hspace{1mm} \dots  \hspace{1mm}|  \hspace{1mm} d_{r}$.\\

\noindent Lémbrese que se pode escribir $M = T(M) \oplus L$, onde:
\begin{align*}
    T(M) & =  Dy_{1} \oplus \dots \oplus Dy_{r} \\
    L & = Dy_{r+1} \oplus \dots \oplus Dy_{n}
\end{align*}

\noindent e como $(0:y_{i}) = \{0\}$ para cada $i \in \{r+1, \dots, n\}$, $L \simeq D \oplus \overset{^{n-r}}{\dots} \oplus D = D^{n-r}$. Así, o rango de $L$ coincide co de $D^{n-r}$, que é $n-r$.\\

\noindent \textbf{Observación 4.6}. Ao comezo deste tema afirmábase que o \hyperref[th4.2]{\magbf{Teorema de estrutura}} é un resultado equivalente ao teorema da forma canónica de Jordan para endomorfismos de espazos vectoriais de dimensión finita. En \cite{agustin} explícase como é aplicable a teoría desta unidade á demostración dese resultado.\\

\newpage

\clearpage % end title page
\begingroup
  \pagestyle{fancy}
  \null
  \newpage
\endgroup


\noindent {\Huge \textbf{Bibliografía}}

\thispagestyle{noheader}

\renewcommand{\chapter}[2]{}% for other classes - eliminar titulo da bibliografía

\vspace{20pt}

\begin{thebibliography}{99}

\bibitem[1]{notasclase}
Rosa M. Fernández, Antonio García (2019).
\newblock{Notas de clase da materia \textit{Estruturas Alxébricas}.}

\bibitem[2]{rotman}
Joseph J. Rotman (1994). 
\newblock{\em An Introduction to the Theory of Groups.}
\newblock{Springer [4.ª edición].}

\bibitem[3]{fraleigh}
John B. Fraleigh (2003).
\newblock{\em A First Course in Abstract Algebra.}
\newblock{Addison-Wesley.}

\bibitem[4]{hartley}
Brian Hartley, Trevor O. Hawkes (1970).
\newblock{\em Rings, Modules and Linear Algebra.}
\newblock{Chapman and Hall.}

\bibitem[5]{gamboa}
José M. Gamboa, Jesús M. Ruiz (2002).
\newblock{\em Anillos y Cuerpos Conmutativos.}
\newblock{UNED [edición dixital].}

\bibitem[6]{viray}
Bianca Viray (2015).
\newblock{\em Ring Homomorphisms and The Isomorphism Theorems.}
\newblock{[Online] Consultado \href{https://sites.math.washington.edu/~bviray/teaching/RingHomomorphismsAndIsomorphisms.pdf}{\textcolor{magenta}{aquí}} por última vez o 18 de decembro de 2019.}

\bibitem[7]{agustin}
Agustín García (2003).
\newblock{\em Módulos sobre un dominio de ideales principales.}
\newblock{[Online] Consultado \href{http://www.mate.unlp.edu.ar/~demetrio/Monografias/Materias/EA/2.\%20Modulos\%20FG\%20sobre\%20un\%20dominio\%20principal\%20-\%20A.\%20Garcia\%20Iglesias\%20-\%202003.pdf}{\textcolor{magenta}{aquí}} por última vez o 28 de xullo de 2020.}
\end{thebibliography}

\end{document}
